% Possible use http://www.latextemplates.com/template/the-legrand-orange-book as template? See https://www.overleaf.com/9174958nyjxxdxbchks#/33024595/

\documentclass[11pt,letterpaper,fleqn]{memoir} % Default font size and left-justified equations

\usepackage[top=3cm,bottom=3cm,left=3cm,right=3cm,headsep=10pt,letterpaper]{geometry} % Page margins

% Theorem definitions using amsthm

\usepackage{amsthm}
\usepackage{amsmath}
\usepackage{amssymb}

% Remove line spaces between items of enumerate and itemize
\usepackage{enumitem}
\setlist{noitemsep}


% Adds double bracket symbols
\usepackage{stmaryrd}

% Latex symbol guide at http://mirrors.ibiblio.org/CTAN/info/symbols/comprehensive/symbols-letter.pdf

% LOGIC symbols
% -------------

% Allows to create negation symbols
\usepackage{MnSymbol}

\DeclareMathOperator{\truth}{truth}
\DeclareMathOperator{\poss}{possibilities}
\DeclareMathOperator{\result}{result}

\def\TRUE{\textsc{true}}
\def\FALSE{\textsc{false}}

\def\SUCCESS{\textsc{success}}
\def\FAILURE{\textsc{failure}}
\def\UNDEF{\textsc{undefined}}


% Symbols for tautology and contradiction
\def\tautology{\top}
\def\contradiction{\bot}

% Symbols for "compatibility" and "incompatibility"
\def\comp{\doublefrown}
\def\ncomp{\ndoublefrown}

% Symbols for "narrower" and "wider"
\def\narrower{\subseteq}
\def\nnarrower{\nsubseteq}
\def\broader{\supseteq}
\def\nbroader{\nsupseteq}


% Symbol for "independent" and "correlated"
\def\indep{\upmodels}
\def\nindep{\nupmodels}

% Aliases for logical operations
\def\AND{\wedge}
\def\bigAND{\bigwedge}
\def\OR{\vee}
\def\bigOR{\bigvee}
\def\NOT{\neg}


% Formatting for statements
\newcommand{\stmt}[1][s] {\mathsf{#1}}
% Formatting for experimental tests
\newcommand{\expt}[1][e] {\mathsf{#1}}
% Formatting for observations
\newcommand{\obs}[1] {\mathsf{#1}}
% Formatting for observation definition
\newcommand{\obsdef}[2] {\llparenthesis #1, #2 \rrparenthesis}

% Formatting for experimental domain
\newcommand{\edomain}[1][D] {\mathcal{#1}}

% Formatting for theoretical domain
\newcommand{\tdomain}[1][D] {\bar{\mathcal{#1}}}


% Formatting for sentence statements
\newcommand{\statement}[1] {\emph{``#1"}}


\usepackage{xcolor} % Required for specifying colors by name
\definecolor{sectionNumbers}{RGB}{44, 103, 0}


\renewcommand\thesubsection{\thesection.\Alph{subsection}}
\renewcommand{\theequation}{\thechapter.\arabic{equation}}

\newtheorem{assump}{Assumption}
\renewcommand*{\theassump}{\Roman{assump}}

\newtheorem{defn}[equation]{Definition}
\newtheorem{prop}[equation]{Proposition}

%\theoremstyle{definition}

\newenvironment{rationale}{\emph{Rationale}.}{\qed}
\newenvironment{justification}{\emph{Justification}.}{\qed}
\renewenvironment{proof}{\emph{Proof}.}{\qed}

% Style for math section
\RequirePackage[framemethod=default]{mdframed} % Required for creating the theorem, definition, exercise and corollary boxes
\newmdenv[skipabove=7pt,
skipbelow=7pt,
rightline=false,
leftline=true,
topline=false,
bottomline=false,
linecolor=sectionNumbers,
backgroundcolor=black!2,
innerleftmargin=5pt,
innerrightmargin=5pt,
innertopmargin=5pt,
leftmargin=0cm,
rightmargin=0cm,
linewidth=4pt,
innerbottommargin=5pt]{mathSection}


%----------------------------------------------------------------------------------------
%	SECTION NUMBERING IN THE MARGIN
%----------------------------------------------------------------------------------------

\makeatletter
\renewcommand{\@seccntformat}[1]{\llap{\textcolor{sectionNumbers}{\csname the#1\endcsname}\hspace{1em}}}                    
\renewcommand{\section}{\@startsection{section}{1}{\z@}
	{-4ex \@plus -1ex \@minus -.4ex}
	{1ex \@plus.2ex }
	{\normalfont\large\sffamily\bfseries}}
\renewcommand{\subsection}{\@startsection {subsection}{2}{\z@}
	{-3ex \@plus -0.1ex \@minus -.4ex}
	{0.5ex \@plus.2ex }
	{\normalfont\sffamily\bfseries}}
\renewcommand{\subsubsection}{\@startsection {subsubsection}{3}{\z@}
	{-2ex \@plus -0.1ex \@minus -.2ex}
	{.2ex \@plus.2ex }
	{\normalfont\small\sffamily\bfseries}}                        
\renewcommand\paragraph{\@startsection{paragraph}{4}{\z@}
	{-2ex \@plus-.2ex \@minus .2ex}
	{.1ex}
	{\normalfont\small\sffamily\bfseries}} % Loads the book formatting

\begin{document}
	
\chapter{Experimental distinguishability}

\section{Experimental observations}

As science is the systematic study of the physical world through observation and experimentation, it seems appropriate to start our endeavor by characterizing the basic properties of experimental observations.

First we should realize that not all observations are of a scientific nature. For example, \emph{``jazz is marvelous"} or \emph{``green and red go well together"} are not scientific statements as they are subjective. That is, there is no agreed upon definition or procedure for what constitutes marvelous music or good color combination. This does not mean that these are somehow statements of a lesser nature. In fact, most of the things that make life worth living (e.g. love, friendship, arts, purpose and so on) defy objective characterization and it would be a very dull world if they didn't. It simply means: they are not the subject of scientific inquiry.

While we have excluded opinions, not all facts are experimental either. For example, \emph{``the square of the hypotenuse is equal to the sum of the squares of the other two sides"} or \emph{``God is eternal"} are not experimental statements as they describe properties of entities that are not physical in nature. As such they will also depend on the axioms we take as a starting point. Again, this does not mean these concepts are of less significance. In fact, one may be more interested in them precisely because of their abstract, and therefore less transient, nature. Nonetheless, they are not the subject of scientific inquiry.

TODO: add examples of statements of physical objects that are nonsense anyway (e.g. the electron is not green)

Yet, even facts about physical objects may still not be experimentally well defined. For example, \emph{``there is no extra-terrestrial life"} or \emph{``the mass of the electron is exactly $9.109 \times 10^{-31}$ kg"} are not something we can validate experimentally. In the first case, we will never know whether there is life outside the observable universe; in the second, we will always have an error bar, however small.

I encourage you to take a minute and come up with your own examples of what may or may not be a statement that can be validated experimentally. Either because of the subject or the lack of an objective and repeatable verification procedure. It should help realize that science is no golden hammer and not everything is a nail. Science is only one of the intellectual tools we have at our disposal and it does not supersede the others.

We capture this discussion with the following definition:

\begin{defn}
	An \textbf{experimental test} $\mathsf{e}$ is a repeatable procedure that can terminate successfully in a finite amount of time and that can be stopped at any time.
\end{defn}

\begin{defn}
	An \textbf{experimental observation} is a tuple $\mathsf{o} = (\mathsf{s}, \mathsf{e})$ where $\mathsf{s}$ is a statement and $\mathsf{e}$ an experimental test that, if successful, verifies the statement.
\end{defn}

\begin{defn}
	Two experimental observations $\mathsf{o}_1$ and $\mathsf{o}_2$ are \textbf{equivalent} $\mathsf{o}_1 = \mathsf{o}_2$ if each is verified when the other one is.
\end{defn}

\begin{defn}
	Two experimental observations $\mathsf{o}_1$ and $\mathsf{o}_2$ are \textbf{mutually exclusive} if neither is ever verified when the other one is.
\end{defn}

\section{Logical operations}

Once we have set of verified observations, we can use logic to create others or check for consistency. For example, once we have established that \emph{``cold reduces swelling"} and that \emph{``ice is cold"} we may conclude that \emph{``ice 
reduces swelling"}. Note that this type of inference relies on the meaning of the actual observations. That is: we understand that cold is a property that has an effect and that ice posses it. As such, it cannot be generalize to all experimental observations.

What we want to understand is how experimental observations work under the common logical operations. These do not depend on the specifics of the observation and therefore have a more general character.
\begin{table}[h]
	\centering
	\begin{tabular}{p{0.2\textwidth} p{0.1\textwidth} p{0.1\textwidth} p{0.5\textwidth}}
		Operator & Gate & Symbol & Example \\ 
		\hline 
		Negation & NOT & $\neg A$ &  \emph{``the sauce is not sweet"} \\ 
		Conjunction & AND & A $\wedge B$ & \emph{``the sauce is sweet and sour"} \\ 
		Disjunction & OR & $A \vee B$ & \emph{``the sauce is at least sweet or sour"}\\
		\multicolumn{4}{c}{  $A$ = \emph{``the sauce is sweet"} and $B$ = \emph{``the sauce is sour"}}
	\end{tabular} 
	\caption{Boolean operations on experimental observations}
\end{table}

A somewhat surprising result is that the negation of a experimental observation is not always a experimental observation. For example, \emph{``there is extra-terrestrial life"} and \emph{``the mass of the electron is not exactly $9.109 \times 10^{-31}$ kg"} can be validated experimentally. For the first, we may send a probe to Mars and find bacteria; for the second, we may measure the mass to be within a range that excludes zero. The negation of the two statements,  \emph{``there is no extra-terrestrial life"} and \emph{``the mass of the electron is exactly $9.109 \times 10^{-31}$ kg"} are not for the reason we outlined before.

The ability to verify a statement, therefore, does not imply the ability to verify its negation. It implies, though, the ability to refute its negation. That is, if I verified that \emph{``there is extra-terrestrial life"} then I have refuted that \emph{``there is no extra-terrestrial life"}. This means that we can always re-frame a verification problem into a refutation problem and vice-versa. As such, we can only concentrate on verification and that is why our definition of experimental observation is solely based on that.

Conversely, note that not being able to verifying a statement does not mean it is refuted. After seeing a flying object in the sky, you may not be have been able to verify that \emph{``it is a duck"} because it was too far and you concluded that \emph{``it is a bird"}. Yet, it may still be a duck. In other words, failure to verify does not provide any information.

These subtleties are why we are framing the problem in terms of observations and not questions. A question such that \emph{``is there extra-terrestrial life?"} only admits ``Yes" and ``I don't know" as experimental answers. A question such that \emph{``is the mass of the electron exactly $9.109 \times 10^{-31}$ kg?"} only admits ``No" and ``I don't know" as experimental answers. A question such that \emph{``what is that animal?"} may have answers at different degree of precision (\emph{``it is an animal"}, \emph{``it is a bird"}, \emph{``it is a duck"}) that are not mutually exclusive. Therefore, in this framework, we enumerate all the possible answers and keep track of the ones that can be verified experimentally.

Combining observations with conjunction (i.e. the logical AND) is more straightforward. To verify that \emph{``the sauce is sweet and sour"} we can first test that \emph{``the sauce is sweet"} and then that \emph{``the sauce is sour"}. If both are verified, then we have verified the conjunction. That is: if we have two or more experimental observations we can always verify the logical AND just by verifying each element one at a time. Yet, the number of observations needs to be finite or we would never end the verification process.

Combining observations with disjunction, i.e. the logical OR, is also straightforward. To verify that \emph{``the sauce is at least sweet or sour"} we can first test that \emph{``the sauce is sweet"}. If that is verified that's enough: the sauce is at least sour. If not, we test that \emph{``the sauce is sour"}. That is: if we have two or more experimental observations we can always verify the logical OR just by stopping at the first element that is verified. Because we stop at the first verification, the number of observations we combine in a logical OR can be countably infinite. As long as one verification succeeds, which will always be the case when the overall verification succeeds, it does not matter how many elements we are not going to verify later.

This discussion leads to the following definitions and properties.

\begin{defn}
	An \textbf{experimental counter-observation} is a tuple $\mathsf{o}^c=(\mathsf{s}, \mathsf{e})$ where $\mathsf{s}$ is a statement and $\mathsf{e}$ an experimental test that, if successful, refutes the statement.
\end{defn}

\begin{defn}
	The \textbf{negation of an experimental observation} $\mathsf{o}=(\mathsf{s}, \mathsf{e})$ is the experimental counter-observation $\neg \mathsf{o}=(\neg \mathsf{s}, \mathsf{e})$ where the negation of the statement $\mathsf{s}$ can be refuted by the experimental test $\mathsf{e}$.
\end{defn}

\begin{defn}
	The \textbf{conjunction of two experimental observations} (logical AND) $\mathsf{o}_1=(\mathsf{s}_1, \mathsf{e}_1)$ and  $\mathsf{o}_2=(\mathsf{s}_2, \mathsf{e}_2)$ is the experimental observation $\mathsf{o}_1 \wedge \mathsf{o}_2 =(\mathsf{s}_1 \wedge \mathsf{s}_2, \mathsf{e}_{\wedge}(\mathsf{e}_1, \mathsf{e}_2))$ formed by the conjunction (logical AND) of the respective statements and the experimental test $\mathsf{e}_\wedge$ that executes both $\mathsf{e}_1$ and $\mathsf{e}_2$ and terminates successfully only when both $\mathsf{e}_1$ and $\mathsf{e}_2$ terminated successfully.
\end{defn}

\begin{defn}
	The \textbf{disjunction of two experimental observations} (logical OR) $\mathsf{o}_1=(\mathsf{s}_1, \mathsf{e}_1)$ and  $\mathsf{o}_2=(\mathsf{s}_2, \mathsf{e}_2)$ is the experimental observation $\mathsf{o}_1 \vee \mathsf{o}_2 =(\mathsf{s}_1 \vee \mathsf{s}_2, \mathsf{e}_{\vee}(\mathsf{e}_1, \mathsf{e}_2))$ formed by the disjunction (logical OR)  of the respective statements and the experimental test $\mathsf{e}_{\vee}$ that executes both $\mathsf{e}_1$ and $\mathsf{e}_2$ and terminates successfully when any of them terminated successfully.
\end{defn}


\begin{prop}
	Let $\mathsf{o}_1, \mathsf{o}_2, ... , \mathsf{o}_n, ...$ be a countable (possibly infinite) sequence of experimental observations. The finite conjunction $\bigwedge\limits_{i=1}^{n} \mathsf{o}_i$ and the countable disjunction  $\bigvee\limits_{i=1}^{\infty} \mathsf{o}_i$ are experimental observations.
\end{prop}

\begin{proof}
	We claim the finite conjunction of experimental observations is an experimental observation. Let $\mathsf{e}_{\wedge}(\mathsf{e}_i)$ be the procedure that executes all experimental tests $\mathsf{e}_i$ and succeeds if all $\mathsf{e}_i$ succeed. If $\mathsf{e}_{\wedge}(\mathsf{e}_i)$ succeeds then all $\mathsf{e}_i$ have succeeded using a finite amount of resources as they are experimental tests. $\mathsf{e}_{\wedge}(\mathsf{e}_i)$ succeeds using at maximum the sum of the resources used by a finite number of experimental tests, which is finite. $\mathsf{e}_{\wedge}(\mathsf{e}_i)$ is an experimental test. $\bigwedge\limits_{i=1}^{n} \mathsf{o}_i = (\bigwedge\limits_{i=1}^{n} \mathsf{s}_i, \mathsf{e}_{\wedge}(\mathsf{e}_i))$ is an experimental observation.
	
	We claim the infinite disjunction of experimental observations is an experimental observation. Let $\mathsf{e}_{\vee}(\mathsf{e}_i)$ be the experimental procedure that proceeds as follows.
	\begin{enumerate}
		\item initialize $n$ to 1
		\item for each $i=1..n$
		\begin{enumerate}
		\item run the test $\mathsf{e}_i$ for $n$ seconds
		\item if $\mathsf{e}_i$ terminated successfully, terminate successfully
		\end{enumerate}
		\item increment $n$ and go to step 2
	\end{enumerate}
	Suppose any of the individual experimental
	tests terminates successfully, it will be do so in finite time, so the overall process will find it. The overall process terminates, at which point only finite time has passed in total. $\mathsf{e}_{\vee}(\mathsf{e}_i)$ terminates successfully in finite time and is therefore an experimental test. $\bigvee\limits_{i=1}^{\infty} \mathsf{o}_i = (\bigvee\limits_{i=1}^{\infty} \mathsf{s}_i, \mathsf{e}_{\vee}(\mathsf{e}_i))$ is an experimental observation.
\end{proof}

\section{Questions and answers}

\begin{defn}
	An experimental question is a set of observations closed under finite conjunction and countable disjunction.
\end{defn}

\begin{defn}
	An experimental question is completely answerable if we can identify the set of al verified observations by testing them in finite time.
\end{defn}

\begin{prop}
	An experimental question allows a countable base where each element has only a finite set of elements with which is mutually exclusive.
\end{prop}

\begin{defn}
	An experimental question is completely answerable if we can identify the set of all verified observations by testing them in finite time.
\end{defn}

\begin{prop}
	(Tentative) An experimental question allows a countable base where each element has only a finite set of elements with which is mutually exclusive.
\end{prop}

\begin{defn}
	An experimental question is approximately answerable if we can identify a sequence of sets of verified observations, each by testing in finite time, such that the limit is well defined and corresponds to all observations that are verified.
\end{defn}

\begin{prop}
	(Tentative) An experimental question allows a countable base.
\end{prop}


\section{Experimental identification}

The most common and important type of experimental observation is when we want to identify a specific case from a set of possible ones. For example, we pick ``this is a duck" from all possible animals or ``the position of the ball is $3 \pm 0.5$ m" from all possible values of position.

As we hinted before, observations do not necessarily identify a single element and they are not necessarily mutually exclusive. For example, ``it is an egg-laying animal" and ``it is a mammal" still allow for the platypus and the echidna. ``the position of the ball is $3 \pm 0.5$ m" and ``the momentum of the ball is $2 \pm 0.1$ kg m/s" limits the possible states to a rectangular region. An experimental observation that identifies an object, then, is not in general associated with a single element, but with a set of elements compatible with that observation. For example, the observation ``it is a bird" will be compatible with duck, penguin, hawk and so on while it will be incompatible with cat, moose, shrimp and so on.

Not only each experimental observation will be associated with a set, but two observations that share the same set are, for our purposes, equivalent. For example, ``it is a feathered animal" or ``it is a bird" will give us the same set of animals as all birds have feathers and all feathered animals are birds. The information they give us is the same. Therefore, we can treat an experimental observation as if it were equivalent to its associated set of elements. That is, if $U \subseteq X$ is the set of elements associated with the observation (e.g. all the birds), the observation may as well be ``the element is in U" (e.g. the element is within the set of birds).

But not all sets of elements are necessarily associated to an experimental observation. As we saw ``the position of the ball is precisely $1$ m" is not an observation because we can't measure position with infinite precision. Therefore the set of states for which position is exactly $1$ m is not associated with an experimental observation. The whole problem, then, is to be able to keep track of which set of elements is associated with an experimental observation and which is not.

At the very least, we need to be able to tell whether an object is or is not one of the elements in our set $X$. For example, we must be able to verify that ``it is an animal" or ``it is not an animal". In the same way, if we can't even verify whether something is a ball or not, it does not make sense to distinguish between the states of a ball. Therefore, to identify elements within a set $X$, there will be an experimental observation associated with the full set $X$, equivalent to ``that is an element of the set", and another experimental observation associated to the empty set $\emptyset$, equivalent to ``that is not an element of the set".

As we saw before, we can always take the conjunction (logical AND) and disjunction (logical OR) of two experimental observation. How do these operations transpose in terms of sets? Suppose we have $U \subseteq X$ (e.g. the set of flying animals) with an associated experimental observation (e.g. ``it is a flying animal"). Suppose we have $V \subseteq X$ (e.g. the set of all insects) with an associated experimental observation (e.g. ``that is an insect"). We combine the two observations with a logical AND (e.g. ``that is a flying insect"). Only the elements that are both in $U$ and $V$ are compatible with it. That is: only the elements in the intersection $U \cap V$. This means that if $U$ and $V$ are two sets each associated with an experimental observation, then their intersection $U \cap V$ is also associated with an experimental observation. Similarly, if $U$ and $V$ are two sets each associated with an experimental observation, we can combine these with a logical OR. All elements either in $U$ or $V$ are compatible with it. That is: elements in the union $U \cup V$. This means the union $U \cup V$ is also associated with an experimental observation.

To sum up, let $X$ be a set of elements and let $\mathsf{T}$ be the collection of the sets associated with some experimental observations we can use to identify among them. We have the following:
\begin{itemize}
	\item Both $X$ and $\emptyset$ can be found in $\mathsf{T}$ as we must be able to verify whether an object is or is not within the ones we can identify
	\item Any intersection of a finite number of sets in $\mathsf{T}$ can also be found in $\mathsf{T}$ as the finite conjunction (i.e. logical AND) of experimental observations is also an experimental observation
	\item Any union of a countable number of sets in $\mathsf{T}$ can also be found in $\mathsf{T}$ as the countable disjunction (i.e. logical OR) of experimental observations is also an experimental observation
\end{itemize}
This is the mathematical definition of a topology: a set of sets that contains both the whole set and the empty set, that is closed under finite intersection and under countable union.

\begin{defn}
	A \textbf{set of possibilities} $X$ is a collection of possible elements within which we identify an object.
\end{defn}

\begin{defn}
	A \textbf{verifiable set} $U \subseteq X$ is a subset of possibilities for which there exists an experimental observation $\mathsf{o} = (\text{``The object is in } U \text{"}, \mathsf{e}_\in(U))$ where $\mathsf{e}_\in(U)$ is an experimental test that succeeds only if the object to identify is an element of $U$.
\end{defn}

\begin{defn}
	A \textbf{refutable set} $U \subseteq X$ is a subset of possibilities for which there exists an experimental counter-observation $\mathsf{o} = (\text{``The object is in } U \text{"}, \mathsf{e}_{\notin}(U))$ where $\mathsf{e}_{\notin}(U)$ is an experimental test that succeeds only if the object to identify is not an element of $U$.
\end{defn}

\begin{prop}
	The complement $U^C$ of a verifiable set $U \subseteq X$ is a refutable set.
\end{prop}

\begin{proof}
	Let $U\subset X$ be a verifiable set. There exists an experimental observation $\mathsf{o} = (\text{``The object is in } U \text{"}, \mathsf{e}_\in(U))$. Consider $\neg \mathsf{o} = (\neg \text{``The object is in } U \text{"}, \mathsf{e}_\in(U)) = ($``The object  is not in $ U \text{"}, \mathsf{e}_{\notin}(U^C)) = ($``The object  is in $ U^C \text{"}, \mathsf{e}_{\notin}(U^C))$. That is, there exists a counter-observation $($``The object  is in $ U^C \text{"}, \mathsf{e}_{\notin}(U^C))$: $U^C$ is a refutable set.
\end{proof}

\begin{prop}
	Let $U_1, U_2, ... , U_n, ...$ be a countable infinite sequence of verifiable sets. The finite intersection $\bigcap\limits_{i=1}^{n} U_i$ and the countable union  $\bigcup\limits_{i=1}^{\infty} U_i$ are verifiable sets.
\end{prop}

\begin{proof}
	We claim the finite intersection of verifiable sets is a verifiable set. Let $U_1, U_2, ... , U_n \subseteq X$ be n verifiable sets. For each $U_i$ there exists an experimental observation $\mathsf{o}_i = ($``The object is in $ U_i \text{"}, \mathsf{e}_\in(U_i))$. Consider $\mathsf{o} = \bigwedge\limits_{i=1}^{n} \mathsf{o}_i = (\bigwedge\limits_{i=1}^{n} $``The object is in $ U_i \text{"}, \mathsf{e}_{\wedge}(\mathsf{e}_\in(U_i)))=( $``The object is in $ \bigcap\limits_{i=1}^{n} U_i \text{"}, \mathsf{e}_\in(\bigcap\limits_{i=1}^{n} U_i))$ is an experimental observation. $\bigcap\limits_{i=1}^{n} U_i$ is a verifiable set.
	
	We claim the countable union of verifiable sets is a verifiable set. Let $U_1, U_2, ... , U_n, ... \subseteq X$ be an infinite sequence of verifiable sets. For each $U_i$ there exists an experimental observation $\mathsf{o}_i = ($``The object is in $ U_i \text{"}, \mathsf{e}_\in(U_i))$. Consider $\mathsf{o} = \bigvee\limits_{i=1}^{\infty} \mathsf{o}_i = (\bigvee\limits_{i=1}^{\infty} $``The object is in $ U_i \text{"}, \mathsf{e}_{\vee}(\mathsf{e}_\in(U_i)))=( $``The object is in $ \bigcup\limits_{i=1}^{\infty} U_i \text{"}, \mathsf{e}_\in(\bigcup\limits_{i=1}^{\infty} U_i))$ is an experimental observation. $\bigcup\limits_{i=1}^{\infty} U_i$ is a verifiable set.
\end{proof}

\begin{stmt}
	Let $\mathsf{T}$ be a collection of experimental observations that identifies elements of $X$. $\mathsf{T}$ establishes a topology for $X$.
\end{stmt}


\begin{justification}
	FIXME: copied from previous work
	
	We claim $\mathcal{S}$ has a topology $\mathsf{T}$. Consider the set of all possible physical outcomes associated with all physical processes. Each possible outcome is associated with a set of states that are compatible with that outcome. Let $\mathsf{T}$ be the set of all sets associated with all physical outcomes. $\mathcal{S} \in \mathsf{T}$ and is associated with the outcome ``the system exists." $\varnothing \in \mathsf{T}$ and is associated with the outcome ``the system doesn't exist." Let $V_1, V_2 \in \mathsf{T}$. Then, by definition, there exists a process $P_1$ that admits $V_1$ as an outcome and a process $P_2$ that admits $V_2$ as an outcome. Consider the process $P$ that combines the outcomes of $P_1$ and $P_2$ with a logical \textsc{and}. This always exists physically as we can prepare the same state multiple times and let it interact with each process separately. $P$ will have a possible outcome $V$ corresponding to the case where $P_1$ gave outcome $V_1$ and $P_2$ gave outcome $V_2$. The states compatible with $V$ must be in both $V_1$ and $V_2$, that is $V = V_1 \cap V_2$. Therefore $\mathsf{T}$ is closed under intersection. Let $V_1, V_2 \in \mathsf{T}$. If they are physically distinguishable, then there exists a physical process $P$ that admits both as outcomes. Given $P$, we can always construct the process $P_0$ that combines $V_1, V_2$ with a logical \textsc{or} into a single outcome $V$, by ``forgetting" which of the two was given. The states compatible with $V$ must be in either $V_1$ or $V_2$, that is $V = V_1 \cup V_2$. Therefore $\mathsf{T}$ is closed under union. $\mathsf{T}$ is a topology by definition.
\end{justification}


\section{Experimental distinguishability}

What we have done so far is to assign to a set of elements $X$ a collection of experimental observations $\mathsf{T}$. This is not enough. For example, the observations ``it is an animal" and ``it is not an animal" form a collection closed under logical operators (i.e. it forms the trivial topology) but it is hardly useful.

We call $X$ a set of experimentally distinguishable elements if we can associate with it a set of experimental observations rich enough to be able to distinguish each element from the other. That is, given two possible elements (e.g. two possible animals or two possible positions) we can always find an experimental observation that can tell them apart. This property is a basic requirement to treat these elements in a scientific context: if we are not even able to tell them apart experimentally, they cannot be the subject of scientific investigation.

If two elements are experimentally distinguishable from each other there we must have an experimental observation that can tell them apart. For example, to be able to define the house sparrow (Passer domesticus) as a separate animal from the Italian sparrow (Passer italiae) we have at least two observations (``it has a dark gray crown", ``it has a chestnut crown") that are mutually exclusive, each compatible with just one element.

That is, for each pair of elements $x_1, x_2 \in X$, we must be able to find two sets $U,V \in \mathsf{T}$ each associated with an experimental observation such that no element is in both (i.e. their intersection $U \cap V=\emptyset$ is empty) and one element is in one set (i.e. $x_1 \in U$ and $x_2 \in V$). This is the mathematical definition of a Hausdorff space: one in which distinct elements have disjoint neighborhoods. 

This solves one problem, there exists a way to distinguish any two elements, but it does not solve the other problem: we must be able to actually find it. As we saw before, we can only test a finite amount of experimental observations. So we must make sure we can distinguish any two elements in a finite number of steps.

Note that, once we have identified an element, we are able to tell which experimental observations are verified: the ones associated with a set that includes the element. For example, once we have identified the house sparrow, we have verified ``it has a gray crown", ``it is a bird", ``it has wings" and so on. Conversely, once we know all the experimental observations that can be verified, we have identified the element: the only one that is included in all sets associated with verified observations. That is, once we have verified that ``it has a gray crown", ``it is a bird", ``it has wings" and so on, we have identified the house sparrow. In other words: identifying an element is equivalent to being able to verify all possible observations.

Fortunately we don't have to test all possible observations: just enough to be able to calculate all other cases. For example, we can simply test each animal one at a time (i.e. ``it is a cat", ``it is a dog", ``it is a duck" and so on) until we find the correct one. Once we verify one, we will know whether ``it is a bird" or ``it is a mammal". We define sub-base a set of experimental observations from which all others can be obtained through conjunction (i.e. logical AND) and disjunction (i.e. logical OR). In particular, we consider the smallest possible sub-base. This gives us the smallest number of experimental observation to test in order to distinguish an element from all others.

If the smallest sub-base is finite, it is possible to test all experimental observations. This is the case of the set of animals: there are a large but finite number of species. If it's infinite, we cannot. But if the smallest sub-base is countably infinite, we can get as close as we want. This is the case for the position of a ball: we can in principle always increase the precision. Suppose, in fact, that we have a large but finite number of elements $n$ among which we want to distinguish. We need a set of $n$ disjoint sets, each including one element. Note that the union will not make two overlapping sets disjoint, so these sets will only be intersections of elements of the sub-base. Intersections can only be finite, therefore we will need only a finite number of members of the sub-basis. If the sub-basis is countable, at some point we will get to the last member needed and we can stop: we can distinguish between the $n$ elements. If the sub-basis is not countable, instead, we will never stop: we can't experimentally distinguish between a set of arbitrary elements in a finite amount of time.

If there is a countable sub-base, then there is a countable base as well. This is the definition of a second countable space: one that allows a countable bases for its topology.

\begin{defn}
	A set of experimental possibilities $X$ is \textbf{physically distinguishable} if given n possibilities $x_i \in X$ with $i=1..n$ we can always find n mutually exclusive experimental observation $\mathsf{o}_i$ (i.e. n disjoint verifiable sets $U_i$) such that each possibility $x_i$ is compatible with the respective experimental observation (i.e. $x_i \subset U_i$).
\end{defn}


\begin{prop}
	A set $X$ of physically distinguishable experimental possibilities is a Hausdorff and second countable topological space with the topology $\mathsf{T}(X)$ formed by all distinguishable sets.
\end{prop}

\begin{justification}
	FIXME: copied from previous work
	
	We claim that $\mathsf{T}$ is Hausdorff. Let $s_1, s_2 \in \mathcal{S}$. As states are physically distinguishable, there must exist a physical process with two possible outcomes $V_1, V_2 \in \mathsf{T}$ for which $s_1 \in V_1, s_2 \in V_2, V_1 \cap V_2 = \varnothing$. $\mathcal{S}$ is Hausdorff by definition.
\end{justification}

\section{Connections between topology and experimental distinguishability}

Now that we have made a tight link between topology and experimental distinguishability, we can go through some of the mathematical vocabulary and give it a more precise physical meaning.

The topology defined on a set $X$, which will note as $\mathsf{T}(X)$, describes the collection of all possible experimental observations we can perform to identify an element of the set. Each of these observation can be identified by a subset $U \subseteq X$ which represents all the elements that are compatible with the observation.

An open set $U \in \mathsf{T}(X)$, a set in the topology, is a set for which there exist a way to  experimentally verify that an element is in that set. Conversely, a closed set $V = X \setminus U $, a set whose complement is in the topology, is a set for which there exists a way to experimentally refute that an element is in that set. For example, in the standard topology for $\mathbb{R}$ the interval $(2.5, 3.5)$ is an open set because $3 \pm 0.5$ is a valid measurement for a continuous quantity while $\{3\}$ is a closed set because, while we can't measure a real number with infinite precision, we can exclude it.

The discrete topology, the one for which each singleton $\{x\}$ (i.e. set of one element $x \in X$) is both open and closed, corresponds to the ability to verify and refute each element individually. Any finite set\footnote{TODO what about countable?} that is Hausdorff and second countable has a discrete topology. This is probably why it is not as intuitive to think that something verifiable may not be refutable: it only happens in infinite sets.

The standard topology for $\mathbb{R}$, the one generate by all open intervals, corresponds to the ability of measuring continuous value only with finite precision. This topology is Hausdorff and second countable. While we can give a discrete topology to $\mathbb{R}$, this would no longer be second countable so it would violate our requirement for experimental distinguishability.

An interesting mathematical result is the following.

\begin{prop}
	A Hausdorff and second countable space $X$ has at most cardinality of continuum.
\end{prop}

\begin{proof}
	TODO
\end{proof}

This means that, no matter what technique we use now or in the future, the elements that we can properly define experimentally are at most infinite like the continuum. This already gives us a first simple and basic requirement a set of mathematical objects need to pass to be of scientific interest.

For example, these objects have cardinality of continuum, and therefore are good candidates:
\begin{itemize}
	\item Euclidean space $\mathbb{R}^n$
	\item all continuous functions from $\mathbb{R}$ to $\mathbb{R}$
	\item all open sets in $\mathbb{R}^n$
	\item all subsets of $\mathbb{N}$
\end{itemize}

These, instead, have cardinality greater then continuum, and therefore are not good candidates:
\begin{itemize}
	\item all functions from $\mathbb{R}$ to $\mathbb{R}$
	\item all subsets of $\mathbb{R}$
\end{itemize}


\section{Continuous functions}

We now turn our attention to relationships between two sets $X$ and $Y$ of experimentally distinguishable elements. There are two ways of characterize them and we want to show that they are equivalent.

Suppose we have two sets $X$ and $Y$ of experimentally distinguishable elements (e.g. the temperature and height of a mercury column in a thermometer). Let's assume we have a causal relationship $f: X \rightarrow Y$ between the first and the second (e.g. the temperature determines how high is the mercury column). We assume the relationship is valid on the whole set without loss of generality: if it is not, just redefine $X$ and $Y$ to be the appropriate regions (e.g. the valid ranges of temperature and height of a mercury column in which the thermometer can operate).

The relationship $f$ can also be used to infer observations on $X$ from observations of $Y$. Suppose we verify that $y$ is within a set $V \in \mathsf{T}(Y)$ (e.g. ``the height of the mercury column is between 24 and 25 millimeters"). Then we can infer that $x$ is within the reverse image $U=f^{-1}(V)$ (e.g. ``the temperature is 24 and 25 degrees Celsius"). $U$ is therefore associated with an experimental observation: $U \in \mathsf{T}(X)$ must be a set in the topology (e.g. if we measure the height of the column with finite precision, we cannot end up inferring the value of temperature with infinite precision).

In other words, to each causal relationship $f: X \rightarrow Y$ between two set of experimentally distinguishable elements (e.g. if the temperature is $x$ the height of the mercury column will be $f(x)$) we have an associated reverse inference relationship $g  : \mathsf{T}(Y) \rightarrow \mathsf{T}(X) = f^{-1}$ (e.g. if the height of the mercury column is within $V$ then the temperature is within $f^{-1}(V)$ ). This is true even if the function is not monotonic (e.g. if ``the water density is between 999.8 and 999.9 kg/$m^3$" then ``the water temperature is between 0 and 0.52 or between 7.6 and 9.12 degrees Celsius" as water is most dense at 4 degrees Celsius).  The reverse image $U=f^{-1}(V)$ of a set associated with an experimental observation (i.e. an open set) is also a set associated with an experimental observation (i.e. an open set). This is the definition of a continuous function in topology: one for which the reverse image of an open set is an open set. Note that, when using the standard topology on $\mathbb{R}^n$, topological continuity is equivalent to analytical continuity.

Now suppose we start with the inference relationship $g  : \mathsf{T}(Y) \rightarrow \mathsf{T}(X)$ that for each verified experimental observation on $Y$ gives us a verified experimental observation on $X$ (e.g. if ``the height of the mercury column is within $V$" then ``the temperature is within $g(V)$"). For it to be a valid inference relationship, it will have to be compatible with logical operations (e.g. if ``the height of the mercury column is between the 23 and 25 millimeters" and ``the height of the mercury column is between the 24 and 26 millimeters" then ``the temperature is between 23 and 25 degree Celsius" and  ``the temperature is between 24 and 26 degree Celsius"). This means the relationship must also be compatible with the set operations that correspond to the logical operations: $g(V_1 \cup V_2)=g(V_1)\cup g(V_2)$ and $g(V_1 \cap V_2)=g(V_1)\cap g(V_2)$.

If nothing is known about $y$ (e.g. if ``the height of the mercury column is in a valid range") then nothing should be known about $x$ (e.g. ``the temperature is in a valid range"): $g(Y)=X$. If we excluded all values in $Y$ (e.g. ``the height of the mercury column is not in a valid range of the thermometer") then we excluded all values in $X$ (e.g. ``the temperature is not in a valid range of the thermometer"): $g(\emptyset) = \emptyset$. That is, for the inference relationship to be valid we shouldn't be able to infer something from nothing or nothing from something.

Under these conditions, one can show that for any such relationship there exists a continuous function $f: X \rightarrow Y$ such that $g=f^{-1}$ is its inverse image.

\begin{prop}
	Let $g: \mathsf{T}(Y) \rightarrow \mathsf{T}(X)$ such that:
	\begin{enumerate}
		\item it is compatible with union and intersection $\forall V_1, V_2 \in Y$ $g(V_1 \cup V_2)=g(V_1)\cup g(V_2)$ and $g(V_1 \cap V_2)=g(V_1)\cap g(V_2)$
		\item $g(\emptyset) = \emptyset$
		\item $g(Y) = X$
	\end{enumerate}
    There exists a unique continuous function $f: X \rightarrow Y$ such that $g = f^{-1} |_{\mathsf{T}(Y)}$.
\end{prop}

\begin{proof}
	TODO
\end{proof}

\section{Distinguishability of functions}

For all of this to be self consistent, we must require that functions between experimentally distinguishable elements are  experimentally distinguishable themselves. That is, we must show that the set of continuous function $C(X,Y)$ from $X$ to $Y$ is a Hausdorff and second countable topological space with a suitable topology. Note that $C(X,Y)$ has the cardinality of continuum, therefore we already know it allows Hausdorff and second countable topologies, but we need to make sure we can give one that is actually physically meaningful in terms of experimental observations.

Suppose we have two sets $X$ and $Y$ of experimentally distinguishable elements (e.g. time and space). Suppose we need to distinguish elements within the set of all continuous function $C(X,Y)$ from $X$ to $Y$ (e.g. all possible trajectories). From a sub-base of $X$ (e.g. all open time intervals between rational numbers) we pick a set $U$ (e.g. between 1 and 2 seconds). From a sub-base of $Y$ (e.g. all open spatial intervals between rational numbers) we pick a set $V$ (e.g. between 1 and 2 meters). We can define the set $S(U,V) = \{f: X \rightarrow Y | f \in C(X,Y), f(U) \subset V\}$ of all the continuous functions such that their value remains within $V$ over the domain $U$ (e.g. all trajectories that between $1$ and $2$ seconds remain within $1$ and $2$ meters).

If we take all possible sets of functions constructed this way, and combine them with finite intersections and countable unions, we obtain a topology for the continuous function. The collections of all sets $S(U,V)$ forms a sub-base for this topology and is constructed from the sub-bases of $X$ and $Y$. This new sub-base is countable because the choices for $U$ and $V$ are countable themselves. This means that $C(X,Y)$ is second countable with this topology.

Now we need to show that we can distinguish between any two functions. Say we have two different functions $f_1$ and $f_2$ (i.e. two different trajectories). Since they are different, there will be at least one value $x \in X$ such that the values $f_1(x)\neq f_2(x)$ will be different (e.g. at $1.2$ seconds the first trajectory is at $1.1$ meters while the second trajectory is at $1.2$ meters). We can now find two disjoint elements $V_1$ and $V_2$ of the sub-base for $Y$ such that each includes either $f_1(x)$ or $f_2(x)$ (e.g. the intervals $(1.095, 1.105)$ and $(1.195, 1.205)$ meters). Since the functions are continuous, we can find an element $U$ of the sub-base for $X$ such that both functions remain within those ranges (e.g. between $(1.197, 1.203)$ seconds the first trajectory remains within $(1.095, 1.105)$ meters and the second trajectory remains within $(1.195, 1.205)$ meters). The sets $S(U, V_1)$ (e.g. all trajectories that between $(1.197, 1.203)$ seconds that remain within $(1.095, 1.105)$ meters) and $S(U, V_2)$ (e.g. all trajectories that between $(1.197, 1.203)$ seconds that remain within $(1.195, 1.205)$ meters) are disjoint and they each contain $f_1$ and $f_2$ respectively. We can distinguish between functions: the topology is Hausdorff.

This confirms that all the conceptual infrastructure is solid and self consistent. Sets of experimentally distinguishable elements are Hausdorff and second countable topological spaces. Maps between elements of such spaces are continuous maps as they preserve experimental distinguishability. The maps themselves are experimentally distinguishable and the can be given a Hausdorff and second countable topology. We can continues this recursively by constructing maps of maps (e.g. a map from a state to a trajectory) without ever going outside our definitions: everything remains experimental distinguishable.

\section{Summary}

\begin{table}[h]
	\centering
\begin{tabular}{p{0.20\textwidth} p{0.7\textwidth}}
	Math/Topology & Science/Physics \\ 
	\hline 
	Open set & Verifiable set. We can verify experimentally that an element is within the set  \\ 
	Closed set & Refutable set. We can verify experimentally that an element is not in the set \\ 
	Continuous \newline function &  A function between two sets of physically distinguishable elements, a function that preserves distinguishability \\
	Homeomorphism &  A bijective function between two sets of physically distinguishable elements \\
\end{tabular} 
	\caption{Topology to physics dictionary}
\end{table}

	
\end{document}