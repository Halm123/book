% Possible use http://www.latextemplates.com/template/the-legrand-orange-book as template? See https://www.overleaf.com/9174958nyjxxdxbchks#/33024595/

\documentclass[11pt,letterpaper,fleqn]{memoir} % Default font size and left-justified equations

\usepackage[top=3cm,bottom=3cm,left=3cm,right=3cm,headsep=10pt,letterpaper]{geometry} % Page margins

% Theorem definitions using amsthm

\usepackage{amsthm}
\usepackage{amsmath}
\usepackage{amssymb}

% Remove line spaces between items of enumerate and itemize
\usepackage{enumitem}
\setlist{noitemsep}


% Adds double bracket symbols
\usepackage{stmaryrd}

% Latex symbol guide at http://mirrors.ibiblio.org/CTAN/info/symbols/comprehensive/symbols-letter.pdf

% LOGIC symbols
% -------------

% Allows to create negation symbols
\usepackage{MnSymbol}

\DeclareMathOperator{\truth}{truth}
\DeclareMathOperator{\poss}{possibilities}
\DeclareMathOperator{\result}{result}

\def\TRUE{\textsc{true}}
\def\FALSE{\textsc{false}}

\def\SUCCESS{\textsc{success}}
\def\FAILURE{\textsc{failure}}
\def\UNDEF{\textsc{undefined}}


% Symbols for tautology and contradiction
\def\tautology{\top}
\def\contradiction{\bot}

% Symbols for "compatibility" and "incompatibility"
\def\comp{\doublefrown}
\def\ncomp{\ndoublefrown}

% Symbols for "narrower" and "wider"
\def\narrower{\subseteq}
\def\nnarrower{\nsubseteq}
\def\broader{\supseteq}
\def\nbroader{\nsupseteq}


% Symbol for "independent" and "correlated"
\def\indep{\upmodels}
\def\nindep{\nupmodels}

% Aliases for logical operations
\def\AND{\wedge}
\def\bigAND{\bigwedge}
\def\OR{\vee}
\def\bigOR{\bigvee}
\def\NOT{\neg}


% Formatting for statements
\newcommand{\stmt}[1][s] {\mathsf{#1}}
% Formatting for experimental tests
\newcommand{\expt}[1][e] {\mathsf{#1}}
% Formatting for observations
\newcommand{\obs}[1] {\mathsf{#1}}
% Formatting for observation definition
\newcommand{\obsdef}[2] {\llparenthesis #1, #2 \rrparenthesis}

% Formatting for experimental domain
\newcommand{\edomain}[1][D] {\mathcal{#1}}

% Formatting for theoretical domain
\newcommand{\tdomain}[1][D] {\bar{\mathcal{#1}}}


% Formatting for sentence statements
\newcommand{\statement}[1] {\emph{``#1"}}


\usepackage{xcolor} % Required for specifying colors by name
\definecolor{sectionNumbers}{RGB}{44, 103, 0}


\renewcommand\thesubsection{\thesection.\Alph{subsection}}
\renewcommand{\theequation}{\thechapter.\arabic{equation}}

\newtheorem{assump}{Assumption}
\renewcommand*{\theassump}{\Roman{assump}}

\newtheorem{defn}[equation]{Definition}
\newtheorem{prop}[equation]{Proposition}

%\theoremstyle{definition}

\newenvironment{rationale}{\emph{Rationale}.}{\qed}
\newenvironment{justification}{\emph{Justification}.}{\qed}
\renewenvironment{proof}{\emph{Proof}.}{\qed}

% Style for math section
\RequirePackage[framemethod=default]{mdframed} % Required for creating the theorem, definition, exercise and corollary boxes
\newmdenv[skipabove=7pt,
skipbelow=7pt,
rightline=false,
leftline=true,
topline=false,
bottomline=false,
linecolor=sectionNumbers,
backgroundcolor=black!2,
innerleftmargin=5pt,
innerrightmargin=5pt,
innertopmargin=5pt,
leftmargin=0cm,
rightmargin=0cm,
linewidth=4pt,
innerbottommargin=5pt]{mathSection}


%----------------------------------------------------------------------------------------
%	SECTION NUMBERING IN THE MARGIN
%----------------------------------------------------------------------------------------

\makeatletter
\renewcommand{\@seccntformat}[1]{\llap{\textcolor{sectionNumbers}{\csname the#1\endcsname}\hspace{1em}}}                    
\renewcommand{\section}{\@startsection{section}{1}{\z@}
	{-4ex \@plus -1ex \@minus -.4ex}
	{1ex \@plus.2ex }
	{\normalfont\large\sffamily\bfseries}}
\renewcommand{\subsection}{\@startsection {subsection}{2}{\z@}
	{-3ex \@plus -0.1ex \@minus -.4ex}
	{0.5ex \@plus.2ex }
	{\normalfont\sffamily\bfseries}}
\renewcommand{\subsubsection}{\@startsection {subsubsection}{3}{\z@}
	{-2ex \@plus -0.1ex \@minus -.2ex}
	{.2ex \@plus.2ex }
	{\normalfont\small\sffamily\bfseries}}                        
\renewcommand\paragraph{\@startsection{paragraph}{4}{\z@}
	{-2ex \@plus-.2ex \@minus .2ex}
	{.1ex}
	{\normalfont\small\sffamily\bfseries}} % Loads the book formatting

\begin{document}
	
\chapter{Experimental distinguishability}

\section{Experimental observations}

As science is the systematic study of the physical world through observation and experimentation, it seems appropriate to start our endeavor by characterizing the basic properties of experimental observations.

First we should realize that not all observations are of a scientific nature. For example, \emph{``jazz is marvelous"} or \emph{``green and red go well together"} are not scientific statements as they are subjective. That is, there is no agreed upon definition or procedure for what constitutes marvelous music or good color combination. This does not mean that these are somehow statements of a lesser nature. In fact, most of the things that make life worth living (e.g. love, passion, friendship, arts, creativity, purpose and so on) defy objective characterization and it would be a very dull world if they didn't. It simply means: they are not the subject of scientific inquiry.

While we have excluded opinions, not all facts are experimental either. For example, \emph{``the square of the hypotenuse is equal to the sum of the squares of the other two sides"} or \emph{``God is eternal"} are not experimental statements as they describe properties of entities that are not physical in nature. Again, this does not mean these concepts are of less significance. In fact, one may be more interested in them precisely because of their abstract, and therefore less transient, nature. Nonetheless, they are not the subject of scientific inquiry.

Yet, even facts about physical objects may still not be experimentally well defined. For example, \emph{``there is no extra-terrestrial life"} or \emph{``the mass of the photon is exactly zero"} are not something we can validate experimentally. In the first case, we will never know whether there is life outside the observable universe; in the second, we will always have an error bar, however small.

I encourage you to take a minute and come up with your own examples of what may or may not be a statement that can be validated experimentally. Either because of the subject or the lack of an objective and repeatable verification procedure. It should help realize that science is no golden hammer and not everything is a nail. Science is only one of the intellectual tools we have at our disposal and it does not supersede the others.

We capture this discussion with the following definition:
\begin{defn}
An \textbf{experimental observation} is a fact that can be experimentally verified. That is, there is an agreed upon procedure that, if successful, confirms the validity of the statement.
\end{defn}

\section{Logical operations}

Once we have set of verified observations, we can use logic to infer others or check for consistency. For example, once we have established that \emph{``cold reduces swelling"} and that \emph{``ice is cold"} we may infer that \emph{``ice 
reduces swelling"}. Note that this type of inference relies on the meaning of the actual observations. That is: we understand that cold is a property that has an effect and that ice posses it. As such, it cannot be generalize to all experimental observations.

What we want to understand is how experimental observations work under the common logical operations. These do not depend on the specifics of the observation and therefore have a more general character.
\begin{table}[h]
	\centering
	\begin{tabular}{p{0.2\textwidth} p{0.1\textwidth} p{0.1\textwidth} p{0.5\textwidth}}
		Operator & Gate & Symbol & Example \\ 
		\hline 
		Negation & NOT & $\neg A$ &  \emph{``the sauce is not sweet"} \\ 
		Conjunction & AND & A $\wedge B$ & \emph{``the sauce is sweet and sour"} \\ 
		Disjunction & OR & $A \vee B$ & \emph{``the sauce is at least sweet or sour"}\\
		\multicolumn{4}{c}{  $A$ = \emph{``the sauce is sweet"} and $B$ = \emph{``the sauce is sour"}}
	\end{tabular} 
	\caption{Boolean operations on experimental observations}
\end{table}

A somewhat surprising result is that the negation of a experimental observation is not always a experimental observation. For example, \emph{``there is extra-terrestrial life"} and \emph{``the mass of the neutrino is not exactly zero"} can be validated experimentally. For the first, we may send a probe to Mars and find bacteria; for the second, we may measure the mass to be within a range that excludes zero. The negation of the two statements,  \emph{``there is no extra-terrestrial life"} and \emph{``the mass of the neutrino is exactly zero"} are not for the reason we outlined before.

The ability to verify a statement, therefore, does not implies the ability to verify its negation. It implies, though, the ability to refute the negation. That is, if I verified that \emph{``there is extra-terrestrial life"} then I have refuted that \emph{``there is no extra-terrestrial life"}. This means that we can always re-frame a verification problem into a refutation problem and vice-versa. As such, we can only concentrate on verification and that is why our definition of experimental observation is solely based on that.

Conversely, note that not verifying a statement does not mean it is refuted. After seeing a flying object in the sky, you may not be have been able to verify that \emph{``it is a duck"} because it was too far and you concluded that \emph{``it is a bird"}. Yet, it may still be a duck.

These subtleties are why we are framing the problem in terms of observations and not questions. A question such that \emph{``is there extra-terrestrial life?"} only admits ``Yes" and ``I don't know" as experimental answers. A question such that \emph{``is the mass of the neutrino exactly zero?"} only admits ``No" and ``I don't know" as experimental answers. A question such that \emph{``what is that in the sky?"} may have answers at different degree of precision (\emph{``it is an animal"}, \emph{``it is a bird"}, \emph{``it is a duck"}) that are not mutually exclusive. Therefore, in this framework, we enumerate all the possible answers and keep track of the ones that can be verified experimentally.

Combining observations with conjunction, i.e. the logical AND, is more straightforward. To verify that \emph{``the sauce is sweet and sour"} we can first verify that \emph{``the sauce is sweet"} and then that \emph{``the sauce is sour"}. If both are successful, then we have verified the conjunction. That is: if we have two or more experimental observations we can always verify the logical AND just by verifying each element one at a time. Yet, the number of observations needs to be finite or we would never end the verification process.

Combining observations with disjunction, i.e. the logical OR, is also straightforward. To verify that \emph{``the sauce is at least sweet or sour"} we can first verify that \emph{``the sauce is sweet"}. If that succeeds that's enough: the sauce is at least sour. If not, we verify that \emph{``the sauce is sour"}. That is: if we have two or more experimental observations we can always verify the logical OR just by verifying each element one at a time. As soon as one is verified, the disjunction is verified. Because we stop at the first verification, the number of observations we combine in a logical OR can be countably infinite. As long as one verification succeeds, which will always be the case when the overall verification succeeds, it does not matter how many elements we are not going to verify later.

This discussion leads to the following definitions and properties.

\begin{defn}
	Let $A_1, A_2, ... , A_n, ...$ be an infinite sequence of n scientific observations. We define $\bigwedge\limits_{i=1}^{n} A_i$ as the scientific observation that is verified by verifying the first $n$ scientific observations in the sequence. We define $\bigvee\limits_{i=1}^{\infty} A_i$ as the scientific observation that is verified by verifying at least one element in the infinite sequence.
\end{defn}

%\section{Scientific questions}
%
%We now turn our attention to scientific questions: these are questions that allow one or more answers can be verified experimentally, they are experimental observations. ``Is there extra-terrestrial life?" is a scientific question because ``there is extra-terrestrial life" is a experimental observation. ``What is the mass of this particle?" is a scientific questions because ``the mass is $9.1 \pm 0.05 \times 10^{-31}$ kg" is an experimental observation. ``How does the sauce taste?" can be answered by the experimental observation ``the sauce is sweet" or by the opinion ``the sauce is pretty good".
%
%Again we stress two aspects that may be counter intuitive. First is that not all answers to a scientific questions are experimental observations. ``The electron mass is precisely $9.1 \times 10^{-31}$ kg" is an answer to ``what is the mass of the electron?" but is not an experimental observation since we can't measure mass with infinite precision. Second is that not all answer are mutually exclusive. ``What animal is that?" allows ``that is a bird" and ``that is a duck" as answers, both of which are experimental observations.
%
%Consider now the set of all experimental observations that constitute a valid answer to a specific question. We can always take two or more of them and combine them using conjunction (i.e. logical AND) and disjunction (i.e. logical OR). These are also valid answers to the same question: if ``the sauce is sweet" and ``the sauce is sour" are both valid answers to ``how does the sauce taste?" then ``the sauce is sweet and sour" is also a valid answer.
%
%Mathematically we say that the set of all experimental observations to a specific answer is closed under finite conjunction and countable disjunction. We therefore can write the following:
%
%\begin{prop}
%	Let $A$ be the set of all experimental observations that can be used to answer a scientific question. $A$ is closed under finite conjunction and countable disjunction. That is, for all $A_1, A_2, ... , A_n, ... \in A$, $\bigwedge\limits_{i=1}^{n} A_i \in A$ and $\bigvee\limits_{i=1}^{\infty} A_i \in A$.
%\end{prop}

\section{Experimental distinguishability}

The most common and important type of experimental observation is when we want to identify a specific case from a set of possible ones. For example, we pick ``this is a duck" from all possible animals or ``the position of the ball is $3 \pm 0.5$ m" from all possible values of position.

We call $X$ a set of experimentally distinguishable elements if we can associate with it a set of experimental observations rich enough to be able to distinguish each element from the other. That is, given two possible elements (e.g. two possible animals or two possible positions) there always exists an experimental observation that can tell them apart. This property is a basic requirement to treat these elements in a scientific context: if we are not even able to tell them apart experimentally, they cannot be the subject of scientific investigation.

As we hinted before, observations do not necessarily identify a single element and they are not necessarily mutually exclusive. For example, ``that is an egg-laying animal" and ``this is a mammal" still allow for the platypus and the echidna. ``the position of the ball is $3 \pm 0.5$ m" and ``the momentum of the ball is $2 \pm 0.1$ kg m/s" limits the possible states to a rectangular region. In general, given an observation there is going to be a set of states that is compatible with that observation. For example, duck will be compatible with ``that is a bird" while cat will not.

Not only each experimental observation will be associated with a set, but two observations that share the same set are, for our purposes, equivalent. For example, ``that is a feathered animal" or ``that is a bird" will give us the same set of animals as all birds have feathers and all feathered animals are birds. The information they give us is the same. Therefore, we can treat an experimental observation as if it were equivalent to its associated set of elements. That is, if $U \subseteq X$ is the set of elements associated with the observation (e.g. all the birds), the observation may as well be ``the element is in U" (e.g. the element is within the set of birds).

But not all sets of elements are necessarily associated to an experimental observation. As we saw ``the position of the ball is precisely $1$ m" is not an observation because we can't measure position with infinite precision. Therefore the set of states for which position is exactly $1$ m is not associated with an experimental observation. The whole problem, then, is to be able to keep track of which set of elements is associated with an experimental observation and which is not.

At the very least, we need to be able to tell whether an object is or is not one of the elements in our set $X$. For example, we must be able to verify that ``that is an animal" or ``that is not an animal". In the same way, if we can't even verify whether something is a ball or not, it does not make sense to distinguish between the states of a ball. Therefore, if our set $X$ is truly formed by distinguishable elements, there will be an experimental observation associated with the full set $X$, equivalent to ``that is an element of the set", and another experimental observation assocaited to the empty set $\emptyset$, equivalent to ``that is not an element of the set".

As we saw before, we can always take the conjunction (logical AND) and disjunction (logical OR) of two experimental observation. How do these operations transpose in terms of sets? Suppose we have $U \subseteq X$ (e.g. the set of flying animals) with an associated experimental observation (e.g. ``that is a flying animal"). Suppose we have $V \subseteq X$ (e.g. the set of all insects) with an associated experimental observation (e.g. ``that is an insect"). We combine the two observations with a logical AND (e.g. ``that is a flying insect"). Only the elements that are both in $U$ and $V$ are compatible with it. That is: only the elements in the intersection $U \cap V$. This means that if $U$ and $V$ are two sets each associated with an experimental observation, then their intersection $U \cap V$ is also associated with an experimental observation. Similarly, if $U$ and $V$ are two sets each associated with an experimental observation, we can combine these with a logical OR. All elements either in $U$ or $V$ are compatible with it. That is: elements in the union $U \cup V$. This means the union $U \cup V$ is also associated with an experimental observation.

To sum up, let $X$ be a set of distinguishable elements, let $\mathsf{T}$ be the collection of all sets associated with an experimental observation. We have the following:
\begin{itemize}
	\item Both $X$ and $\emptyset$ can be found in $\mathsf{T}$ as we must be able to verify whether an object can or cannot be identified
	\item Any intersection of a finite number of sets in $\mathsf{T}$ can also be found in $\mathsf{T}$ as the finite conjunction (i.e. logical AND) of experimental observations is also an experimental observation
	\item Any union of a countable number of sets in $\mathsf{T}$ can also be found in $\mathsf{T}$ as the countable disjunction (i.e. logical OR) of experimental observations is also an experimental observation
\end{itemize}
This is the mathematical definition of a topology: a set of sets that contains both the whole set and the empty set, that is closed under finite intersection and under countable union.

\begin{prop}
	Let $X$ be a set of experimentally distinguishable elements. The set of experimental observations that can identify elements in $X$ induces a topology in $X$.
\end{prop}

\section{TBD}

What we have done so far is to assign to a set of distinguishable element $X$ a set of experimental observations $\mathsf{T}$. This is not enough as we currently have no guarantee that those observations allow to distinguish one element from all others. We break this requirement into two parts. First we will make sure that we can always distinguish between any two elements and then we will make sure that there aren't too many experimental observations such that it would be impossible to distinguish one element from the rest.

If two elements are experimentally distinguishable there we must have an experimental observation that can tell them apart. For example, to be able to define the house sparrow (Passer domesticus) as a separate animal from the Italian sparrow (Passer italiae) we have at least two observations (``it has a dark gray crown", ``it has a chestnut crown") that are mutually exclusive, each compatible with just one element.

That is, for each pair of elements $x_1, x_2 \in X$, we must be able to find two sets $U,V \in \mathsf{T}$ each associated with an experimental observation such that no element is in both (i.e. their intersection $U \cap V=\emptyset$ is empty) and one element is in one set (i.e. $x_1 \in U$ and $x_2 \in V$). This is the mathematical definition of a Hausdorff space: one in which distinct elements have disjoint neighborhoods. This guarantees we are able to make pair-wise distinctions.

Note that, once we have identified an element, we are able to tell which experimental observations are verified: the ones associated with a set that includes the element. For example, once we have identified the house sparrow, we have verified ``it has a gray crown", ``it is a bird", ``it has wings" and so on. Conversely, once we know all the experimental observations that can be verified, we have identified the element: the only one that is included in all sets. That is, once we have verified that ``it has a gray crown", ``it is a bird", ``it has wings" and so on, we have identified the house sparrow. In other words: identifying an element is equivalent to being able to verify all possible observations.

Fortunately we don't have to try to verify all observations: just enough to be able to calculate all other cases. For example, we can simply identify each animal one at a time (i.e. ``it is a cat", ``it is a dog", ``it is a duck" and so on) until we find the correct one. After that, we will know whether ``that is a bird" or ``that is a mammal". We define sub-base a set of experimental observations from which all others can be obtained through conjunction (i.e. logical AND) and disjunction (i.e. logical OR). In particular, we consider the smallest possible sub-base. This gives us the smallest number of experimental observation to test in order to distinguish an element from all others.

If the smallest sub-base is finite, it is possible to check all experimental observations. This is the case of the set of animals: there are a large but finite number of species. If it's infinite, we cannot. But if the smallest sub-base is countably infinite, we can get as close as we want. This is the case for the position of a ball: we can in principle always increase the precision. Suppose, in fact, that we have a large but finite number of elements $n$ among which we want to distinguish. We need a set of $n$ disjoint sets, each including one element. Note that the union will not make two overlapping sets disjoint, so these sets will only be intersections of elements of the sub-base. Intersections can only be finite, therefore we will need only a finite number of members of the sub-basis. If the sub-basis is countable, at some point we will get to the last member needed and we can stop: we can distinguish between the $n$ elements. If the sub-basis is not countable, instead, we will never stop: we can't experimentally distinguish between a set of arbitrary elements in a finite amount of time.

If there is a countable sub-base, then there is a countable base as well. This is the definition of a second countable space: one that allows a countable bases for its topology.

\begin{prop}
	Let $X$ be a topological space of experimentally distinguishable elements and the topology induced by the associated experimental observations. $X$ is Hausdorff and second countable.
\end{prop}

\section{Connections between topology and experimental distinguishability}

Now that we have made a tight link between topology and experimental distinguishability, we can go through some of the mathematical vocabulary and give it a more precise physical meaning.

An open set, a set in the topology, is a set for which there exists a way to  experimentally verify that an element is in that set. Conversely, a closed set, a set whose complement is in the topology, is a set for which there exists a way to experimentally refute that an element is in that set. For example, in the standard topology for $\mathbb{R}$ the interval $(2.5, 3.5)$ is an open set because $3 \pm 0.5$ is a valid measurement for a continuous quantity while $(3)$ is a closed set because, while we can measure a real number with infinite precision, we can exclude it.

The discrete topology, the one for which each singleton (i.e. set of one element) is both open and closed, corresponds to the ability to verify and refute each element individually. Any finite set\footnote{TODO what about countable?} that is Hausdorff and second countable has a discrete topology. This is probably why it is not as intuitive to think that something verifiable may not be refutable: it only happens in infinite sets.

The standard topology for $\mathbb{R}$, the one generate by all open intervals, corresponds to the ability of measuring continuous value only with finite precision. This topology is Hausdorff and second countable. While we can give a discrete topology to $\mathbb{R}$, this would no longer be second countable so it would violate our requirement for experimental distinguishability.

An interesting mathematical result is the following.

\begin{prop}
	A set $X$ of experimentally distinguishable elements has at most cardinality of continuum.
\end{prop}

This means that, no matter what technique we use now or in the future, the elements that we can properly define experimentally are at most infinite like the continuum. This already gives us a first simple and basic requirement a set of mathematical objects need to pass to be of scientific interest.

For example, these objects have cardinality of continuum, and therefore are good candidates:
\begin{itemize}
	\item Euclidean space $\mathbb{R}^n$
	\item all continuous functions from $\mathbb{R}$ to $\mathbb{R}$
	\item all open sets in $\mathbb{R}^n$
	\item all subsets of $\mathbb{N}$
\end{itemize}

These, instead, have cardinality greater then continuum, and therefore are not good candidates:
\begin{itemize}
	\item all functions from $\mathbb{R}$ to $\mathbb{R}$
	\item all subsets of $\mathbb{R}$
\end{itemize}


\section{Continuous functions}

Continuous functions are the only connections that preserve physical distinguishability.

\section{Distinguishability of functions}

General way to create a topology for continuous functions. Prove that it is Hausdorff and second countable

\section{Summary}

\begin{table}[h]
	\centering
\begin{tabular}{p{0.20\textwidth} p{0.7\textwidth}}
	Math/Topology & Science/Physics \\ 
	\hline 
	Open set & Verifiable set. We can verify experimentally that an element is within the set  \\ 
	Closed set & Refutable set. We can verify experimentally that an element is not in the set \\ 
	Continuous \newline function &  A function between two sets of physically distinguishable elements, a function that preserves distinguishability \\
	Homeomorphism &  A bijective function between two sets of physically distinguishable elements \\
\end{tabular} 
	\caption{Topology to physics dictionary}
\end{table}

	
\end{document}