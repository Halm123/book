% Possible use http://www.latextemplates.com/template/the-legrand-orange-book as template? See https://www.overleaf.com/9174958nyjxxdxbchks#/33024595/

\documentclass[11pt,letterpaper,fleqn]{memoir} % Default font size and left-justified equations

\usepackage[top=3cm,bottom=3cm,left=3cm,right=3cm,headsep=10pt,letterpaper]{geometry} % Page margins

% Theorem definitions using amsthm

\usepackage{amsthm}
\usepackage{amsmath}
\usepackage{amssymb}

% Remove line spaces between items of enumerate and itemize
\usepackage{enumitem}
\setlist{noitemsep}


% Adds double bracket symbols
\usepackage{stmaryrd}

% Latex symbol guide at http://mirrors.ibiblio.org/CTAN/info/symbols/comprehensive/symbols-letter.pdf

% LOGIC symbols
% -------------

% Allows to create negation symbols
\usepackage{MnSymbol}

\DeclareMathOperator{\truth}{truth}
\DeclareMathOperator{\poss}{possibilities}
\DeclareMathOperator{\result}{result}

\def\TRUE{\textsc{true}}
\def\FALSE{\textsc{false}}

\def\SUCCESS{\textsc{success}}
\def\FAILURE{\textsc{failure}}
\def\UNDEF{\textsc{undefined}}


% Symbols for tautology and contradiction
\def\tautology{\top}
\def\contradiction{\bot}

% Symbols for "compatibility" and "incompatibility"
\def\comp{\doublefrown}
\def\ncomp{\ndoublefrown}

% Symbols for "narrower" and "wider"
\def\narrower{\subseteq}
\def\nnarrower{\nsubseteq}
\def\broader{\supseteq}
\def\nbroader{\nsupseteq}


% Symbol for "independent" and "correlated"
\def\indep{\upmodels}
\def\nindep{\nupmodels}

% Aliases for logical operations
\def\AND{\wedge}
\def\bigAND{\bigwedge}
\def\OR{\vee}
\def\bigOR{\bigvee}
\def\NOT{\neg}


% Formatting for statements
\newcommand{\stmt}[1][s] {\mathsf{#1}}
% Formatting for experimental tests
\newcommand{\expt}[1][e] {\mathsf{#1}}
% Formatting for observations
\newcommand{\obs}[1] {\mathsf{#1}}
% Formatting for observation definition
\newcommand{\obsdef}[2] {\llparenthesis #1, #2 \rrparenthesis}

% Formatting for experimental domain
\newcommand{\edomain}[1][D] {\mathcal{#1}}

% Formatting for theoretical domain
\newcommand{\tdomain}[1][D] {\bar{\mathcal{#1}}}


% Formatting for sentence statements
\newcommand{\statement}[1] {\emph{``#1"}}


\usepackage{xcolor} % Required for specifying colors by name
\definecolor{sectionNumbers}{RGB}{44, 103, 0}


\renewcommand\thesubsection{\thesection.\Alph{subsection}}
\renewcommand{\theequation}{\thechapter.\arabic{equation}}

\newtheorem{assump}{Assumption}
\renewcommand*{\theassump}{\Roman{assump}}

\newtheorem{defn}[equation]{Definition}
\newtheorem{prop}[equation]{Proposition}

%\theoremstyle{definition}

\newenvironment{rationale}{\emph{Rationale}.}{\qed}
\newenvironment{justification}{\emph{Justification}.}{\qed}
\renewenvironment{proof}{\emph{Proof}.}{\qed}

% Style for math section
\RequirePackage[framemethod=default]{mdframed} % Required for creating the theorem, definition, exercise and corollary boxes
\newmdenv[skipabove=7pt,
skipbelow=7pt,
rightline=false,
leftline=true,
topline=false,
bottomline=false,
linecolor=sectionNumbers,
backgroundcolor=black!2,
innerleftmargin=5pt,
innerrightmargin=5pt,
innertopmargin=5pt,
leftmargin=0cm,
rightmargin=0cm,
linewidth=4pt,
innerbottommargin=5pt]{mathSection}


%----------------------------------------------------------------------------------------
%	SECTION NUMBERING IN THE MARGIN
%----------------------------------------------------------------------------------------

\makeatletter
\renewcommand{\@seccntformat}[1]{\llap{\textcolor{sectionNumbers}{\csname the#1\endcsname}\hspace{1em}}}                    
\renewcommand{\section}{\@startsection{section}{1}{\z@}
	{-4ex \@plus -1ex \@minus -.4ex}
	{1ex \@plus.2ex }
	{\normalfont\large\sffamily\bfseries}}
\renewcommand{\subsection}{\@startsection {subsection}{2}{\z@}
	{-3ex \@plus -0.1ex \@minus -.4ex}
	{0.5ex \@plus.2ex }
	{\normalfont\sffamily\bfseries}}
\renewcommand{\subsubsection}{\@startsection {subsubsection}{3}{\z@}
	{-2ex \@plus -0.1ex \@minus -.2ex}
	{.2ex \@plus.2ex }
	{\normalfont\small\sffamily\bfseries}}                        
\renewcommand\paragraph{\@startsection{paragraph}{4}{\z@}
	{-2ex \@plus-.2ex \@minus .2ex}
	{.1ex}
	{\normalfont\small\sffamily\bfseries}} % Loads the book formatting

\begin{document}
	
% Set theory (expected) \in, \subset, injective/surjective, cartesian product

% Closure of a set under an operation

% Need examples of topology and sigma algebra

% Note that sigma algebra are also closed under intersections

\chapter*{Assumptions of physics}

\textbf{This book is a work in progress}. This draft is a development copy built on \today. You can get the latest version for http://assumptionsofphysics.org. It is provided as-is for the purpose of early review and feedback. 

\chapter{Verifiable statements and experimental domains}

In this chapter we lay the foundations for our general mathematical theory of experimental science: a formalism that is broad enough to be applied to any area of scientific investigation. It is based on the idea of \textbf{verifiable statements}, assertions that are either true or false and that can be validated experimentally. Whether it is physics or chemistry, economics or psychology, medicine or biology, the goal is always to objective find some truth about the natural world that is supported by experimental data.

We group verifiable statements into \textbf{experimental domains} which represent the list of all possible testable answers to a particular scientific question. From those, we define \textbf{theoretical domains} which add those statements that, though not directly testable, can be used to describe predictions. Within each theoretical domain, we find those particular statements that, if true, predict the truthfulness or falsehood of all other statements. We call these the \textbf{possibilities} of the domain as they identify the complete descriptions that are admissible. To answer a scientific question, then, is to find which possibility is consistent with experimental data: the one that correctly predicts the result of all experimental tests.

We'll see how the above organization always exists on any given set of verifiable statements. That is, it is a fundamental structure for all sciences. We'll also see that these concepts are deeply intertwined with fundamental mathematical tools: experimental domains map to topologies while theoretical domains map to $\sigma$-algebras. These two core mathematical structures provide the foundation for differential geometry, Lie algebras, measure theory, probability theory and other mathematical branches that are heavily used in physics and other sciences.

As a consequence of this connection, we can build a more precise, intuitive and insightful understanding of what these mathematical structures are meant to represent in the scientific world. It also reveals why these mathematical tools are so pervasive and successful in science.

\section{Statements}

Statements will be the cornerstone of our general mathematical theory of experimental science. Statements are declarative sentences that are either true or false, like \statement{the mass of the electron is $511 \pm 0.5$ keV} or \statement{that animal is a cat}. In this section we will outline the basic definitions that allow us to combine statements into other statements (e.g. \statement{that animal is black} and \statement{that animal a cat} gives \statement{that animal is a black cat}) and to compare their content (e.g. \statement{the mass of the electron is $511 \pm 0.5$ keV} and \statement{the mass of the electron is $0.511 \pm 0.0005$ MeV} are equivalent).

Science is the systematic study of the physical world through observation and experimentation. Therefore we start our discussion introducing the following principle that formally captures that idea.

\begin{mathSection}
	\textbf{Principle of scientific objectivity}.
		Science is universal, non-contradictory and evidence based.
\end{mathSection}

Consider assertions like \statement{jazz is marvelous} or \statement{green and red go well together}. These are not objective: there is no agreed upon definition or procedure for what constitutes marvelous music or good color combination. Because of their nature, they can't be the subject of scientific inquiry. This does not mean that marvelous music or good color combinations do not exist or are not worth studying.\footnote{In fact, one can argue that most of the things that make life worth living (e.g. love, friendship, arts, purpose and so on) defy objective characterization and, therefore, that science gives us certain truth about trifling matters.} What the principle tells us is simply that if we choose to do science, we are limiting ourselves to those assertions that are either true or false (i.e. non-contradictory) for everybody (i.e. universal): assertions that have a single truth value. We call these assertions statements: they are the basic building blocks of a scientific description as only these can be studied scientifically.

\begin{mathSection}
\begin{defn}
	The \textbf{Boolean domain} is the set $\mathbb{B} = \{\FALSE, \TRUE\}$ of all possible truth values.
\end{defn}


\begin{axiom}\label{def_statement}
	A \textbf{statement} $\stmt$ is a declarative sentence that is either true or false. Formally, a statement is an element of the set $\mathcal{S}$ of all statements upon which is defined a function $\truth: \mathcal{S} \to \mathbb{B}$ that returns the truth value for each element.
\end{axiom}

\end{mathSection}

The idea of statements has their origin in the philosophical tradition of classical logic, \statement{Socrates is a man} being a classic example. Any language can be used to form them, formal or natural, as indeed any language is used in practice. This means we are not going to care what particular syntax (i.e. symbols and grammar rules) is used.\footnote{In general, statements in this context are not necessarily well formed formulas, predicates or similar concepts in the context of mathematical or propositional logic. Scientific investigation in the broad sense of learning from experimentation predates math and formal languages: information about agriculture, astronomy, metallurgy, botany and the like were collected and used even before the written word. Moreover, cognitive scientists have shown that children start using deliberate experimentation at a very young age to understand the world around them, even before their speech is fully developed. Ultimately, that knowledge is encoded in the language of electrical and biochemical signals. Formal languages are indeed extremely helpful in that they allow us to be more precise and to better keep track of possible inconsistencies, but ultimately one always needs natural language to give meaning and context to the mathematical symbols.} In fact, even a grammatically incorrect statement is fine as long as the intent is clear.

On the other hand, we are going to care about the semantics of the statements (i.e. their content and meaning). For example, when we say \statement{Socrates is a man} it has to be clear who is Socrates and what a man is. If it didn't, we would have no idea what to experimentally test and how.

At the very least, we should be able to know whether there is anything to test in the first place. Consider the statement \statement{that cat is a bird}. There is nothing to experimentally test here: based on the definitions of cat and bird we know the statement can never be true, no matter what particular cat we are considering. The statement provides no new information. Consider, instead, \statement{that cat is an animal}. This statement can never be false. Each statement, then, not only has a truth value but also a set of possibilities: the set of possible truth values the statement is allowed to take based on its content.

We will call a tautology a statement, like \statement{that cat is an animal}, that can only be true and a contradiction a statement, \statement{that cat is a bird}, that can only be false.

\begin{mathSection}
	
	\begin{axiom}\label{def_possibilities}
		The \textbf{possibilities} of a statement $\stmt$ are the possible truth values allowed by the content of the statement. Formally, on the set $\mathcal{S}$ of all statements is also defined a function $\possFn: \mathcal{S} \to \{\{\FALSE, \TRUE\},\{\FALSE\},\{\TRUE\}\}$ such that $\truth(\stmt) \in \possFn(\stmt)$ for all $\stmt \in \mathcal{S}$.
	\end{axiom}
	
	\begin{defn}
		A \textbf{tautology} $\tautology$ is a statement that must be true simply because of its content. That is, $\possFn(\stmt) = \{\TRUE\}$.
	\end{defn}
	
	\begin{defn}
		A \textbf{contradiction} $\contradiction$ is a statement that must be false simply because of its content. That is, $\possFn(\stmt) = \{\FALSE\}$.
	\end{defn}
	
\end{mathSection}

Next we want to keep track of statements whose truths are a function of the truth of other statements. Consider \statement{that animal is a cat} and \statement{that animal is not a cat}: if the first one is true then the second is false and vice-versa. In this sense, the second statement is a function of the first. Therefore, in general, a statement is a function of other statements if its truth is determined by the truth values of the other statements.

Since statements are intangible, there are no limits to the number of arguments one statement may depend on. For example, consider the statement \statement{the mass of the electron is $511 \pm 0.5$ keV} and the set of all the statements of the form \statement{the mass of the electron is exactly $x$ keV} with $510.5 < x < 511.5$. If any of the latter is true then the original statement is true as well. Given that $x$ is a real number, that is uncountably many statements so the original statement can be seen as a function of uncountably many statements.

At first one may expect that the possibilities of the final statement will be determined by the possibilities of the original statement, but this is not the case in general. Consider the following statements: \statement{Felix is a cat}, \statement{Felix is black} and \statement{Felix is a dog}. All three can independently be true or false. But just knowing that does not allow us to know whether they can be true at the same time. It is because we know the meaning of cat, black and dog that we know that \statement{Felix is a cat and a dog} is a contradiction while \statement{Felix is a cat and black} is not. If the statements where written in a language we didn't understand, we would have no clue. There are, however, some constraints that the possibilities have to satisfy, though we leave the details in the justification that follows as they are a bit technical. The important point is that looking at the possibilities of combined statements tells us something more. Each statement can tell us whether it can be true or false individually while functions of statements can tell us whether those statement can be true or false at the same time.

\begin{mathSection}
	\begin{axiom}\label{def_functions_of_statement}
		We can always construct a statement whose truth value arbitrarily depends on an arbitrary set of statements. Formally: given an arbitrary truth function $f : \mathbb{B}^n \to \mathbb{B}$ and a set of statements $\{\stmt_1, ..., \stmt_n\} \subseteq \mathcal{S}$  whose cardinality matches the arity (i.e. number of arguments) of the function, there exists a unique statement in $\mathcal{S}$, noted as $f(\stmt_1, ..., \stmt_n)$, such that:
		\begin{itemize}
			\item its truth value is the result of the truth function $f$: \newline $\truth(f(\stmt_1, ..., \stmt_n)) = f(\truth(\stmt_1), ..., \truth(\stmt_n))$
			\item its possibilities are consistent with the truth function $f$: \newline $\possFn(f(\stmt_1, ..., \stmt_n)) \subseteq f(\possFn(\stmt_1) \times ... \times \possFn(\stmt_n))$
			\item all possibilities for all arguments are allowed:
			\newline let $1 \leq i \leq n$, for every truth value $t_i \in \possFn(\stmt_i) \subseteq \mathbb{B}$ there exists a combination of possibilities $(t_1, ..., t_n) \in \possFn(\stmt_1) \times ... \times \possFn(\stmt_n)$ such that the value for the $i$-th argument is $t_i$ and  $f(t_1, ..., t_n) \in \possFn(f(\stmt_1, ..., \stmt_n))$.
		\end{itemize}
	This also holds in the case of infinite, possibly uncountable, arguments.
	\end{axiom}
	\begin{justification}
		Since this is an axiom, there is technically nothing to prove. However, we need to make sure the starting points of our framework are well justified.
		
		The idea that we can use statements to construct other statements is something well established in logic. The first property simply states that the truth of the final statement is calculable from the truth of the arguments.
		
		On the other hand, it is not possible in general to calculate the possibilities of the result given only the truth value and possibilities of the arguments. A few examples will illustrate the problem.
		
		Let $\stmt_1$ be a statement such that $\truth(\stmt_1) = \TRUE$ and $\possFn(\stmt_1) = \{\FALSE, \TRUE\}$. Let $\stmt_2$ be a statement such that $\truth(\stmt_2) = \FALSE$ and $\possFn(\stmt_2) = \{\FALSE, \TRUE\}$. Let $f : \mathbb{B} \times \mathbb{B} \to \mathbb{B}$ be the truth function that returns $\TRUE$ only if the arguments are $\TRUE$ (i.e. the logical conjunction as we'll see later). Consider $f(\stmt_1, \stmt_2)$. We would expect $\truth(f(\stmt_1, \stmt_2)) = \FALSE$ since one of the statement is not $\TRUE$. This is guaranteed by the first property.
		
		The possibilities, though, cannot be known without further information. Suppose $\stmt_1$ is \statement{Felix is a cat} and that $\stmt_2$ is \statement{Felix is black}. If these are the actual statements, then we would expect $\possFn(f(\stmt_1, \stmt_2)) = \{\FALSE, \TRUE\}$ since Felix could be a black cat or not a black cat. However, suppose $\stmt_1$ is \statement{Felix is a cat} and that $\stmt_2$ is \statement{Felix is a dog}. If these are the actual statements, then we would expect $\possFn(f(\stmt_1, \stmt_2)) = \{\FALSE\}$ since Felix cannot be both a cat and a dog. Depending on the extra information we reach different results.
		
		While we are not always able to calculate the possibilities, we can put some bounds. Suppose $f$ is as defined before, and both $\stmt_1$ and $\stmt_2$ are tautologies. In this case we would expect $\FALSE \notin \possFn(f(\stmt_1, \stmt_2))$ since both statements cannot be false and $f$ only returns false when one of the inputs is false. This justifies the second property: the result can only be possibly true or false if the possibilities of the arguments applied to the truth functions allow for it.
		
		Now suppose $f$ is as before, $\stmt_1$ is a tautology and $\possFn(\stmt_2) = \{\FALSE, \TRUE\}$. We would expect that  $\possFn(f(\stmt_1, \stmt_2)) = \{\FALSE, \TRUE\}$. Suppose, in fact, that $f(\stmt_1, \stmt_2)$ could only be true. Then we would expect that the arguments can never be false, since $f$ only returns true when both inputs are true. But this can't be since $\stmt_2$ is not a tautology. On the other hand, suppose that $f(\stmt_1, \stmt_2)$ could only be false. This would mean that $\stmt_2$ can only be false, since $f$ only returns false when one of the inputs is false and $\stmt_1$ can only be true. But this, again, can't be. This justifies the third property: the possibilities of the result can never limit the possibilities of the arguments in a way that is contradictory.	
	\end{justification}
\end{mathSection}

To better characterize truth functions, we borrow ideas and definitions from Boolean algebra which is the branch of algebra that operates on truth values. Boolean algebra is fundamental in logic and computer science, since every digital circuit ultimately is implemented on two-state systems (e.g. high/low voltage, up/down magnetization).  One of its main results is that all truth functions can be expressed by combining the following three simple operations: negation (i.e. logical NOT), conjunction (i.e. logical AND) and disjunction (i.e. logical OR).

Suppose $\stmt_1$ = \emph{``the sauce is sweet"} and $\stmt_2$ = \emph{``the sauce is sour"}. We can apply the three operations to make this table:

\begin{table}[h]
	\centering
	\begin{tabular}{p{0.2\textwidth} p{0.1\textwidth} p{0.1\textwidth} p{0.5\textwidth}}
		Operator & Gate & Symbol & Example \\ 
		\hline 
		Negation & NOT & $\NOT \stmt_1$ &  \emph{``the sauce is not sweet"} \\ 
		Conjunction & AND & $\stmt_1 \AND \stmt_2$ & \emph{``the sauce is sweet and sour"} \\ 
		Disjunction & OR & $\stmt_1 \OR \stmt_2$ & \emph{``the sauce is at least sweet or sour"}
	\end{tabular} 
	\caption{Boolean operations on statements.}
\end{table}

Most languages typically already provide similar operations, as the examples show. Technically, though, we should consider the ones defined here as meta-operations that are defined outside the language of the statements. For example, \statement{x is the position of a ball}$\AND$\statement{$\,\frac{d^2 x}{dt^2} = - g$} stitches together an English statement with a calculus statement into a new statement that is neither. This kind of mix should be allowed as it does happen in practice.

\begin{mathSection}
	\begin{defn}
		The \textbf{negation or logical NOT} is the function $\NOT : \mathbb{B} \to \mathbb{B}$ that takes a truth value and returns its opposite. That is: $\NOT \TRUE = \FALSE$ and $\NOT \FALSE = \TRUE$. The negation $\NOT \stmt$ of a statement $\stmt$ is the statement whose truth is the negation of the truth of $\stmt$.
	\end{defn}
	
	\begin{defn}
		The \textbf{conjunction or logical AND} is the function $\AND : \mathbb{B} \times \mathbb{B} \to \mathbb{B}$ that returns $\TRUE$ only if all the arguments are $\TRUE$. That is: $\TRUE \AND \TRUE = \TRUE$ and $\TRUE \AND \FALSE =\FALSE \AND \TRUE =\FALSE \AND \FALSE = \FALSE$. The conjunction $\stmt_1 \AND \stmt_2$ is the statement whose truth is the conjunction of the truths of $\stmt_1$ and $\stmt_2$.
	\end{defn}
	
	\begin{defn}
		The \textbf{disjunction or logical OR} is the function $\OR : \mathbb{B} \times \mathbb{B} \to \mathbb{B}$ that returns $\FALSE$ only if all the arguments are $\FALSE$. That is: $\FALSE \OR \FALSE = \FALSE$ and $\TRUE \OR \FALSE =\FALSE \OR \TRUE =\TRUE \OR \TRUE = \TRUE$. The disjunction $\stmt_1 \OR \stmt_2$ is the statement whose truth is the disjunction of the truths of $\stmt_1$ and $\stmt_2$.
	\end{defn}
\end{mathSection}

Negation, conjunction and disjunction as we have defined them form the two-values Boolean algebra and they are sufficient to express any truth function. Consider the statement $\stmt=$\statement{the sauce is sweet and sour or it is neither}. This is a function of the two statements $\stmt_1$ and $\stmt_2$ defined before: if we know whether $\stmt_1$ and $\stmt_2$ are true, we can tell whether $\stmt$ is true as well. The idea is that we can express the function as all possible cases that make the result true. For example, $\stmt$ will be true if the sauce is sweet and sour or if it is not sweet and not sour. That is: $(\stmt_1 \AND \stmt_2) \OR (\NOT \stmt_1 \AND \NOT \stmt_2)$. Similarly, the statement \statement{the sauce is not sweet and sour} can be expressed as $(\stmt_1 \AND \NOT \stmt_2) \OR (\NOT \stmt_1 \AND \stmt_2) \OR (\NOT \stmt_1 \AND \NOT \stmt_2)$ since is going to be true in all cases except the one where the sauce is sweet and sour.

Each of the cases is the conjunction of all arguments where each one appears only once, negated or not. We call these expressions minterms. A function can always be expressed as the disjunction of all the minterms for which the function is true. This is called its disjunctive normal form.%TODO maybe explain the term?

\begin{mathSection}
	\begin{defn}\label{def_minterm}
		Let $\{t_1, ..., t_n\} \subseteq \mathbb{B}^n$ be a set of truth values. A \textbf{minterm} of $\{t_1, ..., t_n\}$ is a conjunction where each element appears only once, either negated or not. That is, it can be written as $m = \bigAND \limits_{i=1}^n (\NOT)^{a_{i}} \, t_i$ where $a_{i} \in \mathbb{B}$, $\NOT ^ \TRUE \, t_i = t_i$ and $\NOT ^ \FALSE \, t_i = \NOT t_i$. In this notation, $m = \TRUE$ if and only if $t_i = a_i$ for all $i=1...n$. This extends to arbitrary sets of truth values. Similarly, we call a minterm of a set of statements a conjunction where each statement appears only once, either negated or not.
	\end{defn}
	
	\begin{prop}\label{prop_disjunctive_normal_form}
		Any function $f : \mathbb{B}^n \to \mathbb{B}$ that takes $n$ truth values and returns a truth value, which we call a \textbf{truth function}, can be expressed in its \textbf{disjunctive normal form} as a disjuction of minterms of the arguments. Formally, $f(t_1, ..., t_n) =\bigOR \limits_{i=1}^m \left( \bigAND \limits_{j=1}^n (\NOT)^{a_{ij}} \, t_j \right)$ where $m \in \mathbb{N}$ and  $a_{ij} \in \mathbb{B}$. This extends to functions of arbitrary arguments. Similarly, we call disjunctive normal form of a function of statements the analogous construction in terms of statements.
	\end{prop}
	\begin{proof}
		We first show that this can be done for a function that returns $\TRUE$ for a unique combination of values. Let $a_1, ..., a_n \in \mathbb{B}$ be $n$ truth values. Let $f_1: \mathbb{B}^n \to \mathbb{B}$ be a function such that $f_1(t_1, ..., t_n) = \TRUE$ if and only if $t_j = a_j$ for all $j=1...n$. Consider the minterm $\bigAND \limits_{j=1}^n (\NOT)^{a_{j}} \, t_j$. It will be $\TRUE$ if and only if $t_j \equal a_j$ for all $j=1...n$. Then we have $f_1(t_1, ..., t_n) = \bigAND \limits_{j=1}^n (\NOT)^{a_{j}} \, t_j$ since both functions return the same values for the same arguments.
		
		Now we generalize the result for arbitrary functions. Let $f : \mathbb{B}^n \to \mathbb{B}$ be a truth function. Let $m \in \mathbb{N}$ be the number of value combinations for which $f(t_1, ..., t_n) = \TRUE$. For each value combination $i=1...m$ let $a_{ij} \in \mathbb{B}$ be the sequence of values. Then $f_i = \bigAND \limits_{j=1}^n (\NOT)^{a_{ij}} \, t_j$ is the minterm for each value combination. Consider $\bigOR \limits_{i=1}^m f_i$. This function returns $\TRUE$ if and only if the arguments match one of the value combinations for which $f$ returns $\TRUE$. Then we have $f(t_1, ..., t_n) =\bigOR \limits_{i=1}^m \left( \bigAND \limits_{j=1}^n (\NOT)^{a_{ij}} \, t_j \right)$ since they return the same values for the same arguments.
		
		This procedure can be generalized to the case where the number of arguments of $f$ and the number of minterms is infinite.
	\end{proof}
\end{mathSection}

Since the disjunctive normal form allows us to write all functions in terms of just negation, conjunction and disjunction, it is sufficient to study those operations to understand the properties of all functions. Given the previous definitions, we can deduce the following.

\begin{mathSection}
	\begin{prop}
		The set of all statements $\mathcal{S}$ is closed under negation, arbitrary conjunction and arbitrary disjunction.
	\end{prop}
	\begin{proof}
		Negation, arbitrary intersection and arbitrary union are particular truth functions. The statement associated with them always exists by axiom \eqref{def_functions_of_statement}.
	\end{proof}
	\begin{prop}\label{boolean_properties}
		The set of all statements $\mathcal{S}$ satisfies the following properties:
		\begin{itemize}
			\item associativity: $a \OR (b \OR c) = (a \OR b) \OR c$, $a \AND (b \AND c) = (a \AND b) \AND c$
			\item commutativity: $a \OR b = b \OR a$, $a \AND b = b \AND a$
			\item absorption: $a \OR (a \AND b) = a$, $a \AND (a \OR b) = a$
			\item identity: $a \OR \contradiction = a
			$, $a \AND \tautology = a$
			\item distributivity: $a \OR (b \AND c) = (a \OR b) \AND (a \OR c)$, $a \AND (b \OR c) = (a \AND b) \OR (a \AND c)$
			\item complements: $a \OR \NOT a = \tautology$, $a \AND \NOT a = \contradiction$
			\item De Morgan: $\NOT a \OR \NOT b = \NOT (a \AND b)$, $\NOT a \AND \NOT b = \NOT (a \OR b)$
		\end{itemize}
		This, by definition, means $\mathcal{S}$ is a \textbf{Boolean algebra}.
	\end{prop}
	\begin{proof}
		The left and right expressions for each equality correspond to the same truth function applied to the same statements. Therefore, by axiom  \eqref{def_functions_of_statement}, they correspond to the same statement.
	\end{proof}
\end{mathSection}

These operations and properties define the \textbf{algebra of statements}. The relationships we found are the logical identities of classical logic. These are exactly what we need to make sure the truth values of all our statements are consistent. Note, though, that in our framework they are simply inherited from the properties of truth functions and are not separate axioms.

Now we have all the elements to define when two statements have the same content. Consider the two statements \statement{that animal is a cat} and \statement{quell'animale \`{e} un gatto}: they mean the same thing but in different languages. Consider \statement{the mass of the electron is $511 \pm 0.5$ keV} and \statement{the mass of the electron is $0.511 \pm 0.0005$ MeV}: they represent the same measurement but in different units. So, how can we express the fact that two statements $\stmt_1$ and $\stmt_2$ mean the same thing? The idea is that they can never have opposite truth values. If the first is true, then the second must be true. If the first is false, then the second must be false. That is: either $\stmt_1 \AND \stmt_2$ is true or $\NOT\stmt_1 \AND \NOT\stmt_2$ is true. In other words $(\stmt_1 \AND \stmt_2) \OR (\NOT\stmt_1 \AND \NOT\stmt_2)$ can only be true: it's a tautology.

\begin{mathSection}

\begin{defn}
	Two statements $\stmt_1$ and $\stmt_2$ are \textbf{equivalent} $\stmt_1 \equiv \stmt_2$ if they must be equally true or false simply because of their content. That is, $(\stmt_1 \AND \stmt_2) \OR (\NOT\stmt_1 \AND \NOT\stmt_2)$ is a tautology.
\end{defn}

\begin{prop}
	Statement equivalence satifies the following properties:
	\begin{itemize}
		\item reflexivity: $s \equiv s$
		\item symmetry: $s_1 \equiv s_2$ if and only if $s_2 \equiv s_1$
		\item transitivity: if $s_1 \equiv s_2$ and $s_2 \equiv s_3$ then $s_1 \equiv s_3$
	\end{itemize}
	and is therefore an equivalence relation.
\end{prop}
\begin{proof}
	For reflexivity, we have $(\stmt \AND \stmt) \OR (\NOT\stmt \AND \NOT\stmt) = (\stmt) \OR (\NOT\stmt) = \tautology$. Therefore $s \equiv s$.
	
	For symmetry, we have $s_1 \equiv s_2$ implies $\tautology = (\stmt_1 \AND \stmt_2) \OR (\NOT\stmt_1 \AND \NOT\stmt_2) = (\stmt_2 \AND \stmt_1) \OR (\NOT\stmt_2 \AND \NOT\stmt_1)$. Therefore $s_2 \equiv s_1$.
	
	For transitivity, consider the statements:
	\begin{align*}
\stmt_{12} &= (\stmt_1 \AND \stmt_2) \OR (\NOT\stmt_1 \AND \NOT\stmt_2) \\
\stmt_{23} &= (\stmt_2 \AND \stmt_3) \OR (\NOT\stmt_2 \AND \NOT\stmt_3) \\
\stmt_{13} &= (\stmt_1 \AND \stmt_3) \OR (\NOT\stmt_1 \AND \NOT\stmt_3) \\
\stmt_{123} &= (\stmt_1 \AND \stmt_2 \AND \stmt_3) \OR (\NOT\stmt_1 \AND \NOT\stmt_2 \AND \NOT\stmt_3).
\end{align*}
These statements are true if and only if the values of the respective arguments are the same. Note that $\stmt_{12} \AND \stmt_{23} = \stmt_{123}$. In fact:
	\begin{align*}
\stmt_{12} \AND \stmt_{23}  &= ((\stmt_1 \AND \stmt_2) \OR (\NOT\stmt_1 \AND \NOT\stmt_2)) \AND ((\stmt_2 \AND \stmt_3) \OR (\NOT\stmt_2 \AND \NOT\stmt_3)) \\
&= ((\stmt_1 \AND \stmt_2) \AND (\stmt_2 \AND \stmt_3)) \OR ((\NOT\stmt_1 \AND \NOT\stmt_2) \AND (\stmt_2 \AND \stmt_3)) \OR \\
& ((\stmt_1 \AND \stmt_2) \AND (\NOT\stmt_2 \AND \NOT\stmt_3)) \OR ((\NOT\stmt_1 \AND \NOT\stmt_2) \AND (\NOT\stmt_2 \AND \NOT\stmt_3)) \\
&= (\stmt_1 \AND \stmt_2 \AND \stmt_3) \OR (\NOT\stmt_1 \AND \NOT\stmt_2 \AND \NOT\stmt_3) = \stmt_{123}.
\end{align*}
That is, if the truth values are equal pairwise then they are all equal. Similarly, $\stmt_{13} \AND \stmt_{23} = \stmt_{123}$. If $s_1 \equiv s_2$ and $s_2 \equiv s_3$ then $\possFn(\stmt_{12})=\possFn(\stmt_{23})=\{\TRUE\}$. By the second property in axiom \ref{def_functions_of_statement} we also have $\possFn(\stmt_{123}) = \possFn(\stmt_{12} \AND \stmt_{23}) \subseteq \possFn(\stmt_{12}) \AND \possFn(\stmt_{23}) = \{\TRUE\} \AND \{\TRUE\} = \{\TRUE\}$. Therefore $\stmt_{123}$ is a tautology, meaning all three statements cannot possibly have different truth values.

Consider now $\stmt_{13} \AND \stmt_{123} = \stmt_{13} \AND \stmt_{13} \AND \stmt_{23} = \stmt_{13} \AND \stmt_{23} = \stmt_{123}$. By the third property in axiom \ref{def_functions_of_statement}, for each $t_{13} \in \possFn(\stmt_{13})$ we must find suitable $t_{123} \in \possFn(\stmt_{123})$ and $\hat{t}_{123} \in \possFn(\stmt_{123})$ such that $t_{13} \AND t_{123} = \hat{t}_{123}$. But since $\possFn(\stmt_{123}) = \{\TRUE\}$, then we must have $t_{13} \AND \TRUE = \TRUE$ which cannot be satisfied if $t_{13}=\FALSE$. Therefore $\FALSE \notin \possFn(\stmt_{13})$, $\possFn(\stmt_{13})=\{\TRUE\}$, $\stmt_{13}$ is a tautology and $s_1 \equiv s_3$.
\end{proof}

\end{mathSection}

Again, we want to stress that this notion of equivalence is not based on the truth value (i.e. whether the statements happen to be both true or false) or on properties in \ref{boolean_properties} (i.e. whether they are the same statement): it is based on the possibilities (i.e. whether the two statements can possibly have a different truth) and therefore on the content of the statement. Also note that the semantics defines the possibilities but not the truth value, unless the statement is a tautology or a contradiction. For example, even if the meaning of \statement{the next race is going to be won by Secretariat} is clear, we may be none the wiser about its truthfulness. Intuitively, the equivalence we defined here answers the question: do these two statements carry the same information? Is experimentally testing the first the same as experimentally testing the second? If that's the case, they are essentially equivalent to us. So much so, that from now on we will implicitly assume two different statements to be inequivalent.\footnote{Technically, when we'll say that $\stmt$ is a statement we actually mean $\stmt$ is an equivalence class of statements. We are not going to be explicit about the distinction, though, as we feel it simply distracts without adding greater clarity. We'll let the context determine what is meant.}

Equivalence is not the only semantic relationship that we want to capture. Consider the contents of the following:
\begin{description}
	\item $\stmt_1=$\statement{that animal is a cat}
	\item $\stmt_2=$\statement{that animal is a mammal}
	\item $\stmt_3=$\statement{that animal is a dog}
	\item $\stmt_4=$\statement{that animal is black}
\end{description}
The second will be true whenever the first is true. In this case we say the first statement is narrower than the second ($\stmt_1$ $\narrower$ $\stmt_2$) because \statement{that animal is a cat} is more specific than \statement{that animal is a mammal}. The third will be false whenever the first is true. In this case we say that they are incompatible ($\stmt_1$ $\ncomp$ $\stmt_3$) because \statement{that animal is a dog} and \statement{that animal is a cat} can never be true at the same time. The fourth will be true or false regardless of whether the first is true. In this case we say that they are independent ($\stmt_1$ $\indep$ $\stmt_4$) because knowing whether \statement{that animal is a cat} tells us nothing about whether \statement{that animal is black}. As for equivalence, we can define these relationships upon the previous definitions.

\begin{mathSection}

\begin{defn}
	Given two statement $\stmt_1$ and $\stmt_2$, we say that:
	\begin{itemize}
		\item $\stmt_1$ \textbf{is narrower than} $\stmt_2$ (noted $\stmt_1 \narrower \stmt_2$) if $\stmt_2$ is true whenever $\stmt_1$ is true simply because of their content. That is, $\stmt_1 \AND \NOT \stmt_2 \equiv \contradiction$.
		\item $\stmt_1$ \textbf{is broader than} $\stmt_2$ (noted $\stmt_1 \broader \stmt_2$) if $\stmt_2 \narrower \stmt_1$.
		\item $\stmt_1$ \textbf{is compatible to} $\stmt_2$ (noted $\stmt_1 \comp \stmt_2$) if their content allows them to be true at the same time. That is, $\stmt_1 \AND \stmt_2 \nequiv \contradiction$.

	\end{itemize}
	The negation of these properties will be noted by $\nnarrower$, $\nbroader$ , $\ncomp$ respectively.
\end{defn}
\begin{defn}
	The elements of a set of statements $S \subseteq \mathcal{S}$ are said to be \textbf{independent} (noted $\stmt_1 \indep \stmt_2$ for a set of two) if their content is such that any combination of their possibilities is allowed. That is, $\possFn(f(S)) = f(\bigtimes\limits_{\stmt \in S} \possFn(\stmt))$ for any truth function $f : \mathbb{B}^{|S|} \to \mathbb{B}$. The negation of independence, will be noted by $\nindep$.
\end{defn}

\end{mathSection}

Depending on context, a statement could be narrower than another even if it is describing different qualities. For example, \statement{this harp seal is white} is narrower than \statement{this harp seal is less than one year old}. Since harp seals have white fur only for their first month, the first one can never be true while the second is not. Intuitively, if a statement implies another then it is narrower than the other. For example, \statement{this harp seal is white} implies \statement{this harp seal is less than one year old} so it is narrower.\footnote{We considered using the term implication directly, but it seems that it leads to confusion. Implication in classical logic is something different: it is simply another truth function. Moreover, saying that a contradiction is narrower than all other statements sounds better than saying that a contradiction implies all other statements.}

Note that independence is not transitive and pair-wise independence is not sufficient. Consider the following statements for an ideal gas:
\begin{enumerate}
	\item \statement{the pressure is $101\pm1$ kPa}
	\item \statement{the volume is $1\pm0.1$ $m^3$}
	\item \statement{the temperature is $293\pm1$ Kelvin}
\end{enumerate}
Since the three quantities are linked by the equation of state $PV=nRT$, any two statements are independent but the three together aren't. This notion of independence is similar, and in fact related, to statistical independence and linear independence.

Finally, we prove the following propositions that will be useful in later proofs. The first states that functions of equivalent statements are equivalent, which should be intuitively obvious. The others are simple mathematical relationships that provide no particular insight but are useful for calculations.

\begin{mathSection}
	
	\begin{prop}
		Two statements constructed using the same truth function and two sets of equivalent statements are equivalent. That is: given $f : \mathbb{B}^n \to \mathbb{B}$ then $f(\stmt_1, ..., \stmt_n) \equiv f(\bar{\stmt}_1, ..., \bar{\stmt}_n)$ if $\stmt_i \equiv \bar{\stmt}_i$ for all $1 \leq i \leq n$. This extends to functions of arbitrary arguments.
	\end{prop}
	\begin{proof}
		We first show the proposition holds when $f$ is the negation. Let $\stmt, \bar{\stmt} \in \mathcal{S}$ be two equivalent statements. Since $\stmt \equiv \bar{\stmt}$ we have $\{\TRUE\} = \possFn((\stmt \AND \bar{\stmt}) \OR (\NOT \stmt \AND \NOT \bar{\stmt})) = \possFn((\NOT \NOT \stmt \AND \NOT \NOT \bar{\stmt}) \OR (\NOT \stmt \AND \NOT \bar{\stmt})) = \possFn((\NOT \stmt \AND \NOT \bar{\stmt}) \OR (\NOT (\NOT \stmt) \AND \NOT (\NOT \bar{\stmt})))$. Which means  $\NOT \stmt \equiv \NOT \bar{\stmt}$.
		
		We then show the proposition holds when $f$ is the conjunction. Let $\stmt_1, \stmt_2, \bar{\stmt}_1, \bar{\stmt}_2 \in \mathcal{S}$ be four statements such that $\stmt_1 \equiv \bar{\stmt}_1$ and $\stmt_2 \equiv \bar{\stmt}_2$. Consider
		\begin{align*}
\stmt[a] &= ((\stmt_1 \AND \bar{\stmt}_1) \OR (\NOT \stmt_1 \AND \NOT \bar{\stmt}_1)) \AND ((\stmt_2 \AND \bar{\stmt}_2) \OR (\NOT \stmt_2 \AND \NOT \bar{\stmt}_2)) \\
&=(\stmt_1 \AND \bar{\stmt}_1 \AND \stmt_2 \AND \bar{\stmt}_2) \OR (\stmt_1 \AND \bar{\stmt}_1 \AND \NOT \stmt_2 \AND \NOT \bar{\stmt}_2) \OR (\NOT \stmt_1 \AND \NOT \bar{\stmt}_1 \AND \stmt_2 \AND \bar{\stmt}_2) \OR (\NOT \stmt_1 \AND \NOT \bar{\stmt}_1 \AND \NOT \stmt_2 \AND \NOT \bar{\stmt}_2).
		\end{align*}
		Since $\stmt_1 \equiv \bar{\stmt}_1$ and $\stmt_2 \equiv \bar{\stmt}_2$, $\stmt[a]$ is the conjunction of two tautologies so it is a tautology as well. Consider
		\begin{align*}
		\stmt[b] &= ((\stmt_1 \AND \stmt_2) \AND (\bar{\stmt}_1 \AND \bar{\stmt}_2)) \OR  (\NOT (\stmt_1 \AND \stmt_2) \AND \NOT (\bar{\stmt}_1 \AND \bar{\stmt}_2)) \\
		&= 	
		(\stmt_1 \AND \stmt_2 \AND \bar{\stmt}_1 \AND \bar{\stmt}_2) \OR ((\NOT \stmt_1 \OR \NOT \stmt_2) \AND (\NOT \bar{\stmt}_1 \OR \NOT \bar{\stmt}_2)) \\
		&= (\stmt_1 \AND \stmt_2 \AND \bar{\stmt}_1 \AND \bar{\stmt}_2) \OR (\NOT \stmt_1 \AND \NOT \bar{\stmt}_1) \OR (\NOT \stmt_1 \AND \NOT \bar{\stmt}_2) \OR (\NOT \stmt_2 \AND \NOT \bar{\stmt}_1) \OR (\NOT \stmt_2 \AND \NOT \bar{\stmt}_2).
		\end{align*}
		Note that the conjunction of $\stmt[b]$ with any minterm of $\stmt[a]$ is the same minterm. Therefore we have  $\stmt[a] \AND \stmt[b] = \stmt[a]$. But if $\stmt[a]$ is a tautology, then $\stmt[a] \AND \stmt[b]$ is a tautology because it is equal to $\stmt[a]$, and $\stmt[b]$ is tautology as well or we would violate the third property of \ref{def_functions_of_statement}. Since $\stmt[b]$ is a tautology, $\stmt_1 \AND \stmt_2 \equiv \bar{\stmt}_1 \AND \bar{\stmt}_2$.
		
		Given that all functions can be generated from negation, conjunction and disjunction, and given that disjunction can be generated from negation and conjunction, the result proven generalizes to all functions.
	\end{proof}
	
	\begin{prop}
		$\stmt_1 \narrower \stmt_2$ if and only if $\stmt_1 \AND \stmt_2 \equiv \stmt_1$.
	\end{prop}
	
	\begin{proof}
		Consider $\stmt_1 \AND \stmt_2 = \stmt_1 \AND \stmt_2 \OR \contradiction$. Since $\stmt_1 \narrower \stmt_2$, $\stmt_1 \AND \NOT \stmt_2 \equiv \contradiction$. We have $\stmt_1 \AND \stmt_2 \OR \contradiction \equiv ( \stmt_1 \AND \stmt_2 ) \OR (\stmt_1 \AND \NOT \stmt_2) = \stmt_1 \AND (\stmt_2 \OR \NOT \stmt_2) = \stmt_1 \AND \tautology = \stmt_1$. Therefore $\stmt_1 \AND \stmt_2 \equiv \stmt_1$. The same logic can be applied in reverse.
	\end{proof}	
	
	\begin{prop}
		$\stmt_1 \ncomp \stmt_2$ if and only if $\stmt_1 \AND \NOT \stmt_2 \equiv \stmt_1$.
	\end{prop}
	
	\begin{proof}
		Consider $\stmt_1 \AND \NOT \stmt_2 = \stmt_1 \AND \stmt_2 \OR \contradiction$. Since $\stmt_1 \ncomp \stmt_2$, $\stmt_1 \AND \NOT \stmt_2 \equiv \contradiction$. We have $\stmt_1 \AND \NOT \stmt_2 \OR \contradiction \equiv ( \stmt_1 \AND \NOT \stmt_2 ) \OR (\stmt_1 \AND \stmt_2) = \stmt_1 \AND (\NOT \stmt_2 \OR \stmt_2) = \stmt_1 \AND \tautology = \stmt_1$. Therefore $\stmt_1 \AND \NOT \stmt_2 \equiv \stmt_1$. The same logic can be applied in reverse.
	\end{proof}	
	
	\begin{prop}
		$\stmt_1 \narrower \stmt_2$ and $\stmt_1 \broader \stmt_2$ if and only if $\stmt_1 \equiv \stmt_2$.
	\end{prop}
	
	\begin{proof}
		Suppose $\stmt_1 \narrower \stmt_2$ and $\stmt_1 \broader \stmt_2$. Then $\stmt_1 \AND \stmt_2 \equiv \stmt_1$ since $\stmt_1 \narrower \stmt_2$ and $\stmt_1 \AND \stmt_2 \equiv \stmt_2$ since $\stmt_2 \narrower \stmt_1$. Therefore $\stmt_1 \equiv \stmt_2$. Conversely, suppose $\stmt_1 \equiv \stmt_2$. Then $\stmt_1 \AND \stmt_2 \equiv \stmt_1 \equiv \stmt_2$. Therefore $\stmt_1 \narrower \stmt_2$ and $\stmt_1 \broader \stmt_2$.
	\end{proof}	
\end{mathSection}

With these tools in place we are in a position to formulate models that are universal and non-contradictory. These models will be a collection of statements with a well defined content, whose truth value will be discovered experimentally.

\section{Verifiable statements and experimental domains}

We now focus on those statements that are verifiable: we have a way to experimentally confirm that the statement is true. The main result of this section is that not all functions of verifiable statements are verifiable statements. For example, since a test has to finish in a finite amount of time we are not going to be able to verify a statement that is the conjunction (i.e. logical AND) of infinitely many statements. We are also going to group verifiable statements into experimental domains which represent all the experimental evidence about a scientific subject that can be acquired in an indefinite amount of time.

The previous section took care of universality and non-contradiction, but the principle of scientific objectivity requires science to be evidence based. Consider the statements \statement{the square of the hypotenuse is equal to the sum of the squares of the other two sides} or \statement{God is eternal}. They deal with abstract concepts that cannot solely be defined experimentally and therefore cannot be experimentally tested. Again, this does not mean these concepts are of less significance, just that they cannot be the subject of scientific inquiry.\footnote{In fact, one may be more interested in them precisely because of their abstract, and therefore less transient, nature.}

Limiting the scope of our discussion to objects and properties that are well defined physically is also not enough. For example, \emph{``the electron is green"} or \emph{``1 meter is equal to 5 Kelvin"} are still not suitable scientific statements as the relationships established are not physically meaningful. Even when the relationship is meaningful, we may still not be able to validate it experimentally. For example, \emph{``there is no extra-terrestrial life"} or \emph{``the mass of the electron is exactly $9.109 \times 10^{-31}$ kg"} are not statements that can be verified in practice. In the first case, we would need to check every corner of the universe and find none, with the closest galaxy like ours, Andromeda, being 2.5 million light-years away; in the second case, we will always have an uncertainty associated with the measurement, however small.

So we have to narrow the scope to those and only those statements that can be verified experimentally. That is, we have to provide an experimental test: a repeatable experimental procedure (i.e. evidence based) that anyone (i.e. universal) can in principle execute and obtain consistent results (i.e. non-contradictory). This is both the power and the limit of scientific inquiry: it gives us a way to construct a coherent description of the physical world but it is limited to those aspects that can be reliably studied experimentally.

\begin{mathSection}
\begin{axiom}\label{def_experimental_tests}
	An \textbf{experimental test} $\expt$ is a repeatable procedure (i.e. it can be restarted and stopped at any time) that anybody can execute and will always either terminate successfully, terminate unsuccessfully or never terminate. Formally, an experimental test is an element of the set $\mathcal{E}$ of all experimental tests upon which is defined a function $\result: \mathcal{E} \to \{\SUCCESS, \FAILURE, \UNDEF\}$ that returns the result for each experimental test.
\end{axiom}
\begin{defn}
	A \textbf{verifiable statement} is a statement that can be shown to be true experimentally. Formally, given a verifiable statement $\stmt \in \mathcal{S}$ we can find an experimental test $\expt \in \mathcal{E}$ such that $\truth(\stmt)=\TRUE$ if and only if $\result(\expt)=\SUCCESS$.
\end{defn}
\end{mathSection}

Experimental tests are the second and last building block of our general mathematical theory of experimental science. As with statements, any language (e.g. natural, formal, engineering drawings, computer programs, ...) can in principle be used to describe the procedure, which can be arbitrarily complicated. It may require building detectors, gathering large amounts of data and performing complicated computations. We are not going to care how these procedures are described, just that it is done in a way that allows us to execute the test.\footnote{Trying to formalize a universal language for experimental tests is not only impractical but also conceptually problematic. To know what we can test experimentally is to know what is physically possible, which is equivalent to knowing the laws of physics, which is what we are trying to construct a framework for.}

As an example, consider the following procedure:
\begin{enumerate}
	\item find a swan
	\item if it's black terminate successfully
	\item go to step 1
\end{enumerate}
If a black swan exists, at some point we'll find it and the test will be successful. If a black swan does not exist, then the procedure will never terminate and the result is undefined. This is something anybody can do and will always provide the same result: it is an experimental test. It also terminates successfully if and only if a black swan exists, so the statement \statement{black swans exist} is verifiable.

Note that, in principle, science can also study statements that can be refuted experimentally. But the negation of those is a statement that can be verified experimentally. Therefore we lose nothing by only focusing on verification.\footnote{Mathematically, the spaces of verifiable statements and refutable statements are dual to each other.}

In the previous section we saw that we can combine statements into new statements. How about verifiable statements? Can we always combine verifiable statements into other verifiable statements? Since all truth functions can be constructed from the three basic Boolean operations, the question becomes: can we construct experimental tests for the negation, conjunction and disjunction of verifiable statements?

The first important result is that the negation of an experimental test, an experimental test that is successful when the first is not successful, does not necessarily exist. Consider our black swan example, an experimental test for the negation would be a procedure that terminates successfully if black swans do not exist. But the given procedure never finishes in that case, so it is not just a matter of switching success with failure. Because of non-termination, not-successful does not necessarily mean failure.\footnote{In this case, the old adage ``absence of evidence is not evidence of absence" applies.} Moreover, it is a result of computability theory that some problems are undecidable: they do not allow the construction of an algorithm that always terminates with a correct yes-or-no answer. So we know that in some cases this is not actually possible.

In the same vein we are able to confirm experimentally that \statement{the mass of this particle is not zero} but not that \statement{the mass of this particle is exactly zero} since we always have uncertainty in our measurements of mass. Even if we could continue shrinking the uncertainty arbitrarily, we would ideally need infinite time to shrink it to zero. What this means is that not all answers to the same question can be equally verified. Is the mass of the photon exactly zero? We can either give a precise ``no" or an imprecise ``it's within this range." Is there extra-terrestrial life? We can either give a precise ``yes" or an imprecise ``we haven't found it so far."\footnote{Note that we are on purpose avoiding induction. It does not play any role in our general mathematical theory of experimental science since the decision of when and how to apply induction violates the principle of scientific objectivity.}

\begin{mathSection}
	\emph{Remark}. The \textbf{negation or logical NOT} of a verifiable statement is not necessarily a verifiable statement.
\end{mathSection}

While this is true in general, we can still test the negation of many verifiable statements. Consider the statement \statement{this swan is black}. It allows the following experimental test:
\begin{enumerate}
	\item look at the swan
	\item if it's black terminate successfully
	\item terminate unsuccessfully
\end{enumerate}
Note that, since the test always terminates, we can switch failure to success and vice-versa. In this case we can test the negation and we say that the statement is decidable: we can decide experimentally whether it is true or false. It is precisely when and only when the test is guaranteed to terminate, that we can test the negation.

\begin{mathSection}
	\begin{defn}
		A \textbf{decidable statement} is a statement that can be shown to be either true or false experimentally. Formally, given a decidable statement we can find an experimental test $\expt \in \mathcal{E}$ such that $\truth(\stmt)=\TRUE$ if and only if $\result(\expt)=\SUCCESS$ and such that $\truth(\stmt)=\FALSE$ if and only if $\result(\expt)=\FAILURE$.
	\end{defn}
	\begin{axiom}
		Let $\stmt \in \mathcal{S}$ be a verifiable statement. Then $\NOT\stmt$ is verifiable if and only if $\stmt$ is decidable.
	\end{axiom}
	\begin{justification}
		Mathematically, we are simply assuming the negation of a decidable statement is decidable. However, we want to show that the axiom is well justified in practice.
		
		Let $\stmt \in \mathcal{S}$ be a decidable statement and $\expt \in \mathcal{E}$ an experimental test that verifies whether the statement is true or false. Consider the procedure $\expt_\NOT(\expt)$ defined as follows:
		\begin{enumerate}
			\item run test $\expt$
			\item if unsuccessful terminate successfully
			\item if successful terminate unsuccessfully.
		\end{enumerate}
		Since $\expt$ is repeatable and can be executed by anybody, $\expt_\NOT(\expt)$ is also repeatable and can be executed by anybody. Since $\stmt$ is decidable, $\expt$ always terminates and therefore $\expt_\NOT(\expt)$ always terminates.  Therefore $\expt_\NOT(\expt)$ is an experimental test such that $\truth(\stmt)=\FALSE$ if and only if $\result(\expt_\NOT(\expt))=\SUCCESS$. Which means $\truth(\NOT\stmt)=\TRUE$ if and only if $\result(\expt_\NOT(\expt))=\SUCCESS$ and $\NOT\stmt$ is verifiable.
		
		Let $\stmt \in \mathcal{S}$ be a verifiable statement such that $\NOT\stmt$ is also verifiable. Let $\expt, \expt_\NOT \in \mathcal{E}$ be their respective experimental tests. We have to be careful as $\expt$ and $\expt_\NOT$ may not terminate. Consider the procedure $\hat{\expt}$ defined as follows:
		\begin{enumerate}
			\item initialize $n$ to 1
			\item for each $i=1..n$
			\begin{enumerate}
				\item run the test $\expt$ for $n$ seconds
				\item if successful, return successfully
				\item run the test $\expt_\NOT$ for $n$ seconds
				\item if successful, return unsuccessfully
			\end{enumerate}
			\item increment $n$ and go to step 2
		\end{enumerate}
	    The procedure is repeatable and can be executed by anybody. Both $\expt$ and $\expt_\NOT$ are eventually run an arbitrarily long amount of time. If $\stmt$ is true, $\expt$ will eventually terminate successfully and $\hat{\expt}$ will do so as well. If $\stmt$ is false, $\expt_\NOT$ will eventually terminate successfully making $\hat{\expt}$ terminate unsuccessfully. Therefore $\stmt$ is decidable.
	\end{justification}

\end{mathSection}

Decidable statements are not critical elements of our theory. We introduce them here because their definition and related axiom clarify what happens during negation. They will be used again only much later when talking about discrete quantities.

Combining verifiable statements with conjunction (i.e. the logical AND) is more straightforward. If we are able to verify that \emph{``that animal is a swan"} and that \emph{``that animal is black"}, we can verify that \emph{``that animal is a black swan"} by verifying both. If the tests for both are successful, then the test for the conjunction is successful. That is, if we have two or more verifiable statements, we can always construct an experimental test for the logical AND by running all tests one at a time and check if they are successful. Yet, the number of tests needs to be finite or we would never terminate, so we are limited to the conjunction of a finite number of verifiable statements.

\begin{mathSection}
	\begin{axiom}\label{def_experimental_test_AND}
	The conjunction of a finite collection of verifiable statements is a verifiable statement. Formally, for any finite collection of experimental tests $\{\expt_i\}_{i=1}^{n} \subseteq \mathcal{E}$ we can construct the experimental test $ \bigAND\limits_{i=1}^{n} \expt_i \in \mathcal{E}$ such that $\result(\bigAND\limits_{i=1}^{n} \expt_i) = \SUCCESS$ if and only if $\result(\expt_i) = \SUCCESS$ for all $i=1..n$, which makes the finite conjunction of verifiable statements a verifiable statement.
	\end{axiom}
	\begin{justification}
		Mathematically, we are simply assuming that the finite conjunction exists, so in principle there is nothing to prove. However, we want to show that the axiom is well justified in practice.
		
		Let $\bigAND\limits_{i=1}^{n} \expt_i$ be the experimental procedure defined as follows:
		\begin{enumerate}
			\item for each $i=1..n$ run the test $\mathsf{e}_i$
			\item if all tests terminate successfully then terminate successfully.
		\end{enumerate}
		The experimental procedure so defined is repeatable, can be executed by anybody and terminates successfully if and only if all $\mathsf{e}_i$ terminate successfully as per the definition.
	\end{justification}
\end{mathSection}	

Combining verifiable statements with disjunction (i.e. the logical OR) is also straightforward. To verify that \emph{``the swan is black or white"} we can first test that \emph{``the swan is black"}. If that is verified that's enough: the swan is black or white. If not, we test that \emph{``the swan is white"}. That is, if we have two or more verifiable statements we can always construct an experimental test for the logical OR by running all tests and stopping at the first one that is successful. Because we stop at the first success, the number of tests can be countably infinite. As long as one test succeeds, which will always be the case when the overall test succeeds, it does not matter how many elements we are not going to verify later. But it cannot be more than countably infinite since the only way we have to find if one experimental test in the set is successful is testing them all one by one. Therefore we are limited to the disjunction of a countable number of verifiable statements.

\begin{mathSection}
	\begin{axiom}\label{def_experimental_test_OR}
	The disjunction of a countable collection of verifiable statements is a verifiable statement. Formally, for any countable collection of experimental tests $\{\expt_i\}_{i=1}^{\infty} \in \mathcal{E}$ we can construct an experimental test $ \bigOR\limits_{i=1}^{\infty} \expt_i \in \mathcal{E}$ such that $\result(\bigOR\limits_{i=1}^{\infty} \expt_i) = \SUCCESS$ if and only if $\result(\expt_i) = \SUCCESS$ for some $i \geq 1$, which makes the countable disjunction of verifiable statements a verifiable statement.
	\end{axiom}
	\begin{justification}
		Mathematically, we are simply assuming that the countable disjunction exists, so in principle there is nothing to prove. However, we want to show that the axiom is well justified in practice.
		
		In this case, we have to be careful to handle tests that may not terminate.
		Let $\bigOR\limits_{i=1}^{\infty} \expt_i$ be the experimental procedure defined as follows:
		\begin{enumerate}
			\item initialize $n$ to 1
			\item for each $i=1..n$
			\begin{enumerate}
				\item run the test $\mathsf{e}_i$ for $n$ seconds
				\item if $\mathsf{e}_i$ terminates successfully then terminate successfully
			\end{enumerate}
			\item increment $n$ and go to step 2
		\end{enumerate}
		The procedure will eventually run all tests for an arbitrarily long amount of time. Suppose there exists an $i \geq 1$ such that $\mathsf{e}_i$ will terminate successfully. Then the above procedure will eventually run that test for a time long enough and terminate successfully.
	\end{justification}
\end{mathSection}

Taken as a whole, finite conjunction and countable disjunction define the \textbf{algebra of verifiable statements}. It is limited compared to the algebra of statements and it tells us that, in practice, we are not going to be able in general to construct an experimental test whose success is an arbitrary function of the success of other tests.

\begin{table}[h]
	\centering
%	\begin{tabular}{p{0.2\textwidth} p{0.1\textwidth} p{0.1\textwidth} p{0.5\textwidth}}
\begin{tabular}{p{0.15\textwidth} p{0.15\textwidth} p{0.2\textwidth} p{0.3\textwidth}}
	Operator & Gate & Statements & Verifiable Statements \\ 
	\hline 
	Negation & NOT & allowed & disallowed \\ 
	Conjunction & AND & arbitrary  & finite \\ 
	Disjunction & OR & arbitrary  & countable \\ 
\end{tabular}
	\caption{Comparing algebras of statements.}
\end{table}

Before we continue, it is interesting and useful to stop and understand the interplay between scientific and mathematical constructs.\footnote{It took us many many confusing years to fully understand where the scientific argument ends and the mathematical argument begins, what makes sense to assume physically and what makes sense to prove rigorously. Part of the confusion is that this line is not objective but it is based on what is considered ``precise" by a mathematician, which has evolved considerably through the centuries. The rule we follow is: in the mathematical formalism the only objects that can be left unspecified are the elements of a set.} Technically, definitions \ref{def_statement}, \ref{def_possibilities}, \ref{def_functions_of_statement}, \ref{def_experimental_tests}, \ref{def_experimental_test_AND} and \ref{def_experimental_test_OR} are the axioms of our mathematical formalism for statements and experimental tests. Note that the actual content of the statements and the procedure for the tests are not formally defined: the math treats them simply as a label that we use for identification. The only assumptions are that statements exist (each with a set of possible truth values and an actual truth value), that experimental tests exist (each with a well defined result) and that they both admit the associated algebra. The mathematical formalism does not know what these objects actually are: they may as well be pieces of cardboard painted black or white. Therefore the math does not know whether the properties we assigned make actual sense: it can only guarantee their non-contradiction. In other words, the way that we are making the framework precise is not by making everything precise: it is by omitting the details that are not amenable to a precise specification.

We should stress this for a couple of reasons. First, the part that is not formalized is \emph{the most important part}. Discovering new science is exactly finding new things to study (i.e. new statements) or devising new measurement techniques (i.e. new experimental tests). The content of the statements and the procedure of the experimental tests \emph{is} the actual science. Everything that follows is, in a sense, the trivial bit and that is why it can be done generally. Which leads to the second reason: understanding whether statements and experimental tests \emph{actually} follow the algebras we defined is crucial. The math just takes it at face value, it does not prove it. The justifications for our axioms, then, are the most critical part of this work and they are not mathematical proofs. If we botch them, we'll have a nice, consistent, rich but meaningless mathematical framework. Lastly, it has to be clear that something gets lost in the formalization. The mathematical framework cannot carry all the physics content: we removed the most important part! Different systems may have the same mathematical description, so the scientific content can never be entirely reconstructed from the math. That is why we always have to carefully bring it along.

Now that we have characterized verifiable statements we want to understand how to characterize groups of them. Consider the verifiable statements
\begin{description}
	\item \statement{that animal is a duck}
	\item \statement{that animal is a swan}
	\item \statement{that animal is white}
	\item \statement{that animal is black}
	\item \statement{that animal is a black swan}
	\item \statement{that animal is a white duck}
	\item \statement{that animal is a duck or a swan}
\end{description}
Since some are functions of others, we do not need to actually run all the tests. Once we have tested the first four we have gathered enough information for the others. We call this set a basis.\footnote{The term basis is used in general to define a set of objects from which, through a series of operations, one can construct the full space. It is the same for a vector space: from a basis one can construct any other vector through linear combination. What changes is what objects are combined and what operations are used.}

\begin{mathSection}
	\begin{defn}
		Given a set $\edomain$ of verifiable statements, $\basis \subseteq \edomain$ is a \textbf{basis} if verifying its elements is equivalent to verifying the whole set. Formally, all elements of $\edomain$ can be generated from $\basis$ using finite conjunction and countable disjunction.
	\end{defn}
\end{mathSection}

Note that once we have tested the basis, we have tested any other verifiable statement that can be constructed from it. In the example before, once we tested the first four we have implicitly tested \statement{that animal is a black duck}. It is also true that contradictions and tautologies don't really need to be tested. We already know that \statement{that animal is a duck and a swan} is false. The idea, then, is to group verifiable statements into experimental domains that can be seen as all the experimental information one can gather for a particular subject. These will include the tautology, the contradiction and any other verifiable statement that can be constructed from a basis. The basis, though, has to be countable so that, by running one test at a time, we can hope to eventually reach any element.

\begin{mathSection}
\begin{defn}
	An \textbf{experimental domain} $\edomain$ represents all the experimental evidence that can be acquired about a scientific subject in an indefinite amount of time. Formally, it is a set of statements, closed under finite conjunction and countable disjunction, that includes the tautology, the contradiction, and a set of verifiable statements that can be generated from a countable basis.
\end{defn}
\begin{justification}
	As usual, let's make sure the formal definition is consistent with the informal one. The set consists only of verifiable statements, which can be tested experimentally, and the tautology and contradiction, which don't need to be verified experimentally. Functions of verifiable statements represent experimental evidence about the same subject, therefore it is appropriate that the experimental domain is closed under finite conjunction and countable disjunction. Given that each test terminates successfully in a finite amount of time, we can only test a countable set in an indefinite amount of time. Therefore, if the experimental domain didn't allow for a countable basis, we would always have verifiable statements we would never be able to test.
\end{justification}
\end{mathSection}

We can think of an experimental domain as the enumeration of all possible verifiable answers to a scientific question. For example, the domain related to the question ``what is that animal?" would include \emph{``it is a mammal"}, \emph{``it is a dog"}, \emph{``it is an animal with feathers"} and so on. If two statements are possible answers to that question, then their conjunction and disjunction will also be possible answers. For example: \emph{``it is a dog or a cat"} or \emph{``it is a mammal and it lays eggs"}.

While each statement only needs finite time to be verified, we allow indefinite time for the domain because we want to capture those questions that can be answered only approximately. The idea is that, given more time, we can always get a better answer so, in principle, we have an infinite sequence of tests to perform and continue indefinitely.

The basis not only serves as a way to constrain the size of the experimental domain, but most of the time it will also serve to define the experimental domain itself. We will typically start by characterizing a set of verifiable statements (e.g. a set of characteristics of animals and how to identify them) and then consider the domain of all the verifiable statements that can be constructed from them (e.g. the set of all animals and groups of animals we can identify).

\section{Theoretical domains and possibilities}

The basis for an experimental domain allows us to create a procedure that will eventually test any verifiable statement. But to fully characterize a domain we want to find those statements, like \statement{that animal is a cat} or \statement{the mass of the photon is exactly 0 eV}, that, if true, determine the truth value of all other verifiable statements in the domain. The main result of this section is that these statements, which we call possibilities for the domain, are not necessarily verifiable themselves. We will therefore need to introduce theoretical domains which consist of those statements we can use to give predictions for an experimental domain. We will also be able to conclude that the set of possibilities for an arbitrary experimental domain has at most the cardinality of the continuum, thus putting a hard constraint on what type of mathematical objects are useful in science.

Suppose $\edomain_X$ is the domain of animal species identification. It will contain verifiable statements such as \statement{that animal has feathers}, \statement{that animal has claws}, \statement{that animal has paws}. Some statements are broader and some are narrower. But some statements, like \statement{that animal is a swan} or \statement{that animal is a duck}, are special because if we verify those then we are able to know which other statements are true or false. Once we verify those we are essentially done. These are what we call the possibilities of the domain and enumerating them means characterizing the experimental domain.

Unfortunately, not all possibilities are verifiable statements. Consider the statements $s_1=$\statement{there is extra-terrestrial life} and $s_2$=\statement{there is no extra-terrestrial life}. We can create an experimental test for the first (i.e. find extra-terrestrial life somewhere) but not for the second (it would require us to check every place in the universe which is something we cannot do). So, for this question, the experimental domain $\edomain_X = \{s_1, \tautology, \contradiction\}$ is composed of the first statement, the tautology and the contradiction. But $s_2$ is conceptually still one of the possibilities: if true we have a complete answer for the domain.

What happens is that while the negation of a verifiable statement is not always a verifiable statement, it is still a statement that can be used to characterize the outcome of some experimental test. While we cannot verify non termination of an experimental test, we can still predict it. To be able to find all possibilities, then, we have to create the set of statements that can be used to make predictions which include negations. We call this set the theoretical domain and theoretical statements its elements.

\begin{mathSection}
\begin{defn}
	The \textbf{theoretical domain} $\tdomain$ of an experimental domain $\edomain$ is the set of statements that we can use to state predictions, which is constructed from $\edomain$ by allowing negation. We call \textbf{theoretical statement} a statement that is part of a theoretical domain. More formally, $\tdomain$ is the set of all statements generated from $\edomain$ using negation, finite conjunction and countable disjunction.
\end{defn}
\end{mathSection}

Because of its construction, the theoretical domain will also include all the limits of all the sequences of verifiable statements. Consider the experimental domain for the mass of the photon. It will contain verifiable statements such as
\begin{description}
	\item $\stmt_1$ =\statement{the mass of the photon is smaller than $10^{-1}$ eV}
	\item $\stmt_2$ =\statement{the mass of the photon is smaller than $10^{-2}$ eV}
	\item $\stmt_3$ =\statement{the mass of the photon is smaller than $10^{-3}$ eV}
	\item ...
\end{description}
It will not contain the statement $\stmt=$\statement{the mass of the photon is exactly 0 eV}, though, as we cannot measure a continuous quantity with infinite precision.

Note $\stmt$ can be seen as the limit of the sequence of ever increasing precision, but can also be seen as the conjunction for all those statements $\stmt=\bigAND\limits_{i=1}^{\infty} \stmt_i$. In fact, the mass of the photon is exactly 0 if and only if all the finite precision measurements will contain 0 in the range. It makes sense, then, that it is not part of the experimental domain because only finite conjunctions of verifiable statements are verifiable. But we expect $\stmt$ to be a possibility for the mass of the photon. Why should it be in the theoretical domain?

Because of the De Morgan properties in \ref{boolean_properties}, we can express conjunctions in terms of negation and disjuction. So we have $\stmt=\bigAND\limits_{i=1}^{\infty} \stmt_i=\NOT \bigOR\limits_{i=1}^{\infty} \NOT\stmt_i$. Therefore by allowing negation we are also allowing countable conjunction and therefore we are including all the limits of sequences of verifiable statements.

% Not being used at this point, but keep around to cannibalize later. As we saw before, not all possibilities are verifiable statements. For example, if we are able to only verify \statement{there is extra-terrestrial life}, the opposite possibility will never have experimental confirmation. The possibility \statement{the mass of the photon is exactly 0 eV} is also not verifiable since we can only measure mass with finite precision. The second case, though, is different from the first. Given two different values of mass, we can in principle always find a resolution such that we can tell the two values apart experimentally. Similarly, we can tell the house sparrow (Passer domesticus) apart from the Italian sparrow (Passer italiae) because we have at least two verifiable statements (``it has a dark gray crown", ``it has a chestnut crown") that are incompatible with each other, but each compatible with one bird. In these cases, we say that the domain is experimentally distinguishable.

\begin{mathSection}
	\begin{prop}
		All theoretical domains are closed under countable conjunction.
	\end{prop}
	
	\begin{proof}
		Any countable conjunction $\mathsf{s} = \bigAND\limits_{i=1}^{\infty} \mathsf{s}_i$ is equivalent to the negation of disjunction of the negation: $\mathsf{s} = \NOT\bigOR\limits_{i=1}^{\infty} \NOT\mathsf{s}_i$. As the theoretical domain is closed under negation and countable disjunction, so it is closed under countable conjunction.  
	\end{proof}

	\begin{defn}\label{def_approximately_verifiable}
		A theoretical statement $\stmt \in \tdomain$ is \textbf{approximately verifiable} if it is the limit of some sequence of verifiable statements. Formally, if there exists a sequence $\{\obs_i\}_{i=1}^{\infty} \in \edomain$ such that $\stmt = \bigAND\limits_{i=1}^{\infty} \obs_i$.
	\end{defn}
\end{mathSection}

Note that not all theoretical statements can be approximated with a sequence of verifiable statements. Consider $\stmt=$\statement{the mass of the photon is rational as expressed in eV}. It is the disjunction of all possibilities with rational numbers, which is countable, and therefore is a theoretical statement. Any finite precision version of this statement will always cover all possible values since any finite range will include infinitely many rational (and irrational) numbers. This means we cannot construct a sequence of finite precision verifiable statements that, in the limit, will correspond to $\stmt$. It also means we predict non termination when trying to experimentally verify $\stmt$ or its negation. Therefore we have to be cautious in giving physical significance to all theoretical statements as the prediction for some is that they may not be established experimentally.

Also note that we are closed under countable operations and not arbitrary. Therefore there could be statements that can be constructed from verifiable statements that are not even theoretical statements. Consider a set $U$ of possible mass values for a particle that is uncountable, has an uncountable complement, and where the elements are picked arbitrarily and not according to a simple rule.\footnote{Mathematically, we are looking for a set of real numbers that is not a Borel set.} The statement \statement{the mass of the particle expressed in eV is in the set $U$} can only be tested by checking each value individually. But since the set in uncountable and a procedure can only be made of countable steps, it will be impossible to construct a test for such a statement. It's not that the procedure to test may not terminate, like for a theoretical statement: it's that we can't even write the procedure in the first place.

To sum up, verifiable statements are the only ones that, perhaps under some simplifying assumption, we can think as tangible scientific objects. For example, the idea that we can measure mass with finite precision. From these we construct theoretical statements that represent our idealizations. For example, the infinitely precise value for the mass of a particle or the idea that said value is a rational number expressed in a particular unit. And then there are statements that are not even physically meaningful as they have no well defined experimental consequences.\footnote{It will be a recurrent theme of this work to make a precise distinction between mathematical objects that represent physical entities (e.g. verifiable statements), those that represent idealizations of physical entities (e.g. theoretical statements) and those that do not have scientific standing (e.g. statements that are neither verifiable nor theoretical). The particular status of real numbers will be explored more precisely when discussing arbitrary precision quantities.}

While the theoretical domain contains more statements, it does not contain more information. That is, if we knew which verifiable statements are true and which aren't, we would automatically know which theoretical statements would be true or not. So it is not adding extra cases. It is essentially completing the list of answers by adding a ``no" if only ``yes" can be experimentally verified and vice-versa. To see that this is true, we can show that a basis for an experimental domain $\edomain$ is also a basis for its theoretical domain $\tdomain$. That is, all verifiable and theoretical statements can be expressed as functions of the same basis. The difference is that a theoretical statement can be a function of the negation of the element of a basis.

\begin{mathSection}
\begin{prop}
	The truth values of the statements of a basis $\basis$ for an experimental domain $\edomain$ are enough to determine the truth values for all statements in the associated theoretical domain $\tdomain$. More formally, all statements in the theoretical domain $\tdomain$ can be generated by negation, countable conjunction and countable disjunction from a basis $\basis$ of $\edomain$.
\end{prop}

\begin{proof}
	By definition of basis, any verifiable statement within the experimental domain $\edomain$ can be generated from $\basis$ using only finite conjunction and countable disjunction. The tautology may be generated through negation and disjunction from any verifiable statement therefore $\basis$ generates $\edomain$ which in turn generates $\tdomain$ by definition. This means that $\basis$, through negation, countable conjunction and countable disjunction, generates all of $\tdomain$.
\end{proof}
\end{mathSection}

Having defined what a theoretical domain is, we can finally define what the possibilities of a domain are: those statements that if known to be true determine the truth value of all other statements.

\begin{mathSection}

\begin{defn}
	A \textbf{possibility} for an experimental domain $\edomain$ is a statement $x \in \tdomain$ that, when true, determines the truth value for all statements in the theoretical domain. That is, $x \nequiv \contradiction$ and for each $\mathsf{s} \in \tdomain$, either $x \narrower \mathsf{s}$ or $x \ncomp \mathsf{s}$. The \textbf{full possibilities}, or simply the \textbf{possibilities}, $X$ for $\edomain$ are the collection of all possibilities.
\end{defn}

\end{mathSection}

A possibility represents a complete answer for a scientific question. Only one of them can be true and one of them must be true since the theoretical domain contains all negations. But how can we construct them? Suppose $\edomain$ is the experimental domain for animal species identification. Suppose $\basis \subset \edomain$ is a basis of statements, like \statement{that animal has feathers}, \statement{that animal has claws}, \statement{that animal has paws} and so on, that allow us to fully identify the animal species. Consider a minterm, a conjunction where each statement appears once either negated or not. For example, \statement{that animal has feathers}$\AND$\statement{that animal has claws}$\AND\NOT$\statement{that animal has paws}$\AND$... . If that statement is true, it will determine the truth value of all the basis, and therefore of all verifiable statements. Then a possibility for the domain is simply a minterm of a basis.

\begin{mathSection}
	
\begin{prop}\label{prop_poss_is_minterm}
	Let $\edomain$ be an experimental domain. A possibility for $\edomain$ is any minterm of a basis that is not a contradiction.
\end{prop}

\begin{proof}
	Let $\basis \subseteq \edomain$ be a basis for $\edomain$. Let $x$ be a minterm of $\basis$. Any theoretical statement $\stmt \in \tdomain$ can be expressed as the disjunction of minterms of $\basis$ by \ref{prop_disjunctive_normal_form}. $x$ is either within the minterms needed to express $\stmt$, or not. If it is, $x \AND \stmt \equiv x$ and therefore $x \narrower \stmt$. If it's not, $x \AND \NOT \stmt \equiv x$ and therefore $x \ncomp \stmt$. Therefore a minterm is either narrower or incompatible with all theoretical statements and, if it is not a contradiction, it is a possibility by definition.
	
	Conversely, suppose $x \in \tdomain$ is a possibility. As it is a theoretical statement, it can be expressed as a disjunction of minterms of a basis $\basis$. Suppose it is the disjunction of more than one minterm. Then each minterm would be narrower than $x$, which cannot be. $x$ must be expressed by a single minterm. Therefore any possibility is a minterm.
\end{proof}

\begin{prop}[No other possibilities]
	All statements that determine and only determine the truth value of all statements in a theoretical domain $\tdomain$ are possibilities of $\edomain$.
\end{prop}

\begin{proof}
	Let $x \in \mathcal{S}$ be a statement that determines and only determines all truth values of all statements in a theoretical domain $\tdomain$. This is equivalent to determining the truth values and only the truth values of all elements of a basis $\basis \subseteq \edomain$. As we can find a countable basis, the statement $x$ is equivalent to the countable conjunction of statements of $\basis$ or their negation. Therefore $x \in \tdomain$ as it is generated by the statements of the basis by negation and countable conjunction. But a statement in $\tdomain$ that determines all truth values of the statements in $\tdomain$ is a possibility by definition. Therefore $x$ is a possibility.
\end{proof}
\end{mathSection}

There is one possibility that is often forgotten and sometimes needs special handling. Suppose one is trying to identify an illness by going through a series of known markers. It may happen that no match for the disease is found because we are dealing with a new kind of illness. In the same way, we may fail to identify a plant or animal from the standard taxonomy as we may have found a new species. In other words, it may be possible that none of our tests succeed and none of the verifiable statements is verified. We call this possibility the residual because it's what remains after we went through all the cases we already know.

Note, though, that the residual possibility does not exist for all domains. Suppose we have a basket of fruit and we want to count how many whole apples there are. There can only be a finite number of them, and we can successfully identify all finite numbers: there is no ``something else" to be found in this case.

\begin{mathSection}
	\begin{prop}
		The \textbf{residual possibility} $\mathring{x}$ for an experimental domain $\edomain$ is, if it exists, the possibility that predicts that no test will be successful. That is, let $\basis \subseteq \edomain$ be a basis then  $\mathring{x} = \bigAND\limits_{\stmt[e] \in \basis} \NOT e$ if it is not a contradiction. An experimental domain is \textbf{complete} if it doesn't admit a residual possibility.
	\end{prop}

	\begin{prop}
	The \textbf{established possibilities} $\dot{X}=X\setminus\{\mathring{x}\}$ for an experimental domain is the set of all possibilities excluding the residual possibility.
\end{prop}
\end{mathSection}

As we refine our understanding and techniques for a domain of knowledge, we may find that the residual possibility actually corresponds to multiple cases that weren't previously cataloged. So, intuitively, it is better thought of as a bucket that contains all that is yet to be experimentally discovered within the particular domain of knowledge. Because of its special nature, the residual possibility sometimes behaves differently than the other possibilities. Therefore we will find that some theorems are more elegantly expressed in terms of the established possibilities and some in terms of the full possibilities.

With our definitions in mind, we can answer the following fundamental question: what is the maximum number of possibilities that an experimental domain can have? In other words: what is the maximum number of cases among which we can distinguish experimentally? As we saw before, a possibility is a statement that defines the truth values of a basis. Since a basis is countable, we can uniquely identify a possibility by a countable sequence of $\TRUE$ or $\FALSE$. Note that a real number expressed in a binary basis (e.g. 0.10110001...) will also be uniquely identified by such a sequence: the cardinality of the possibility is at most the one of the continuum.

\begin{mathSection}
	\begin{thrm}
		The possibilities $X$ for an experimental domain $\edomain$ have at most the cardinality of the continuum.
	\end{thrm}
	
	\begin{proof}
		Let $\basis = \{\obs[e]_i\}_{i=1}^{\infty} \subseteq \edomain$ be a countable basis. Let $2^{\mathbb{N}}$ denote the set of infinite binary sequences. We define the function $F:X\to2^{\mathbb{N}}$ such that $F(x) = \{F(x)_i\}_{i=1}^{\infty}$ is given by: 
		$$
		F(x)_i = 
		\begin{cases}
		1 & x \comp \obs[e]_i \\
		0 & x \ncomp \obs[e]_i
		\end{cases}
		$$
		For each $x \in X$ we have $x = \bigAND\limits_{i=1}^{\infty} \NOT^{F(x)_i} \obs[e]_i$. Suppose $x_1 \neq x_2$, then $F(x_1)_i \neq F(x_2)_i$ for some $i$, therefore $F$ is injective. We then have $|X| \leq |2^{\mathbb{N}}|=|\mathbb{R}|$. $X$ has at most the cardinality of the continuum.
	\end{proof}
\end{mathSection}

This means we have an upper bound on how many cases can be distinguished experimentally: only up to the continuum. We are not going to be able to tell apart experimentally more possibilities than those. This result gives us a basic requirement for any mathematical object we want to use in a scientific theory: if the cardinality is greater than the continuum, it cannot have a well defined experimental meaning. For example, while the set of all continuous functions between real numbers has the cardinality of the continuum, the set of all functions (including discontinuous ones) has greater cardinality. We can already conclude that the first set may be useful to represent physical objects and the second may not.

We have now presented the fundamental objects of our general mathematical theory of experimental science. In our framework, a ``scientific theory" or a ``scientific model" \emph{is} an experimental domain: a set of statements, what they mean and how to verify them experimentally.

The starting point is often a basis: a set of verifiable statements which defines all the domain knowledge that we can gather experimentally. The content of the statements determines what combinations can be true at the same time, which defines the possibilities for our domain. Everything in the domain is grounded within the verifiable statements: there is nothing else in it. This also maps well to the practice of most scientific fields, where one defines states and other physical objects based on what can be measured.

As we'll see later, some verifiable statements may be idealizations. For example, we may assume that a quantity can be measured with arbitrary level of precision, which we know not to be strictly true. We may assume a volume of gas to have a well defined temperature, which we know not be true if it is not at equilibrium. This type of simplification makes a domain applicable only within the realm of validity of those idealizations, but does not change the formal structure we have identified here. In fact, the focus of much of this work will be deriving the details of different experimental domains under different physical assumptions.

The main point of our framework is that this conceptual structure is inescapable once we set the principle of scientific objectivity. We will always need a set of statements and a way to test them and, if we are given those, we have all we need. These elements are necessary and sufficient to be able to do science. And by being clear about which statements can be considered physically meaningful, and which are an idealization, we can then be more precise on the physical status of each component of a particular scientific theory.

\section{Topological spaces}

Now that we have defined what experimental domains are, we want to explore the link between them and some fundamental mathematical structures. The main result of this section is that an experimental domain provides a natural topology for its possibilities. Each verifiable statement can be seen as the disjunction of a set of possibilities. Performing finite conjunction and countable disjunction of verifiable statements means performing finite union and countable intersection on those sets of possibilities.

Topological spaces were developed in the first half of the 1900s as a generalization of metric spaces. The idea is to define a notion of closeness without having to define an actual distance.\footnote{In science and engineering, one talks about topology also when discussing the structure of a molecule, an electronic circuit or a computer network. These types of structures (i.e. nodes connected by vertices) are studied by graph theory and should not be confused with point-set topology.} Other branches of math (e.g. metric spaces, differential geometry, Lie algebras) now see their foundation on topological spaces, which therefore play a very important role in mathematics as a whole. In our case, this notion of closeness will map to how hard it is to tell possibilities apart. That is, possibilities that are topologically closer are more difficult to distinguish experimentally.

Let's first review what a topology is. The general idea is that we have a set $X$ of elements, which we call points, and a collection of subsets of $X$ such that it is closed under finite intersection and arbitrary union, contains the empty set and contains the whole set $X$. For example, suppose $X=\{1,2,3\}$ then $\{\{\}, \{1\}, \{2\},\{1,2,3\}\}$ is not a topology while $\{\{\}, \{1\}, \{2\},\{1,2\},\{1,2,3\}\}$ is. The first one is missing the union of $\{1\}$ and $\{2\}$.

\begin{mathSection}
	\begin{defn}
		Let $X$ be a set. A \textbf{topology} on $X$ is a collection $\mathsf{T}_X$ of subsets of $X$ closed under finite intersection and arbitrary union such that it contains $X$ and $\emptyset$. A \textbf{topological space} is a tuple $(X, \mathsf{T}_X)$ of a set and a topology defined on it.
	\end{defn}
\end{mathSection}

Mathematicians designed this abstract mathematical structure because it is a useful and general tool to study the notion of continuity. It also happens that all the mathematical structures used in science are topological spaces. Why is that? What is it that topological spaces capture?

Let's go back to our verifiable statements and possibilities. For example, consider $\stmt_1=$\statement{the mass of the photon is less than $10^{-10}$ eV}. This can be expressed as $\stmt_1=\bigOR\limits_{0\leq x<10^{-10}}$\statement{the mass of the photon is precisely x eV}: the precise value must be in the given range of possibilities. Consider $\stmt_2=$\statement{the mass of the photon is greater than $10^{-20}$ eV}$=\bigOR\limits_{x>10^{-20}}$\statement{the mass of the photon is precisely x eV}. The conjunction is the intersection of the possible values: $\stmt_1\AND\stmt_2=$\statement{the mass of the photon is between $10^{-20}$ and $10^{-10}$ eV}$=\bigOR\limits_{10^{-20}< x<10^{-10}}$\statement{the mass of the photon is precisely x eV}. The disjunction is the union of the possible values: $\stmt_1\OR\stmt_2=$\statement{the mass of the photon can be anything}$=\bigOR\limits_{x\geq0}$\statement{the mass of the photon is precisely x eV}.

This is something that works in general. In Proposition \ref{prop_disjunctive_normal_form} we saw that, if a statement is a function of other statements, it can be expressed as the disjunction of minterms of the arguments. A verifiable statement is a function of basis, so it can be expressed as the disjunction of minterms of a basis. But we have also seen that the minterms of a basis are the possibilities, so each verifiable statement can be expressed as the disjunction of possibilities.


Therefore each statement in the experimental domain defines a set of possibilities, which we call a verifiable set. Since tautology and contradiction are in the domain, the empty set and the full set of possibilities are verifiable sets. Since we can take finite conjunction and countable disjunction of verifiable statements, we can take finite intersection and countable union of verifiable sets. The collection of all verifiable sets forms a topology on the set of possibilities.

\begin{mathSection}
	
\begin{defn}
	Let $\edomain$ be an experimental domain and $X$ its possibilities. We define the map $U : \edomain \rightarrow 2^X$ that for each statement $\obs \in \edomain$ returns the set of possibilities compatible with it. That is: $U(\obs)\equiv\{ x \in X \, | \, x \comp \obs\}$. We call $U(\obs)$ the \textbf{verifiable set} of possibilities associated with $\obs$.
\end{defn}

\begin{prop}
	A statement $\obs \in \edomain$ is the disjunction of the possibilities in its verifiable set $U(\obs)$. That is, $\obs=\bigOR\limits_{x \in U(\obs)} x$.
\end{prop}
\begin{proof}
	First we show each statement is the disjunction of some set of possibilities. Let $\edomain$ be an experimental domain, $\obs \in \edomain$ a verifiable statement and $\basis \subseteq \edomain$ a basis. Since $\obs$ is a function of the basis, it can be expressed as a disjunction of minterms of $\basis$. The minterms of $\basis$ that are contradictions can be ignored since $\obs \OR \contradiction \equiv \obs$. But the minterms of $\basis$ that are not contradictions are possibilities of $\edomain$ so $\obs=\bigOR\limits_{x \in U} x$ for some $U \subseteq X$.
	
	Now we show it is the disjunction of its verifiable set. Let $x \in X$ be a possibility and consider $x \AND \obs$. If $x\in U$ then $x \AND \obs = x \AND \bigOR\limits_{\hat{x} \in U} \hat{x} = \bigOR\limits_{\hat{x} \in U} ( x \AND \hat{x}) \equiv x \nequiv \contradiction$. Therefore $x \comp \obs$. If $x \notin U$ then $x \AND \obs \equiv \contradiction$. Therefore $x \ncomp \obs$. This means $U=U(\obs)$ as it contains and only contains all the possibilities compatible with $\obs$.
\end{proof}

\begin{prop}
	Let $X$ be the set of possibilities for an experimental domain $\edomain$. $X$ has a natural topology given by the collection of all verifiable sets $\mathsf{T}_X=U(\edomain)$.
\end{prop}

\begin{proof}
	The verifiable sets for the tautology and the contradiction correspond to the full set and empty set respectively. Formally, $U(\tautology) = \{ x \in X \, | \, x \comp \tautology\} = X$ while $U(\contradiction) = \{ x \in X \, | \, x \comp \contradiction\} = \emptyset$. Therefore $X, \emptyset \in U(\edomain)$ since $\tautology, \contradiction \in \edomain$.

	The finite intersection of verifiable sets corresponds to the verifiable set of the finite conjunction and therefore it is a verifiable set. Formally, $U(\obs_1\AND\obs_2) = \{ x \in X \, | \, x \comp (\obs_1\AND\obs_2)\} =  \{ x \in X \, | \, x \comp \obs_1 \, and \, x \comp \obs_2\} = \{ x \in X \, | \, x \comp \obs_1\} \cap \{ x \in X \, | \, x \comp \obs_2\} = U(\obs_1) \cap U(\obs_2)$.

	The countable union of verifiable sets corresponds to the verifiable set of the countable disjunction and therefore it is a verifiable set. Formally, $U(\obs_1\OR\obs_2) = \{ x \in X \, | \, x \comp (\obs_1\OR\obs_2)\} =  \{ x \in X \, | \, x \comp \obs_1 \, or \, x \comp \obs_2\} = \{ x \in X \, | \, x \comp \obs_1\} \cup \{ x \in X \, | \, x \comp \obs_2\} = U(\obs_1) \cup U(\obs_2)$. This generalizes to countable disjunctions. Arbitrary disjunctions can be re-expressed as countable disjunctions, since any verifiable statement can always be expressed in terms of a countable basis.

	The collection $\mathsf{T}_X=U(\edomain)$ is therefore a topology by definition since it satisfies all its properties.
\end{proof}
\end{mathSection}

Mainly for historical reasons, the sets in a topology are called open sets. The complements of open sets are called closed sets. In metric spaces, such as the Euclidean space with the standard topology, these will map to the standard notion of open and closed intervals. But, in general, they do not and this may lead to confusion. For example, if we take the integers with their standard topology, any subset is both open and closed.

Given that we are only interested in the natural topologies of possibilities, we are going to refer to the sets in our topology as verifiable sets and we will occasionally call refutable sets their complements. For example, when counting apples a subset of the integers is both verifiable and refutable: we can test whether the apple count is within or outside that set of possible numbers. While this terminology does not follow math convention, we find it more intuitive and meaningful in the context of this work.

We can also re-express the semantic relationships between statements in terms of set operations on the verifiable sets. For example, \statement{that animal is a cat} is narrower than \statement{that animal is a mammal} because the set of possibilities for which the first is true is a subset of the possibilities for which the second is true. Conversely, \statement{that animal is a cat} and \statement{that animal is a dog} are incompatible because the set of possibilities in which both are true is empty.

In the following table we summarize how statement operations and relationships are expressed in terms of operations and relationships between sets of possibilities.

\begin{table}[h]
	\centering
	\begin{tabular}{p{0.075\textwidth} p{0.275\textwidth} p{0.2\textwidth} p{0.3\textwidth}}
		& Statement relationship & & Set relationship  \\ 
		\hline 
		$\stmt_1 \AND \stmt_2$ & (Conjunction) & $U(\stmt_1) \cap U(\stmt_2)$ & (Intersection) \\ 
		$\stmt_1 \OR \stmt_2$ & (Disjunction) & $U(\stmt_1) \cup U(\stmt_2)$ & (Union) \\ 
		$\NOT \stmt$ & (Negation) & $U(\stmt)^C$ & (Complement) \\ 
		$\stmt_1 \equiv \stmt_2$ & (Equivalence) & $U(\stmt_1) = U(\stmt_2)$ & (Equality) \\ 
		$\stmt_1 \narrower \stmt_2$ & (Narrower than) & $U(\stmt_1) \subseteq U(\stmt_2)$ & (Subset) \\ 
		$\stmt_1 \broader \stmt_2$ & (Broader than) & $U(\stmt_1) \supseteq U(\stmt_2)$ & (Superset) \\ 
		$\stmt_1 \comp \stmt_2$ & (Compatibility) & $U(\stmt_1) \cap U(\stmt_2) \neq \emptyset$ & (Intersection not empty)
	\end{tabular} 
	\caption{Correspondence between statement operators and set operators.}
\end{table}

Let's review the definition of basis and sub-basis for a topology: a collection of sets from which we can generate the whole topology through finite intersection and countable union (for a sub-basis) or just through countable union (for a basis). Bases are important since they are often used in proofs and calculations. Moreover, many properties of topologies can be shown to be equivalent to properties of one of their bases. Countability, and in particular second-countability, is one such property which characterizes the number of verifiable sets in the topology.

\begin{mathSection}
\begin{defn}
	A collection $\mathcal{B} \subseteq \mathsf{T}_X$ of verifiable sets of $X$ is a \textbf{sub-basis} if every verifiable set in $X$ is the union of finite intersections of $\mathcal{B}$. It is a \textbf{basis} if every verifiable set in $X$ is the union of elements of $\mathcal{B}$.
\end{defn}
\begin{defn}
	A topology for $X$ is \textbf{second-countable} if it admits a countable basis.
\end{defn}
\end{mathSection}

There is a link between the basis of an experimental domain and a sub-basis of a topology. If every statement in the experimental domain can be constructed from a basis through finite conjunction and countable disjunction, then each corresponding verifiable set can be generated through intersection and union of the verifiable set corresponding to the basis. Therefore the verifiable set corresponding to the basis of the experimental domain forms a sub-basis in the topology. Since experimental domains must have a countable basis to make sure we can test any verifiable statement given enough time, the topologies we'll be interested in must be second-countable.

\begin{mathSection}
	\begin{prop}
		Let $X$ be the set of possibilities of an experimental domain $\edomain$. Let $\basis \subseteq \edomain$ be a basis for the domain, then the collection of verifiable sets $U(\basis)\cup\{X\}$ forms a sub-basis for the natural topology of $X$.
	\end{prop}
	\begin{proof}
		Since every verifiable statement of a domain can be generated by finite conjunction and countable disjucntion from a basis $\basis \subseteq \edomain$, its corresponding verifiable set can be generated by finite intersection and countable union from the verifiable sets $U(\basis)$ corresponding to the basis. Note that the tautology, though, is not necessarily the union of $U(\basis)$: if the domain is not complete, the residual possibility is not contained in any verifiable sets. Therefore $U(\basis)\cup\{X\}$ can generate all verifiable sets, including the one for the tautology, and is a sub-basis.
	\end{proof}
	\begin{prop}
		The natural topology for the possibilities of an experimental domain is second-countable.
	\end{prop}
	\begin{proof}
		Since each experimental domain admits a countable basis, its verifiable sets form a countable sub-basis for the natural topology. We can close the sub-basis over finite intersection, forming a countable basis for the topology. The natural topology is therefore second-countable as it admits a countable basis.
\end{proof}
\end{mathSection}

Another important property to classify topological spaces is the degree of separation of their elements: how well one can use verifiable sets to tell points and sets apart. A Kolmogorov space is one in which for every pair of points there is always a verifiable set that contains one but not the other. This property is significant because it allows all points to be distinguished through verifiable sets. A Hausdorff space is one in which for every pair of points there are always two disjoint verifiable sets each containing one. It is significant because it implies the uniqueness of limits of sequences of points.

\begin{mathSection}
	\begin{defn}
		A topology for $X$ is \textbf{Kolmogorov} (or $\mathsf{T}_0$) if for every two elements $x_1, x_2 \in X$ there exists a verifiable set $U \in \mathsf{T}_X$ containing one element but not the other. That is: either $x_1 \in U$ while $x_2 \notin U$ or $x_1 \notin U$ while $x_2 \in U$.
	\end{defn}
	\begin{defn}
	A topology for $X$ is \textbf{Hausdorff} (or $\mathsf{T}_2$) if for every two elements $x_1, x_2 \in X$ there exist two disjoint verifiable sets $U_1, U_2 \in \mathsf{T}_X$ each containing one element. That is: $U_1 \cap U_2 = \emptyset$, $x_1 
	\in U_1$ and $x_2 \in U_2$.
\end{defn}

\end{mathSection}

How do these properties relate to experimental domains? Consider two possibilities for a domain, for example \statement{that is a cat} and \statement{that is a swan}. We can always find a verifiable statement, such as \statement{that animal has feathers}, that we can use to distinguish one possibility from the other. This means that, given two different possibilities, we can always find a verifiable set that contains one and not the other: the natural topology for any set of possibilities is always Kolmogorov.

Now suppose two possibilities are approximately verifiable as we defined in Definition \ref{def_approximately_verifiable}. For example, \statement{the mass of the photon is exactly 0 eV} or \statement{the mass of the photon is exactly $10^{-20}$ eV}. We can find two verifiable statements \statement{the mass of the photon is less than $10^{-25}$ eV} and \statement{the mass of the photon is more than $10^{-25}$ eV} that are incompatible with each other, but each compatible with one possibility. This means that, given two approximately verifiable possibilities, we can find two disjoint (i.e. incompatible) verifiable sets each containing one possibility: if all possibilities are approximately verifiable then the natural topology is Hausdorff.

\begin{mathSection}
	\begin{prop}
	The natural topology of a set of possibilities is Kolmogorov (or $\mathsf{T}_0$).
\end{prop}
\begin{proof}
	Let $X$ be the set of possibilities for an experimental domain $\edomain$. Let $x_1, x_2 \in X$ be two distinct possibilities. Each of them can be expressed as a minterm of a basis $\basis \subseteq \edomain$. Since the two possibilities are distinct, there must exist a verifiable statement $\obs[e] \in \basis$ that appears negated in one conjunction but not the other. That is, $\obs[e]$ is compatible with only one possibility. Since the verifiable set associated with a verifiable statement contains only the possibilities compatible with said statement, the verifiable set of $\obs[e]$ either contains $x_1$ or $x_2$ but not both. The topology is therefore Kolmogorov (or $\mathsf{T}_0$).
\end{proof}
	\begin{prop}
	The natural topology of a set of possibilities is Hausdorff (or $\mathsf{T}_2$) if and only if all possibilities are approximately verifiable.
\end{prop}
\begin{proof}
	Suppose all possibilities in $X$ for an experimental domain $\edomain_X$ are approximately verifiable. Let $x_1, x_2 \in X$ be two possibilities, then we can find two sequences of verifiable statements $\{\obs_i^1\}_{i=1}^\infty, \{\obs_j^2\}_{j=1}^\infty \in \edomain_X$ such that $x_1=\bigAND\limits_{i=1}^\infty \obs_i^1$ and $x_2=\bigAND\limits_{j=1}^\infty \obs_j^2$. We can assume the sequences are monotone with respect to narrowness, that is $\obs_{i+1}^1 \narrower \obs_i^1$, as we can always create a monotone sequence from one that is not by taking the sequence of finite conjunction, that is $\hat{\obs}_k^1=\bigAND\limits_{i=1}^k \obs_i^1$. If $x_1 \neq x_2$, then $x_1 \AND x_2 \equiv \contradiction$ since different possibilities are incompatible. Therefore we must have $\obs_i^1 \AND \obs_j^2 \equiv \contradiction$ from some $i,j \geq 1$ or the limits would not be incompatible. In terms of verifiable sets we have $U(\obs_i^1) \cap U(\obs_j^2) = \emptyset$. For any two distinct possibilities we can find two disjoint verifiable sets each containing one: the natural topology is Hausdorff.

	Conversely, suppose the natural topology $\mathsf{T}_X$ for the possibilities $X$ for an experimental domain $\edomain_X$ is Hausdorff. Let $x \in X$ be a possibility. Consider the collection, not necessarily countable, of all verifiable sets $\{U_i\}_{i \in I} \subset \mathsf{T}_X$ such that they contain $x$. Consider their intersection $U_x = \bigcap\limits_{i \in I} U_i$. It will contain $x$ since all $U_i$ contain $x$. It will not contain anything else: since the topology is Hausdorff, for every other possibility $\hat{x}$ there is always an open set $U_i$ that does not contain it. Therefore $U_x = \{x\}$. Because the natural topology is second countable, we can find a countable basis $\basis$ and rewrite the arbitrary intersection into $\{x\} = \bigcap\limits_{i=1}^\infty V_i$ a countable intersection of elements $V_i \in \basis$ of the basis. Let $\{\obs_i\}_{i=1}^\infty$ be the sequence of verifiable statements such that $U(\obs_i) = V_i$ for every $i$. Then $x=\bigAND\limits_{i=1}^\infty \obs_i$ which means $x$ is approximately verifiable.

\end{proof}
\end{mathSection}

\section{Sigma-algebras}

In the same way that experimental domains find a natural mathematical representation as topological spaces, theoretical domains find a natural mathematical representation in $\sigma$-algebras. The main result of this section is that a theoretical domain provides a natural $\sigma$-algebra on its possibilities.

Like topologies, $\sigma$-algebras are fundamental in mathematics as they allow us to construct measures (i.e. assigning sizes to sets), limits for sequences of sets and probability spaces. It is again fitting that theoretical domains are associated to such a fundamental mathematical structure.

Let's first review what a $\sigma$-algebra is. The general idea is that we have a set $X$ of elements which we call points, and we have a collection of subsets of $X$ such that it is closed under complement and countable union, contains the empty set and contains the whole set $X$. For example, suppose $X = \{1,2,3\}$ then  $\{\{\},\{1\},\{1,2,3\}\}$ is not a $\sigma$-algebra while $\{\{\},\{1\}, \{2,3\},\{1,2,3\}\}$ is. The first one is missing the complement of $\{1\}$.

\begin{mathSection}
	\begin{defn}
		Let $X$ be a set. A \textbf{$\sigma$-algebra} on $X$ is a collection $\Sigma_X$ of subsets of $X$ closed under complement and countable union such that it contains $X$.
	\end{defn}
\end{mathSection}

Note that $\sigma$-algebras are also closed under countable intersections, since these can be expressed in terms of negation and countable unions.

In the previous section we saw how each verifiable statement can be expressed as the conjunction of a set of possibilities, how the operations on statements can be expressed as operations on the verifiable sets and how all the verifiable sets form a topology. The same is true for theoretical statements, with the only difference being that we will end up with a collection of sets that is closed under complement and countable union since the theoretical domain is closed under negation and countable disjunction.

\begin{mathSection}
	
	\begin{defn}
		Let $\tdomain$ be a theoretical domain and $X$ its possibilities. We define the map $A : \tdomain \rightarrow 2^X$ that for each theoretical statement $\stmt \in \tdomain$ returns the set of possibilities compatible with it. That is, $A(\stmt)\equiv\{ x \in X \, | \, x \comp \stmt\}$. We call $A(\stmt)$ the \textbf{theoretical set} of possibilities associated with $\stmt$
	\end{defn}
	
	\begin{prop}
		Let $X$ be the set of possibilities for a theoretical domain $\tdomain$. $X$ has a natural $\sigma$-algebra given by the collection of all theoretical sets $\Sigma_X=A(\tdomain)$.
	\end{prop}
	
	\begin{proof}
	The theoretical sets for the tautology and the contradiction correspond to the full set and empty set respectively. Formally, $A(\tautology) = \{ x \in X \, | \, x \comp \tautology\} = X$ while $A(\contradiction) = \{ x \in X \, | \, x \comp \contradiction\} = \emptyset$. Therefore $X, \emptyset \in A(\tdomain)$ since $\tautology, \contradiction \in \tdomain$.

	The complement of a theoretical set corresponds to the theoretical set of the negation and therefore it is a theoretical set. Formally, $A(\stmt)^C = \{ x \in X \, | \, x \ncomp \stmt\} =  \{ x \in X \, | \, x \comp \NOT\stmt\} = A(\NOT\stmt)$.

	The countable union of verifiable sets corresponds to the verifiable set of the countable disjunction and therefore it is a theoretical set. Formally, $A(\stmt_1\OR\stmt_2) = \{ x \in X \, | \, x \comp \stmt_1\OR\stmt_2\} =  \{ x \in X \, | \, x \comp \stmt_1 \, or \, x \comp \stmt_2\} = \{ x \in X \, | \, x \comp \stmt_1\} \cup \{ x \in X \, | \, x \comp \stmt_2\} = A(\stmt_1) \cup A(\stmt_2)$. This generalizes to countable disjunctions.

	The collection $\Sigma_X=A(\tdomain)$ is therefore a $\sigma$-algebra by definition since it satisfies all its properties.
	\end{proof}
\end{mathSection}

There is also a special link between topologies and $\sigma$-algebras. As one may want to construct measures and probability spaces on topological spaces, there is a standard way to construct a $\sigma$-algebra from a topology. This object, called Borel algebra, is the smallest $\sigma$-algebra that contains all verifiable sets defined by the topology. The $\sigma$-algebra defined by a theoretical domain is none other than the Borel algebra of the topology defined by the corresponding experimental domain.

\begin{mathSection}
	
	\begin{defn}
		Let $(X, \mathsf{T})$ be a topological space. Its \textbf{Borel algebra} is the collection $\Sigma_X$ of subsets of $X$ generated by countable union, countable intersection and complement from the verifiable sets.
	\end{defn}
	
	\begin{prop}
		The natural $\sigma$-algebra for a set of possibilities is the Borel algebra of its natural topology.
	\end{prop}
	
	\begin{proof}
		Since the theoretical domain can be generated by a basis of the experimental domain, the natural $\sigma$-algebra can be generated by a sub-basis of the natural topology. This means that it is also generated by countable union, countable intersection and negation from the verifiable sets of the natural topology.
	\end{proof}
\end{mathSection}

This fundamental link between experimental domains and topology on one side and theoretical domains and $\sigma$-algebra on the other is important for multiple reasons.

From a practical standpoint, it guarantees that these mathematical tools can always be used in science. Since experimental and theoretical domains are general constructs, any branch of scientific investigation can use techniques and results from topology and $\sigma$-algebras for calculations or for characterizing the domain at hand.

From a conceptual standpoint it provides a Rosetta stone, i.e a way to translate, between the mathematical concepts and the scientific ones. It gives a precise scientific meaning to the mathematical tools and everything built on top of them. Every single step in a calculation, every single argument in a proof can be given a clear, and possibly insightful, physical meaning. It grounds the abstract mathematical language in more concrete scientific objects. This in turn helps clarify the science described by common mathematical tools, unearthing possible hidden assumptions or simplifications about the physical systems being studied.

This connection explains why these mathematical tools have found such successful application in the physical sciences.

\section{Summary}

In this first chapter we have laid down the foundations for our general mathematical theory of experimental science. We have seen how it is grounded in the logic of verifiable statements, which is more limited than the logic of pure statements as it has to deal with the practical constraints introduced by the termination of the tests.

We saw that we can group verifiable statements into experimental domains which must have a countable basis to allow us to test any statement within an indefinite amount of time. We saw how to construct theoretical domains to find all the theoretical statements that we can use as predictions. And we saw how the possibilities are those statements that, if true, give a complete prediction for all statements in the domain.

We have seen that, because of the disjunctive normal form, each verifiable and theoretical statement is equivalent to a set of possibilities and how logic operations and relationships become set operations and relationships. As such, the experimental and theoretical domains respectively provide a natural topology and $\sigma$-algebra for the possibilities.

What we have ended up with is a conceptual framework that captures the necessary elements of scientific practice and codifies them into a symbolic representation with a well defined meaning. There is no guesswork as to what the points of our spaces are: they are the possibilities, statements that provide a complete prediction for the domain. We do not have to provide an ``interpretation" as to what the sets of a topology represent: they correspond to verifiable statements. All the objects have a clear definition and meaning from the start, we know which ones are necessary and to what extent they are physical or idealized. This will provide a much more solid foundation to the rest of the work, which will ultimately allow us to understand much better the fundamental physical theories and the connections between them and to other areas of scientific thought.

\chapter{Relationships and combinations of domains}

\section{Dependence and equivalence between domains}

Now that we have characterized experimental domains individually, we want to study relationships betwee different domains. For example, consider the domains for the temperature and height of a mercury column or the domains for the temperature and density of water. How do we express, in this framework, the fact that these domains are connected?

We have two ways to define these relationships between domains. The first is in terms of inference: any measurement on the height of a mercury column is an indirect measurement on its temperature; any experimental test on the density of water is an indirect experimental test on its temperature. The second is in terms of causes: the height of the mercury column depends on its temperature; the density of water is a function of its temperature. The main result of this section is to show that these definitions are equivalent.

Suppose $\edomain_X$ represents the domain for the temperature of a mercury column while $\edomain_Y$ represents the domain for its height. Since we know that an increase in temperature makes the metal expand, we can infer the temperature of the mercury column by looking at its height. For example, if we verify that \statement{the height of the mercury column is between 24 and 25 millimeters} we will be able to infer that \statement{the temperature is between 24 and 25 degrees Celsius}. That is, given a verifiable statement $\obs_Y$ we have another verifiable statement $\obs_X$ that is going to be true if and only if the first one is, that is $\obs_Y\equiv\obs_X$.

Note that the inference is between verifiable statements and not intervals. For example, the verifiable statement \statement{the water density is between 999.8 and 999.9 kg/$m^3$} will correspond to \statement{the water temperature is between 0 and 0.52 Celsius}$\OR$\statement{the water temperature is between 7.6 and 9.12 degrees Celsius} as water is most dense at 4 degrees Celsius. The disjuction of verifiable statements is still a verifiable statement so we are still inferring one verifiable statement from the other. For each verifiable statement in $\edomain_Y$ we can find a verifiable statement in $\edomain_X$ that is verified if and only the first is. That is: an inference relationship is a map from $\edomain_Y$ to $\edomain_X$ that preserves equivalence.

\begin{mathSection}
	\begin{defn}
		An \textbf{inference relationship} between two experimental domains establishes that testing a verifiable statement in one means testing a verifiable statement in the other. More formally, it is a map $\erel: \edomain_Y \to \edomain_X$ between two experimental domains $\edomain_X$ and $\edomain_Y$ such that $\erel(\obs_Y) \equiv \obs_Y$. In other words: it is an equivalence-preserving map between experimental domains.
	\end{defn}
	\begin{defn}
		An experimental domain $\edomain_Y$ is \textbf{dependent} on another experimental domain $\edomain_X$ if there exists an inference relationship $\erel: \edomain_Y \to \edomain_X$.
	\end{defn}
	\begin{defn}
		Two experimental domains $\edomain_X$ and $\edomain_Y$ are \textbf{equivalent} $\edomain_X \equiv \edomain_Y$ if $\edomain_X$ depends on $\edomain_Y$ and vice-versa.
	\end{defn}
\end{mathSection}

It should be evident that we cannot impose inference relationships between any two domains: it's something that the domains allow. The domains for the temperature of two different mercury columns are in general not related: testing the value of one does not tell us anything about the other. The topologies of the two domains, however, are going to be the same because we'll have the same possible values and the same way to experimentally test them. Equivalence between experimental domains is a much stronger relationship than equivalence of the natural topology. It carries enough of the semantic to be able to tell what spaces are truly scientifically equivalent.

Let's continue with our example. We can re-express the relationship between domains in terms of causal relationship between the two domains. If $x$ is the value of temperature of the mercury column (i.e. a possibility for $\edomain_X$) and $y$ is the height of the mercury column (i.e. a possibility for $\edomain_Y$), then we can write $y=f(x)$ since the height is determined by the temperature.

Note that the direction of the causal relationship is the opposite of the inference. $X$ causes $Y$ and we can infer $Y$ from $X$. Chains of events are in terms of possibilities and start with the cause and end with the effect. Chains of inferences are in terms of verifiable statements and start with the result and end with the origin.

The reverse does not work in general. Even if we know the final possibility, we may not be able to reconstruct the initial possibility: if the water density is exactly 999.9 kg/$m^3$, the temperature could be either 0.52 or 7.6 Celsius because density peaks at 4 Celsius. For the same reason, a measurement of the cause is not equivalent to a measurement of the effect: verifying that \statement{the water temperature is between 0 and 0.52 Celsius} will mean that \statement{the water density is between 999.8 and 999.9 kg/$m^3$} but not the other way around. Because of the peak in density, we can learn more about the temperature by measuring it directly.

Another important consideration is that, in order to be consistent, the function $y=f(x)$ has to be continuous. The general idea is the following: if we say that we can only measure both the temperature and height of a mercury column with finite precision, we have to make sure that when we use the causal relationship for inference, a finite precision measurement of height will correspond to a finite precision measurement of temperature. This means a small change in height has to correspond to a small change in temperature: the function is continuous.

More precisely, consider the verifiable statement $\obs_Y=$\statement{the height of the mercury column is between 24 and 25 millimeters}. The height of the mercury column $y$ is within the verifiable set $U(\obs_Y) = (24, 25)$ millimeters. We can then infer that the temperature must be in the reverse image of the possible heights $f^{-1}(U_Y(\obs_Y))=(24,25)$ Celsius. But this means that, indirectly, we have experimentally verified that $x$ is in $f^{-1}(U_Y(\obs_Y))$. And if $\edomain_X$ is really the domain of the verifiable statements for the temperature, then it must contain one that matches \statement{the temperature of the mercury column is between 24 and 25 Celsius}. In other words, $f^{-1}(U_Y(\obs_Y))$ must be a set in the topology of $X$ and the function is continuous.\footnote{In topology, continuity is defined in terms of the sets in the topology and not in terms of small changes as in analysis. When using the standard topology on real numbers, the two coincides but not in general.}

\begin{mathSection}
	\begin{defn}
		Let $(X, \mathsf{T}_X)$ and $(Y, \mathsf{T}_Y)$ be two topological spaces. A \textbf{continuous function} is a map $f: X \to Y$ such that given any open set $U_Y \in \mathsf{T}_Y$ its reverse image $f^{-1}(U_Y) \in \mathsf{T}_X$ is an open set. A \textbf{homeomorphism} is a continuous bijective map such that its inverse is also continuous.
	\end{defn}
	\begin{defn}
		Let $\edomain_X$ and $\edomain_Y$ be two experimental domains. A \textbf{causal relationship} from $\edomain_X$ to $\edomain_Y$ is a continuous function $f : X \to Y$ between the possibilities of the domains such that $x \narrower f(x)$.
	\end{defn}
	\begin{justification}
		Formally, we simply define causal relationships to be continuous but we still need a physical justification. An experimental test for the verifiable set $U_Y$ will indirectly test that the cause of $Y$ was in the set $f^{-1}(U_Y)$. This means that $f^{-1}(U_Y)$ can be associated with an experimental test and is therefore a verifiable set. If we want to be consistent, then, we need to make sure $f^{-1}(U_Y)$ is in the natural topology for $\edomain_X$.
	\end{justification}
	\begin{thrm}[Experimental Relationship Theorem]
		Inference and causal relationships are equivalent. More formally, let $\edomain_X$ and $\edomain_Y$ be two experimental domains. An inference relationship $\erel: \edomain_Y \to \edomain_X$ exists between them if and only if a causal relationship $f: \edomain_X \to \edomain_Y$ also exists.
	\end{thrm}
	\begin{proof}
		First we show that a causal relationship exists between independent and dependent domain. Suppose $\edomain_Y$ depends on $\edomain_X$. Given that for each statement in $\edomain_Y$ there exists an equivalent statement in $\edomain_X$, $\edomain_Y$ is effectively a subset of $\edomain_X$. By the same token, the theoretical domain $\tdomain_Y$ is effectively a subset of $\tdomain_X$. This means that a possibility $x \in \tdomain_X$, if true, will determine all the truth values of  all statements in $\tdomain_Y$, including its possibilities. But only one possibility $y \in \tdomain_Y$ can be compatible with $x$ since the possibilities of a domain are all incompatible with each other. Therefore we can define $f : X \to Y$ the function that given a possibility $x \in X$ returns the only possibility $y=f(x) \broader x$ that is compatible with it.
		
		We still need to show that $f$ is continuous. Consider a verifiable statement $\obs_Y \in \edomain_Y$. Let $U_Y(\obs_Y) \in \mathsf{T}_Y$ be its verifiable set. Since $\edomain_Y$ depends on $\edomain_X$, we can find $\obs_X \in \edomain_X$ such that $\obs_X \equiv \obs_Y$. Let $U_X(\obs_X) \in \mathsf{T}_X$ be its verifiable set. This is also the set of all possibilities in $X$ that is compatible with $\obs_Y$, which means $U_X(\obs_X)$ contains all the possibilities that are compatible with one possibility in $U_Y(\obs_Y)$. Since $f$ returns the only possibility in $Y$ compatible with a possibility in $X$, $f^{-1}(U_Y(\obs_Y))$ will return all the possibilities in $X$ that are compatible with one possibility in $U_Y(\obs_Y)$. That means $f^{-1}(U_Y(\obs_Y)) = U_X(\obs_X)$ and that $f^{-1}$ maps verifiable sets to verifiable sets. Therefore $f$ is continuous.
		
		Now we show that a causal relationship implies dependence between domains. Suppose we have a causal relationship $f: X \to Y$ between $\edomain_X$ and $\edomain_Y$. Let $y \in Y$ be a possibility for $\edomain_Y$. Consider   $f^{-1}(\{y\})$: this is the set of all possibilities in $X$ that are compatible with $y$. Since $\truth(y)=\TRUE$ if and only if $\truth(x)=\TRUE$ for some $x \in f^{-1}(\{y\})$, we must have $y \equiv \bigOR\limits_{x \in f^{-1}(\{y\})} x$. Now consider a verifiable statement $\obs_Y \in \edomain_Y$. We have $\obs_Y \equiv \bigOR\limits_{y \in U_Y(\obs_Y)} y \equiv \bigOR\limits_{y \in U_Y(\obs_Y)} \bigOR\limits_{x \in f^{-1}(\{y\})} x \equiv \bigOR\limits_{x \in f^{-1}(U_Y(\obs_Y))} x$. Because $f$ is continuous, the reverse image of a verifiable set is a verifiable set. Therefore there is a $\obs_X \in \edomain_X$ such that $U_X(\obs_X) = f^{-1}(U_Y(\obs_Y))$. The two verifiable statements $\obs_X \equiv \bigOR\limits_{x \in U_X(\obs_X)} x \equiv \obs_Y$ are equivalent. For each $\obs_Y \in \edomain_Y$ we can find an equivalent $\obs_X \in \edomain_X$ so $\edomain_Y$ depends on $\edomain_X$.
	\end{proof}
\end{mathSection}

Given the equivalence, we will simply use experimental relationship to describe the link between the two domains.

\section{Experimental domain for experimental relationships}

Now that we have seen how to describe relationships between domains, we should ask: are experimental relationships themselves something we can experimentally verify? We may know that there is a relationship between the temperature of a mercury column and its height, but how can we confirm experimentally which one it is?

The main result of this section is to show that, given two experimental domains, we can always construct from them another experimental domain for which the possibilities are continuous functions between the possibilities of the original domain. This means that, since we can recursively create relationship domains about relationship domains, the universe of discourse of our mathematical framework is closed.

Suppose we have a way to verify experimentally statements of the type \statement{if the temperature of the mercury column is between 24 and 25 Celsius then its height is between 24 and 25 millimeters}. That is, we can verify that whenever $\stmt_X=$\statement{the temperature of the mercury column is between 24 and 25 Celsius} is verified, then $\stmt_Y=$\statement{the height of the mercury column is between 24 and 25 millimeter} is verified. In that case, we can explore the connection between the two domain at different ranges at ever increasing precision. This means we can narrow the range of possible functions in the same way that we can narrow the range of possible values for a quantity. In other words, those type of statements form an experimental domain where each possibility corresponds to a possible continuous function between the two initial experimental domains.

\begin{mathSection}
	\begin{defn}
		Let $\edomain_X$ and $\edomain_Y$ be two experimental domains such that the second is dependent on the first. Let $\basis_X \subseteq \edomain_X$,  $\basis_Y \subseteq \edomain_Y$ be the two countable basis of the respective domains. Suppose the statements of the form
		$$\obs[v](\obs[e]_x, \obs[e]_y) = \statement{if $\obs[e]_x$ is true then $\obs[e]_y$ must also be true}=``\obs[e]_x \narrower \obs[e]_y"$$
		where $\obs[e]_x \in \basis_X$ and $ \obs[e]_y \in \basis_Y$ are verifiable. Then the \textbf{relationship domain} $\edomain_{C(X,Y)}$ is the experimental domain generated by the said verifiable statements.
	\end{defn}
	\begin{proof}
		TODO: sufficient to show it has a countable basis.
	\end{proof}
\begin{prop}
	The possibilities for a relationship domain $\edomain_{C(X,Y)}$ are the continuous functions between the possibilities $X$ and $Y$.
\end{prop}
\begin{proof}
	TODO: tentative strategy is to show that a possibility needs to identify a possible causal relationship and that a causal relationship will determine a possibility (as it will say which terms in the basis are true and which ones are false).
\end{proof}
\end{mathSection}

Note that the definition does not assume that we always have a way to verify such if/then statements. So when can we do it? Or better, what do we need in practice to be able to build the necessary confidence? Let's think how we would test the relationship between temperature and height of a mercury column. We would prepare different mercury samples with different values of temperatures in many different conditions, measure height and temperature with different devices, repeat many times, ask someone else to do it independently, compare results, and so on. At some point we will have explored enough of the possible cases, checked and tried anything that could invalidate the result and we will have the confidence to say that \statement{if the temperature of the mercury column is between 24 and 25 Celsius then its height is between 24 and 25 millimeters}. We should stress that the procedure is not at all to observe a few values and then generalize. It is the ability to prepare and control the system in many different conditions and our inability to violate the relationship that gives us the confidence needed to reach the conclusion.

Suppose, in fact, that we want to experimentally verify the link between inflation and money supply. As long as we cannot create new countries in different economic conditions, the only thing we can do is gather data for as many nations as we can throughout history.\footnote{Computer simulations can sometimes alleviate this problem, though they are only as good as the model one uses.} Since we cannot purposely explore different conditions and we can't even replicate older ones, the best we can do is show that there was a correlation between those specific values. It is the inability to freely and fully explore the problem space that may not enable us to experimentally verify the causal relationship.\footnote{In this sense, experimental sciences allow for more rigorous results than observational science precisely because it is more feasible to experimentally test relationships between domains.}

The point is that we are not going to try to formalize how to construct experimental tests for the relationships starting from the experimental tests of the individual domain.\footnote{That is, we tried to do it in several ways and realized it couldn't be done.} We cannot formalize in general what it means that we have explored the space ``enough" to consider the relationship verified, no more than we can formalize when the data collected is ``enough" to consider a statement verified or when a statement is specific ``enough" to consider the semantics well defined. This is where the practice of experimental science comes in.

But, while we may not know in general whether we can experimentally verify a relationship between two specific domains, we do know that the relationship domain can always be constructed in principle and therefore our mathematical framework is complete. That is: we can take two experimental domains (e.g. $\edomain_t$ for time and $\edomain_x$ for position), construct a relationship domain between them (e.g. $\edomain_{x(t)}$ for trajectories in space), then take another experimental domain (e.g. $\edomain_{(q,p)}$ for the states) and construct another relationship domain between the two (e.g. $\edomain_{(q,p)\to x(t)}$ for the relationship between states and trajectories). Our general mathematical theory of experimental science is therefore closed, since we can recursively create relationship domains about relationship domains indefinitely while remaining within its bounds. In other words, we cannot create relationships between domains that lie outside of our theory.

\section{Properties and quantities}

Most of the time we characterize a system by qualifying one of its properties, especially quantifiable ones. For examples, \statement{the mass of the electron is $9.109 \times 10^{-31}$ kg} of \statement{the world population is 7.6 billion people}. Therefore we need to properly define what we mean by quantities in our framework and how they relate to the mathematical structures defined before.

First we'll introduce the idea of using properties to differentiate a possibility among the others. A property can be either a physical one (e.g. the beak color of a bird) or one we define by convention (e.g. the taxonomy name for an animal species). Quantities are properties that have a magnitude: we can compare two different values and determine which one is greater or smaller. For example, groups of individuals can be labeled by their count and distances can be labeled in meters. In particular, we will define discrete quantities, those that correspond to integers, and continuous quantities, those that correspond to real numbers.

Suppose $\edomain_X$ is the domain for animal identification and $X$, its possibilities, are all animal species. Providing good names and definitions for species is a whole scientific subject by itself (i.e. taxonomy) with its own rules (i.e. the International Code of Zoological Nomenclature). The ICZN assignes each species a name composed of two latin words. For example, \statement{Passer domesticus} is the official name for the house sparrow while \statement{Passer italiae} is the one for the Italian sparrow. So, if $\mathcal{Q}$ is the set of all the names for all species, we have a function $q: X \to \mathcal{Q}$ from the species to its name. Since each species is given a unique name the function is invertible as well. This is the most discriminating type of property: one that covers the whole range of possibilities and fully identifies each of them. But this is a special case.

The ICZN also assignes a genus (pl. genera) to each species, which corresponds to the first part of the species name. For example, both aforementioned sparrows are of the genus \statement{Passer} while all swans, black or white, are of the genus \statement{Cygnus}. Now suppose that $\mathcal{Q}$ is the set of all names for all genera and $q: X \to \mathcal{Q}$ is the function that gives us the genus name for each species. We can still use this as a label for our possibilities, but it does not fully identify them. On a different note, we could decide to distinguish species by their morphological attributes. For example, if $\mathcal{Q}$ is the set of all colors, we can imagine a partial function $q: X \to \mathcal{Q}$ that gives us the beak color for each species, if they have one. While this is less general, it is still a valid label. In fact, it is with such labels that animal classification began.

Other examples of properties include: postal addresses for buildings, tax ID numbers for people, generation for fundamental particles, position for states of a point particles. In all these cases, the general pattern is that we assign a label to a possibility from an established set.

\begin{mathSection}
	\begin{defn}
		A \textbf{property} for an experimental domain $\edomain_X$ is any attribute we can use to distinguish between its possibilities. Formally, it is a tuple $(\mathcal{Q}, q)$ where $\mathcal{Q}$ is a topological space and $q : U \to \mathcal{Q}$ is continuous function where $U \in \mathsf{T}_X$ is a verifiable set of possibilities.
	\end{defn}
	\begin{justification}
		For the property to be physically meaningful, the domain of $q$ must be in the topology as we must be able to tell experimentally when the property is defined or not. For example, we must be able to tell whether an animal has a beak if we want to define the beak color. Therefore there must be a verifiable statement that is true if the property is defined. This corresponds to the verifiable set $U$, the domain of the property.
		
		Similarly, $\mathcal{Q}$ must be a topological space since we must be able to experimentally test the value for the property. For example, we can look at the beak color or we can look up the species name on a bird guide. Furthermore $q$ must be continuous since there must be a causal relationship between the possibility and the value of the property. For example, the species of the animal determines the color of the beak.
	\end{justification}
	\begin{defn}
		An experimental domain is \textbf{fully identified} by a property if its value is enough to uniquely determine a possibility. Formally, let $(\mathcal{Q}, q)$ be a property for an experimental domain $\edomain_X$. Then $\edomain_X$ is fully identified if $q: X \to \mathcal{Q}$ is injective.
	\end{defn}

	\begin{defn}
		An experimental domain is \textbf{fully characterized} by a property if its range of values and how they can be measured fully characterizes the domain. Formally, let $(\mathcal{Q}, q)$ be a property for an experimental domain $\edomain_X$. Then it is fully characterized if $q: X \to \mathcal{Q}$ is a homeomorphism.
	\end{defn}
\end{mathSection}

The difference between properties that fully identify and fully characterize the domain is subtle. Consider the experimental domain for identifying massive fundamental particles. The value of the mass in eV is enough to determine the particle, therefore it fully identifies the domain. But knowing that the mass can be expressed in eV is not enough to know which particular values of mass correspond to an actual particle. In this sense, the mass as a continuous quantity expressed in eV in not enough to fully characterize the experimental domain, which is in fact discrete. Somebody would have to tell you what are the allowed values. On the other hand, consider the experimental domain for the position of the center of mass of a particle along a particular direction. The distance from a fixed point is enough to identify the position, therefore it fully identifies the domain. But for each value of the distance we also have a possible position of the particle: the quantity fully characterizes the domain.

Let's now turn our attention to those properties that can be quantified. The number of elements of a group can be quantified by an integer, the distance between two objects can be quantified in meters, the force acting on a body can be quantified by the magnitude of a vector expressed in Newtons. The defining characteristic for a quantity is that a value can be greater, smaller or equal to another. That is: we have an ordering defined on our labels.
	
\begin{mathSection}
	\begin{defn}
		A \textbf{total order} on a set $\mathcal{Q}$ is a relationship $\leq : \mathcal{Q} \times \mathcal{Q} \to \mathbb{B}$ such that:
		\begin{enumerate}
			\item (antisymmetry) if $q_1 \leq q_2$ and $q_1 \leq q_2$ then $q_1 = q_2$
			\item (transitivity) if $q_1 \leq q_2$ and $q_2 \leq q_3$ then $q_1 \leq q_3$
			\item (total) at least $q_1 \leq q_2$ or $q_2 \leq q_1$
		\end{enumerate}
	\end{defn}
\begin{defn}
	Let $\mathcal{Q}$ be a topological space and $\leq : \mathcal{Q} \times \mathcal{Q} \to \mathbb{B}$ a total order. The \textbf{order topology} is the topology generated by the collections of sets of the form:
	$$(a, \infty) = \{x \in \mathcal{Q} \, | \, a < x\} \;,\; (-\infty, b) = \{x \in \mathcal{Q} \, | \, x < b\}.$$
\end{defn}
\begin{defn}
	A \textbf{quantity} for an experimental domain $\edomain_X$ is an ordered property. Formally, it is a tuple $(\mathcal{Q}, \leq, q)$ where $(\mathcal{Q}, q)$ is a property, $\leq : \mathcal{Q} \times \mathcal{Q} \to \mathbb{B}$ is a total order and $\mathcal{Q}$ is a topological space with the order topology with respect to $\leq$.
\end{defn}
\end{mathSection}

\subsection{Decidable domains and discrete quantities}

If the possible values of a quantity are the integers (positive or negative) we say the quantity is discrete. For example, the number of chromosomes for a species, the number of abitants of a country or the proton number for an element are all discrete quantities.

\begin{mathSection}
\begin{defn}
	A \textbf{discrete quantity} for an experimental domain $\edomain_X$ is a quantity $(\mathbb{Z}, \leq, q)$ where $\mathbb{Z}$ is the set of integers.
\end{defn}
\end{mathSection}

There is a special relationship between discrete quantities and decidability, the ability to test experimentally both the truthfullness and falsehood of a statement. Consider the examples above of discrete quantities: in each case we can experimentally test whether we have a particular value or not. For example, we are always able to tell whether an element has three protons or not.\footnote{Recall that this is not the case with continuous quantities. Because of uncertainty of measurement, we are able to exclude that a given particle has exactly zero mass but it is not possible to conclusevely show that it has zero mass.} It turns out that whenever we have a domain consisting of only decidable statements, we can always create a discrete quantity that fully characterizes the experimental domain.

\begin{mathSection}
	
	\begin{defn}
		An experimental domain $\edomain_X$ is \textbf{decidable} if all statements in the domain are decidable. Formally, for every $\obs \in \edomain$ we have $\NOT\obs \in \edomain$.
	\end{defn}

	\begin{prop}
		The set of possibilities $X$ for a decidable domain $\edomain_X$ is a countable basis.
	\end{prop}
	
	\begin{proof}
		First we show that $X$ is a basis. The decidable domain $\edomain_X$ is already closed under negation, so it coincides with the theoretical domain $\tdomain_X$. The set of possibilities $X$ is therefore a set of verifiable statements.
		
		Since the possibilities can generate all other statements through disjunction, we just need to show that $X$ is countable. Consider a countable basis $\basis \subseteq \edomain_X$. Because the possibilities are verifiable statements, they can be generated from $\basis$ by finite conjunction and countable disjunction. Moreover, since the possibilities are narrowest statement that are not contradictions, they can be generated from $\basis$ using finite conjunction only. Since $\basis$ is countable and $X$ is generated by $\basis$ through finite conjunction, $X$ can be at most countable. Therefore $X$ is a countable basis.
	\end{proof}
		
	\begin{defn}
		A topology $\mathsf{T}_X$ on a set $X$ is said \textbf{discrete} if it contains every subset of $X$.
	\end{defn}
	
\begin{thrm}[Decidabily is discreteness]\label{thrm_decidablity_is_discreteness}
	The natural topology of the possibilities $X$ for a domain $\edomain_X$ is discrete if and only if the domain is decidable.
\end{thrm}
\begin{proof}
	Suppose $\edomain_X$ is decidable. Let $U \subseteq X$ be a subset of possibilities. The statement $\stmt = \bigOR\limits_{x \in U} x$ is generated from $X$ through countable disjunction. Since $\edomain_X$ is decidable, $X$ is a countable basis and $\stmt$ is verifiable. Therefore $U$ is a verifiable set and it is contained in the natural topology. The natural topology of $X$ is discrete by definition.
	
	Now suppose $\edomain_X$ is such that the natural topology for its possibilities $X$ is discrete. Let $\stmt = \bigOR\limits_{x \in U} x$ be a statement. Since the topology is discrete, $U$ is part of the topology and $\stmt$ is verifiable. Consider its negation $\NOT\stmt = \bigOR\limits_{x \in U^C} x$. Since the topology is discrete, $U^C$ is also part of the topology and $\NOT\stmt$ is verifiable. This means $\stmt$ is decidable. Since every statement in $\edomain_X$ is decidable, the domain is decidable.
\end{proof}
	
\begin{prop}
	The order topology for the integers is discrete.
\end{prop}
\begin{proof}
	Each singleton $\{z\} \subseteq \mathbb{Z}$ is in the order topology since $\{z\} = (z-1, \infty) \cap (-\infty, z+1)$. Each arbitrary set of integers is the union of singletons and is therefore in the order topology as well. The order topology on the integers is discrete.
\end{proof}
	
	\begin{prop}
		An experimental domain is decidable if and only if it is fully characterized by a discrete quantity.
	\end{prop}
	
	\begin{proof}
		Let $\edomain_X$ be a decidable domain. Then the set of possibilities $X$ is countable and the natural topology is discrete. Since $X$ is countable, there exists a bijective map $q: X \to \mathbb{Z}$. The map is a homeomorphism since the topology on both $X$ and $\mathbb{Z}$ is discrete. The domain $\edomain_X$ is fully identify by $(\mathbb{Z}, \leq, q)$.
		
		Let $\edomain_X$ be fully characterized by $(\mathbb{Z}, \leq, q)$. This means that $q : X \to \mathbb{Z}$ is a homeomorphism. The natural topology for $X$ is therefore discrete and the domain is decidable by \ref{thrm_decidablity_is_discreteness}.
	\end{proof}	
	
\end{mathSection}

Note that for the link between decidability and discrete quantities to apply, it is crucial the quantity is measurable: that we can actually experimentally ascertain its values. Consider the domain with the possibilities \statement{there is no extra-terrestrial life} and \statement{there is extra-terrestrial life}. We can label 0 the former and 1 the latter. But since we cannot verify the first statement, we cannot really measure 0. In that case, the domain is fully identified by the discrete quantity, but not fully characterized: we still need to say what values are measurable.

\subsection{Arbitrary precision and continuous quantities}

If the possible values of a quantity are the real numbers we say the quantity is continuous. For example, the average wingspan for a species, the population density of a country or the mass of a proton are all continuous quantities.

\begin{mathSection}
	\begin{defn}
		A \textbf{continuous quantity} for an experimental domain $\edomain_X$ is a quantity $(\mathbb{R}, \leq, q)$ where $\mathbb{R}$ is the set of the real numbers.
	\end{defn}
\end{mathSection}

There is a special relationship between continuous quantities and the ability to measure the value with arbitrary precision. Consider the examples above of continuous quantities: in each can we can never experimentally ascertain the precise value, but we can (in principle) measure it to the desired level of precision.\footnote{Open problem: we define arbitrary precision using rational numbers but it may not be necessary. It may be sufficient to have a set of references, on which is defined an order. The verifiable statement would be of the form \statement{the value is between reference A and reference B}. We would require that there is always one reference between two references.} It turns out that whenever we can measure a quantity with arbitrary precision, we can always assign a continuous value to each possibility.

\begin{mathSection}
	
	\begin{defn}
		An experimental domain $\edomain_X$ is said to allow \textbf{arbitrary precision} if it describes a quantity measurable with arbitrary precision. Formally, there exists a countable basis $\basis \subseteq \edomain_X$ such that:
		\begin{itemize}
			\item we can define a function $\operatorname{bounds} : \basis \to \mathbb{Q} \times \mathbb{Q}$ such that $\operatorname{bounds}(\stmt) = (a, b)$ is a tuple where $a < b$ for all $\stmt \in E$
			\item if $\stmt = \stmt_1 \AND \stmt_2$, $\operatorname{bounds}(\stmt_1) = (a_1, b_1)$ and $\operatorname{bounds}(\stmt_2) = (a_2, b_2)$, then $\stmt \equiv \contradiction$ if $b_1 \leq a_2$ or $b_2 \leq a_1$ otherwise $\stmt \nequiv \contradiction$ and $\operatorname{bounds}(\stmt) = (\max(a_1, a_2), \min(b_1, b_2))$
		\end{itemize}
	\end{defn}
	
	\begin{defn}
		We call \textbf{standard topology} for the real numbers $\mathbb{R}$ the one generated by the collections of sets $\mathcal{B} = \{ (a,b) \subset \mathbb{R} \; | \; a,b \in \mathbb{Q} \}$ of all open intervals between rational numbers $\mathbb{Q}$.
	\end{defn}
	
	\begin{thrm}[Arbitrary precision is continuity]\label{thrm_arbitrary_precision_is_continuity}
		The natural topology of the possibilities $X$ for a domain $\edomain_X$ is homeomorphic to the reals with the standard topology if and only if the domain allows arbitrary precision.
	\end{thrm}
	\begin{proof}
		Let $\edomain_X$ allow arbitrary precision and let $\basis \subseteq \edomain_X$ be a countable basis on which the rational bounds are defined. A possibility $x$ is a minterm of such statements. Let $\basis_x =\{\obs[e]_i\}_{i=1}^{\infty} \subset \basis$ be the statements that are compatible with the possibility $x$. The set is countable since $\basis$ is countable.
		
		Consider the sequence $\{\stmt_j\}_{j=1}^{\infty}$ where $\stmt_j = \bigAND\limits_{i=1}^j\{\obs[e]^x_i\}$ is the conjunction of the first $j$ elements. For each element we have $\operatorname{bound}(\stmt_j) = (a_j, b_j)$. The sequence formed by $\{a_j\}_{j=1}^{\infty}$ is monotonically increasing as the lower bound cannot decrease during conjunction. The sequence formed by $\{b_j\}_{j=1}^{\infty}$ is monotonically decreasing as the upper bound cannot increase during conjunction. For each $j$ we have $a_j < b_j$, therefore both sequences are bound which means they both converge. Consider the sequence $c_j = b_j - a_j$. It must converge since it's the difference between two converging sequences. It cannot converge to a negative number since $a_j < b_j$ for all $j$. It cannot converge to a positive number: suppose it would, then we would be able to construct a statement strictly narrower than the limit of the sequence $\{\stmt_j\}_{j=1}^{\infty}$, but that is not possible since the limit is the possibility $x$ and only the contradiction is strictly narrower than a possibility. Therefore $c_j$ converges to zero and $a_j$ and $b_j$ are two equivalent Cauchy sequences of rational numbers. We can repeat the construction by reordering the elements of $\basis_x$. We would obtain different Cauchy sequences that are also equivalent to the first two. Each possibility $x$, then, is an equivalence class for the Cauchy sequences of the rational numbers. But this is a standard way to construct the real numbers therefore each possibility $x$ can be assigned to a real number.
		
		The verifiable set for each element of $\basis$ is the set of real numbers contained between the bounds: it is the open interval $(a,b)$. This means that the natural topology is the one generated by the collections of sets $\mathcal{B} = \{ (a,b) \subset \mathbb{R} \; | \; a,b \in \mathbb{Q} \}$ of all open sets of rational numbers $\mathbb{Q}$. The natural topology is the standard topology for the real numbers.
		
		Conversely, suppose let $\edomain_X$ be an experimental domain such that the possibilities are the real number and the natural topology is the standard topology. Since the collection of all open sets of rational numbers forms a subbasis of the topology, the statements associated with those verifiable sets is basis of the experimental domain. To each statement in the set we can associate the bounds of the open set satisfying the definition of an arbitrary precision domain.
	\end{proof}
	
	\begin{prop}
		The order topology for the real numbers is the standard topology.
	\end{prop}
	\begin{proof}
		To show that they are equivalent, we show that the subbase of one generates the subbase of the other. Let $a,b \in \mathbb{Q}$ be two rationals. The sets $(-\infty, b)$ and $(a, \infty)$ are in the subbase of the order topology. Their intersection is the set $(a, b)$ of the standard topology. The subbase of the standard topology can be generated by the order topology.
		
		Conversely, let $a \in \mathbb{R}$ be a real number. Let $\{U_i\}_{i \in I}$ the collection of all sets $U_i = (a_i, b_i)$ such that $a_i, b_i \in \mathbb{Q}$ and $a < a_i$. These sets are in the subbase of the standard topology. We have $\bigcup\limits_{i \in I} U_i = (a, \infty)$. In the same way, let $b \in \mathbb{R}$ be a real number. Let $\{V_j\}_{j \in J}$ the collection of all sets $V_j = (a_j, b_j)$ such that $a_j, b_j \in \mathbb{Q}$ and $b_j < b$. These sets are in the subbase of the standard topology. We have $\bigcup\limits_{j \in J} V_j = (-\infty, b)$. The subbase of the order topology can be generated by the standard topology.
	\end{proof}
	
	\begin{prop}
		An experimental domain allows arbitrary precision if and only if it is fully characterized by a continuous quantity.
	\end{prop}

	\begin{proof}
		If an experimental domain allows arbitrary precision then the set of possibilities with the natural topolofy is homeomorphic to the real numbers with the standard topology, which is also the order topology. Therefore we can construct a continuous quantity that fully identifies the possibilities. Conversely, if the domain is fully characterizes by a continuous quantity, then the set of real numbers with the order topology, which is also the standard topology, is homeomophic to the set of possibilities with the natural topology. Therefore the domain allows arbitrary precision by \ref{thrm_arbitrary_precision_is_continuity}.
	\end{proof}	
\end{mathSection}

The construction we presented here clarifies a couple of imporant aspects about continuous quantities. First of all, finite precision measurements are the starting point. They are what actually exists in practice. Second, the idea that we can always refine the precision of our measurements is clearly an idealization: at some point we are bound to encounter the physical limitation of either our measurement process or of our definitions. For example, measuring the inside of a gas container with a ruler can only get us so far and at some point, since a gas is made of molecules and a lot of empty space in between, defining very precisely when the gas ends and the container begins becomes problematic (and probably pointless). So, while the coarse precision measurement are physically meaningful, the finer measurements may not. Third, real numbers (i.e. the possibilities of an arbitrary precision domain) are abstractions based on the idealization of an arbitrary precision measurement. They are not verifiable statement: we cannot measure a continuous quantity with infinite precision. They exist only in so far that our idealization holds. They are indeed useful ways to model the domain but we should be very cautious to assign them a real tangible status.

This may be contrary to the way many people see the relationship between mathematical and physical objects. Some may feel that the geometric description, with its infinite precision, is the perfect one while the physical one, with the inherent measurement uncertainties, is the less precise one. Actually, it is quite the opposite: the bounds of a measurement better qualify our description and knowledge while the geometrical description provides a simplified, idealized and therefore less precise account. In other words, $3.14 \pm 0.005$ is something that actually exists while $\pi$ an the approximation.

Why, then, are continuous quantities so successful in science? We will look at this question in detail when we introduce the assumption of deterministic and reversible evolution. Without going too much into details, we can anticipate that the role of time is crucial. Time is the quantity that we want to assume continuous (i.e. that there is always an instant between other two). Because we want to describe deterministic and reversible systems, we will have an experimental relationship between time and all other quantities during the evolution. Which means that the way we assume time can be measured (i.e. its topology) will need to be consistent with the way those other quantities can be measured as well during deterministic and reversible evolution.

To sum up, we should regard the real number for what they are: the limit of an infinite process of subdivision (i.e. the limit of a Cauchy sequence). But this process, in reality, cannot go on forever.\footnote{How this process should continue in reality is not clear to us. We know it cannot just stop cold because a continuous topology is not the limit of a discrete topology with uncountable points. We know that arbitrary precision is a good model at large scales, so that should remain. We can imagine that the total ordering at small scale as the precision gets fuzzier.}


\section{Combining domains}

In this section we want to understand what happens when we combine statements from different domains. For example, suppose we have the experimental domain for the pressure of an ideal gas and the experimental domain for its temperature. We can mix verifiable statements with conjunction and disjunction as in \statement{the pressure is between 1 and 1.1 KPa}$\AND$\statement{the temperature is between 20 and 21 C} creating a new domain. How can we characterize this combined experimental domain?

The main result of this section is that the possibilities of the combined domain depend on how compatible are the verifiable statements of the domains. We will explore three particular cases: combing independent domains will give use the scalar product of the possibilities, combining dependent domains will leave the possibilities unchanged while combing incompatible domains will give the disjoint union of the established possibilities.

Suppose $\edomain_X$ is the experimental domain generated by the two verifiable statements \statement{the patient is dead} and the \statement{the patient is alive} and a second one $\edomain_Y$ generated by the two verifiable statements \statement{the patient is not in a coma} and \statement{the patient is in a coma}. Both domains are decidable and the given verifiable statements also correspond to the possibilities for the respective domain.

We can construct the combined domain $\edomain_X \times \edomain_Y$ by taking all possible disjunctions and conjunctions. What are the possibilities for the new domain? Since by \ref{prop_poss_is_minterm} the possibilities are minterms, we have the following cases to consider:
\begin{itemize}
	\item \statement{the patient is alive} $\AND$ \statement{the patient is in a coma}
	\item \statement{the patient is alive} $\AND$ \statement{the patient is not in a coma}
	\item \statement{the patient is dead} $\AND$ \statement{the patient is in a coma}
	\item \statement{the patient is dead} $\AND$ \statement{the patient is not in a coma}
\end{itemize}
The third one is a contradiction: the patient cannot be dead and in a coma. Therefore the combined domain has only three possibilities. The possibilities of the combined domain is, in general, the subset of all possible combination (i.e. the scalar product) of the possibilities of the domains we are combining that do not lead to contradictions.

\begin{mathSection}
	
	\begin{defn}
		Let $\{\edomain_{X_i}\}_{i=1}^{\infty}$ be a countable set of experimental domains. The \textbf{combined experimental domain} $\edomain_{X} = \bigtimes\limits_{i=1}^{\infty} \edomain_{X_i}$ is the experimental domain generated from all statements in $\{\edomain_{X_i}\}_{i=1}^{\infty}$ by finite conjunction and countable disjunction.
	\end{defn}
	\begin{proof}
		We need to show that the combined experimental domain is indeed an experimental domain. It will contain the tautology and the contradiction since any of the original experimental domains contains them. It is closed under finite conjunction and countable disjunction by construction. To show that it has a countable basis, for each $i=1..\infty$ let $\basis_i \in \edomain_{X_i}$ be a countable basis. Consider $\basis=\bigcup\limits_{i=1}^\infty \basis_i$. From this set we can generate any $\edomain_{X_i}$ and therefore we can also generate all of $\edomain_{X}$. $\basis$ is a basis and it is countable since it is the union of a countable set of countable elements.
	\end{proof}

\begin{prop}\label{prop_combined_possibility}
	The possibilities $X$ for a combined domain $\edomain_{X} = \bigtimes\limits_{i=1}^{\infty} \edomain_{X_i}$ is set of all possible conjunctions of possibilities $x_i \in X_i$ of the original domain that are not contradictions. That is $X = \{ x = \bigAND\limits_{i=1}^{\infty} x_i \, | \, x_i \in X_i, \, x \nequiv \contradiction \}$.
\end{prop}   
\begin{proof}
	A possibility $x$ of the combined domain is a minterm of a basis $\basis \subseteq \edomain_{X}$. Since we can choose $\basis=\bigcup\limits_{i=1}^\infty \basis_i$ where $\basis_i \subseteq \edomain_{X_i}$ is a countable basis for each domain, $x$ is the conjunction $x=\bigAND\limits_{i=1}^{\infty}x_i$ of minterms $x_i$ of $\basis_i$. Since $x$ is a possibility, it is not a contradiction and therefore neither of the $x_i$ can be a contradiction. Since each $x_i$ is a minterm of the respective basis $\basis_i$ that is not a contradiction, it is a possibility by \ref{prop_poss_is_minterm}. Therefore a possibility $x$ of the combined domain is the conjunction of the possibilities $x_i$ of the original domains that is not a contradiction.
\end{proof}
\end{mathSection}

\subsection{Independent domains}

A special case is when combining two independent domains. For example, the domain for the pressure and the domain for the volume of an ideal gas are indepent because a measurement on one tells us nothing about the other. Similarly, the domain for the shape and the domain for the color of an object are independent. In these cases, we can have any combination of the possibilties: any pressure with any volume or any color with any shape.

In terms of topology, the possibilities of the combined domain is the cartesian product of the possibilities of the original domains and its natural topology is the product topology.\footnote{Note that the topology is quite naturally the product topology and not the box topology. The box topology would require countable conjunction and is therefore discarded. The fact that the correct topology is the one most natural to define confirms again the appropriatedness how our framework.}

\begin{mathSection}
	\begin{defn}
		The experimental domains of a countable set $\{\edomain_{X_i}\}_{i=1}^{\infty}$ are \textbf{independent} if taking one verifiable statement $\obs_i \in \edomain_{X_i}$ from each domain always gives an independent set of statements.
	\end{defn}
	\begin{prop}
		Let $\{\edomain_{X_i}\}_{i=1}^{\infty}$ be a countable set of independent experimental domains and $X_i$ their respective possibilities. The set of possibilities $X$ of the combined experimental domain $\bigtimes\limits_{i=1}^{\infty} \edomain_{X_i}$ consists of all the possible conjunctions of the possibilities of each domain. That is: $X = \{ \bigAND\limits_{i=1}^{\infty} x_i \, | \, x_i \in X_i \}$. Notationally, we write $\edomain_X=\edomain_{\bigtimes\limits_{i=1}^{\infty} X_i}$.
	\end{prop}
	\begin{proof}
		A possibility $x=\bigAND\limits_{i=1}^{\infty}x_i$ for the combined domain is the conjunction of possibilities of each individual domain by \ref{prop_combined_possibility}. Since the domains are independent and since possibilities are neither tautologies or contradictions, we have $\possFn(x) = \possFn(\bigAND\limits_{i=1}^{\infty}x_i) = \bigAND\limits_{i=1}^{\infty}\possFn(x_i)= \bigAND\limits_{i=1}^{\infty} \{\FALSE, \TRUE\} = \{\FALSE, \TRUE\}$. That is, each conjunction $x=\bigAND\limits_{i=1}^{\infty}x_i$ is not a contradiction and therefore is a possibility.
	\end{proof}
	\begin{defn}
	Let $\{(X_i, \mathsf{T}_i)\}_{i=1}^{\infty}$ be a countable set of topological spaces. Let $X=\bigtimes\limits_{i=1}^{\infty} X_i$ be the Cartesian product of the points. Let $\mathcal{B}$ the collection of sets of the form $\bigtimes\limits_{i=1}^{\infty} U_{i}$, with $U_i \in \mathsf{T}_i$ and $U_i \neq X_i$ only finitely many times. The topology generated by $\mathcal{B}$ is called the \textbf{product topology}.
\end{defn}
	\begin{prop}
	Let $\{\edomain_{X_i}\}_{i=1}^{\infty}$ be a countable set of independent experimental domains. The natural topology for the possibilities of the combined experimental domain $\edomain_{\bigtimes\limits_{i=1}^{\infty} X_i}$ is the product topology of the natural topology for the possibilities of each domain.
\end{prop}
\begin{proof}
	Let $U_i : \edomain_{X_i} \to \mathsf{T}_{X_i}$ be the map from a verifiable statement of a domain to its verifiable set in the respective topology. Let $U : \edomain_{X} \to \mathsf{T}_X$ the same map for the combined domain. Let $\obs_i \in \edomain_{X_i}$ a verifiable statement from a particular domain and $U_i(\obs_i)$ its verifiable set in that domain. Since we also have $\obs_i \in \edomain_{X}$, the statement is also associate with the verifiable set $U(\obs_i)$ in the combined domain. Because the domains are independent, every possibility in $U_i(\obs_i)$ is compatible with any possibility $x_j \in X_j$ for all $j \neq i$. This means that $U(\obs_i)=X_1\times ... \times X_{i-1} \times U_i(\obs_i) \times X_{i+1} \times ...$ . Given that a verifiable statements in the combined domain can be generated using finite conjunction and countable disjunction from the verifiable statements of the independent domains, the topology of the combined space can be generated by all sets of the form $\bigtimes\limits_{i=1}^{\infty} U_{i}$, with $U_i \in \mathsf{T}_i$ and $U_i \neq X_i$ only once. Using finite intersection, this includes those sets where $U_i \neq X_i$ finitely many times. The natural topology of the combined domain is the product topology by definition.
\end{proof}
\end{mathSection}

\subsection{Dependent domains}

Another special case is combining a domain $\edomain_X$ with another $\edomain_Y$ that is dependent on it. For example, combining the domain for the temperature of a mercury column with the domain for its height. Since the height can be determined by the temperature, no new possibilities are added. The combined domain is equivalent to the independent domain $\edomain_X$ since all the verifiable statements in $\edomain_Y$ have all equivalents in it.

\begin{mathSection}
	\begin{prop}
		Let $\edomain_X$ and $\edomain_Y$ be two experimental domain such that the second depends on the first. Then $\edomain_X \times \edomain_Y \equiv \edomain_X$.
	\end{prop}
	\begin{proof}
		Since $\edomain_Y$ is dependent on $\edomain_X$, any statement in $\edomain_Y$ is equivalent to one in $\edomain_X$. The combined domain $\edomain_X \times \edomain_Y$ is generated by all statements in $\edomain_X$ and by another set of statements equivalent to a subset of $\edomain_X$. Therefore $\edomain_X \times \edomain_Y$ must be equivalent to $\edomain_X$. 
	\end{proof}
\end{mathSection}

\subsection{Incompatible domains}

The last special case we look into is when the domains are incompatible, that is all verifiable statements of one are incompatible with the verifiable statements of the other. As you will see, this is one of the cases where the residual possibility behaves differently from all the others.\footnote{In fact, this is what prompted us to introduce the residual possibility.} Suppose $\edomain_X$ is the domain to classify a particual specimen as an animal and $\edomain_Y$ is the domain to classify it as a plant. If we take a verifiable statement from the first, such as \statement{that specimen has fur}, then it will be incompatible with a verifiable statement from the other, such as \statement{that specimen has lobed leaves}. The only way we can combine the possibilities is to take an established possibility of one (e.g. \statement{this specimen is a cat}) and combine it with the residual possibility of the other (e.g. \statement{this specimen is not a plant}). In other words, the combined possibilities are the union of the possibilities of the two domains (e.g. all possible plants and all possible animals).

In terms of the topology, the established possibilities of the combined domain is the disjoint union of the established possibilities of the original domains and its natural topology is the disjoint union topology (or coproduct topology). 

\begin{mathSection}
	\begin{defn}
		Two experimental domains $\edomain_X$ and $\edomain_Y$ are \textbf{incompatible} if all verifiable statements in one are incompatible to all verifiable statements of the other. Formally, $\obs_X \ncomp \obs_Y$ for each pair of verifiable statements $\obs_X \in \edomain_X$ and $\obs_Y \in \edomain_Y$.
	\end{defn}
	\begin{prop}
		Let $\edomain_{X_1}$ and $\edomain_{X_2}$ be two incompatible experimental domains. Then the established possibilities of the combined domain $\edomain_X = \edomain_{X_1} \times \edomain_{X_2}$ is equal to the disjoint union of the established possibilities of the individual domains. That is $\dot{X} = \dot{X}_1 \cup \dot{X}_2$. Notationally, we write $\edomain_{X_1} \times \edomain_{X_2} = \edomain_{X_1 \coprod X_2}$.
	\end{prop}
	\begin{proof}
		TODO
	\end{proof}
\end{mathSection}

Before ending this section, a note on category theory for those already familiar with it. A morphism of an experimental domain is an experimental relationship: $\erel : \edomain_1 \to \edomain_2$ such that $f(\obs) \equiv \obs$. It is a map that preserves the semantic content of each statement and therefore what is verified by an experimental test.

As we noted before, though, for experimental domains we do not get to choose what morphisms we have on the different domains. If we take two one dimensional euclidean spaces as topological spaces, we can choose what continuous function (if any) we have between them.  If we take the two experimental domain for temperature and height of the same mercury column, we have no choice. So the set of morphisms between two experimental domains either has one element or none. Indeed, this maps to what happens in science: whether there is a relationships between two physical quantities is something that we have to find out, not arbitrarily decide.

The same applies to the combination of domains. If we have two one dimensional euclidean spaces we can decide whether to take their product (i.e. the plane) or their coproduct (i.e. the disjoint union of the two lines). For experimental domain we do not choose: it is what it is. There is no way to combine the two experimental domain for temperature and height of the same mercury column and get the Cartesian product of their possibilities. Though combining two independent domains mimics the categorical product and combining two incompatible domains mimics the categorical coproduct, it is the semantic (and ultimately physical) relationship between them that decides which product we have.

In later chapters, though, we will find some limited use for category concepts by ``forgetting" the semantic. Topological spaces will only care how the possibilities are distinguished in terms of verifiable sets. Therefore, while the domains for temperature of two different thermometers are not equivalent, their natural topologies are equivalent because we are assuming that they are both fully characterized by a scalar quantity that can be measured with arbitrary precision. In the same vein, Abelian groups will only care how distributions can be composed into other distributions. Non-abelian groups will only care how transformations can be composed into other transformations. And so on. But it is the pattern defined by the combination of experimental domain that is really giving us the meta-structure common to all categories.

\section{Summary}

\chapter{Second chapter}

The first chapter was about testing truths about a single domain. The second chapter should about testing truths of multiple domain.

It should address
\begin{itemize}
	\item Define some domain for general statements of other domains. For example \statement{the position of this electron is between 1 and 2 meters} and \statement{an electron can have position between 1 and 2 meters}.
	\item Define the domain for relationships between two domains so we can show that experimental relationships can be themselves tested experimentally.
	\item Define processes that produce a sequence of experimental domains. For example, I produce electrons within a certain range of momentum.
	\item Define (statistical?) equivalence for processes. Define ``regular" or ``at equilibrium" processes as onces that are self-similar. Define probability
	\item Define differentiable structure on a manifold. Define distributions, measures, etc...
\end{itemize}

(Moved from previous chapter) Now that we have characterized the nature of experimental relationships, we need to show that these relationship can be tested experimentally. That is, we must be able to create an experimental domain such that the possibilities are the continuous functions between two domains. We have to do this two show that our framework is complete.

\begin{mathSection}
	\begin{axiom}
		Given two experimental tests $\expt_1$ and $\expt_2$ we can always construct another experimental test that is always successful if and only if the success of the first test $\expt_1$ implies the success of the second test is $\expt_2$. More formally, let $\expt_1, \expt_2 \in \exptSet$, there exists $\expt_3 \in \exptSet$ such that $\result(\expt_3, i) = \SUCCESS$ for all $i \in \mathbb{N}$ if and only if for all $j \in \mathbb{N}$ where $\result(\expt_1, j) = \SUCCESS$ we have $\result(\expt_2, j) = \SUCCESS$.
	\end{axiom}
	\begin{defn}
		Let $\edomain_X$ and $\edomain_Y$ be two experimental domains such that the second is dependent on the first. Let $E_X \subseteq \edomain_X$,  $E_Y \subseteq \edomain_Y$ be the two countable basis of the respective domains. The relationiship domain $\edomain_{C(X,Y)}$ is the experimental domain generated by the experimental statements
		$$\obs[v](\obs[e]_x, \obs[e]_y) = \statement{if $\obs[e]_x$ is true then $\obs[e]_y$ must also be true}=``\obs[e]_x \narrower \obs[e]_y"$$
		where $\obs[e]_x \in E_X$ and $ \obs[e]_y \in E_Y$.
	\end{defn}
	\begin{proof}
		We have to show that the statements $\obs[v](\obs[e]_x, \obs[e]_y)$ are indeed verifiable statements.
	\end{proof}
\end{mathSection}

\chapter{Other}

Notes on sequences/probability.

A source? is a sequence of truth values $s=\{t_i\}$. Two sources $s=\{t_i\}$ and $s=\{u_i\}$ are similar $s_1 \sim s_2$ if given $\epsilon > 0$ we can find an $N \in \mathbb{N}$ such that for all $a,b>N$ we have $\left| \frac{1}{a} \sum\limits_{i=1}^a t_i - \frac{1}{b} \sum\limits_{i=1}^b u_i \right| < \epsilon$.

A source? is stable? if it self-similar $s \sim s$. The average of a stable source converges as it satisfies the Cauchy condition.

	
\end{document}