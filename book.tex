% Possible use http://www.latextemplates.com/template/the-legrand-orange-book as template? See https://www.overleaf.com/9174958nyjxxdxbchks#/33024595/

\documentclass[11pt,letterpaper,fleqn]{memoir} % Default font size and left-justified equations

\usepackage[top=3cm,bottom=3cm,left=3cm,right=3cm,headsep=10pt,letterpaper]{geometry} % Page margins

% Theorem definitions using amsthm

\usepackage{amsthm}
\usepackage{amsmath}
\usepackage{amssymb}

% Remove line spaces between items of enumerate and itemize
\usepackage{enumitem}
\setlist{noitemsep}


% Adds double bracket symbols
\usepackage{stmaryrd}

% Latex symbol guide at http://mirrors.ibiblio.org/CTAN/info/symbols/comprehensive/symbols-letter.pdf

% LOGIC symbols
% -------------

% Allows to create negation symbols
\usepackage{MnSymbol}

\DeclareMathOperator{\truth}{truth}
\DeclareMathOperator{\poss}{possibilities}
\DeclareMathOperator{\result}{result}

\def\TRUE{\textsc{true}}
\def\FALSE{\textsc{false}}

\def\SUCCESS{\textsc{success}}
\def\FAILURE{\textsc{failure}}
\def\UNDEF{\textsc{undefined}}


% Symbols for tautology and contradiction
\def\tautology{\top}
\def\contradiction{\bot}

% Symbols for "compatibility" and "incompatibility"
\def\comp{\doublefrown}
\def\ncomp{\ndoublefrown}

% Symbols for "narrower" and "wider"
\def\narrower{\subseteq}
\def\nnarrower{\nsubseteq}
\def\broader{\supseteq}
\def\nbroader{\nsupseteq}


% Symbol for "independent" and "correlated"
\def\indep{\upmodels}
\def\nindep{\nupmodels}

% Aliases for logical operations
\def\AND{\wedge}
\def\bigAND{\bigwedge}
\def\OR{\vee}
\def\bigOR{\bigvee}
\def\NOT{\neg}


% Formatting for statements
\newcommand{\stmt}[1][s] {\mathsf{#1}}
% Formatting for experimental tests
\newcommand{\expt}[1][e] {\mathsf{#1}}
% Formatting for observations
\newcommand{\obs}[1] {\mathsf{#1}}
% Formatting for observation definition
\newcommand{\obsdef}[2] {\llparenthesis #1, #2 \rrparenthesis}

% Formatting for experimental domain
\newcommand{\edomain}[1][D] {\mathcal{#1}}

% Formatting for theoretical domain
\newcommand{\tdomain}[1][D] {\bar{\mathcal{#1}}}


% Formatting for sentence statements
\newcommand{\statement}[1] {\emph{``#1"}}


\usepackage{xcolor} % Required for specifying colors by name
\definecolor{sectionNumbers}{RGB}{44, 103, 0}


\renewcommand\thesubsection{\thesection.\Alph{subsection}}
\renewcommand{\theequation}{\thechapter.\arabic{equation}}

\newtheorem{assump}{Assumption}
\renewcommand*{\theassump}{\Roman{assump}}

\newtheorem{defn}[equation]{Definition}
\newtheorem{prop}[equation]{Proposition}

%\theoremstyle{definition}

\newenvironment{rationale}{\emph{Rationale}.}{\qed}
\newenvironment{justification}{\emph{Justification}.}{\qed}
\renewenvironment{proof}{\emph{Proof}.}{\qed}

% Style for math section
\RequirePackage[framemethod=default]{mdframed} % Required for creating the theorem, definition, exercise and corollary boxes
\newmdenv[skipabove=7pt,
skipbelow=7pt,
rightline=false,
leftline=true,
topline=false,
bottomline=false,
linecolor=sectionNumbers,
backgroundcolor=black!2,
innerleftmargin=5pt,
innerrightmargin=5pt,
innertopmargin=5pt,
leftmargin=0cm,
rightmargin=0cm,
linewidth=4pt,
innerbottommargin=5pt]{mathSection}


%----------------------------------------------------------------------------------------
%	SECTION NUMBERING IN THE MARGIN
%----------------------------------------------------------------------------------------

\makeatletter
\renewcommand{\@seccntformat}[1]{\llap{\textcolor{sectionNumbers}{\csname the#1\endcsname}\hspace{1em}}}                    
\renewcommand{\section}{\@startsection{section}{1}{\z@}
	{-4ex \@plus -1ex \@minus -.4ex}
	{1ex \@plus.2ex }
	{\normalfont\large\sffamily\bfseries}}
\renewcommand{\subsection}{\@startsection {subsection}{2}{\z@}
	{-3ex \@plus -0.1ex \@minus -.4ex}
	{0.5ex \@plus.2ex }
	{\normalfont\sffamily\bfseries}}
\renewcommand{\subsubsection}{\@startsection {subsubsection}{3}{\z@}
	{-2ex \@plus -0.1ex \@minus -.2ex}
	{.2ex \@plus.2ex }
	{\normalfont\small\sffamily\bfseries}}                        
\renewcommand\paragraph{\@startsection{paragraph}{4}{\z@}
	{-2ex \@plus-.2ex \@minus .2ex}
	{.1ex}
	{\normalfont\small\sffamily\bfseries}} % Loads the book formatting

\begin{document}
	
\chapter{Experimental observations and distinguishability}

In this chapter we want to establish the basic building blocks that are necessary to capture the nature of scientific descriptions. While mathematics finds much of its foundation in classical logic and set theory, we want to find equivalent ideas for science. In essence, we need to define what is a scientific truth and what it means to define elements and collections in the physical world.

The most elementary concept we introduce is the one of \textbf{experimental observation} which is a statement that can be experimentally verified in a finite amount of time. While, in logic, we can always combine propositions using negation (logical NOT), conjunction (logical AND) and disjunction (logical OR), we cannot do the same with experimental observations. For example, given that we can only measure a real value with finite precision, excluding a particular value can be done in finite time but verifying it with infinite precision is not possible.

Next we group experimental observations into an \textbf{experimental domain} which is the collection of possible answers to a question that can be settled experimentally. This is guaranteed to contain a \textbf{basis}: a countable set of experimental observations that, once tested, will tell us which other observations are verified. Without such basis, we would not be able to find an answer to the question even in the limit of infinite time.

Then we define \textbf{experimental characterization/identification/categorization/recognition} as an observation that places an object within a set possibilities. A set of possibilities is said to be \textbf{experimentally distinguishable} if all possible experimental recognitions within that set forms an experimental domain and are able to tell all the elements apart. The logical operations (conjunction and disjunction) on the experimental recognitions become set operations (intersection and union) on the set of possibilities. Mathematically, this induces a \textbf{topology} on the set of possibilities.

Lastly, we define \textbf{experimental relationships} between experimentally distinguishable objects. As these need to be compatible with the experimental distinguishability of the objects, the associated mathematical functions must be compatible with the topology: they must be \textbf{continuous functions}. These, a set, are themselves experimentally distinguishable which means that we can describe relationships of arbitrarily higher-order (i.e. functions of functions of functions ...) without ever losing experimental distinguishability.

The overall result is that all sets of objects that are of scientific interest are mathematically described by a Hausdorff and second countable topological space. All relationships that are of scientific interest are continuous functions as defined on the respective topologies. This explains at the core why topological space and continuous functions are indeed so prevalent within scientific theories.

\section{Experimental observations}

Science is the systematic study of the physical world through observation and experimentation. This means that everything we want to talk about scientifically must be well defined experimentally in a way that can be independently verified by anyone. The foundation of any scientific theory needs to start there and this is exactly where we are going to start.

First we should realize that not all statements can be studied in science. For example, \emph{``jazz is marvelous"} or \emph{``green and red go well together"} are not suitable scientific statements as they are subjective. That is, there is no agreed upon definition or procedure for what constitutes marvelous music or good color combination. This does not mean that these are somehow statements of a lesser nature. In fact, most of the things that make life worth living (e.g. love, friendship, arts, purpose and so on) defy objective characterization and it would be a very dull world if they didn't. It simply means: they are not the subject of scientific inquiry.

While we have excluded subjective statements, not all objective statements are experimental either. For example, \emph{``the square of the hypotenuse is equal to the sum of the squares of the other two sides"} or \emph{``God is eternal"} are not suitable scientific statements as they involve objects that are not physical in nature. As we cannot define those objects experimentally, they depend on what basic axioms or definitions we take as a starting point. For example, the first statement is not true in non-Euclidean geometry and the concept of God as evolved significantly throughout the millennia. Again, this does not mean these concepts are of less significance. In fact, one may be more interested in them precisely because of their abstract, and therefore less transient, nature. Nonetheless, they are not the subject of scientific inquiry.

Limiting the scope of our discussion to objects and properties that are well defined physically is still not enough. For example, \emph{``the electron is green"} or \emph{``1 meter is equal to 5 Kelvin"} are still not suitable scientific statements as the relationships established are not physically meaningful. Even when the relationship is meaningful, we may still not be able to validate it experimentally. For example, \emph{``there is no extra-terrestrial life"} or \emph{``the mass of the electron is exactly $9.109 \times 10^{-31}$ kg"} are not something we can validate experimentally. In the first case, we would need to check every corner of the universe and find none, with the closest galaxy like ours, Andromeda, being 2.5 millions light-years; in the second, we will always have an error bar, however small.

So what is the property that allows a statement to be studied scientifically? The ability to verify it experimentally. Science is limited to study only those statements for which we have available an experimental procedure to objectively and repeatedly verify them. This is both the power and the limit of scientific inquiry: it gives us a way to construct a coherent description of the physical world but it is limited to those aspects that can be reliably studied experimentally. Therefore, science is only one of the intellectual tools we have at our disposal and it can only complement the others (such as mathematics, philosophy, theology and art).

\begin{mathSection}

To formally capture the idea of experimentally verifiable statements, we introduce the following definitions.

\begin{defn}
	A \textbf{statement} $\mathsf{s}$ is a declarative sentence that is either true or false, as in classical logic.
\end{defn}

\begin{defn}
	An \textbf{experimental test} $\mathsf{e}$ is a repeatable procedure (i.e. it can be restarted and stopped an arbitrary number of times) that, if successful, terminates in a finite amount of time.
\end{defn}

\begin{defn}
	An \textbf{experimental observation} is a tuple $\mathsf{o} = \llparenthesis \mathsf{s}, \mathsf{e} \rrparenthesis$ composed by a statement $\mathsf{s}$ and an experimental test $\mathsf{e}$ such that the statement is true if and only if the  experimental test is always successful. The experimental observation is \textbf{verified} if the statement is true.
\end{defn}

\begin{defn}
	An \textbf{experimental counter-observation} is a tuple $\mathsf{o}^c= \llbracket \mathsf{s}, \mathsf{e} \rrbracket$ composed by a statement $\mathsf{s}$ and an experimental test $\mathsf{e}$ such that the statement is false if and only if the experimental test is always successful. The experimental counter-observation is \textbf{refuted} if the statement is false.
\end{defn}

Experimental observations are the core object of the scientific discourse. We now define some basic relationships between experimental observations.

\begin{defn}
	Given two experimental observations $\mathsf{o}_1$ and $\mathsf{o}_2$ we define:
	\begin{itemize}
	\item \textbf{implication}: $\mathsf{o}_1 \imp \mathsf{o}_2$ ($\mathsf{o}_1$ implies $\mathsf{o}_2$) if $\mathsf{o}_2$ is always verified whenever $\mathsf{o}_1$ is verified
	\item \textbf{equivalence}: $\mathsf{o}_1 = \mathsf{o}_2$ ($\mathsf{o}_1$ is equal to $\mathsf{o}_2$) if $\mathsf{o}_1 \imp \mathsf{o}_2$ and $\mathsf{o}_2 \imp \mathsf{o}_1$
	\item \textbf{compatibility}: $\mathsf{o}_1 \comp \mathsf{o}_2$ ($\mathsf{o}_1$ is compatible with $\mathsf{o}_2$) if there exists a case in which $\mathsf{o}_1$ and $\mathsf{o}_2$ are jointly verified
	\end{itemize}
    The negation of these properties will be noted by $\nimp$, $\neq$ , $\ncomp$ respectively.
\end{defn}

A note on the language: even if $\mathsf{o}_1$ implies $\mathsf{o}_2$ this does not give us a way to verify $\mathsf{o}_2$. There may be cases where the $\mathsf{o}_2$ is verified and $\mathsf{o}_1$ is not. Therefore it is improper to say that one observation verifies another: only experimental tests verify observations.

Lastly, we define a special type of observation. 

\begin{defn}
	The \textbf{null observation} is any experimental observation that is never verified (i.e. the experimental test will never terminate successfully). 
\end{defn}

The null observation is the degenerate case and has the following, maybe counterintuitive, properties. Since it is never verified, it can never rule out any other observations: the null observation implies all other observations. Since it is never verified, it can never be jointly verified with another observation: the null observation is incompatible with every experimental observation, including itself.

\end{mathSection}

\section{Logical operations on experimental observations}

Now that we have defined experimental observations, we want to use logic to form other observations or to check for consistency. For example, once we have established that \emph{``cold reduces swelling"} and that \emph{``ice is cold"} we may conclude that \emph{``ice 
reduces swelling"}. Note that this type of inference relies on the meaning of the actual observations. That is: we understand that cold is a property that has an effect and that ice posses it. Therefore we cannot be generalize it to all experimental observations.

What we want to do is study how experimental observations behave under the common logical operations. These do not depend on the specifics of the observation and therefore can be applied generally.
\begin{table}[h]
	\centering
	\begin{tabular}{p{0.2\textwidth} p{0.1\textwidth} p{0.1\textwidth} p{0.5\textwidth}}
		Operator & Gate & Symbol & Example \\ 
		\hline 
		Negation & NOT & $\neg A$ &  \emph{``the sauce is not sweet"} \\ 
		Conjunction & AND & A $\wedge B$ & \emph{``the sauce is sweet and sour"} \\ 
		Disjunction & OR & $A \vee B$ & \emph{``the sauce is at least one of sweet or sour"}\\
		\multicolumn{4}{c}{  where $A$ = \emph{``the sauce is sweet"} and $B$ = \emph{``the sauce is sour"}}
	\end{tabular} 
	\caption{Boolean operations on experimental observations}
\end{table}

A somewhat surprising result is that the negation of an experimental observation is not always a experimental observation. For example, \emph{``there is extra-terrestrial life"} and \emph{``the mass of the electron is not exactly $9.109 \times 10^{-31}$ kg"} can be verified experimentally. For the first, we may send a probe to Mars and find bacteria; for the second, we may measure the mass to be within a range that excludes that number. The negation of the two statements,  \emph{``there is no extra-terrestrial life"} and \emph{``the mass of the electron is exactly $9.109 \times 10^{-31}$ kg"} are not for the reason we outlined before.

The ability to verify a statement, therefore, does not imply the ability to verify its negation. It implies, though, the ability to refute its negation. That is, if I verified that \emph{``there is extra-terrestrial life"} then I have refuted that \emph{``there is no extra-terrestrial life"}. This means that we can always re-frame a verification problem into a refutation problem and vice-versa. As such, we can only concentrate on verification and that is why our definition of experimental observation is solely based on that.

Conversely, note that not being able to verifying a statement does not mean it is refuted. After seeing a flying object in the sky, you may not be have been able to verify that \emph{``it is a duck"} because it was too far and you concluded that \emph{``it is a bird"}. Yet, it may still be a duck. In other words, failure to verify does not provide any information.

These subtleties are why we are framing the problem in terms of observations and not questions. A question such that \emph{``is there extra-terrestrial life?"} only admits ``Yes" and ``I don't know" as experimental answers. A question such that \emph{``is the mass of the electron exactly $9.109 \times 10^{-31}$ kg?"} only admits ``No" and ``I don't know" as experimental answers. A question such that \emph{``what is that animal?"} may have answers at different degree of precision (\emph{``it is an animal"}, \emph{``it is a bird"}, \emph{``it is a duck"}) that are compatible. Therefore, in this framework, we enumerate all the possible answers and keep track of the ones that can be verified experimentally.

Combining observations with conjunction (i.e. the logical AND) is more straightforward. To verify that \emph{``the sauce is sweet and sour"} we can first test that \emph{``the sauce is sweet"} and then that \emph{``the sauce is sour"}. If both are verified, then we have verified the conjunction. That is: if we have two or more experimental observations we can always verify the logical AND just by verifying each element one at a time. Yet, the number of observations needs to be finite or we would never end the verification process.

Combining observations with disjunction, i.e. the logical OR, is also straightforward. To verify that \emph{``the sauce is at least sweet or sour"} we can first test that \emph{``the sauce is sweet"}. If that is verified that's enough: the sauce is at least sour. If not, we test that \emph{``the sauce is sour"}. That is: if we have two or more experimental observations we can always verify the logical OR just by stopping at the first element that is verified. Because we stop at the first verification, the number of observations we combine in a logical OR can be countably infinite. As long as one verification succeeds, which will always be the case when the overall verification succeeds, it does not matter how many elements we are not going to verify later.

\begin{mathSection}

We formally capture these properties of experimental observations by defining the following logical operations. Taken as whole, they define the \textbf{algebra of experimental observations}. With each definition, we will prove that the resulting object is an experimental (counter-)observation. 

\begin{defn}
	The \textbf{negation or logical NOT} of an experimental observation $\mathsf{o}=\llparenthesis \mathsf{s}, \mathsf{e}\rrparenthesis$ is the experimental counter-observation $\neg \mathsf{o}=\llbracket\neg \mathsf{s}, \mathsf{e}\rrbracket$ where the negation of the statement $\mathsf{s}$ can be refuted by the experimental test $\mathsf{e}$.
\end{defn}

\begin{proof}
	To show that the negation exists and is well defined, let $\mathsf{o}=\llparenthesis \mathsf{s}, \mathsf{e}\rrparenthesis$ be an experimental observation. The experimental test $\mathsf{e}$ succeeds if and only if $\mathsf{s}$ is true or, equivalently, if and only if $\neg \mathsf{s}$ is false. So the tuple $\llbracket\neg \mathsf{s}, \mathsf{e}\rrbracket$ forms an experimental counter-observation.
\end{proof}

\begin{defn}
	The \textbf{conjunction or logical AND} of a finite collection  of experimental observations $\{\mathsf{o}_i\}_{i=1}^{n}=\{\llparenthesis \mathsf{s}_i, \mathsf{e}_i\rrparenthesis\}_{i=1}^{n}$ is the experimental observation $\bigwedge\limits_{i=1}^{n} \mathsf{o}_i = \llparenthesis \mathsf{s}, \mathsf{e}\rrparenthesis$ where $\mathsf{s} = \bigwedge\limits_{i=1}^{n} \mathsf{s}_i$ is the conjunction of the respective statements and $\mathsf{e} = \mathsf{e}_\wedge(\{\mathsf{e}_i\}_{i=1}^{n})$ is the experimental test that successfully terminates if and only if all $\{\mathsf{e}_i\}_{i=1}^{n}$ successfully terminate.
\end{defn}

\begin{proof}
	To show that the conjunction exists and is well defined, let $\mathsf{e}_\wedge(\{\mathsf{e}_i\}_{i=1}^{n})$ be the experimental procedure defined as follows:
	\begin{enumerate}
	\item for each $i=1..n$ run the test $\mathsf{e}_i$
	\item if all tests terminated successfully terminate successfully.
	\end{enumerate}
	The experimental procedure so defined terminates successfully if and only if all $\mathsf{e}_i$ terminate successfully. It will do so in finite time as each $\mathsf{e}_i$ succeeds in finite time and there only a finite number of them. Therefore $\mathsf{e}_\wedge(\{\mathsf{e}_i\}_{i=1}^{n})$ is an experimental test that is successful if and only if all statements $\{\mathsf{s}_i\}_{i=1}^{n}$ are successful. So $\bigwedge\limits_{i=1}^{n} \mathsf{o}_i = \llparenthesis\bigwedge\limits_{i=1}^{n} \mathsf{s}_i, \mathsf{e}_{\wedge}(\mathsf{e}_i)\rrparenthesis$ is an experimental observation.
\end{proof}

\begin{defn}
	The \textbf{disjunction or logical OR} of a countable (finite or infinite) collection of experimental observations $\{\mathsf{o}_i\}_{i=1}^{\infty}=\{\llparenthesis \mathsf{s}_i, \mathsf{e}_i\rrparenthesis\}_{i=1}^{\infty}$ is the experimental observation $\bigvee\limits_{i=1}^{\infty} \mathsf{o}_i = \llparenthesis \mathsf{s}, \mathsf{e}\rrparenthesis$ where $\mathsf{s} = \bigvee\limits_{i=1}^{\infty} \mathsf{s}_i$ is the disjunction of the respective statements and $\mathsf{e} = \mathsf{e}_\vee(\{\mathsf{e}_i\}_{i=1}^{\infty})$ is the experimental test that successfully terminates if and only if at least one experimental test in $\{\mathsf{e}_i\}_{i=1}^{\infty}$ successfully terminates.
\end{defn}

\begin{proof}
	To show that the disjunction exists and is well defined, let $\mathsf{e}_\vee(\{\mathsf{e}_i\}_{i=1}^{\infty})$ be the experimental procedure defined as follows:
	\begin{enumerate}
	\item initialize $n$ to 1
	\item for each $i=1..n$
	\begin{enumerate}
		\item run the test $\mathsf{e}_i$ for $n$ seconds
		\item if $\mathsf{e}_i$ terminated successfully, terminate successfully
	\end{enumerate}
	\item increment $n$ and go to step 2
	\end{enumerate}
	Suppose there exists an $i \in \mathbb{Z}^+$ such that $\mathsf{e}_i$ will terminate successfully, then the above procedure will eventually run that test for a time long enough and terminate successfully. It will do so in finite time as it will have run finitely many tests finitely many times each for a finite amount of time. Therefore $\mathsf{e}_\vee(\{\mathsf{e}_i\}_{i=1}^{\infty})$ is an experimental test that is successful if and only if at least one statement in $\{\mathsf{s}_i\}_{i=1}^{n}$ is successful. So $\bigvee\limits_{i=1}^{\infty} \mathsf{o}_i =\llparenthesis\bigvee\limits_{i=1}^{\infty} \mathsf{s}_i, \mathsf{e}_{\wedge}(\{\mathsf{e}_i\}_{i=1}^{\infty})\rrparenthesis$ is an experimental observation.
\end{proof}

\end{mathSection}

\section{Experimental domain}

\begin{mathSection}

\begin{defn}
	An \textbf{experimental domain} is a set of observations closed under finite conjunction and countable disjunction such that all observations can be tested in infinite time. 
\end{defn}

\begin{defn}
	A \textbf{sub-basis} of an experimental domain is any subset that can generate all others via finite conjunction and countable disjunction. A \textbf{basis} of an experimental domain is any subset that can generate all others by countable disjunctions. 
\end{defn}

Clearly, given a sub-basis one can generate a basis by taking all finite conjunctions of the observations. Note that any infinite sub-basis will generate a basis of the same cardinality. We claim that it is only possible to verify all members of an experimental domain with a countable (sub-)basis even given infinite time. 

\begin{prop}
Let $\mathcal{S}$ be an experimental domain. Then there exists a countable basis (equivalently, sub-basis) $\mathcal{B}$ of $\mathcal{S}$.
\end{prop}

\begin{proof}
If there exists a countable basis $\mathcal{B}$, then given infinite time one can test all observations in $\mathcal{B}$. Given all truth values for elements of the basis, one can deduce the truth values for the other observations in $\mathcal{S}$ (again using infinite time) by computing the appropriate conjunctions and disjunctions. 

If there does not exist a countable basis, then by definition there does not exist a sequence of experimental observations in $\mathcal{S}$ from which one can deduce all other observations in $\mathcal{S}$. Hence it is impossible to test all members of $\mathcal{S}$.
\end{proof}

\end{mathSection}

\section{Experimental identification}

The most common and important type of experimental observation is when we want to identify a specific case from a set of possible ones. For example, we pick ``this is a duck" from all possible animals or ``the position of the ball is $3 \pm 0.5$ m" from all possible values of position.

As we hinted before, observations do not necessarily identify a single element and they are not necessarily mutually exclusive. For example, ``it is an egg-laying animal" and ``it is a mammal" still allow for the platypus and the echidna. ``the position of the ball is $3 \pm 0.5$ m" and ``the momentum of the ball is $2 \pm 0.1$ kg m/s" limits the possible states to a rectangular region. An experimental observation that identifies an object, then, is not in general associated with a single element, but with a set of elements compatible with that observation. For example, the observation ``it is a bird" will be compatible with duck, penguin, hawk and so on while it will be incompatible with cat, moose, shrimp and so on.

Not only each experimental observation will be associated with a set, but two observations that share the same set are, for our purposes, equivalent. For example, ``it is a feathered animal" or ``it is a bird" will give us the same set of animals as all birds have feathers and all feathered animals are birds. The information they give us is the same. Therefore, we can treat an experimental observation as if it were equivalent to its associated set of elements. That is, if $U \subseteq X$ is the set of elements associated with the observation (e.g. all the birds), the observation may as well be ``the element is in U" (e.g. the element is within the set of birds).

But not all sets of elements are necessarily associated to an experimental observation. As we saw ``the position of the ball is precisely $1$ m" is not an observation because we can't measure position with infinite precision. Therefore the set of states for which position is exactly $1$ m is not associated with an experimental observation. The whole problem, then, is to be able to keep track of which set of elements is associated with an experimental observation and which is not.

At the very least, we need to be able to tell whether an object is or is not one of the elements in our set $X$. For example, we must be able to verify that ``it is an animal" or ``it is not an animal". In the same way, if we can't even verify whether something is a ball or not, it does not make sense to distinguish between the states of a ball. Therefore, to identify elements within a set $X$, there will be an experimental observation associated with the full set $X$, equivalent to ``that is an element of the set", and another experimental observation associated to the empty set $\emptyset$, equivalent to ``that is not an element of the set".

As we saw before, we can always take the conjunction (logical AND) and disjunction (logical OR) of two experimental observation. How do these operations transpose in terms of sets? Suppose we have $U \subseteq X$ (e.g. the set of flying animals) with an associated experimental observation (e.g. ``it is a flying animal"). Suppose we have $V \subseteq X$ (e.g. the set of all insects) with an associated experimental observation (e.g. ``that is an insect"). We combine the two observations with a logical AND (e.g. ``that is a flying insect"). Only the elements that are both in $U$ and $V$ are compatible with it. That is: only the elements in the intersection $U \cap V$. This means that if $U$ and $V$ are two sets each associated with an experimental observation, then their intersection $U \cap V$ is also associated with an experimental observation. Similarly, if $U$ and $V$ are two sets each associated with an experimental observation, we can combine these with a logical OR. All elements either in $U$ or $V$ are compatible with it. That is: elements in the union $U \cup V$. This means the union $U \cup V$ is also associated with an experimental observation.

To sum up, let $X$ be a set of elements and let $\mathsf{T}$ be the collection of the sets associated with some experimental observations we can use to identify among them. We have the following:
\begin{itemize}
	\item Both $X$ and $\emptyset$ can be found in $\mathsf{T}$ as we must be able to verify whether an object is or is not within the ones we can identify
	\item Any intersection of a finite number of sets in $\mathsf{T}$ can also be found in $\mathsf{T}$ as the finite conjunction (i.e. logical AND) of experimental observations is also an experimental observation
	\item Any union of a countable number of sets in $\mathsf{T}$ can also be found in $\mathsf{T}$ as the countable disjunction (i.e. logical OR) of experimental observations is also an experimental observation
\end{itemize}
This is the mathematical definition of a topology: a set of sets that contains both the whole set and the empty set, that is closed under finite intersection and under countable union. 

\begin{mathSection}

We will now use our notions of experimental observations and experimental domains to construct a new space with a topology which will more rigorously capture our notion of possibilities for an experimental domain. We will begin with a given experimental domain $\mathcal{S}$.

\begin{defn}
	A \textbf{possibility} for $\mathcal{S}$ is a subset of experimental observations compatible with each other but incompatible with the rest of the domain. That is $x \subseteq \mathcal{S}$ such that:
	\begin{enumerate}
		\item $\mathsf{o_1} \in x$ if and only if $\mathsf{o_1} \comp \mathsf{o}_2$ for all $\mathsf{o_2} \in x$ 
		\item $\mathsf{o_1} \notin x$ if and only if $\mathsf{o_1} \ncomp \mathsf{o}_2$ for some $\mathsf{o_2} \in x$ 
	\end{enumerate}
\end{defn}

This represents a maximal set of experimental observations that can be verified without contradiction.

\begin{prop}
	Let $x$ be a possibility for $\mathcal{S}$. If $\mathsf{o_1} \in x$ and $\mathsf{o_1} \imp \mathsf{o_2} \in \mathcal{S}$ then $\mathsf{o_2} \in x$. If $\mathsf{o_1}, \mathsf{o_2} \in x$ then $\mathsf{o_1} \cap \mathsf{o_2} \in x$. If $\mathsf{o_1} \cup \mathsf{o_2} \in x$ then $\mathsf{o_1} \in x$ and/or $\mathsf{o_2} \in x$.
\end{prop}
\begin{proof}
	TODO
\end{proof}

\begin{defn}
	The \textbf{set of possibilities} $X$ for an experimental domain is the collection of all possibilities. An experimental domain is \textbf{trivial} if it admits less than two possibilities (i.e. $|X| < 2$).
\end{defn}

\begin{defn}
	Let $\mathcal{S}$ be an experimental domain and $X$ its possibilities. For each experimental observation $\mathsf{o} \in \mathcal{S}$ we define its \textbf{compatible possibilities} $U(\mathsf{o}) \subseteq X$ as those that contain $\mathsf{o}$ as part of the answer. That is: $U(\mathsf{o}) = \{x \in X \, | \, \mathsf{o} \in x\}$.
\end{defn}

\begin{prop}
	Let $U : \mathcal{S} \rightarrow 2^X$ the function that maps an observation to its compatible possibilities. Let $\{\mathsf{o}_i\}_{i=1}^{n} \subseteq \mathcal{S}$ a finite collection of experimental observations. Then we have $U(\bigwedge\limits_{i=1}^{n} \mathsf{o}_i) = \bigcap\limits_{i=1}^{n} U (\mathsf{o}_i)$.
\end{prop}
\begin{proof}
	Let $\mathsf{o} = \bigwedge\limits_{i=1}^{n} \mathsf{o}_i$. Suppose $x \in U(\mathsf{o})$. Then $\mathsf{o} \in x$. Since $\mathsf{o} \imp \mathsf{o}_i$ for all $i$ then $\mathsf{o}_i \in x$ for all $i$. Then $x \in U(\mathsf{o}_i)$ for all $i$: $x \in\bigcap\limits_{i=1}^{n} U (\mathsf{o}_i)$. Conversely suppose $x \in\bigcap\limits_{i=1}^{n} U (\mathsf{o}_i)$. Then $\mathsf{o}_i \in x$ for all $i$. Then $\mathsf{o} = \bigwedge\limits_{i=1}^{n} \mathsf{o}_i \in x$: $x \in U(\bigwedge\limits_{i=1}^{n} \mathsf{o}_i)$.
\end{proof}

\begin{prop}
	Let $U : \mathcal{S} \rightarrow 2^X$ the function that maps an observation to its compatible possibilities. Let $\{\mathsf{o}_i\}_{i=1}^{\infty} \subseteq \mathcal{S}$ a countable collection of experimental observations. Then we have $U(\bigvee\limits_{i=1}^{\infty} \mathsf{o}_i) = \bigcup\limits_{i=1}^{\infty} U (\mathsf{o}_i)$.
\end{prop}
\begin{proof}
	Let $\mathsf{o} = \bigvee\limits_{i=1}^{\infty} \mathsf{o}_i$. Suppose $x \in U(\mathsf{o})$. Then $\mathsf{o} \in x$. Since $\mathsf{o} = \bigvee\limits_{i=1}^{\infty} \mathsf{o}_i$ for at least some $i$ we must have $\mathsf{o}_i \in x$. Then $x \in U(\mathsf{o}_i)$ for some $i$: $x \in\bigcup\limits_{i=1}^{\infty} U (\mathsf{o}_i)$. Conversely suppose $x \in\bigcup\limits_{i=1}^{\infty} U (\mathsf{o}_i)$. Then $x \in U(\mathsf{o}_i)$ and $\mathsf{o}_i \in x$ for some $i$.  Since $\mathsf{o}_i \imp \mathsf{o}$ we have $\mathsf{o} \in x$: $x \in U(\bigvee\limits_{i=1}^{\infty} \mathsf{o}_i)$.
\end{proof}

END NEW DEFINITIONS

\begin{defn}
A \textbf{generalized exact outcome} for $\mathcal{S}$ is a sequence of experimental observations $\mathcal{P} = \{\mathsf{o}_i\}_{i=1}^{\infty}\subset\mathcal{S}$ with the following property. For all experimental observations $\mathsf{o}\in\mathcal{S}$, there exists some positive integer $N$ such that for all integers $n>N$, one of the following must happen:
\begin{enumerate}
\item $\mathsf{o}_n \me \mathsf{o}$ 
\item $\mathsf{o}_n \implies \mathsf{o}$
\end{enumerate}
\end{defn}

\begin{defn}
An \textbf{exact outcome} for $\mathcal{S}$ is a generalized exact outcome $\mathcal{P} = \{\mathsf{o}_i\}_{i=1}^{\infty}$ such that for all $i\geq1$, we have $\mathsf{o}_{i+1}\implies\mathsf{o}_i$.
\end{defn}
Note that given a generalized exact outcome, one can build an exact outcome by simply replacing each $\mathsf{o}_n$ with the (finite) conjunction $\bigvee_{i=1}^n\mathsf{o}_i$. We will often denote the collection of (generalized) exact outcomes by $\mathcal{C}$. Next, we note that different sequences can result in ``equivalent" outcomes. 

\begin{defn}
We say two exact outcomes $\mathcal{P} = \{\mathsf{o}_i\}_{i=1}^{\infty}$ and $\mathcal{P}' = \{\mathsf{o}'_i\}_{j=1}^{\infty}$ are \textbf{equivalent}, denoted $\mathcal{P}\sim\mathcal{P}'$, if for all $i,j$, we have that $\mathsf{o}_i$ and $\mathsf{o}'_j$ are not mutually exclusive. 
\end{defn}

Note that given a generalized exact outcome, the exact outcome built from it will be equivalent. Importantly, this notion of equivalence is in fact an equivalence relation (i.e. a relation with reflexivity, symmetry, and transitivity):

\begin{prop}
The equivalence $\sim$ above is an equivalence relation.
\end{prop}
\begin{proof}
Let $\mathcal{P},\mathcal{P}',\mathcal{P}''$ be exact outcomes. Reflexivity and symmetry are immediate since not being mutually exclusive is reflexive and symmetric. 

For transitivity, suppose $\mathcal{P}\sim\mathcal{P}'$ and $\mathcal{P}'\sim\mathcal{P}''$, and pick some positive integers $m$ and $n$. We will show that $\mathsf{o}_m$ is not mutually exclusive to $\mathsf{o}''_n$. By hypothesis, for every $\mathsf{o}'_j\in\mathcal{P}'$, we have $\mathsf{o}_m$ and $\mathsf{o}'_j$ are not mutually exclusive and $\mathsf{o}'_j$ and $\mathsf{o}''_n$ are not mutually exclusive. Because $\mathcal{P}'$ is an exact outcome and neither $\mathsf{o}_m$ nor $\mathsf{o}''_n$ are ever mutually exclusive from any $\mathsf{o}'_j$, we may select an integer $J$ large enough so that $\mathsf{o}'_J\implies\mathsf{o}_m$ and $\mathsf{o}'_J\implies\mathsf{o}''_n$. Hence whenever $\mathsf{o}'_J$ is verified, both $\mathsf{o}_m$ and $\mathsf{o}''_n$ are verified, so they are not mutually exclusive. Because this holds for all positive integers $m,n$, we have that $\mathcal{P}_1\sim\mathcal{P}_3$. 
\end{proof}

Next, we will define our universe of discourse. 


\begin{defn}
	A \textbf{set of possibilities} $X$ for an experimental domain $\mathcal{S}$ is the collection $\mathcal{C}$ of all exact outcomes for $\mathcal{S}$, modulo the equivalence relation $\sim$ defined above. That is, $X=\mathcal{C}/\sim$. 
\end{defn}

Equivalently, a set of possibilities $X$ for $\mathcal{S}$ may be specified by choosing one exact outcome from each equivalence class. We will now provide a notion of containment for exact outcomes being ``within" an experimental observation. 

\begin{defn}
An exact outcome $\mathcal{P}=\{\mathsf{o}_i\}_{i=1}^{\infty}$ for experimental domain $\mathcal{S}$ is said to be an element of an experimental observation $\mathsf{o}\in\mathcal{S}$ if there exists some $N$ such that for all $n>N$, we have $\mathsf{o}_n\implies\mathsf{o}$. We denote this as $\mathcal{P}\in\mathsf{o}$. 
\end{defn}

In light of the above definition, we will refer to experimental observations either as observations or as sets. 

\begin{prop}
Let $\mathcal{P}=\{\mathsf{o}_i\}_{i=1}^{\infty}$ be an exact outcome which is an element of observation $\mathsf{o}$. Suppose $\mathcal{P}'$ is another exact outcome $\mathcal{P}'=\{\mathsf{o}'_j\}_{j=1}^{\infty}$ such that $\mathcal{P}\sim\mathcal{P}'$. Then $\mathcal{P}'\in\mathsf{o}$. 
\end{prop}
\begin{proof}
Suppose not - then we must be in case (1) of the definition of (generalized) exact outcome. Then there exists some $N_1$ such that for all $n>N_1$, we have $\mathsf{o}'_n\me\mathsf{o}$. By hypothesis, there exists some $N_2$ such that for all $m>N_2$, we have $\mathsf{o}_m\implies\mathsf{o}$. Let $N=\max(N_1,N_2)$. But then for all $k>N$, we have that $\mathsf{o}_k \nme \mathsf{o}'_k$. This contradicts $\mathsf{o}'_k\me\mathsf{o}$. 
\end{proof}

In particular, we immediately extend the definition of ``exact outcome containment" for elements of $\mathcal{S}$ to containment of equivalence classes of exact outcomes in $\mathcal{C}/\sim$. By the above, we can use containment of an individual exact outcome interchangeably with its equivalence class. In practice, most of our work will use individual exact outcomes representing an equivalence class, while the technical definitions will use their equivalence classes.

\begin{prop}
Let $\{\mathsf{o}_i\}_{i=1}^{n}$ be some collection of experimental observations. Then the conjunction $\bigwedge\limits_{i=1}^{n} \mathsf{o}_i = \llparenthesis\bigwedge\limits_{i=1}^{n} \mathsf{s}_i, \mathsf{e}_{\wedge}(\mathsf{e}_i)\rrparenthesis$ contains precisely the exact outcomes that are in all of the $\mathsf{o}_i$'s. 
\end{prop}
\begin{proof}
Let $\mathcal{P} = \{\mathsf{o}'_i\}_{i=1}^{\infty}$ be an exact outcome, and suppose $\mathcal{P}\in\mathsf{o}_i$ for $i=1,\ldots,n$. Then there exists $N$ large enough so that $\mathsf{o}'_m\implies\mathsf{o}_i$ for all $m>N$ and all $i=1,\ldots,n$, which is possible since $n$ is finite. It follows that $\mathsf{o}'_m\implies\bigwedge\limits_{i=1}^{n} \mathsf{o}_i$ for all $m>N$, so $\mathcal{P}\in\bigwedge\limits_{i=1}^{n} \mathsf{o}_i$, as desired. 

Next, let $\mathcal{P}\in\bigwedge\limits_{i=1}^{n} \mathsf{o}_i$. It follows that $\mathcal{P}\in\mathsf{o}_i$ for each $i$ immediately from the definitions. 
\end{proof}


\begin{prop}
Let $\{\mathsf{o}_i\}_{i=1}^{\infty}$ be some collection of experimental observations. Then the disjunction $\bigvee\limits_{i=1}^{n} \mathsf{o}_i = \llparenthesis\bigvee\limits_{i=1}^{n} \mathsf{s}_i, \mathsf{e}_{\vee}(\mathsf{e}_i)\rrparenthesis$ contains precisely the exact outcomes that are in any of the $\mathsf{o}_i$'s. 
\end{prop}
\begin{proof}
Let $\mathcal{P} = \{\mathsf{o}'_i\}_{i=1}^{\infty}$ be an exact outcome, and suppose $\mathcal{P}\in\mathsf{o}_k$ for some $k\geq1$. Then there exists $N$ large enough so that $\mathsf{o}'_m\implies\mathsf{o}_k$ for all $m>N$. It follows that $\mathsf{o}'_m\implies\bigvee\limits_{i=1}^{\infty} \mathsf{o}_i$ for all $m>N$, so $\mathcal{P}\in\bigvee\limits_{i=1}^{\infty} \mathsf{o}_i$, as desired. 

Next, let $\mathcal{P}\in\bigvee\limits_{i=1}^{\infty} \mathsf{o}_i$. Then there exists $N$ such that for all $m>N$, we have $\mathsf{o}'_m\implies \bigvee\limits_{i=1}^{\infty} \mathsf{o}_i$. Pick some such $m$, and so in particular, there exists $k$ such that $\mathsf{o}'_m\implies\mathsf{o}_k$. By definition of (non-generalized) exact outcome, this means $\mathcal{P}\in \mathsf{o}_k$, as required. 
\end{proof}


TODO: check this. the definition of experimental observation and their OR and AND are a little hard to use as needed here. 

TODO: maybe make a new notation for observation as a set of outcomes. 


TODO: a lot of the indices/notation might be a little hard to read/follow. 




We are now ready to prove one of the main results of this chapter. The following theorem links the philosophical definitions at the beginning of the chapter to the mathematical structure of a topological space.

TODO: maybe add a theorem environment for this one

\begin{prop}
Let $\mathcal{S}$ be an experimental domain with (by definition countable) basis $\mathcal{B}$, and let $X = \mathcal{C}/\sim$ be the set of possibilities. Then $(X,\mathcal{S})$ is a Hausdorff, second-countable topological space with points $X$ and open sets $\mathcal{S}$, where union and intersection are given by conjunction and disjunction of experimental observations, respectively. 
\end{prop}
\begin{proof}
We first will show that the collection of sets $\mathsf{o}\in\mathcal{S}$ do indeed form a topology on $X$. First, the experimental observation formed by the (countable) disjunction of all observations in $\mathcal{B}$ is an observation which contains all others; that is, this set contains all of $X$. The empty observation contains no elements of $X$, so serves as the empty set. 

Next, let $\mathsf{o}_1,\mathsf{o}_2\in\mathcal{S}$. Then by the previous proposition about conjunction, we have that the set of exact outcomes corresponding to $\mathsf{o}_1\wedge\mathsf{o}_2$ is precisely $\mathsf{o}_1\cap\mathsf{o}_2$. For a countable subset of $\mathcal{S}$, the same argument shows that their union (as sets of exact outcomes) matches the disjunction (as experimental observations). For arbitrary collections, because $\mathcal{B}$ is countable and generates all of the experimental observations, we can rewrite an arbitrary union of the corresponding subsets of exact outcomes as a countable disjunction of experimental observations, and we are reduced to the countable case. 

This proves that $(X,\mathcal{S})$ is a topology. Further, we can see that $\mathcal{B}$ is a countable basis for the topology (because conjunction and disjunction correspond to intersection and union), so it is second-countable. Next, we prove that $(X,\mathcal{S})$ is Hausdorff. 

Let $\mathcal{P} = \{\mathsf{o}_i\}_{i=1}^{\infty}$ and $\mathcal{P}' = \{\mathsf{o}'_i\}_{j=1}^{\infty}$ be two non-equivalent exact outcomes. Then there exist some $m,n$ such that $\mathsf{o}_m\me\mathsf{o}'_n$. But then $\mathcal{P}\in\mathsf{o}_m$ and $\mathcal{P}'\in\mathsf{o}'_n$, which are two disjoint sets in $\mathcal{S}$. This completes the proof. 
\end{proof}

\end{mathSection}


\section{Example of an Experimental Domain}

Consider the collection of statements, ``this object has a mass strictly between $x$ and $y$ kilograms" where $x$ and $y$ are any positive real numbers with $x<y$. Consider the experimental test where we put the object on a balance with an appropriate level of precision for each corresponding statement. Then each observation may be equated to an open interval of positive real numbers $(x,y)\subset\mathbb{R}$. Conjunction and disjunction of these observations will correspond exactly to intersection and union of these intervals, respectively. Further, the collection of intervals of the form $(p,q)$ for $p,q\in\mathbb{Q}$ forms a basis of observations. Next, one can show that any sequence of observations satisfying the conditions of exact outcome corresponds precisely to a real number. In particular, the set of possibilities $X$ for this experimental domain will be the set of positive real numbers with their usual topology. The reader with some experience with elementary real analysis will recognize that this construction is essentially the same as the Cauchy sequence construction of the real numbers, where representative Cauchy sequences come from endpoints of intervals, and the intervals used those corresponding to observations in a sequence of observations yielding an exact outcome. 

TODO: write this out much better. 

















TODO: I am not sure whether we should keep the rest of the section below and change the definitions to reflect the above. and, should we break the above proof into smaller pieces? is the above proof what we want??



\begin{defn}
	A \textbf{verifiable set} $U \subseteq X$ is a subset of possibilities for which there exists an experimental observation $\mathsf{o}\in\mathcal{S}$  = (\text{``The object is in } U \text{"}, $\mathsf{e}_\in(U))$ where $\mathsf{e}_\in(U)$ is an experimental test that succeeds only if the object to identify is an element of $U$.
\end{defn}

\begin{defn}
	A \textbf{refutable set} $U \subseteq X$ is a subset of possibilities for which there exists an experimental counter-observation $\mathsf{o} = (\text{``The object is in } U \text{"}, \mathsf{e}_{\notin}(U))$ where $\mathsf{e}_{\notin}(U)$ is an experimental test that succeeds only if the object to identify is not an element of $U$.
\end{defn}

\begin{prop}
	The complement $U^C$ of a verifiable set $U \subseteq X$ is a refutable set.
\end{prop}

\begin{proof}
	Let $U\subset X$ be a verifiable set. There exists an experimental observation $\mathsf{o} = (\text{``The object is in } U \text{"}, \mathsf{e}_\in(U))$. Consider $\neg \mathsf{o} = (\neg \text{``The object is in } U \text{"}, \mathsf{e}_\in(U)) = ($``The object  is not in $ U \text{"}, \mathsf{e}_{\notin}(U^C)) = ($``The object  is in $ U^C \text{"}, \mathsf{e}_{\notin}(U^C))$. That is, there exists a counter-observation $($``The object  is in $ U^C \text{"}, \mathsf{e}_{\notin}(U^C))$: $U^C$ is a refutable set.
\end{proof}

\begin{prop}
	Let $U_1, U_2, ... , U_n, ...$ be a countable infinite sequence of verifiable sets. The finite intersection $\bigcap\limits_{i=1}^{n} U_i$ and the countable union  $\bigcup\limits_{i=1}^{\infty} U_i$ are verifiable sets.
\end{prop}

\begin{proof}
	We claim the finite intersection of verifiable sets is a verifiable set. Let $U_1, U_2, ... , U_n \subseteq X$ be n verifiable sets. For each $U_i$ there exists an experimental observation $\mathsf{o}_i = ($``The object is in $ U_i \text{"}, \mathsf{e}_\in(U_i))$. Consider $\mathsf{o} = \bigwedge\limits_{i=1}^{n} \mathsf{o}_i = (\bigwedge\limits_{i=1}^{n} $``The object is in $ U_i \text{"}, \mathsf{e}_{\wedge}(\mathsf{e}_\in(U_i)))=( $``The object is in $ \bigcap\limits_{i=1}^{n} U_i \text{"}, \mathsf{e}_\in(\bigcap\limits_{i=1}^{n} U_i))$ is an experimental observation. $\bigcap\limits_{i=1}^{n} U_i$ is a verifiable set.
	
	We claim the countable union of verifiable sets is a verifiable set. Let $U_1, U_2, ... , U_n, ... \subseteq X$ be an infinite sequence of verifiable sets. For each $U_i$ there exists an experimental observation $\mathsf{o}_i = ($``The object is in $ U_i \text{"}, \mathsf{e}_\in(U_i))$. Consider $\mathsf{o} = \bigvee\limits_{i=1}^{\infty} \mathsf{o}_i = (\bigvee\limits_{i=1}^{\infty} $``The object is in $ U_i \text{"}, \mathsf{e}_{\vee}(\mathsf{e}_\in(U_i)))=( $``The object is in $ \bigcup\limits_{i=1}^{\infty} U_i \text{"}, \mathsf{e}_\in(\bigcup\limits_{i=1}^{\infty} U_i))$ is an experimental observation. $\bigcup\limits_{i=1}^{\infty} U_i$ is a verifiable set.
\end{proof}




\section{Experimental distinguishability}

What we have done so far is to assign to a set of elements $X$ a collection of experimental observations $\mathsf{T}$. This is not enough. For example, the observations ``it is an animal" and ``it is not an animal" form a collection closed under logical operators (i.e. it forms the trivial topology) but it is hardly useful.

We call $X$ a set of experimentally distinguishable elements if we can associate with it a set of experimental observations rich enough to be able to distinguish each element from the other. That is, given two possible elements (e.g. two possible animals or two possible positions) we can always find an experimental observation that can tell them apart. This property is a basic requirement to treat these elements in a scientific context: if we are not even able to tell them apart experimentally, they cannot be the subject of scientific investigation.

If two elements are experimentally distinguishable from each other there we must have an experimental observation that can tell them apart. For example, to be able to define the house sparrow (Passer domesticus) as a separate animal from the Italian sparrow (Passer italiae) we have at least two observations (``it has a dark gray crown", ``it has a chestnut crown") that are mutually exclusive, each compatible with just one element.

That is, for each pair of elements $x_1, x_2 \in X$, we must be able to find two sets $U,V \in \mathsf{T}$ each associated with an experimental observation such that no element is in both (i.e. their intersection $U \cap V=\emptyset$ is empty) and one element is in one set (i.e. $x_1 \in U$ and $x_2 \in V$). This is the mathematical definition of a Hausdorff space: one in which distinct elements have disjoint neighborhoods. 

This solves one problem, there exists a way to distinguish any two elements, but it does not solve the other problem: we must be able to actually find it. As we saw before, we can only test a finite amount of experimental observations. So we must make sure we can distinguish any two elements in a finite number of steps.

Note that, once we have identified an element, we are able to tell which experimental observations are verified: the ones associated with a set that includes the element. For example, once we have identified the house sparrow, we have verified ``it has a gray crown", ``it is a bird", ``it has wings" and so on. Conversely, once we know all the experimental observations that can be verified, we have identified the element: the only one that is included in all sets associated with verified observations. That is, once we have verified that ``it has a gray crown", ``it is a bird", ``it has wings" and so on, we have identified the house sparrow. In other words: identifying an element is equivalent to being able to verify all possible observations.

Fortunately we don't have to test all possible observations: just enough to be able to calculate all other cases. For example, we can simply test each animal one at a time (i.e. ``it is a cat", ``it is a dog", ``it is a duck" and so on) until we find the correct one. Once we verify one, we will know whether ``it is a bird" or ``it is a mammal". We define sub-base a set of experimental observations from which all others can be obtained through conjunction (i.e. logical AND) and disjunction (i.e. logical OR). In particular, we consider the smallest possible sub-base. This gives us the smallest number of experimental observation to test in order to distinguish an element from all others.

If the smallest sub-base is finite, it is possible to test all experimental observations. This is the case of the set of animals: there are a large but finite number of species. If it's infinite, we cannot. But if the smallest sub-base is countably infinite, we can get as close as we want. This is the case for the position of a ball: we can in principle always increase the precision. Suppose, in fact, that we have a large but finite number of elements $n$ among which we want to distinguish. We need a set of $n$ disjoint sets, each including one element. Note that the union will not make two overlapping sets disjoint, so these sets will only be intersections of elements of the sub-base. Intersections can only be finite, therefore we will need only a finite number of members of the sub-basis. If the sub-basis is countable, at some point we will get to the last member needed and we can stop: we can distinguish between the $n$ elements. If the sub-basis is not countable, instead, we will never stop: we can't experimentally distinguish between a set of arbitrary elements in a finite amount of time.

If there is a countable sub-base, then there is a countable base as well. This is the definition of a second countable space: one that allows a countable bases for its topology.

\begin{defn}
	A set of experimental possibilities $X$ is \textbf{experimentally distinguishable} if the collection of all possible experimental identifications form an experimental domain and if given two arbitrary possibilities $x_1, x_2 \in X$ we can always find two mutually exclusive experimental observation such that each possibility is compatible with one observation.
\end{defn}


\begin{prop}
	A set $X$ of experimentally distinguishable  possibilities is a Hausdorff and second countable topological space with the topology $\mathsf{T}(X)$ formed by all distinguishable sets.
\end{prop}

\section{Connections between topology and experimental distinguishability}

Now that we have made a tight link between topology and experimental distinguishability, we can go through some of the mathematical vocabulary and give it a more precise physical meaning.

The topology defined on a set $X$, which will note as $\mathsf{T}(X)$, describes the collection of all possible experimental observations we can perform to identify an element of the set. Each of these observation can be identified by a subset $U \subseteq X$ which represents all the elements that are compatible with the observation.

An open set $U \in \mathsf{T}(X)$, a set in the topology, is a set for which there exist a way to  experimentally verify that an element is in that set. Conversely, a closed set $V = X \setminus U $, a set whose complement is in the topology, is a set for which there exists a way to experimentally refute that an element is in that set. For example, in the standard topology for $\mathbb{R}$ the interval $(2.5, 3.5)$ is an open set because $3 \pm 0.5$ is a valid measurement for a continuous quantity while $\{3\}$ is a closed set because, while we can't measure a real number with infinite precision, we can exclude it.

The discrete topology, the one for which each singleton $\{x\}$ (i.e. set of one element $x \in X$) is both open and closed, corresponds to the ability to verify and refute each element individually. Any finite set\footnote{TODO what about countable?} that is Hausdorff and second countable has a discrete topology. This is probably why it is not as intuitive to think that something verifiable may not be refutable: it only happens in infinite sets.

The standard topology for $\mathbb{R}$, the one generate by all open intervals, corresponds to the ability of measuring continuous value only with finite precision. This topology is Hausdorff and second countable. While we can give a discrete topology to $\mathbb{R}$, this would no longer be second countable so it would violate our requirement for experimental distinguishability.

An interesting mathematical result is the following.

\begin{prop}
	A Hausdorff and second countable space $X$ has at most cardinality of continuum.
\end{prop}

\begin{proof}
	We define an injective function $F:X\to2^{\mathbb{N}}$, where $2^{\mathbb{N}}$ denotes all infinite binary sequences.
	
	Since $X$ is second-countable, we can enumerate the countable basis $\mathcal{B}$ as $\mathcal{B} = \{B_i\}_{i=1}^{\infty}$. Let $2^{\mathbb{N}}$ denote all infinite binary sequences. We define $F:X\to2^{\mathbb{N}}$ such that $F(x) = (F(x)_i)_{i=1}^{\infty}$ is the sequence where each element is given by: 
	$$
	F(x)_i = 
	\begin{cases}
	1 & x\in B_i \\
	0 & x\notin B_i
	\end{cases}
	$$
	This is an injective function. Suppose $x_1 \neq x_2$, then since $X$ is Hausdorff there is at least one element of the basis that contains one but not the other.\footnote{In fact, $T_0$ separability would be sufficient.} Therefore $F(x_1) \neq F(x_2)$. As $F$ injects $X$ into $2^{\mathbb{N}}$, we have $|X| \leq |2^{\mathbb{N}}|=|\mathbb{R}|$. $X$ has at most cardinality of continuum.
\end{proof}

This means that, no matter what technique we use now or in the future, the collections of elements that we can properly define experimentally are at most infinite like the continuum. This already gives us a first simple and basic requirement a set of mathematical objects need to pass to be of scientific interest.

For example, these objects have cardinality of continuum, and therefore are good candidates:
\begin{itemize}
	\item Euclidean space $\mathbb{R}^n$
	\item all continuous functions from $\mathbb{R}$ to $\mathbb{R}$
	\item all open sets in $\mathbb{R}^n$
	\item all subsets of $\mathbb{N}$
\end{itemize}

These, instead, have cardinality greater then continuum, and therefore are not good candidates:
\begin{itemize}
	\item all functions from $\mathbb{R}$ to $\mathbb{R}$
	\item all subsets of $\mathbb{R}$
\end{itemize}


\section{Continuous functions}

We now turn our attention to relationships between two sets $X$ and $Y$ of experimentally distinguishable elements. There are two ways of characterize them and we want to show that they are equivalent.

Suppose we have two sets $X$ and $Y$ of experimentally distinguishable elements (e.g. the temperature and height of a mercury column in a thermometer). Let's assume we have a causal relationship $f: X \rightarrow Y$ between the first and the second (e.g. the temperature determines how high is the mercury column). We assume the relationship is valid on the whole set without loss of generality: if it is not, just redefine $X$ and $Y$ to be the appropriate regions (e.g. the valid ranges of temperature and height of a mercury column in which the thermometer can operate).

The relationship $f$ can also be used to infer observations on $X$ from observations of $Y$. Suppose we verify that $y$ is within a set $V \in \mathsf{T}(Y)$ (e.g. ``the height of the mercury column is between 24 and 25 millimeters"). Then we can infer that $x$ is within the reverse image $U=f^{-1}(V)$ (e.g. ``the temperature is 24 and 25 degrees Celsius"). $U$ is therefore associated with an experimental observation: $U \in \mathsf{T}(X)$ must be a set in the topology (e.g. if we measure the height of the column with finite precision, we cannot end up inferring the value of temperature with infinite precision).

In other words, to each causal relationship $f: X \rightarrow Y$ between two set of experimentally distinguishable elements (e.g. if the temperature is $x$ the height of the mercury column will be $f(x)$) we have an associated reverse inference relationship $g  : \mathsf{T}(Y) \rightarrow \mathsf{T}(X) = f^{-1}$ (e.g. if the height of the mercury column is within $V$ then the temperature is within $f^{-1}(V)$ ). This is true even if the function is not monotonic (e.g. if ``the water density is between 999.8 and 999.9 kg/$m^3$" then ``the water temperature is between 0 and 0.52 or between 7.6 and 9.12 degrees Celsius" as water is most dense at 4 degrees Celsius).  The reverse image $U=f^{-1}(V)$ of a set associated with an experimental observation (i.e. an open set) is also a set associated with an experimental observation (i.e. an open set). This is the definition of a continuous function in topology: one for which the reverse image of an open set is an open set. Note that, when using the standard topology on $\mathbb{R}^n$, topological continuity is equivalent to analytical continuity.

Now suppose we start with the inference relationship $g  : \mathsf{T}(Y) \rightarrow \mathsf{T}(X)$ that for each verified experimental observation on $Y$ gives us a verified experimental observation on $X$ (e.g. if ``the height of the mercury column is within $V$" then ``the temperature is within $g(V)$"). For it to be a valid inference relationship, it will have to be compatible with logical operations (e.g. if ``the height of the mercury column is between the 23 and 25 millimeters" and ``the height of the mercury column is between the 24 and 26 millimeters" then ``the temperature is between 23 and 25 degree Celsius" and  ``the temperature is between 24 and 26 degree Celsius"). This means the relationship must also be compatible with the set operations that correspond to the logical operations: $g(V_1 \cup V_2)=g(V_1)\cup g(V_2)$ and $g(V_1 \cap V_2)=g(V_1)\cap g(V_2)$.

If nothing is known about $y$ (e.g. if ``the height of the mercury column is in a valid range") then nothing should be known about $x$ (e.g. ``the temperature is in a valid range"): $g(Y)=X$. If we excluded all values in $Y$ (e.g. ``the height of the mercury column is not in a valid range of the thermometer") then we excluded all values in $X$ (e.g. ``the temperature is not in a valid range of the thermometer"): $g(\emptyset) = \emptyset$. That is, for the inference relationship to be valid we shouldn't be able to infer something from nothing or nothing from something.

Under these conditions, one can show that for any such relationship there exists a continuous function $f: X \rightarrow Y$ such that $g=f^{-1}$ is its inverse image.

\begin{prop}
	\label{setfunctions}
	Let $X$ and $Y$ be two Hausdorff topological spaces, and let $g: \mathsf{T}(Y) \rightarrow \mathsf{T}(X)$ be a mapping such that:
	\begin{enumerate}
		\item It is compatible with union and intersection $\forall V_1, V_2 \in Y$ $g(V_1 \cup V_2)=g(V_1)\cup g(V_2)$ and $g(V_1 \cap V_2)=g(V_1)\cap g(V_2)$
		\item $g(\emptyset) = \emptyset$
		\item $g(Y) = X$
	\end{enumerate}
	Then there exists a unique continuous function $f: X \rightarrow Y$ such that $g = f^{-1} |_{\mathsf{T}(Y)}$.
\end{prop}

\begin{proof}
	We claim there exists a unique extension $\bar{g}:\sigma(Y)\to\sigma(X)$ to the Borel $\sigma$-algebras of $X$ and $Y$, respectively $\sigma(X)$ and $\sigma(Y)$, such that $\bar{g}|_{\mathsf{T}(Y)}=g$ and $\bar{g}$ is compatible with union, intersection and complements. Let $\bar{g}(V) = g(V)$ for all open sets $V \in \mathsf{T}(Y)$. Let $A \in \sigma(Y)$ (not necessarily open) and $A^C$ be its complement. We must have $\bar{g}(A^C) = \bar{g}(A)^C = X\setminus \bar{g}(A)$ for $\bar{g}$ to be compatible with complements. Recall that all Borel sets in $\sigma(Y)$ and $\sigma(X)$ may be written as some combination of unions, intersections, and complements of open sets. Thus, the construction uniquely determine what $\bar{g}$ should output on any Borel set. We need only check that the output is still a Borel set. But by definition of $\bar{g}$, the outputs will be given as unions, intersections, and complements of outputs of $g$, which are open sets, and so the image of $\bar{g}$ is contained in $\sigma(X)$.  $\bar{g}$ is well defined. The function $\bar{g}$ in a sense represents extracting the maximum amount of information possible out of $g$.
	
	We claim we can define $\hat{g}:Y\to\sigma(X)$ such that $\hat{g}(y) = \bar{g}(\{y\})$. Since $Y$ is Hausdorff, every singleton $\{y\}$ is closed and is therefore a Borel set. $\bar{g}(\{y\})$ is well defined and so is $\hat{g}(y)$.
	
	We claim  $\hat{g}(y_1)\cap\hat{g}(y_2) = \emptyset$ if and only if $y_1\neq y_2$ for all $y_1,y_2\in Y$ such that $\hat{g}(y_i)\neq\emptyset$ for $i=1,2$. If $y_1\neq y_2$ we have
	$$
	\hat{g}(y_1)\cap\hat{g}(y_2) = \bar{g}(\{y_1\})\cap\bar{g}(\{y_2\}) = \bar{g}(\{y_1\}\cap\{y_2\}) = \bar{g}(\emptyset) = \emptyset.
	$$
	Conversely, if $y_1 = y_2$ we have
	$$
	\hat{g}(y_1)\cap\hat{g}(y_2) = 	\hat{g}(y_1)\cap\hat{g}(y_1) = 
	\hat{g}(y_1) \neq \emptyset.
	$$
	
	We claim we can define $f: X\to Y$ such that $f(x) = y$ if and only if $x\in \hat{g}(y)$. Since $g(Y)=X$, there exists $y\in Y$ such that $x\in\hat{g}(y)$. By the preceding claim, this $y$ is unique. $f: X\to Y$ is well defined. Note that no arbitrary choice where made so far that lead to the construction of $f$, which is therefore determined uniquely by $g$. 
	
	We claim $g = f^{-1} |_{\mathsf{T}(Y)}$. Let $V\in\mathsf{T}(Y)$. We want to show $f^{-1}(V) = g(V)$. Let $x\in f^{-1}(V)$. Then for some $y \in V$ we have $f(x)=y$. $x\in \hat{g}(y)$ by construction of $f$. $\hat{g}(y) \subset g(V)$ since $\{y\}\subset V$, so $x\in g(V)$. $f^{-1}(V) \subseteq g(V)$. Conversely, let $x\in g(V)=\bar{g}(V)$. Then for some $y\in V$, we have $x\in\bar{g}(\{y\})\subset\bar{g}(V)$. But then by definition we have $f(x)=y$, so $x\in f^{-1}(V)$. $f^{-1}(V) \supseteq g(V)$. $f^{-1}(V) = g(V)$ for all $V\in\mathsf{T}(Y)$ and therefore $f^{-1}|_{\mathsf{T}(Y)}=g$.
	
	We claim $f$ is continuous. It is so since $g = f^{-1} |_{\mathsf{T}(Y)}$ takes open sets to open sets. 
\end{proof}

The above work gives us an approach to reconstruct continuous functions given the behavior of the inverse on open sets. In the context of collecting information on observable phenomena, the function $g$ represents the total of all information possible to gather on the correlation between two variables. The fact that from $g$ we can construct a unique continuous $f$ shows us that in the infinite-resource ideal, we fully obtain the information to give exact relations between variables. Further, the result itself tells us that the interesting phenomena in our framework of experimental observations will be continuous, which is often a baseline assumption in physics. 





TODO: Introduce/define Borel algebra? In fact, we really only need infinite intersections of open sets. We could skip complements all together and just let $\bar{g}$ be defined on all open sets plus all infinite intersections of open sets. This might be slightly simpler but won't make much of a difference either way. Also, it might make more sense to absorb Lemma 1.24 and $\hat{g}$ into the proof of Proposition 1.22. 

\section{Distinguishability of functions}

For all of this to be self consistent, we must require that functions between experimentally distinguishable elements are  experimentally distinguishable themselves. That is, we must show that the set of continuous function $C(X,Y)$ from $X$ to $Y$ is a Hausdorff and second countable topological space with a suitable topology. Note that $C(X,Y)$ has the cardinality of continuum, therefore we already know it allows Hausdorff and second countable topologies, but we need to make sure we can give one that is actually physically meaningful in terms of experimental observations.

Suppose we have two sets $X$ and $Y$ of experimentally distinguishable elements (e.g. time and space). Suppose we need to distinguish elements within the set of all continuous function $C(X,Y)$ from $X$ to $Y$ (e.g. all possible trajectories). From a sub-base of $X$ (e.g. all open time intervals between rational numbers) we pick a set $U$ (e.g. between 1 and 2 seconds). From a sub-base of $Y$ (e.g. all open spatial intervals between rational numbers) we pick a set $V$ (e.g. between 1 and 2 meters). We can define the set $S(U,V) = \{f: X \rightarrow Y | f \in C(X,Y), f(U) \subset V\}$ of all the continuous functions such that their value remains within $V$ over the domain $U$ (e.g. all trajectories that between $1$ and $2$ seconds remain within $1$ and $2$ meters).

If we take all possible sets of functions constructed this way, and combine them with finite intersections and countable unions, we obtain a topology for the continuous function. The collections of all sets $S(U,V)$ forms a sub-base for this topology and is constructed from the sub-bases of $X$ and $Y$. This new sub-base is countable because the choices for $U$ and $V$ are countable themselves. This means that $C(X,Y)$ is second countable with this topology.

Now we need to show that we can distinguish between any two functions. Say we have two different functions $f_1$ and $f_2$ (i.e. two different trajectories). Since they are different, there will be at least one value $x \in X$ such that the values $f_1(x)\neq f_2(x)$ will be different (e.g. at $1.2$ seconds the first trajectory is at $1.1$ meters while the second trajectory is at $1.2$ meters). We can now find two disjoint elements $V_1$ and $V_2$ of the sub-base for $Y$ such that each includes either $f_1(x)$ or $f_2(x)$ (e.g. the intervals $(1.095, 1.105)$ and $(1.195, 1.205)$ meters). Since the functions are continuous, we can find an element $U$ of the sub-base for $X$ such that both functions remain within those ranges (e.g. between $(1.197, 1.203)$ seconds the first trajectory remains within $(1.095, 1.105)$ meters and the second trajectory remains within $(1.195, 1.205)$ meters). The sets $S(U, V_1)$ (e.g. all trajectories that between $(1.197, 1.203)$ seconds that remain within $(1.095, 1.105)$ meters) and $S(U, V_2)$ (e.g. all trajectories that between $(1.197, 1.203)$ seconds that remain within $(1.195, 1.205)$ meters) are disjoint and they each contain $f_1$ and $f_2$ respectively. We can distinguish between functions: the topology is Hausdorff.

This confirms that all the conceptual infrastructure is solid and self consistent. Sets of experimentally distinguishable elements are Hausdorff and second countable topological spaces. Maps between elements of such spaces are continuous maps as they preserve experimental distinguishability. The maps themselves are experimentally distinguishable and the can be given a Hausdorff and second countable topology. We can continues this recursively by constructing maps of maps (e.g. a map from a state to a trajectory) without ever going outside our definitions: everything remains experimental distinguishable.

\begin{mathSection}

For continuous functions to be physically distinguishable, we need to show they can always be given, as a set, given a topology that is Hausdorff and second-countable.

To that end, we introduce the basis-to-basis topology on the set of continuous functions from two topological spaces. This is the topology generated by the sets of functions that map a basis element of one element inside an element of the other space. 

\begin{defn} Let $X$ and $Y$ be two topological spaces. Let $C(X,Y)$ denote the set of all continuous functions from $X$ to $Y$. Let $\mathcal{B}_X$ and $\mathcal{B}_Y$ be two bases for $X$ and $Y$ respectively. The basis-to-basis topology $\mathsf{T}(C(X,Y), \mathcal{B}_X, \mathcal{B}_Y)$ on $C(X,Y)$ with respect to the basis $\mathcal{B}_X$ and $\mathcal{B}_Y$ is the topology generated by all sets of the form 
	$$
	V(U_X, U_Y) = \{f\in C(X,Y) : f(U_X)\subset U_Y\}
	$$
where $U_X \in \mathcal{B}_X$ and $U_Y \in \mathcal{B}_Y$.
\end{defn}

Now we need to show that if the two spaces $X$ and $Y$ are Hausdorff and second-countable, the basis-to-basis topology is Hausdorff and second-countable for a suitable choice of basis.

\begin{prop}
	Let $X$ and $Y$ be two Hausdorff and second-countable topological spaces. Let $C(X,Y)$ denote the set of all continuous functions from $X$ to $Y$. Let $\mathcal{B}_X$ and $\mathcal{B}_Y$ be two countable bases for $X$ and $Y$ respectively. The basis-to-basis topology $\mathsf{T}(C(X,Y), \mathcal{B}_X, \mathcal{B}_Y)$ on $C(X,Y)$ with respect to the basis $\mathcal{B}_X$ and $\mathcal{B}_Y$ is Hausdorff and second-countable. 
\end{prop}
\begin{proof}
	We claim $\mathsf{T}(C(X,Y), \mathcal{B}_X, \mathcal{B}_Y)$ is second-countable. We first note that the sub-basis $\{V(U_X, U_Y) \, |\,   U_X \in \mathcal{B}_X , U_Y \in \mathcal{B}_Y \}$ is countable since $\mathcal{B}_X$ and $\mathcal{B}_Y$ are countable. The collection $\mathcal{B}$ of all finite intersections is still countable, since it is in one-to-one correspondence with the collection of finite subsets of a countable set, which is still countable. Therefore $\mathcal{B}$ is a countable basis, which means $\mathsf{T}(C(X,Y), \mathcal{B}_X, \mathcal{B}_Y)$ is second-countable.
	
	We claim $\mathsf{T}(C(X,Y), \mathcal{B}_X, \mathcal{B}_Y)$ is Hausdorff. Let $f,g:X\to Y$ be distinct continuous functions. Then for some $x\in X$, we have $f(x)\neq g(x)$. Pick $V_1, V_2$ disjoint open subsets of $Y$ with $f(x)\in V_1$ and $g(x)\in V_2$. We may assume (possibly by shrinking $V_1$ or $V_2$) that both are basis elements for the topology of $Y$. Let $U=f^{-1}(V_1)\cap g^{-1}(V_2)$. Then $U$ is an open neighborhood of $x$. We may assume again that $U$ is a basis element for the topology on $X$ by shrinking it if necessary. Now, let $T_1$ to be the (sub-)basis element for basis-to-basis topology corresponding to $U$ and $V_1$. By construction, $f\in T_1$. Similarly, let $T_2$ to be the basis element for the basis-to-basis topology corresponding to $U$ and $V_2$ and containing $g$. Since $V_1$ and $V_2$ are disjoint, so are $T_1$ and $T_2$. $\mathsf{T}(\mathcal{C})$ is Hausdorff.
\end{proof}

We will also show that this is not in general equal to the open-open topology. 

\begin{prop}
	In general, the topology $\mathsf{T}(\mathcal{C})$ defined above depends on the bases for $X$ and $Y$. 
\end{prop}
\begin{proof}
	We will give an example of such a case. Let $X=Y=\mathbb{R}$ with the usual topology on $\mathbb{R}$. Let $\mathcal{B}_1$ be the basis for $\mathbb{R}$ consisting of all open intervals with rational endpoints, and let $\mathcal{B}_2 = \mathcal{B}_1\cup\{(0,\pi)\}$. We find a set open in our topology $\mathsf{T}(\mathcal{C})$ using $\mathcal{B}_2$ as our basis which is not open when we use $\mathcal{B}_1$. Consider the following set, open in our topology when using $\mathcal{B}_2$:
	$$
	V = \{f\in\mathcal{C}| f((0,\pi))\subset(0,1)\}
	$$
	This is clearly open because $(0,\pi)$ and $(0,1)$ are basis elements in $\mathcal{B}_2$. 
	
	Next, we show that when using $\mathcal{B}_1$ as our basis for $X=Y=\mathbb{R}$, no finite intersection and/or infinite union of sub-basis elements for $\mathsf{T}(\mathcal{C})$ will equal $V$. Henceforth, when we say ``open" subsets, unless otherwise stated, we will refer exclusively to the topology $\mathsf{T}(\mathcal{C})$ when using $\mathcal{B}_1$ for $\mathbb{R}$. The smallest sub-basis subsets containing $V$ are those of the form:
	$$
	B_r = \{f\in\mathcal{C} | f((0,r))\subset(0,1)\}
	$$
	for each $r\in\mathbb{Q}$ with $r<\pi$. By ``smallest" we mean any sub-basis elements containing $V$ which are not in the form of $B_r$ will contain some $B_r$. Now, any finite intersection of the $B_r$'s will be another $B_r$, but these strictly contain $V$, so we can never obtain $V$ from this sub-basis. 
	
	Similarly, the largest sub-basis sets contained within $V$ are of the form: 
	$$
	S_r = \{f\in\mathcal{C} | f((0,r))\subset(0,1)\}
	$$
	for each $r\in \mathbb{Q}$ with $r>\pi$. Let $S = \cup_{r>\pi}S_r$. This is the largest open set in our topology which is contained in $V$. Consider the function $f(x) = x/\pi$. This is continuous and is an element of $V$. But notice $f(\pi)=1$, so $f\notin S_r$ for all $r>\pi$, hence $f\notin S$. Thus, $V$ is a set which is open in $\mathsf{T}(\mathcal{C})$ when we use $\mathcal{B}_1$ for $X$ and $Y$, but not open when we use $\mathcal{B}_2$ for $X$ and $Y$. 
\end{proof}

TODO: add corollary environment for this one. 

\begin{prop}
The topology $\mathsf{T}(\mathcal{C})$ on the set of continuous functions $C(\mathbb{R},\mathbb{R})$ from $\mathbb{R}$ to itself is in general not equal to the open-open topology on $C(\mathbb{R},\mathbb{R})$.
\end{prop}
\begin{proof}
In the proof of the previous proposition, note that the basis for the open-open topology is independent of the basis used for the underlying spaces, and in particular is a superset of the basis for $\mathsf{T}(\mathcal{C})$ when using $\mathbb{B}_2$ to construct it. Hence the open-open topology is distinct from $\mathsf{T}(\mathcal{C})$ when using the basis $\mathcal{B}_1$ on $\mathbb{R}$. 
\end{proof}

TODO: the above propositions might generalize to any uncountable, second-countable space. This will be considered during the paper-writing process later. 

Thus we have shown one can start with two Hausdorff, second-countable spaces, and generate a new Hausdorff second-countable space consisting of all functions between them. Topological spaces with these properties make up our class of scientifically interesting spaces, so the space of continuous (distinguishability-preserving) functions between spaces of distinguishable quantities is again a physically distinguishable space. Because one can iterate this construction (since the relevant properties are preserved), this is a way to generate arbitrarily many new physically distinguishable spaces encoding important information about the ``lower-order" spaces (i.e. they consist of functions which encode a conversion between verifiable sets in different spaces).

\end{mathSection}

\section{Summary}

\begin{table}[h]
	\centering
\begin{tabular}{p{0.20\textwidth} p{0.7\textwidth}}
	Math/Topology & Science/Physics \\ 
	\hline 
	Hausdorff, second-countable topological space & Experimentally distinguishable space, whose points are the possible values and whose open sets represent the experimentally attainable levels of precision \\
	Open set & Verifiable set. We can verify experimentally that an element is within the set  \\ 
	Closed set & Refutable set. We can verify experimentally that an element is not in the set \\ 
	Basis of a topology & A collection of verifiable sets such that any verifiable set is determined by the basis sets\\
	Continuous \newline function &  A function between two sets of experimental distinguishable elements that preserves distinguishability \\
	Homeomorphism &  A perfect equivalence between experimentally distinguishable spaces. \\
\end{tabular} 
\caption{Topology to physics dictionary}
\end{table}

	
\end{document}