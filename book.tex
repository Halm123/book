% Possible use http://www.latextemplates.com/template/the-legrand-orange-book as template? See https://www.overleaf.com/9174958nyjxxdxbchks#/33024595/

\documentclass[11pt,letterpaper,fleqn]{memoir} % Default font size and left-justified equations

\usepackage[top=3cm,bottom=3cm,left=3cm,right=3cm,headsep=10pt,letterpaper]{geometry} % Page margins

% Theorem definitions using amsthm

\usepackage{amsthm}
\usepackage{amsmath}
\usepackage{amssymb}

% Remove line spaces between items of enumerate and itemize
\usepackage{enumitem}
\setlist{noitemsep}


% Adds double bracket symbols
\usepackage{stmaryrd}

% Latex symbol guide at http://mirrors.ibiblio.org/CTAN/info/symbols/comprehensive/symbols-letter.pdf

% LOGIC symbols
% -------------

% Allows to create negation symbols
\usepackage{MnSymbol}

\DeclareMathOperator{\truth}{truth}
\DeclareMathOperator{\poss}{possibilities}
\DeclareMathOperator{\result}{result}

\def\TRUE{\textsc{true}}
\def\FALSE{\textsc{false}}

\def\SUCCESS{\textsc{success}}
\def\FAILURE{\textsc{failure}}
\def\UNDEF{\textsc{undefined}}


% Symbols for tautology and contradiction
\def\tautology{\top}
\def\contradiction{\bot}

% Symbols for "compatibility" and "incompatibility"
\def\comp{\doublefrown}
\def\ncomp{\ndoublefrown}

% Symbols for "narrower" and "wider"
\def\narrower{\subseteq}
\def\nnarrower{\nsubseteq}
\def\broader{\supseteq}
\def\nbroader{\nsupseteq}


% Symbol for "independent" and "correlated"
\def\indep{\upmodels}
\def\nindep{\nupmodels}

% Aliases for logical operations
\def\AND{\wedge}
\def\bigAND{\bigwedge}
\def\OR{\vee}
\def\bigOR{\bigvee}
\def\NOT{\neg}


% Formatting for statements
\newcommand{\stmt}[1][s] {\mathsf{#1}}
% Formatting for experimental tests
\newcommand{\expt}[1][e] {\mathsf{#1}}
% Formatting for observations
\newcommand{\obs}[1] {\mathsf{#1}}
% Formatting for observation definition
\newcommand{\obsdef}[2] {\llparenthesis #1, #2 \rrparenthesis}

% Formatting for experimental domain
\newcommand{\edomain}[1][D] {\mathcal{#1}}

% Formatting for theoretical domain
\newcommand{\tdomain}[1][D] {\bar{\mathcal{#1}}}


% Formatting for sentence statements
\newcommand{\statement}[1] {\emph{``#1"}}


\usepackage{xcolor} % Required for specifying colors by name
\definecolor{sectionNumbers}{RGB}{44, 103, 0}


\renewcommand\thesubsection{\thesection.\Alph{subsection}}
\renewcommand{\theequation}{\thechapter.\arabic{equation}}

\newtheorem{assump}{Assumption}
\renewcommand*{\theassump}{\Roman{assump}}

\newtheorem{defn}[equation]{Definition}
\newtheorem{prop}[equation]{Proposition}

%\theoremstyle{definition}

\newenvironment{rationale}{\emph{Rationale}.}{\qed}
\newenvironment{justification}{\emph{Justification}.}{\qed}
\renewenvironment{proof}{\emph{Proof}.}{\qed}

% Style for math section
\RequirePackage[framemethod=default]{mdframed} % Required for creating the theorem, definition, exercise and corollary boxes
\newmdenv[skipabove=7pt,
skipbelow=7pt,
rightline=false,
leftline=true,
topline=false,
bottomline=false,
linecolor=sectionNumbers,
backgroundcolor=black!2,
innerleftmargin=5pt,
innerrightmargin=5pt,
innertopmargin=5pt,
leftmargin=0cm,
rightmargin=0cm,
linewidth=4pt,
innerbottommargin=5pt]{mathSection}


%----------------------------------------------------------------------------------------
%	SECTION NUMBERING IN THE MARGIN
%----------------------------------------------------------------------------------------

\makeatletter
\renewcommand{\@seccntformat}[1]{\llap{\textcolor{sectionNumbers}{\csname the#1\endcsname}\hspace{1em}}}                    
\renewcommand{\section}{\@startsection{section}{1}{\z@}
	{-4ex \@plus -1ex \@minus -.4ex}
	{1ex \@plus.2ex }
	{\normalfont\large\sffamily\bfseries}}
\renewcommand{\subsection}{\@startsection {subsection}{2}{\z@}
	{-3ex \@plus -0.1ex \@minus -.4ex}
	{0.5ex \@plus.2ex }
	{\normalfont\sffamily\bfseries}}
\renewcommand{\subsubsection}{\@startsection {subsubsection}{3}{\z@}
	{-2ex \@plus -0.1ex \@minus -.2ex}
	{.2ex \@plus.2ex }
	{\normalfont\small\sffamily\bfseries}}                        
\renewcommand\paragraph{\@startsection{paragraph}{4}{\z@}
	{-2ex \@plus-.2ex \@minus .2ex}
	{.1ex}
	{\normalfont\small\sffamily\bfseries}} % Loads the book formatting

\begin{document}

\chapter{Experimental observations and distinguishability}

In this chapter we introduce a general theory of science, a formalism that is broad enough to be applied to any area of scientific investigation. The starting point will be two basic concepts: objective statements and a way to experimentally verify them. Whether it is physics or chemistry, economics or psychology, medicine or biology, it is always a matter of finding some truth about the natural world and a way to show that experimental data supports that claim.

Finding those statement and devising a mechanism to verify them is the hard part of science. It is the part that is always evolving, that needs new clever tricks or renewed meticulous effort. This is precisely the part that we are not going to formalize. Personally, I do not even know whether it can be formalized. Therefore we will assume that someone has already done the hard bit and handed us the statements together with a way to experimentally test them.

The question is: what can you do if you are given a set of verifiable statements, or \textbf{experimental observations} as we'll call them? Turns out: a lot. Because the statements must have well defined truth values, we can construct logical operations that allow us to organize them in a well defined structure.

We will group experimental observations into \textbf{experimental domains} which are the list of all possible answers to a particular scientific question that can be experimentally tested. From those, we define \textbf{theoretical domains} that will add those statements that, though not directly testable, would have consequences on the experimental tests. These will include, for example, limits of an experimental procedure. Within each theoretical domains, we find those particular statements that, once verified, they determine the truthfulness or falsehood of all other statements: they give us the full picture. We call those \textbf{possibilities}. To answer a scientific question, then, is to find which possibility is the correct one.

The striking result is that the above organization is always possible on any given set of statements that can be experimentally verified. That is, it is a fundamental structure for all sciences. The other striking result is that these concept map exactly to two mathematical structures that are at the core of mathematics: experimental domains map to topologies whilewhile theoretical domains map to a $\sigma$-algebras. These two mathematical structures provide foundation to differential geometry, Lie algebras, measure theory and other mathematical branches that are heavily used in physics and other sciences.

As a consequence of this connection, we can fully understand why these mathematical ideas are pervasive in science, why only particular types of topologies (i.e. Hausdorff and second-countable) are relevant in science and why all relationships that are of scientific interest are continuous functions as defined on their topologies.

\section{Statements}

Science is the systematic study of the physical world through observation and experimentation. Therefore we start our discussion introducing the following principle that formally captures that idea.

\begin{mathSection}
	\textbf{Principle of scientific objectivity}.
		Science is universal, non-contradictory and evidence based.
\end{mathSection}

The main goal in stating this principle is to make clear what the domain of scientific investigation is and acknowledge its limits. For example, assertions like \statement{jazz is marvelous} or \statement{green and red go well together} cannot be the subject of scientific inquiry as they are not objective. That is, there is no agreed upon definition or procedure for what constitutes marvelous music or good color combination. This does not mean that marvelous music or good color combinations do not exist or are not worth studying. They just can't be the subject of scientific inquiry.\footnote{In fact, one can argue that most of the things that make life worth living (e.g. love, friendship, arts, purpose and so on) defy objective characterization and, therefore, that science gives us certain truth about trifling matters.} If we choose to do science, then, we are limiting ourselves to those assertions that are either true or false (i.e. non-contradictory) for everybody (i.e. universal): assertions that have a single truth value.

\begin{mathSection}
\begin{defn}
	The \textbf{Boolean domain} is the set $\mathbb{B} = \{\FALSE, \TRUE\}$ of all possible truth values.
\end{defn}


\begin{defn}\label{def_statement}
	A \textbf{statement} $\stmt{s}$ is a declarative sentence that is either true or false. Formally, a statement is an element of the set $\mathcal{S}$ of all statements upon which is defined a function $\truth: \mathcal{S} \to \mathbb{B}$ that returns the truth value for each element.
\end{defn}

\end{mathSection}

Statements are the first building block of our general theory of science and we are borrowing them from the established philosophical tradition of classical logic. Any language can be used to form them, formal or natural, as indeed any language is used in practice. Scientific investigation in the broad sense of learning from experimentation predates math and formal languages: information about agriculture, astronomy, metallurgy, botany and the like were collected and used even before the written word. Moreover, cognitive scientists have shown that children start using deliberate experimentation at a very young age to understand the world around them, even before their speech is fully developed. Ultimately, that knowledge is encoded in the language of electrical and biochemical signals. Formal languages are indeed extremely helpful in that they allow to be more precise and to better keep track of possible inconsistencies, but ultimately one always needs natural language to give meaning and context to the mathematical symbols. 

Since any language is allowed, the particular syntax of the language used to write the statements is not of interested. That is, we are not going to care what particular symbols and particular grammar rules are used. In fact, even a grammatically incorrect statement is fine as long as the intent is clear. On the other hand, we are going to care about the semantics of the statements (i.e. their content and meaning). Without a clear semantic it would be impossible to gather experimental evidence in favor or against a statement. Therefore we need to slightly extend its definition to allow a minimal set of operations to compare and contrast the content.

As a first step, we want to avoid assigning truths that are inconsistent with the content of the statement.  We want to avoid saying that contradictory statements such as \statement{that cat is a bird} are true or that obvious statements such as \statement{that cat is an animal} are false. The content of the statement, then, is such that it defines what possible truth values the statement is allowed to take.

\begin{mathSection}
	
	\begin{defn}\label{def_possibilities}
		The \textbf{possibilities} of a statement $\stmt{s}$ are the possible truth values allowed by the content of the statement. Formally, on the set $\mathcal{S}$ of all statements is also defined a function $\poss: \mathcal{S} \to \{\{\FALSE, \TRUE\},\{\FALSE\},\{\TRUE\}\}$ such that $\truth(\stmt{s}) \in \poss(\stmt{s})$ for all $\stmt{s} \in \mathcal{S}$.
	\end{defn}
	
	\begin{defn}
		A \textbf{tautology} $\tautology$ is a statement that must be true simply because of its content. That is, $\poss(\stmt{s}) = \{\TRUE\}$.
	\end{defn}
	
	\begin{defn}
		A \textbf{contradiction} $\contradiction$ is a statement that must be false simply because of its content. That is, $\poss(\stmt{s}) = \{\FALSE\}$.
	\end{defn}
	
\end{mathSection}

Therefore we say \statement{that cat is a bird} is a contradiction while \statement{that cat is an animal} is a tautology.

Next we want to keep track when the truth of a statement is a function of the truth of other statements. For example, \statement{that animal is a cat} and \statement{that animal is not a cat} can't both be true. As there are no limits in the way we can logically combine assertions, we will assume we can create any arbitrary function of any set of statements.

\begin{mathSection}
	\begin{defn}\label{def_functions_of_statement}
		We can always construct a statement whose truth value arbitrarily depends on an arbitrary set of statements. Formally: given an arbitrary set of statements $S \subseteq \mathcal{S}$ and an arbitrary truth function $f : \mathbb{B}^{|S|} \to \mathbb{B}$ whose arity (i.e. number of arguments) matches the cardinality of $S$,  there exists a unique statement in $\mathcal{S}$, noted as $f(S)$, such that:
		\begin{itemize}
			\item its truth value is the result of the truth function: \newline $\truth(f(S)) = f(\truth(S))$
			\item its possibilities are consistent with the truth function: \newline $\poss(f(S)) \subseteq f(\bigtimes\limits_{s\in S}\poss(s))$
			\item all possibilities for all arguments are allowed: for every $t_i \in \poss(s_i)$ of every $s_i \in S$ there exists a combination of possibilities $T \in \bigtimes\limits_{s\in S}\poss(s)$ such that the value for $i$-th argument is $t_i$ and  $f(T) \in \poss(f(S))$.
		\end{itemize}
	\end{defn}
\end{mathSection}

To better characterize truth functions, we borrow ideas and definitions from Boolean algebra which is the branch of algebra that operates on truth values. Boolean algebra is fundamental in logic and computer science, since every digital circuit ultimately is implemented on two-state systems (e.g. high/low voltage, up/down magnetization).  One of its main results is that all truth functions can be expressed by combining the following three simple operations.

\begin{mathSection}
	\begin{defn}
		The \textbf{negation or logical NOT} is the function $\NOT : \mathbb{B} \to \mathbb{B}$ that takes a truth value and returns its opposite. That is: $\NOT \TRUE = \FALSE$ and $\NOT \FALSE = \TRUE$.
	\end{defn}
	
	\begin{defn}
		The \textbf{conjunction or logical AND} is the function $\AND : \mathbb{B} \times \mathbb{B} \to \mathbb{B}$ that returns $\TRUE$ only if all the arguments are $\TRUE$. That is: $\TRUE \AND \TRUE = \TRUE$ and $\TRUE \AND \FALSE =\FALSE \AND \TRUE =\FALSE \AND \FALSE = \FALSE$.
	\end{defn}
	
	\begin{defn}
		The \textbf{disjunction or logical OR} is the function $\OR : \mathbb{B} \times \mathbb{B} \to \mathbb{B}$ that returns $\FALSE$ only if all the arguments are $\FALSE$. That is: $\FALSE \AND \FALSE = \FALSE$ and $\TRUE \AND \FALSE =\FALSE \AND \TRUE =\TRUE \AND \TRUE = \TRUE$.
	\end{defn}
\end{mathSection}

Each of these operations defined on the truth value will have a corresponding operation on statements. As an example, here is a table that show how these logical operations work.
\begin{table}[h]
	\centering
	\begin{tabular}{p{0.2\textwidth} p{0.1\textwidth} p{0.1\textwidth} p{0.5\textwidth}}
		Operator & Gate & Symbol & Example \\ 
		\hline 
		Negation & NOT & $\NOT \stmt{s}_1$ &  \emph{``the sauce is not sweet"} \\ 
		Conjunction & AND & $\stmt{s}_1 \AND \stmt{s}_2$ & \emph{``the sauce is sweet and sour"} \\ 
		Disjunction & OR & $\stmt{s}_1 \OR \stmt{s}_2$ & \emph{``the sauce is at least sweet or sour"}\\
		\multicolumn{4}{c}{  where $\stmt{s}_1$ = \emph{``the sauce is sweet"} and $\stmt{s}_2$ = \emph{``the sauce is sour"}}
	\end{tabular} 
	\caption{Boolean operations on statements}
\end{table}

Negation, conjunction and disjunction form, as a group, the two-values Boolean algebra. Any function can be expressed into its \textbf{disjunctive normal form} by using only these three operations. Or, as we'll say, it can be \emph{generated} via negation, conjunction and disjunction. The idea is to create one term for each combination of arguments that returns $\TRUE$ and using the disjunction to combine them together. For example, the statement \statement{the sauce is sweet and sour or neither} can be expressed as $(\stmt{s}_1 \AND \stmt{s}_2) \OR (\NOT \stmt{s}_1 \AND \NOT \stmt{s}_2)$ and the statement \statement{the sauce is not sweet and sour} can be expressed as $(\stmt{s}_1 \AND \NOT \stmt{s}_2) \OR (\NOT \stmt{s}_1 \AND \stmt{s}_2) \OR (\NOT \stmt{s}_1 \AND \NOT \stmt{s}_2)$.

\begin{mathSection}
	\begin{prop}\label{prop_disjunctive_normal_form}
		Any function $f : \mathbb{B}^n \to \mathbb{B}$ that takes $n$ truth values and returns a truth values, which we call a \textbf{truth function}, can be expressed in its \textbf{disjunctive normal form} by using only the negation, conjunction and disjunction of the arguments. Formally, $f(t_1, ..., t_n) =\bigOR \limits_{i=1}^m \left( \bigAND \limits_{j=1}^n (\NOT)^{a_{ij}} \, t_j \right)$ where $m \in \mathbb{N}$, $a_{ij} \in \mathbb{B}$, $\NOT ^ \TRUE \, t_j = t_j$ and $\NOT ^ \FALSE \, t_j = \NOT t_j$.
	\end{prop}
	\begin{proof}
		We first show that this can be done for a function that returns $\TRUE$ for a unique combination of values. Let $a_1, ..., a_n \in \mathbb{B}$ be $n$ truth values. Let $f_1: \mathbb{B}^n \to \mathbb{B}$ be a function such that $f_1(t_1, ..., t_n) = \TRUE$ if and only if $t_j = a_j$ for all $j=1...n$. Let a \textbf{minterm} be a conjunction where each function argument $t_j$ appears once and only once, negated or not. Consider the minterm $\bigAND \limits_{j=1}^n (\NOT)^{a_{j}} \, t_j$. Each argument $t_j$ appears once and is negated or not depending on the value of $a_j$. If any $t_j \nequal a_j$ then one term will be $\FALSE$ and the whole expression will be $\FALSE$. Then we have $f_1(t_1, ..., t_n) = \bigAND \limits_{j=1}^n (\NOT)^{a_{j}} \, t_j$ since they return the same values for the same arguments.
		
		Now we generalize the result for arbitrary functions. Let $f : \mathbb{B}^n \to \mathbb{B}$ be a truth function. Let $m \in \mathbb{N}$ be the number of value combinations for which $f(t_1, ..., t_n) = \TRUE$. For each value combination $i=1..m$ let $a_{ij} \in \mathbb{B}$ be the sequence of values. Then $f_i = \bigAND \limits_{j=1}^n (\NOT)^{a_{ij}} \, t_j$ is the minterm for each value combination. Consider $\bigOR \limits_{i=1}^m f_i$. This function returns $\TRUE$ if and only if the arguments match one of the value combinations for which $f$ returns $\TRUE$. Then we have $f(t_1, ..., t_n) =\bigOR \limits_{i=1}^m \left( \bigAND \limits_{j=1}^n (\NOT)^{a_{ij}} \, t_j \right)$ since they return the same values for the same arguments.
		
		This procedure can be generalized to the case where the number of arguments of $f$ is infinite.
	\end{proof}
\end{mathSection}

While languages typically have similar operations within them, we should consider the ones defined here as meta-operations that are defined outside the language of the statements. For example, \statement{x is the position of a ball}$\AND$\statement{$\,\frac{d^2 x}{dt^2} = - g$} stitches together an English statement with a calculus statement into a new statement that is neither. This kind of mix should be allowed as it does happen in practice.

Since the disjunctive normal form allows us to write all functions in terms of just negation, conjunction and disjunction, it is sufficient to study those operations to understand the properties of all functions. Given the previous definitions, we can deduce the following.

\begin{mathSection}
	\begin{prop}
		The set of all statements $\mathcal{S}$ is closed under negation, arbitrary intersection and arbitrary union.
	\end{prop}
	\begin{proof}
		Negation, arbitrary intersection and arbitrary union are particular truth functions. The statement associated with them always exists by definition  \eqref{def_functions_of_statement}.
	\end{proof}
	\begin{prop}\label{boolean_properties}
		The set of all statements $\mathcal{S}$ satisfies the following properties:
		\begin{itemize}
			\item associativity: $a \OR (b \OR c) = (a \OR b) \OR c$, $a \AND (b \AND c) = (a \AND b) \AND c$
			\item commutativity: $a \OR b = b \OR a$, $a \AND b = b \AND a$
			\item absorption: $a \OR (a \AND b) = a$, $a \AND (a \OR b) = a$
			\item identity: $a \OR \contradiction = a
			$, $a \AND \tautology = a$
			\item distributivity: $a \OR (b \AND c) = (a \OR b) \AND (a \OR c)$, $a \AND (b \OR c) = (a \AND b) \OR (a \AND c)$
			\item complements: $a \OR \NOT a = \tautology$, $a \AND \NOT a = \contradiction$
		\end{itemize}
		and is therefore a Boolean algebra.
	\end{prop}
	\begin{proof}
		The left and right expression for each property correspond to the same truth function applied to the same statements. Therefore, by definition  \eqref{def_functions_of_statement}, they correspond to the same statement.
	\end{proof}
\end{mathSection}

These operations and properties define the \textbf{algebra of statements}. Note that statements naturally inherit all the logical identities of classical logic from the properties of truth functions. This is exactly what we needed if we are to avoid assigning inconsistent truth values to statements that are related. 

The last step is to keep track when two statements are different but have the same content. For example \statement{that animal is a cat}, \statement{that creature is a cat} and \statement{quell'animale \`{e} un gatto} all mean the same thing. It would be inconsistent to assign different truth values to them.

\begin{mathSection}

\begin{defn}
	Two statements $\stmt{s}_1$ and $\stmt{s}_2$ are \textbf{equivalent} $\stmt{s}_1 \equiv \stmt{s}_2$ if they must be equally true or false simply because of their content. That is, $(\stmt{s}_1 \AND \stmt{s}_2) \OR (\NOT\stmt{s}_1 \AND \NOT\stmt{s}_2)$ is a tautology.
\end{defn}

\begin{prop}
	Statement equivalence satifies the following properties:
	\begin{itemize}
		\item reflexivity: $s \equiv s$
		\item symmetry: $s_1 \equiv s_2$ if and only if $s_2 \equiv s_1$
		\item transitivity: if $s_1 \equiv s_2$ and $s_2 \equiv s_3$ then $s_1 \equiv s_3$
	\end{itemize}
	and is therefore an equivalence relation.
\end{prop}
\begin{proof}
	For reflexivity, we have $(\stmt{s} \AND \stmt{s}) \OR (\NOT\stmt{s} \AND \NOT\stmt{s}) = (\stmt{s}) \OR (\NOT\stmt{s}) = \tautology$. Therefore $s \equiv s$.
	
	For symmetry, we have $s_1 \equiv s_2$ implies $\tautology = (\stmt{s}_1 \AND \stmt{s}_2) \OR (\NOT\stmt{s}_1 \AND \NOT\stmt{s}_2) = (\stmt{s}_2 \AND \stmt{s}_1) \OR (\NOT\stmt{s}_2 \AND \NOT\stmt{s}_1)$. Therefore $s_2 \equiv s_1$.
	
	For transitivity, we have TODO.
\end{proof}

\end{mathSection}

Again, we want to stress that this notion of equivalence is not based on the truth value (i.e. whether the statements happen to be both true or false) or on properties \ref{boolean_properties} (i.e. whether they are the same statement): it is based on the possibilities (i.e. whether the two statement can possibly have a different truth). Also note that the semantics defines the possibilities but not the truth value, unless the statement is a tautology or a contradiction. For example, the content of \statement{the next race is going to be won by Secretariat} may be absolutely clear thus defining a clear set of possibilities, but we may be none the wiser about its truthfulness.\footnote{Therefore our notion of semantics may differ from the notion introduced by others.} Intuitively, the equivalence we defined here answers the question: do these two statement carry the same information? If they do, they are essentially the same to us. So much so, that from now on we will implicitly assume two different statements are inequivalent. Technically, when we'll say that $\stmt{s}$ is a statement we actually mean $\stmt{s}$ is an equivalence class of statement.

Equivalence is not the only semantic relationship that we want to capture. Consider the contents of the following:
\begin{enumerate}
	\item \statement{that animal is a cat}
	\item \statement{that animal is a mammal}
	\item \statement{that animal is a dog}
	\item \statement{that animal is a black}
\end{enumerate}
The second will be true whenever the first is true. The third will be false whenever the first is true. The fourth will be true or false regardless of whether the first is true. Assigning truth values at odds with these relationships would be, again, inconsistent. So we want to give a name to these kind of relationships and track them.

\begin{mathSection}

\begin{defn}
	Given two statement $\stmt{s}_1$ and $\stmt{s}_2$, we say that:
	\begin{itemize}
		\item $\stmt{s}_1$ \textbf{is narrower than} $\stmt{s}_2$ (noted $\stmt{s}_1 \narrower \stmt{s}_2$) if $\stmt{s}_2$ is true whenever $\stmt{s}_1$ is true simply because of their content. That is, $\stmt{s}_1 \AND \NOT \stmt{s}_2 \equiv \contradiction$.
		\item $\stmt{s}_1$ \textbf{is broader than} $\stmt{s}_2$ (noted $\stmt{s}_1 \broader \stmt{s}_2$) if $\stmt{s}_2 \narrower \stmt{s}_1$.
		\item $\stmt{s}_1$ \textbf{is compatible to} $\stmt{s}_2$ (noted $\stmt{s}_1 \comp \stmt{s}_2$) if their content allows them to be true at the same time. That is, $\stmt{s}_1 \AND \stmt{s}_2 \nequiv \contradiction$.

	\end{itemize}
	The negation of these properties will be noted by $\nnarrower$, $\nbroader$ , $\ncomp$ respectively.
\end{defn}
\begin{defn}
	The elements of a set of statements $S \subseteq \mathcal{S}$ are said \textbf{independent} (noted $\stmt{s}_1 \indep \stmt{s}_2$ for a set of two) if their content is such that any combination of their possibilities are allowed. That is, $\poss(f(S)) = f(\bigtimes_{x \in S} \poss(s))$ for any truth function $f : \mathbb{B}^{|S|} \to \mathbb{B}$. The negation of independence, will be noted by $\nindep$.
\end{defn}

\end{mathSection}

In the example before: the first statement is narrower than the second (1 $\narrower$ 2), is incompatible with the third (1 $\ncomp$ 3) and is independent from the fourth (1 $\indep$ 4).

Note that independence is not transitive and pair-wise independence is not sufficient. Consider the following statements for an ideal gas:
\begin{enumerate}
	\item \statement{the pressure is $101\pm1$ kPa}
	\item \statement{the volume is $1\pm0.1$ $m^3$}
	\item \statement{the temperature is $293\pm1$ Kelvin}
\end{enumerate}
Since the three quantities are linked by the equation of state $PV=nRT$, any two statements are independent but the three together aren't.

With these tools in place we are in a position to formulate models that are universal and non-contradictory. These models will be a collection of statements with a well defined content, whose truth value will be discovered experimentally.

\section{Experimental tests}

The previous section took care of universality and non-contradiction, but the principle of scientific objectivity requires science to be evidence based. For example, \statement{the square of the hypotenuse is equal to the sum of the squares of the other two sides} or \statement{God is eternal} are not suitable scientific statements as they involve objects that are not physical in nature. As we cannot define those objects experimentally, they depend on what basic axioms or definitions we take as a starting point. For example, the first statement is not true in non-Euclidean geometry and the concept of God has evolved significantly throughout the millennia. Again, this does not mean these concepts are of less significance, just that they cannot be the subject of scientific inquiry.\footnote{In fact, one may be more interested in them precisely because of their abstract, and therefore less transient, nature.}

Limiting the scope of our discussion to objects and properties that are well defined physically is still not enough. For example, \emph{``the electron is green"} or \emph{``1 meter is equal to 5 Kelvin"} are still not suitable scientific statements as the relationships established are not physically meaningful. Even when the relationship is meaningful, we may still not be able to validate it experimentally. For example, \emph{``there is no extra-terrestrial life"} or \emph{``the mass of the electron is exactly $9.109 \times 10^{-31}$ kg"} are not statements that can be verified in practice. In the first case, we would need to check every corner of the universe and find none, with the closest galaxy like ours, Andromeda, being 2.5 millions light-years away; in the second case, we will always have an uncertainty associated with the measurement, however small.

So we have to narrow the scope to those and only those statements that can be verified experimentally. That is, we have to provide a repeatable (i.e. non-contradictory) experimental procedure (i.e. evidence based) that anyone (i.e. universal) can in principle execute. This is both the power and the limit of scientific inquiry: it gives us a way to construct a coherent description of the physical world but it is limited to those aspects that can be reliably studied experimentally.

\begin{mathSection}
\begin{defn}\label{def_experimental_tests}
	An \textbf{experimental test} $\expt{e}$ is a repeatable procedure (i.e. it can be restarted and stopped an arbitrary number of times) that will always either terminate successfully, terminate unsuccessfully or never terminate. Formally, an experimental test is just an element of the set $\mathcal{E}$ of all experimental tests upon which is defined a function $\result: \mathcal{E} \to \{\SUCCESS, \FAILURE, \UNDEF\}$ that returns the result for each element.
\end{defn}
\end{mathSection}

As an example, the following procedure defines an experimental test:
\begin{enumerate}
	\item find a swan
	\item if it's black terminate successfully
	\item go to step 1
\end{enumerate}
If a black swan exists, at some point we'll find it and the test will be successful. If a black swan does not exist, then the procedure will never terminate and the result is undefined.

Experimental tests are the second and last building block of our general theory of science. As with statements, any language can in principle be used to describe the procedure, which can be arbitrarily complicated. It may require building detectors, gathering large amount of data and performing complicated computations. We are not going to care how these procedure are described, just that it is done in a way that allows us to execute the test.

Similarly to statements, we want to define experimental tests whose success is a function of the success of other experimental tests. This is not as straight forward as with statements: this time we need to make sure we actually can create a suitable experimental procedure in practice and this is not always possible. What we do, then, is to study how experimental tests combine under the three basic Boolean operations.

The first important result is that the negation of an experimental test, an experimental test that is successful when the first is not successful, does not necessarily exist. Going back to the swan example, its negation would be a procedure that terminates successfully if black swans do not exist. But the given procedure never finishes in that case, so it is not just a matter of switching success with failure. Because of non-termination, not-successful does not necessarily mean failure.\footnote{In this case, the old adage ``absence of evidence is not evidence of absence" applies.} Note that in many cases one may construct the negation of an experimental test, but there is nothing that guarantees us that it must exist a priori as we can't construct it in general. Moreover, it is a result of computability theory that some problems are undecidable: they do not allow the construction of an algorithm that always terminates with a correct yes-or-no answer. So we know that in some cases this is not actually possible.

In the same vein we are able confirm experimentally that \statement{the mass of this particle is not zero} but not that \statement{the mass of this particle is exactly zero} since we always have uncertainty in our measurements of mass. Even if we could continue shrinking the uncertainty arbitrarily, we would ideally need infinite time to shrink it to zero. What this means is that not all answers to the same question can be equally verified. Is the mass of the photon exactly zero? We can either give a precise ``no" or an imprecise ``it's within this range." Is there extra-terrestrial life? We can either give a precise ``yes" or an imprecise ``we haven't found it so far."\footnote{Note that we are on purpose avoiding induction. It cannot have any role in a fundamental theory of scientific investigation since the decision of when and how to apply it violates the principle of objectivity. Induction is an extremely useful mental shortcut that can be used to point us in the right direction, but giving it a central role in science just creates confusion. We should think of induction like scaffolding when constructing a building: its removal is necessary to claim that the work is done.}


\begin{mathSection}
	\emph{Remark}. The \textbf{negation or logical NOT} of an experimental test does not necessarily exist.
\end{mathSection}

Combining experimental tests with conjunction (i.e. the logical AND) is more straightforward. If we are able to verify that \emph{``the sauce is sweet"} and that \emph{``the sauce is sour"}, we can verify that \emph{``the sauce is sweet and sour"} by executing both. If both are successful, then the conjunction is successful. That is, if we have two or more experimental tests, we can aways test the logical AND for success by checking one at a time if all tests are successful. Yet, the number of tests needs to be finite or we would never terminate.

\begin{mathSection}
	\begin{defn}\label{def_experimental_test_AND}
	The \textbf{conjunction or logical AND} of a finite collection of experimental tests $\{\expt{e}_i\}_{i=1}^{n}$ is the experimental test $\bigAND\limits_{i=1}^{n} \expt{e}_i$ (or simply $\expt{e}_1 \AND \expt{e}_2$ for two statements) that is successful only when all experimental tests are successful. Formally, for any finite collection $\{\expt{e}_i\}_{i=1}^{n} \subseteq \mathcal{E}$ there exists $ \bigAND\limits_{i=1}^{n} \expt{e}_i \in \mathcal{E}$ such that $\result(\bigAND\limits_{i=1}^{n} \expt{e}_i) = \SUCCESS$ if and only if $\result(\expt{e}_i) = \SUCCESS$ for all $i=1..n$.
	\end{defn}
	\begin{justification}
		Mathematically, we are simply assuming that the finite conjunction exists, so there it nothing to prove. However, we want to convince ourselves that in practice we can construct the test to show that the assumption is well justified. Let $\bigAND\limits_{i=1}^{n} \expt{e}_i$ be the experimental procedure defined as follows:
		\begin{enumerate}
			\item for each $i=1..n$ run the test $\mathsf{e}_i$
			\item if all tests terminated successfully terminate successfully.
		\end{enumerate}
		The experimental procedure so defined terminates successfully if and only if all $\mathsf{e}_i$ terminate successfully as per the definition.
	\end{justification}
\end{mathSection}	

Combining experimental tests with disjunction, i.e. the logical OR, is also straightforward. To verify that \emph{``the sauce is at least sweet or sour"} we can first test that \emph{``the sauce is sweet"}. If that is verified that's enough: the sauce is at least sour. If not, we test that \emph{``the sauce is sour"}. That is, if we have two or more experimental tests we can always test the logical OR for success just by stopping at the first test that is successful. Because we stop at the first success, the number of tests we combine in a logical OR can be countably infinite. As long as one test succeeds, which will always be the case when the overall test succeeds, it does not matter how many elements we are not going to verify later. But it cannot be more than countably infinite since the only way we have to find if one experimental test in the set is successful is testing them all one by one.

\begin{mathSection}
	\begin{defn}\label{def_experimental_test_OR}
	The \textbf{disjunction or logical OR} of a (finite or infinite) countable collection of experimental tests $\{\expt{e}_i\}_{i=1}^{\infty}$ is the experimental test $\bigOR\limits_{i=1}^{\infty} \expt{e}_i$ (or simply $\expt{e}_1 \OR \expt{e}_2$ for two statements) that is successful if and only if at least one experimental test in $E$ is successful. Formally, for any countable collection $\{\expt{e}_i\}_{i=1}^{\infty} \in \mathcal{E}$ there exists $ \bigOR\limits_{i=1}^{\infty} \expt{e}_i \in \mathcal{E}$ such that $\result(\bigOR\limits_{i=1}^{\infty} \expt{e}_i) = \SUCCESS$ if and only if $\result(\expt{e}_i) = \SUCCESS$ for some $1 \leq i \leq n$.
	\end{defn}
	\begin{justification}
		Mathematically, we are simply assuming that the countable disjunction exists, so there it nothing to prove. As before, we want to convince ourselves that in practice we can construct the test to show that the assumption is well justified. This time we have to be more careful to handle tests that may not terminate.
		Let $\bigOR\limits_{i=1}^{\infty} \expt{e}_i$ be the experimental procedure defined as follows:
		\begin{enumerate}
			\item initialize $n$ to 1
			\item for each $i=1..n$
			\begin{enumerate}
				\item run the test $\mathsf{e}_i$ for $n$ seconds
				\item if $\mathsf{e}_i$ terminated successfully, terminate successfully
			\end{enumerate}
			\item increment $n$ and go to step 2
		\end{enumerate}
		The procedure will eventually run all tests for an arbitrary long amount of time. Suppose there exists an $1 \leq i \leq n$ such that $\mathsf{e}_i$ will terminate successfully. Then the above procedure will eventually run that test for a time long enough and terminate successfully.
	\end{justification}
\end{mathSection}

Taken as whole, these logical operations define the \textbf{algebra of experimental tests}. It is much more limited than the algebra of statements and it tells us that, in practice, we are not going to be able in general to write a test whose success if an arbitrary function of the success of other tests.

Before we continue, it is interesting to understand how the interplay between physical and mathematical ideas work. Technically, definitions \ref{def_statement}, \ref{def_possibilities}, \ref{def_functions_of_statement}, \ref{def_experimental_tests}, \ref{def_experimental_test_AND} and \ref{def_experimental_test_OR} are the axioms of our mathematical formalism for statements and experimental tests. Note that the actual content of the statements and the procedure for the tests are left unspecified: the math treats them simply as label that we use for identification. The only assumptions are that statements exist (each with a set of possible truth values and an actual truth value), that experimental tests exists (each with a well defined result) and that they both admit the associated algebra. The mathematical formalism does not know what these objects actually are: they may as well be pieces of cardboard painted black or white. Therefore the math does not know whether the properties they have make actual sense: it just knows its consistent. In other words, the way that we are making the framework precise is not by making everything precise: is by omitting the details that are not amenable to a precise specification.

We should stress this for a couple of reasons. First, the part that is not formalized is \emph{the most important part}. Discovering new science is exactly finding new things to study (i.e. new statements) or devising new measurement techniques (i.e. new experimental tests). The content of the statements and the procedure of the experimental test \emph{is} the actual science. Everything that follows, in a sense, is the trivial bit and that is why it can be done generally. Which leads to the second reason: understanding whether statements and experimental tests \emph{actually} follow the algebras we defined is crucial. The math just takes it at face value, it does not prove it. If we botched the characterization, we'll have a nice, consistent, rich but meaningless mathematical framework. Lastly, it has to be clear that something gets lost in the formalization. The mathematical framework cannot carry all the physics content: we removed the most important part! The physics can never be entirely reconstructed from the math and that is why we always have to carefully bring it along.

\section{Experimental observations and experimental domains}

Now that statement and experimental tests are well defined, we can match them to create a verifiable statement.

\begin{mathSection}
\begin{defn}
	An \textbf{experimental observation} is a tuple $\obs{o} = \llparenthesis \stmt{s}, \expt{e} \rrparenthesis$ composed by a statement $\stmt{s}$ and an experimental test $\expt{e}$ such that the statement is true if and only if the  experimental test is always successful. That is, $\truth(\stmt{s})=\textsc{true}$ if and only if $\result(\expt{e})=\textsc{success}$. The experimental observation is \textbf{verified} if the statement is true.
\end{defn}
\end{mathSection}

Note that, in principle, science can also study statements that can be refuted experimentally. But the negation of those is a statement that can be verified experimentally. Therefore we lose nothing by only focusing on verification.\footnote{Mathematically, the space of verifiable statements and refutable statements are dual to each other.}

As pairing a statement with an experimental test does not affect its truth or meaning, we can extend on experimental observations all the operations that compare the content of the statements. Therefore we'll be able to say that observations and statements are equivalent, incompatible and so on.

\begin{mathSection}
\begin{defn}
	The operations $\equiv$, $\nequiv$, $\subseteq$, $\nsubseteq$, $\supseteq$, $\nsupseteq$, $\comp$, $\ncomp$, $\indep$, $\nindep$ are extended to experimental observations (and between statements and observations) by simply comparing the associated statements.
\end{defn}
\end{mathSection}

The logical operations, though, are affected. To take the conjunction or the disjunction of experimental observations we must make sure that the experimental tests still exist. And since the experimental tests are more restrictive, it is their rules that will determine what logical operations are available.
\begin{mathSection}
	\begin{defn}
	The \textbf{conjunction or logical AND} of a finite collection  of experimental observations $\{\mathsf{o}_i\}_{i=1}^{n}=\{\llparenthesis \stmt{s}_i, \expt{e}_i\rrparenthesis\}_{i=1}^{n}$ is the experimental observation $\bigAND\limits_{i=1}^{n} \mathsf{o}_i = \llparenthesis \bigAND\limits_{i=1}^{n} \stmt{s}_i, \bigAND\limits_{i=1}^{n} \expt{e}_i\rrparenthesis$.
\end{defn}
\begin{proof}
	To check that the definition is valid, we need to make sure that the statement is true if and only if the experimental test is successful. We have $\truth(\bigAND\limits_{i=1}^{n} \stmt{s}_i) = \TRUE$ if and only if $\truth(\stmt{s}_i) = \TRUE$ for all $\stmt{s}_i \in \{\stmt{s}_i\}_{i=1}^{n}$. This is the case if and only $\result(\expt{e}_i) = \SUCCESS$ for all $\expt{e}_i \in \{\expt{e}_i\}_{i=1}^{n}$. This can only happen if and only if   $\result(\bigAND\limits_{i=1}^{n} \expt{e}_i) = \SUCCESS$. Therefore $\bigAND\limits_{i=1}^{n} \mathsf{o}_i$ is a valid experimental observation.
\end{proof}
	\begin{defn}
	The \textbf{disjunction or logical OR} of a (finite or infinite) countable  collection of experimental observations $\{\mathsf{o}_i\}_{i=1}^{\infty}=\{\llparenthesis \stmt{s}_i, \expt{e}_i\rrparenthesis\}_{i=1}^{\infty}$ is the experimental observation $\bigOR\limits_{i=1}^{\infty} \mathsf{o}_i = \llparenthesis \bigOR\limits_{i=1}^{\infty} \stmt{s}_i, \bigOR\limits_{i=1}^{\infty} \expt{e}_i\rrparenthesis$.
\end{defn}
\begin{proof}
	As before, we need to make sure that the statement is true if and only if the experimental test is successful. We have $\truth(\bigOR\limits_{i=1}^{n} \stmt{s}_i) = \TRUE$ if and only if $\truth(\stmt{s}_i) = \TRUE$ for at least one $\stmt{s}_i \in \{\stmt{s}_i\}_{i=1}^{n}$. This is the case if and only $\result(\expt{e}_i) = \SUCCESS$ for the corresponding test $\expt{e}_i \in \{\expt{e}_i\}_{i=1}^{n}$. This can only happen if and only if   $\result(\bigOR\limits_{i=1}^{n} \expt{e}_i) = \SUCCESS$. Therefore $\bigOR\limits_{i=1}^{n} \mathsf{o}_i$ is a valid experimental observation.
\end{proof}
\end{mathSection}

We have defined a way to verify each experimental observation by itself, now we have to understand how can we verify groups of them. On one side, we know it will be impossible to verify an uncountable set of experimental observations by going through them one by one. On the other side, if these observations are correlated then we may not have to verify all of them.

The idea is to group experimental observation into sets where we can perform our logical operations but where we can still, at least in principle, find out which ones are verified.

\begin{mathSection}
\begin{defn}
	An \textbf{experimental domain} $\mathcal{D}$ is a set of observations closed under finite conjunction and countable disjunction such that all observations can be tested in infinite time. 
\end{defn}
\end{mathSection}

An experimental domain represents the enumeration of all possible verifiable answers to a scientific question. For example: ``what is that animal?". The domain would include \emph{``it is a mammal"}, \emph{``it is a dog"}, \emph{``it is an animal with feathers"} and so on. If two statements are possible answers to that question, then their conjunction and disjunction will also be possible answers. For example: \emph{``it is a dog or a cat"} or \emph{``it is a mammal and it lays eggs"}.

We only require infinite time, instead of finite time, for the domain because we want to capture those questions that can be answered only approximately. The idea is that, given more time, we can get a better and better answer, but we may not be able to give a perfect answer.

As the experimental domain contains conjunctions and disjunctions, not all observations are going to be independent. Therefore it may not be necessary to test all the observations to find which ones are verified.

\begin{mathSection}
\begin{defn}
	A \textbf{sub-basis} of an experimental domain $\mathcal{D}$ is any subset that can generate all others via finite conjunction and countable disjunction. A \textbf{basis} of an experimental domain is any subset that can generate all others by countable disjunctions. 
\end{defn}
\end{mathSection}

Since all observations of a domain can be expressed as conjunction and/or disjunction of a given sub-basis (of basis), the truth values for all observations can calculated from the truth values of the element of the sub-basis (or basis). So testing all elements of the basis means testing all the observations. And since we must be able to test all observations of a domain in the limit of infinite time, the domain must have a countable basis.

\begin{mathSection}

\begin{prop}
Let $\mathcal{D}$ be an experimental domain. Then there exists a countable basis (equivalently, sub-basis) $\mathcal{B}$ of $\mathcal{D}$.
\end{prop}

\begin{proof}
	First note that the existence of a countable sub-basis implies the existence of a countable basis and vice-versa. On one side, one can generate a countable basis from a countable sub-basis by adding all finite conjunctions, which will not change the cardinality of the infinite set. On the other side, any basis is also a sub-basis by definition.
	
	Now we show that there must exists a countable basis. If there exists a countable basis $\mathcal{B}$, then given infinite time one can test all observations in $\mathcal{D}$. Given all truth values for elements of the basis, one can deduce the truth values for the other observations in $\mathcal{D}$ (again using infinite time) by computing the appropriate conjunctions and disjunctions. If there does not exist a countable basis, then by definition there does not exist a sequence of experimental observations in $\mathcal{D}$ from which one can deduce all other observations in $\mathcal{D}$. Hence it is impossible to test all members of $\mathcal{D}$.
\end{proof}

\end{mathSection}

\section{Theoretical domains and possibilities}


While the negation of an observation is not an observation, it still gives us a statement. It's not one we can readily verify, but it still is a statement that can be used to answer the question at hand. That is: it is a theoretical statement but not an experimental observation. The idea is to group all theoretical statements one could construct from a given experimental domain.

\begin{mathSection}
\begin{defn}
	The \textbf{theoretical domain} $\bar{\mathcal{D}}$ of an experimental domain $\mathcal{D}$ is the set of statement that could be tested if all negations were verifiable. More formally, let $\mathsf{S}$ be the set of all statements associated with an experimental observation in $\mathcal{D}$. $\bar{\mathcal{D}}$ is the set of all statements generated from $\mathsf{S}$ using negation, finite conjunction and countable disjunction.
\end{defn}
\end{mathSection}

A theoretical domain represents all answers to a scientific question, not just the ones we can verify in practice. For example, if our experimental domain contained the experimental observation \emph{``the mass of the photon is not zero"}, its theoretical domain would contain the theoretical statement \emph{``the mass of the photon is exactly zero"} which cannot be experimentally verified.

While the theoretical statement contains more statements, it does not contain more information. That is, if we knew what experimental observation are verified and which aren't, we would automatically know which theoretical statements would be true or not. So it is not adding extra cases. It is essentially completing the list of answers by adding a ``no" if only ``yes" can be experimentally verified and vice-versa.

\begin{mathSection}
\begin{prop}
	The truth value of the statements of a basis $\mathcal{B}$ for an experimental domain $\mathcal{D}$ are enough to determine the truth value for all statements in the associated experimental domain $\bar{\mathcal{D}}$. More formally, all statements in the theoretical domain $\bar{\mathcal{D}}$ can be generated by negation, countable conjunction and countable disjunction from the statements associated to $\mathcal{B}$.
\end{prop}

\begin{proof}
	By definition of basis, any observation within the experimental domain $\mathcal{D}$ can be generated from $\mathcal{B}$ using only finite conjunction and countable disjunction. Therefore the set $\mathsf{S}$ of all statements associated with an experimental observation in $\mathcal{D}$ can be generated from the set $\mathsf{B}$ of all statements associated with an element of the basis $\mathcal{B}$ using negation, countable conjunction and countable disjunction. As $\mathsf{B}$ generates $\mathsf{S}$, it also generates all elements generated by $\mathsf{S}$. Therefore $\mathsf{B}$ generates $\bar{\mathcal{D}}$.
\end{proof}
\end{mathSection}

A direct consequence of our definitions is that the countable conjunction of theoretical statements is a theoretical statement. The reason is that countable conjunction can be rewritten in terms of negation and countable disjunction.

\begin{mathSection}
	\begin{prop}
		All theoretical domains are closed under countable conjunction.
	\end{prop}

\begin{proof}
	Any countable conjunction $\mathsf{s} = \bigAND\limits_{i=1}^{\infty} \mathsf{s}_i$ is equivalent to the negation of disjunction of the negation: $\mathsf{s} = \NOT\bigOR\limits_{i=1}^{\infty} \NOT\mathsf{s}_i$. As the theoretical domain is closed under negation and countable disjunction, so it is closed under countable conjunction.  
\end{proof}

\end{mathSection}

Both the experimental and the theoretical domains contain answers at different level of precision. For example, \emph{``that animal is a feline"} tells us more than \emph{``that animal is a mammal"}. What we want to find are those theoretical statements that are narrowest: those that, if known to be true, would tell us whether each theoretical statement is true or false. For example, if we knew \emph{``that animal is a cat"} we would know that \emph{``that animal is a mammal"} is true and that \emph{``that animal is a dog"} is false.

\begin{mathSection}

\begin{defn}
	A \textbf{possibility} for $\mathcal{D}$ is a statement $x \in \bar{\mathcal{D}}$ that determines the truth value for all statements in the theoretical domain. That is, given a statement $\mathsf{s} \in \bar{\mathcal{D}}$, either $x \subseteq \mathsf{s}$ or $x \ncomp \mathsf{s}$. The set of \textbf{possibilities} $X$ for $\mathcal{D}$ is the collection of all possibilities.
\end{defn}

\begin{prop}[No other possibilities]
	All statements that determine and only determine the truth value of all statements in a theoretical domain $\bar{\mathcal{D}}$ are possibilities of $\mathcal{D}$.
\end{prop}

\begin{proof}
	Let $\mathsf{s}$ be a statement that determines and only determines all truth values of all statements in a theoretical domain $\bar{\mathcal{D}}$. This is equivalent to determine the truth values and only the truth values of all elements of a basis $\mathcal{B}$ of $\mathcal{D}$. As we can find a countable basis, the statement $\mathsf{s}$ is equivalent to the countable conjunction of statements or negation of statements associated with the basis $\mathcal{B}$. Therefore $\mathsf{s} \in \bar{\mathcal{D}}$ as it is generated by the statements of the basis by countable conjunction. But a statement in $\bar{\mathcal{D}}$ that determines all truth values of the statements in $\bar{\mathcal{D}}$ is a possibility by definition. Therefore $\mathsf{s}$ is a possibility.
\end{proof}
\end{mathSection}

The possibilities represent the true answers to the scientific question. Only one of them can be true and one of them must be true if we are to satisfy our principle of objectivity. There is still something that can go wrong, though: we may not be able to tell different possibilities apart experimentally. For example, if we are able to only verify \emph{``there is extra-terrestrial life"}, the opposite possibility will never have experimental confirmation. Since in most cases we want (and can) tell any two possibilities apart, we want to capture the idea that, at least with some approximation, we are able to experimentally distinguish between possibilities.

\begin{mathSection}
\begin{defn}
	Let $\mathcal{D}$ be an experimental domain with its associated theoretical domain $\bar{\mathcal{D}}$. An \textbf{experimental approximation} of $\textsf{s} \in \bar{\mathcal{D}}$ is any experimental observation $\textsf{o} \in \mathcal{D}$ wider than $\textsf{s}$ (i.e. the observation is always verified if the statement is). That is: $\textsf{o} \supseteq \textsf{s}$.
\end{defn}

\begin{defn}
	The set of possibilities $X$ of a domain $\mathcal{D}$ is \textbf{experimentally distinguishable} if given two distinct possibilities $x_1, x_2 \in X$ we can always find two incompatible approximations $\textsf{o}_1, \textsf{o}_2 \in \mathcal{D}$.
\end{defn}
\end{mathSection}

Experimental domains, with their respective theoretical domains and possibilities, represent the most fundamental structure for our theory of scientific investigation. While the particulars may change, the overall structure remains and it is fully determined by what statements are experimentally verifiable and their content. This also means that there is nothing else in it: it cannot tell us more than what is already defined by the experimental observations.

The main work in science, then, is the creation of experimental domains. Gathering experimental procedures and observations where the possibilities are experimentally distinguishable. That is: where all answers for that particular questions can be explored. It is evident, then, that the availability of new experimental techniques or the application of old techniques to different cases is the engine that drives scientific investigation.

\section{Topological spaces}

Now that we have fully defined our fundamental scientific objects, we want to find useful mathematical structures to capture them. What we want to show is that each set of possibilities has a natural topological structure given by the associated experimental domain. So first we need at least to define what a topology is.

\begin{mathSection}
	\begin{defn}
		Let $X$ be a set. A \textbf{topology} on $X$ is a collection $\mathsf{T}$ of subsets of $X$ closed under finite intersection and arbitrary union such that it contains $X$ and $\emptyset$. A \textbf{topological space} is a tuple $(X, \mathsf{T})$ of a set and a topology defined on it.
	\end{defn}
\end{mathSection}

Topologies are used in math to define a notion of closeness without having to define an actual distance. In our case, this notion of closeness will map to how hard is to tell them apart. That is: points that are closer apart are more difficult to distinguish experimentally.

\begin{mathSection}
	
\begin{defn}
	Let $\mathcal{D}$ be an experimental domain and $X$ its possibilities. We define the map $U : \mathcal{D} \rightarrow 2^X$ that for each experimental observation $\mathsf{o} \in \mathcal{D}$ returns the set of possibilities compatible with it. That is: $U(\mathsf{o})\equiv\{ x \in X \, | \, x \comp \mathsf{o}\}$. We call $U(\mathsf{o})$ the \textbf{verifiable set} of possibilities associated with $\mathsf{o}$
\end{defn}

\end{mathSection}

\begin{mathSection}

\begin{prop}
	Let $X$ be the set of possibilities for an experimental domain $\mathcal{D}$. $X$ has a natural topology given by the collection of all verifiable sets $\mathsf{T}_X=U(\mathcal{D})$.
\end{prop}

\begin{proof}
	Let's show that $U(\mathcal{D})$ contains the empty set $\emptyset$. Let $\mathsf{s} \in \bar{\mathcal{D}}$ be a statement. Then $\NOT\mathsf{s} \in \bar{\mathcal{D}}$ and $\mathsf{s} \AND \NOT\mathsf{s} \equiv \contradiction \in \bar{\mathcal{D}}$. But the contradiction $\contradiction$ itself can't be a possibility and no possibility is narrower than $\contradiction$. Therefore $U(\contradiction) = \emptyset$
\end{proof}
\end{mathSection}

Mainly for historical reason, the sets in a topology are called \textbf{open sets}. The complements of open sets are called \textbf{closed sets}. When using a metric space, such as the Euclidean space with the standard topology, these will map to the natural definition. But, in general, they do not and this may lead to confusion. For example, if we take the integers with their standard topology, any subset is both open and closed.

Given that we are only interested in the natural topologies of possibilities, we are going to refer to the sets in our topology as \textbf{verifiable sets} and we will call \textbf{refutable sets} their complements. As such, a subset of integer is both verifiable and refutable: the number could be in the set or not in the set and we have an observation for both cases.  While this terminology does not follow math convention, we find it more intuitive.

We can define basis and sub-basis on a topology.
\begin{mathSection}
\begin{defn}
	A collection $\mathcal{B}$ of verifiable sets of $X$ is a \textbf{sub-basis} if every verifiable set in $X$ is the union of finite intersections of $\mathcal{B}$. It is a \textbf{basis} if every verifiable set in $X$ is the union of elements of $\mathcal{B}$.
\end{defn}
\begin{defn}
	A topology for $X$ is \textbf{second-countable} if it admits a countable basis.
\end{defn}
\end{mathSection}

We can construct a basis (or sub-basis) on our topology simply bv taking the verifiable sets of a basis (or sub-basis) on the experimental domain.

\begin{mathSection}
	\begin{prop}
		Let $X$ be the set of possibilities of an experimental domain $\mathcal{D}$. Let $\mathcal{B}$ be a basis (or sub-basis) for $\mathcal{D}$, then the verifiable sets $U(\mathcal{B})$ form a basis for the natural topology of $X$.
	\end{prop}
	\begin{proof}
		TODO.
	\end{proof}
	\begin{prop}
	The natural topology of a set of possibility is second-countable.
\end{prop}
\end{mathSection}

When the set of possibilities is experimentally distinguishable, the topology has an important mathematical property called Hausdorff. Physically, this property captures the idea that we are able to tell apart experimentally the possibilities of our space.
\begin{mathSection}
	\begin{defn}
		A topology for $X$ is \textbf{Hausdorff} if for every two elements $x_1, x_2 \in X$ there exist two disjoint verifiable sets $U_1, U_2 \in \mathsf{T}_X$ each containing one element. That is: $U_1 \cap U_2 = \emptyset$, $x_1 
		\in U_1$ and $x_2 \in U_2$.
	\end{defn}
	\begin{prop}
		The natural topology of a set of experimentally distinguishable possibilities is Hausdorff.
\end{prop}
\begin{proof}
	TODO.
\end{proof}
\end{mathSection}

\section{Sigma-algebras}

In the same way that experimental domains find a natural mathematical representation as topological spaces, theoretical domains find a natural mathematical representation in $\sigma$-algebras. Let's first define what that is.

\begin{mathSection}
	\begin{defn}
		Let $X$ be a set. A \textbf{$\sigma$-algebra} on $X$ is a collection $\Sigma$ of subsets of $X$ closed under complement and countable union such that it contains $X$.
	\end{defn}
\end{mathSection}

Like topologies, $\sigma$-algebras are fundamental in mathematics as they allow to construct measures (i.e. assigning sizes to sets), limits for sequences of sets and probabilities spaces. It is again fitting that we can associate such fundamental mathematical structure to theoretical domains.

\begin{mathSection}
	
	\begin{defn}
		Let $\bar{\mathcal{D}}$ be a theoretical domain and $X$ its possibilities. We define the map $A : \bar{\mathcal{D}} \rightarrow 2^X$ that for each theoretical statement $\stmt{s} \in \mathcal{D}$ returns the set of possibilities compatible with it. That is: $A(\stmt{s})\equiv\{ x \in X \, | \, x \comp \stmt{s}\}$. We call $A(\stmt{s})$ the \textbf{theoretical set} of possibilities associated with $\stmt{s}$
	\end{defn}
	
	\begin{prop}
		Let $X$ be the set of possibilities for a theoretical domain $\bar{\mathcal{D}}$. $X$ has a natural $\sigma$-algebra given by the collection of all theoretical sets $\Sigma_X=A(\bar{\mathcal{D}})$.
	\end{prop}
	
	\begin{proof}
		Let's show that $U(\mathcal{D})$ contains the empty set $\emptyset$. Let $\mathsf{s} \in \bar{\mathcal{D}}$ be a statement. Then $\NOT\mathsf{s} \in \bar{\mathcal{D}}$ and $\mathsf{s} \AND \NOT\mathsf{s} \equiv \contradiction \in \bar{\mathcal{D}}$. But the contradiction $\contradiction$ itself can't be a possibility and no possibility is narrower than the $\contradiction$. Therefore $U(\contradiction) = \emptyset$
	\end{proof}
\end{mathSection}

There is also a special link between topologies and $\sigma$-algebra. As one may want to construct measures and probabilities space on topological space, there is a standard way to construct a $\sigma$-algebra from a topology. This object, called Borel algebra, is the smallest $\sigma$-algebra that contains all verifiable sets defined by the topology. The $\sigma$-algebra defined by a theoretical domain is none other than the Borel algebra defined by the corresponding experimental domain.

\begin{mathSection}
	
	\begin{defn}
		Let $(X, \mathsf{T})$ be a topological space. Its \textbf{Borel algebra} is the collection $\Sigma_X$ of subsets of $X$ generated by countable union, countable intersection and complement from the verifiable sets.
	\end{defn}
	
	\begin{prop}
		The natural $\sigma$-algebra for a set of possibilities is the Borel algebra of its natural topology.
	\end{prop}
	
	\begin{proof}
TODO
	\end{proof}
\end{mathSection}

\section{Experimental identification}

The most common and important type of experimental observation is when we want to identify a specific case from a set of possible ones. For example, we pick ``this is a duck" from all possible animals or ``the position of the ball is $3 \pm 0.5$ m" from all possible values of position.

As we hinted before, observations do not necessarily identify a single element and they are not necessarily mutually exclusive. For example, ``it is an egg-laying animal" and ``it is a mammal" still allow for the platypus and the echidna. ``the position of the ball is $3 \pm 0.5$ m" and ``the momentum of the ball is $2 \pm 0.1$ kg m/s" limits the possible states to a rectangular region. An experimental observation that identifies an object, then, is not in general associated with a single element, but with a set of elements compatible with that observation. For example, the observation ``it is a bird" will be compatible with duck, penguin, hawk and so on while it will be incompatible with cat, moose, shrimp and so on.

Not only each experimental observation will be associated with a set, but two observations that share the same set are, for our purposes, equivalent. For example, ``it is a feathered animal" or ``it is a bird" will give us the same set of animals as all birds have feathers and all feathered animals are birds. The information they give us is the same. Therefore, we can treat an experimental observation as if it were equivalent to its associated set of elements. That is, if $U \subseteq X$ is the set of elements associated with the observation (e.g. all the birds), the observation may as well be ``the element is in U" (e.g. the element is within the set of birds).

But not all sets of elements are necessarily associated to an experimental observation. As we saw ``the position of the ball is precisely $1$ m" is not an observation because we can't measure position with infinite precision. Therefore the set of states for which position is exactly $1$ m is not associated with an experimental observation. The whole problem, then, is to be able to keep track of which set of elements is associated with an experimental observation and which is not.

At the very least, we need to be able to tell whether an object is or is not one of the elements in our set $X$. For example, we must be able to verify that ``it is an animal" or ``it is not an animal". In the same way, if we can't even verify whether something is a ball or not, it does not make sense to distinguish between the states of a ball. Therefore, to identify elements within a set $X$, there will be an experimental observation associated with the full set $X$, equivalent to ``that is an element of the set", and another experimental observation associated to the empty set $\emptyset$, equivalent to ``that is not an element of the set".

As we saw before, we can always take the conjunction (logical AND) and disjunction (logical OR) of two experimental observation. How do these operations transpose in terms of sets? Suppose we have $U \subseteq X$ (e.g. the set of flying animals) with an associated experimental observation (e.g. ``it is a flying animal"). Suppose we have $V \subseteq X$ (e.g. the set of all insects) with an associated experimental observation (e.g. ``that is an insect"). We combine the two observations with a logical AND (e.g. ``that is a flying insect"). Only the elements that are both in $U$ and $V$ are compatible with it. That is: only the elements in the intersection $U \cap V$. This means that if $U$ and $V$ are two sets each associated with an experimental observation, then their intersection $U \cap V$ is also associated with an experimental observation. Similarly, if $U$ and $V$ are two sets each associated with an experimental observation, we can combine these with a logical OR. All elements either in $U$ or $V$ are compatible with it. That is: elements in the union $U \cup V$. This means the union $U \cup V$ is also associated with an experimental observation.

To sum up, let $X$ be a set of elements and let $\mathsf{T}$ be the collection of the sets associated with some experimental observations we can use to identify among them. We have the following:
\begin{itemize}
	\item Both $X$ and $\emptyset$ can be found in $\mathsf{T}$ as we must be able to verify whether an object is or is not within the ones we can identify
	\item Any intersection of a finite number of sets in $\mathsf{T}$ can also be found in $\mathsf{T}$ as the finite conjunction (i.e. logical AND) of experimental observations is also an experimental observation
	\item Any union of a countable number of sets in $\mathsf{T}$ can also be found in $\mathsf{T}$ as the countable disjunction (i.e. logical OR) of experimental observations is also an experimental observation
\end{itemize}
This is the mathematical definition of a topology: a set of sets that contains both the whole set and the empty set, that is closed under finite intersection and under countable union. 

\begin{mathSection}

\begin{prop}
	Let $A : \bar{\mathcal{D}} \rightarrow 2^X$ be a map that for each statement $\mathsf{s}$ in the theoretical domain $\bar{\mathcal{D}}$ returns the set of possibilities $A(\mathsf{s})$ compatible with $\mathsf{s}$. That is: $A(\mathsf{s})=\{ x \in X \, | \, x \comp \mathsf{s}\}$ Then $A(\mathcal{D})$. provides a sigma algebra for $X$ that is the Borel algebra of topology defined by $U(\mathcal{D})$.
\end{prop}

\begin{proof}
	TODO
\end{proof}

\end{mathSection}


\begin{mathSection}

\begin{prop}
	If $X$ is an experimentally distinguishable set of possibilities, then its natural topology is Hausdorff.
\end{prop}

%\begin{defn}
%	The theoretical domain $\bar{\mathcal{D}}$ of a domain $\mathcal{D}$ is \textbf{experimentally distinguishable} if given two incompatible statements $\textbf{s}_1, \textbf{s}_2 \in \bar{\mathcal{D}}$ we can always find two incompatible approximations $\textbf{o}_1, \textbf{o}_2 \in \mathcal{D}$.
%\end{defn}	

%\begin{prop}
%	If $\bar{\mathcal{D}}$ is an experimentally distinguishable theoretical domain, then the natural topology on its possibilities is normal and Hausdorff.
%\end{prop}
	
\end{mathSection}


\section{Example of an Experimental Domain}

Consider the collection of statements, ``this object has a mass strictly between $x$ and $y$ kilograms" where $x$ and $y$ are any positive real numbers with $x<y$. Consider the experimental test where we put the object on a balance with an appropriate level of precision for each corresponding statement. Then each observation may be equated to an open interval of positive real numbers $(x,y)\subset\mathbb{R}$. Conjunction and disjunction of these observations will correspond exactly to intersection and union of these intervals, respectively. Further, the collection of intervals of the form $(p,q)$ for $p,q\in\mathbb{Q}$ forms a basis of observations. Next, one can show that any sequence of observations satisfying the conditions of exact outcome corresponds precisely to a real number. In particular, the set of possibilities $X$ for this experimental domain will be the set of positive real numbers with their usual topology. The reader with some experience with elementary real analysis will recognize that this construction is essentially the same as the Cauchy sequence construction of the real numbers, where representative Cauchy sequences come from endpoints of intervals, and the intervals used those corresponding to observations in a sequence of observations yielding an exact outcome. 

TODO: write this out much better. 

















TODO: I am not sure whether we should keep the rest of the section below and change the definitions to reflect the above. and, should we break the above proof into smaller pieces? is the above proof what we want??



\begin{defn}
	A \textbf{verifiable set} $U \subseteq X$ is a subset of possibilities for which there exists an experimental observation $\mathsf{o}\in\mathcal{S}$  = (\text{``The object is in } U \text{"}, $\mathsf{e}_\in(U))$ where $\mathsf{e}_\in(U)$ is an experimental test that succeeds only if the object to identify is an element of $U$.
\end{defn}

\begin{defn}
	A \textbf{refutable set} $U \subseteq X$ is a subset of possibilities for which there exists an experimental counter-observation $\mathsf{o} = (\text{``The object is in } U \text{"}, \mathsf{e}_{\notin}(U))$ where $\mathsf{e}_{\notin}(U)$ is an experimental test that succeeds only if the object to identify is not an element of $U$.
\end{defn}

\begin{prop}
	The complement $U^C$ of a verifiable set $U \subseteq X$ is a refutable set.
\end{prop}

\begin{proof}
	Let $U\subset X$ be a verifiable set. There exists an experimental observation $\mathsf{o} = (\text{``The object is in } U \text{"}, \mathsf{e}_\in(U))$. Consider $\neg \mathsf{o} = (\neg \text{``The object is in } U \text{"}, \mathsf{e}_\in(U)) = ($``The object  is not in $ U \text{"}, \mathsf{e}_{\notin}(U^C)) = ($``The object  is in $ U^C \text{"}, \mathsf{e}_{\notin}(U^C))$. That is, there exists a counter-observation $($``The object  is in $ U^C \text{"}, \mathsf{e}_{\notin}(U^C))$: $U^C$ is a refutable set.
\end{proof}

\begin{prop}
	Let $U_1, U_2, ... , U_n, ...$ be a countable infinite sequence of verifiable sets. The finite intersection $\bigcap\limits_{i=1}^{n} U_i$ and the countable union  $\bigcup\limits_{i=1}^{\infty} U_i$ are verifiable sets.
\end{prop}

\begin{proof}
	We claim the finite intersection of verifiable sets is a verifiable set. Let $U_1, U_2, ... , U_n \subseteq X$ be n verifiable sets. For each $U_i$ there exists an experimental observation $\mathsf{o}_i = ($``The object is in $ U_i \text{"}, \mathsf{e}_\in(U_i))$. Consider $\mathsf{o} = \bigwedge\limits_{i=1}^{n} \mathsf{o}_i = (\bigwedge\limits_{i=1}^{n} $``The object is in $ U_i \text{"}, \mathsf{e}_{\wedge}(\mathsf{e}_\in(U_i)))=( $``The object is in $ \bigcap\limits_{i=1}^{n} U_i \text{"}, \mathsf{e}_\in(\bigcap\limits_{i=1}^{n} U_i))$ is an experimental observation. $\bigcap\limits_{i=1}^{n} U_i$ is a verifiable set.
	
	We claim the countable union of verifiable sets is a verifiable set. Let $U_1, U_2, ... , U_n, ... \subseteq X$ be an infinite sequence of verifiable sets. For each $U_i$ there exists an experimental observation $\mathsf{o}_i = ($``The object is in $ U_i \text{"}, \mathsf{e}_\in(U_i))$. Consider $\mathsf{o} = \bigvee\limits_{i=1}^{\infty} \mathsf{o}_i = (\bigvee\limits_{i=1}^{\infty} $``The object is in $ U_i \text{"}, \mathsf{e}_{\vee}(\mathsf{e}_\in(U_i)))=( $``The object is in $ \bigcup\limits_{i=1}^{\infty} U_i \text{"}, \mathsf{e}_\in(\bigcup\limits_{i=1}^{\infty} U_i))$ is an experimental observation. $\bigcup\limits_{i=1}^{\infty} U_i$ is a verifiable set.
\end{proof}




\section{Experimental distinguishability}

What we have done so far is to assign to a set of elements $X$ a collection of experimental observations $\mathsf{T}$. This is not enough. For example, the observations ``it is an animal" and ``it is not an animal" form a collection closed under logical operators (i.e. it forms the trivial topology) but it is hardly useful.

We call $X$ a set of experimentally distinguishable elements if we can associate with it a set of experimental observations rich enough to be able to distinguish each element from the other. That is, given two possible elements (e.g. two possible animals or two possible positions) we can always find an experimental observation that can tell them apart. This property is a basic requirement to treat these elements in a scientific context: if we are not even able to tell them apart experimentally, they cannot be the subject of scientific investigation.

If two elements are experimentally distinguishable from each other there we must have an experimental observation that can tell them apart. For example, to be able to define the house sparrow (Passer domesticus) as a separate animal from the Italian sparrow (Passer italiae) we have at least two observations (``it has a dark gray crown", ``it has a chestnut crown") that are mutually exclusive, each compatible with just one element.

That is, for each pair of elements $x_1, x_2 \in X$, we must be able to find two sets $U,V \in \mathsf{T}$ each associated with an experimental observation such that no element is in both (i.e. their intersection $U \cap V=\emptyset$ is empty) and one element is in one set (i.e. $x_1 \in U$ and $x_2 \in V$). This is the mathematical definition of a Hausdorff space: one in which distinct elements have disjoint neighborhoods. 

This solves one problem, there exists a way to distinguish any two elements, but it does not solve the other problem: we must be able to actually find it. As we saw before, we can only test a finite amount of experimental observations. So we must make sure we can distinguish any two elements in a finite number of steps.

Note that, once we have identified an element, we are able to tell which experimental observations are verified: the ones associated with a set that includes the element. For example, once we have identified the house sparrow, we have verified ``it has a gray crown", ``it is a bird", ``it has wings" and so on. Conversely, once we know all the experimental observations that can be verified, we have identified the element: the only one that is included in all sets associated with verified observations. That is, once we have verified that ``it has a gray crown", ``it is a bird", ``it has wings" and so on, we have identified the house sparrow. In other words: identifying an element is equivalent to being able to verify all possible observations.

Fortunately we don't have to test all possible observations: just enough to be able to calculate all other cases. For example, we can simply test each animal one at a time (i.e. ``it is a cat", ``it is a dog", ``it is a duck" and so on) until we find the correct one. Once we verify one, we will know whether ``it is a bird" or ``it is a mammal". We define sub-base a set of experimental observations from which all others can be obtained through conjunction (i.e. logical AND) and disjunction (i.e. logical OR). In particular, we consider the smallest possible sub-base. This gives us the smallest number of experimental observation to test in order to distinguish an element from all others.

If the smallest sub-base is finite, it is possible to test all experimental observations. This is the case of the set of animals: there are a large but finite number of species. If it's infinite, we cannot. But if the smallest sub-base is countably infinite, we can get as close as we want. This is the case for the position of a ball: we can in principle always increase the precision. Suppose, in fact, that we have a large but finite number of elements $n$ among which we want to distinguish. We need a set of $n$ disjoint sets, each including one element. Note that the union will not make two overlapping sets disjoint, so these sets will only be intersections of elements of the sub-base. Intersections can only be finite, therefore we will need only a finite number of members of the sub-basis. If the sub-basis is countable, at some point we will get to the last member needed and we can stop: we can distinguish between the $n$ elements. If the sub-basis is not countable, instead, we will never stop: we can't experimentally distinguish between a set of arbitrary elements in a finite amount of time.

If there is a countable sub-base, then there is a countable base as well. This is the definition of a second countable space: one that allows a countable bases for its topology.

\begin{defn}
	A set of experimental possibilities $X$ is \textbf{experimentally distinguishable} if the collection of all possible experimental identifications form an experimental domain and if given two arbitrary possibilities $x_1, x_2 \in X$ we can always find two mutually exclusive experimental observation such that each possibility is compatible with one observation.
\end{defn}


\begin{prop}
	A set $X$ of experimentally distinguishable  possibilities is a Hausdorff and second countable topological space with the topology $\mathsf{T}(X)$ formed by all distinguishable sets.
\end{prop}

\section{Connections between topology and experimental distinguishability}

Now that we have made a tight link between topology and experimental distinguishability, we can go through some of the mathematical vocabulary and give it a more precise physical meaning.

The topology defined on a set $X$, which will note as $\mathsf{T}(X)$, describes the collection of all possible experimental observations we can perform to identify an element of the set. Each of these observation can be identified by a subset $U \subseteq X$ which represents all the elements that are compatible with the observation.

An open set $U \in \mathsf{T}(X)$, a set in the topology, is a set for which there exist a way to  experimentally verify that an element is in that set. Conversely, a closed set $V = X \setminus U $, a set whose complement is in the topology, is a set for which there exists a way to experimentally refute that an element is in that set. For example, in the standard topology for $\mathbb{R}$ the interval $(2.5, 3.5)$ is an open set because $3 \pm 0.5$ is a valid measurement for a continuous quantity while $\{3\}$ is a closed set because, while we can't measure a real number with infinite precision, we can exclude it.

The discrete topology, the one for which each singleton $\{x\}$ (i.e. set of one element $x \in X$) is both open and closed, corresponds to the ability to verify and refute each element individually. Any finite set\footnote{TODO what about countable?} that is Hausdorff and second countable has a discrete topology. This is probably why it is not as intuitive to think that something verifiable may not be refutable: it only happens in infinite sets.

The standard topology for $\mathbb{R}$, the one generate by all open intervals, corresponds to the ability of measuring continuous value only with finite precision. This topology is Hausdorff and second countable. While we can give a discrete topology to $\mathbb{R}$, this would no longer be second countable so it would violate our requirement for experimental distinguishability.

An interesting mathematical result is the following.

\begin{prop}
	A Hausdorff and second countable space $X$ has at most cardinality of continuum.
\end{prop}

\begin{proof}
	We define an injective function $F:X\to2^{\mathbb{N}}$, where $2^{\mathbb{N}}$ denotes all infinite binary sequences.
	
	Since $X$ is second-countable, we can enumerate the countable basis $\mathcal{B}$ as $\mathcal{B} = \{B_i\}_{i=1}^{\infty}$. Let $2^{\mathbb{N}}$ denote all infinite binary sequences. We define $F:X\to2^{\mathbb{N}}$ such that $F(x) = (F(x)_i)_{i=1}^{\infty}$ is the sequence where each element is given by: 
	$$
	F(x)_i = 
	\begin{cases}
	1 & x\in B_i \\
	0 & x\notin B_i
	\end{cases}
	$$
	This is an injective function. Suppose $x_1 \neq x_2$, then since $X$ is Hausdorff there is at least one element of the basis that contains one but not the other.\footnote{In fact, $T_0$ separability would be sufficient.} Therefore $F(x_1) \neq F(x_2)$. As $F$ injects $X$ into $2^{\mathbb{N}}$, we have $|X| \leq |2^{\mathbb{N}}|=|\mathbb{R}|$. $X$ has at most cardinality of continuum.
\end{proof}

This means that, no matter what technique we use now or in the future, the collections of elements that we can properly define experimentally are at most infinite like the continuum. This already gives us a first simple and basic requirement a set of mathematical objects need to pass to be of scientific interest.

For example, these objects have cardinality of continuum, and therefore are good candidates:
\begin{itemize}
	\item Euclidean space $\mathbb{R}^n$
	\item all continuous functions from $\mathbb{R}$ to $\mathbb{R}$
	\item all open sets in $\mathbb{R}^n$
	\item all subsets of $\mathbb{N}$
\end{itemize}

These, instead, have cardinality greater then continuum, and therefore are not good candidates:
\begin{itemize}
	\item all functions from $\mathbb{R}$ to $\mathbb{R}$
	\item all subsets of $\mathbb{R}$
\end{itemize}


\section{Continuous functions}

We now turn our attention to relationships between two sets $X$ and $Y$ of experimentally distinguishable elements. There are two ways of characterize them and we want to show that they are equivalent.

Suppose we have two sets $X$ and $Y$ of experimentally distinguishable elements (e.g. the temperature and height of a mercury column in a thermometer). Let's assume we have a causal relationship $f: X \rightarrow Y$ between the first and the second (e.g. the temperature determines how high is the mercury column). We assume the relationship is valid on the whole set without loss of generality: if it is not, just redefine $X$ and $Y$ to be the appropriate regions (e.g. the valid ranges of temperature and height of a mercury column in which the thermometer can operate).

The relationship $f$ can also be used to infer observations on $X$ from observations of $Y$. Suppose we verify that $y$ is within a set $V \in \mathsf{T}(Y)$ (e.g. ``the height of the mercury column is between 24 and 25 millimeters"). Then we can infer that $x$ is within the reverse image $U=f^{-1}(V)$ (e.g. ``the temperature is 24 and 25 degrees Celsius"). $U$ is therefore associated with an experimental observation: $U \in \mathsf{T}(X)$ must be a set in the topology (e.g. if we measure the height of the column with finite precision, we cannot end up inferring the value of temperature with infinite precision).

In other words, to each causal relationship $f: X \rightarrow Y$ between two set of experimentally distinguishable elements (e.g. if the temperature is $x$ the height of the mercury column will be $f(x)$) we have an associated reverse inference relationship $g  : \mathsf{T}(Y) \rightarrow \mathsf{T}(X) = f^{-1}$ (e.g. if the height of the mercury column is within $V$ then the temperature is within $f^{-1}(V)$ ). This is true even if the function is not monotonic (e.g. if ``the water density is between 999.8 and 999.9 kg/$m^3$" then ``the water temperature is between 0 and 0.52 or between 7.6 and 9.12 degrees Celsius" as water is most dense at 4 degrees Celsius).  The reverse image $U=f^{-1}(V)$ of a set associated with an experimental observation (i.e. an open set) is also a set associated with an experimental observation (i.e. an open set). This is the definition of a continuous function in topology: one for which the reverse image of an open set is an open set. Note that, when using the standard topology on $\mathbb{R}^n$, topological continuity is equivalent to analytical continuity.

Now suppose we start with the inference relationship $g  : \mathsf{T}(Y) \rightarrow \mathsf{T}(X)$ that for each verified experimental observation on $Y$ gives us a verified experimental observation on $X$ (e.g. if ``the height of the mercury column is within $V$" then ``the temperature is within $g(V)$"). For it to be a valid inference relationship, it will have to be compatible with logical operations (e.g. if ``the height of the mercury column is between the 23 and 25 millimeters" and ``the height of the mercury column is between the 24 and 26 millimeters" then ``the temperature is between 23 and 25 degree Celsius" and  ``the temperature is between 24 and 26 degree Celsius"). This means the relationship must also be compatible with the set operations that correspond to the logical operations: $g(V_1 \cup V_2)=g(V_1)\cup g(V_2)$ and $g(V_1 \cap V_2)=g(V_1)\cap g(V_2)$.

If nothing is known about $y$ (e.g. if ``the height of the mercury column is in a valid range") then nothing should be known about $x$ (e.g. ``the temperature is in a valid range"): $g(Y)=X$. If we excluded all values in $Y$ (e.g. ``the height of the mercury column is not in a valid range of the thermometer") then we excluded all values in $X$ (e.g. ``the temperature is not in a valid range of the thermometer"): $g(\emptyset) = \emptyset$. That is, for the inference relationship to be valid we shouldn't be able to infer something from nothing or nothing from something.

Under these conditions, one can show that for any such relationship there exists a continuous function $f: X \rightarrow Y$ such that $g=f^{-1}$ is its inverse image.

\begin{prop}
	\label{setfunctions}
	Let $X$ and $Y$ be two Hausdorff topological spaces, and let $g: \mathsf{T}(Y) \rightarrow \mathsf{T}(X)$ be a mapping such that:
	\begin{enumerate}
		\item It is compatible with union and intersection $\forall V_1, V_2 \in Y$ $g(V_1 \cup V_2)=g(V_1)\cup g(V_2)$ and $g(V_1 \cap V_2)=g(V_1)\cap g(V_2)$
		\item $g(\emptyset) = \emptyset$
		\item $g(Y) = X$
	\end{enumerate}
	Then there exists a unique continuous function $f: X \rightarrow Y$ such that $g = f^{-1} |_{\mathsf{T}(Y)}$.
\end{prop}

\begin{proof}
	We claim there exists a unique extension $\bar{g}:\sigma(Y)\to\sigma(X)$ to the Borel $\sigma$-algebras of $X$ and $Y$, respectively $\sigma(X)$ and $\sigma(Y)$, such that $\bar{g}|_{\mathsf{T}(Y)}=g$ and $\bar{g}$ is compatible with union, intersection and complements. Let $\bar{g}(V) = g(V)$ for all open sets $V \in \mathsf{T}(Y)$. Let $A \in \sigma(Y)$ (not necessarily open) and $A^C$ be its complement. We must have $\bar{g}(A^C) = \bar{g}(A)^C = X\setminus \bar{g}(A)$ for $\bar{g}$ to be compatible with complements. Recall that all Borel sets in $\sigma(Y)$ and $\sigma(X)$ may be written as some combination of unions, intersections, and complements of open sets. Thus, the construction uniquely determine what $\bar{g}$ should output on any Borel set. We need only check that the output is still a Borel set. But by definition of $\bar{g}$, the outputs will be given as unions, intersections, and complements of outputs of $g$, which are open sets, and so the image of $\bar{g}$ is contained in $\sigma(X)$.  $\bar{g}$ is well defined. The function $\bar{g}$ in a sense represents extracting the maximum amount of information possible out of $g$.
	
	We claim we can define $\hat{g}:Y\to\sigma(X)$ such that $\hat{g}(y) = \bar{g}(\{y\})$. Since $Y$ is Hausdorff, every singleton $\{y\}$ is closed and is therefore a Borel set. $\bar{g}(\{y\})$ is well defined and so is $\hat{g}(y)$.
	
	We claim  $\hat{g}(y_1)\cap\hat{g}(y_2) = \emptyset$ if and only if $y_1\neq y_2$ for all $y_1,y_2\in Y$ such that $\hat{g}(y_i)\neq\emptyset$ for $i=1,2$. If $y_1\neq y_2$ we have
	$$
	\hat{g}(y_1)\cap\hat{g}(y_2) = \bar{g}(\{y_1\})\cap\bar{g}(\{y_2\}) = \bar{g}(\{y_1\}\cap\{y_2\}) = \bar{g}(\emptyset) = \emptyset.
	$$
	Conversely, if $y_1 = y_2$ we have
	$$
	\hat{g}(y_1)\cap\hat{g}(y_2) = 	\hat{g}(y_1)\cap\hat{g}(y_1) = 
	\hat{g}(y_1) \neq \emptyset.
	$$
	
	We claim we can define $f: X\to Y$ such that $f(x) = y$ if and only if $x\in \hat{g}(y)$. Since $g(Y)=X$, there exists $y\in Y$ such that $x\in\hat{g}(y)$. By the preceding claim, this $y$ is unique. $f: X\to Y$ is well defined. Note that no arbitrary choice where made so far that lead to the construction of $f$, which is therefore determined uniquely by $g$. 
	
	We claim $g = f^{-1} |_{\mathsf{T}(Y)}$. Let $V\in\mathsf{T}(Y)$. We want to show $f^{-1}(V) = g(V)$. Let $x\in f^{-1}(V)$. Then for some $y \in V$ we have $f(x)=y$. $x\in \hat{g}(y)$ by construction of $f$. $\hat{g}(y) \subset g(V)$ since $\{y\}\subset V$, so $x\in g(V)$. $f^{-1}(V) \subseteq g(V)$. Conversely, let $x\in g(V)=\bar{g}(V)$. Then for some $y\in V$, we have $x\in\bar{g}(\{y\})\subset\bar{g}(V)$. But then by definition we have $f(x)=y$, so $x\in f^{-1}(V)$. $f^{-1}(V) \supseteq g(V)$. $f^{-1}(V) = g(V)$ for all $V\in\mathsf{T}(Y)$ and therefore $f^{-1}|_{\mathsf{T}(Y)}=g$.
	
	We claim $f$ is continuous. It is so since $g = f^{-1} |_{\mathsf{T}(Y)}$ takes open sets to open sets. 
\end{proof}

The above work gives us an approach to reconstruct continuous functions given the behavior of the inverse on open sets. In the context of collecting information on observable phenomena, the function $g$ represents the total of all information possible to gather on the correlation between two variables. The fact that from $g$ we can construct a unique continuous $f$ shows us that in the infinite-resource ideal, we fully obtain the information to give exact relations between variables. Further, the result itself tells us that the interesting phenomena in our framework of experimental observations will be continuous, which is often a baseline assumption in physics. 





TODO: Introduce/define Borel algebra? In fact, we really only need infinite intersections of open sets. We could skip complements all together and just let $\bar{g}$ be defined on all open sets plus all infinite intersections of open sets. This might be slightly simpler but won't make much of a difference either way. Also, it might make more sense to absorb Lemma 1.24 and $\hat{g}$ into the proof of Proposition 1.22. 

\section{Distinguishability of functions}

For all of this to be self consistent, we must require that functions between experimentally distinguishable elements are  experimentally distinguishable themselves. That is, we must show that the set of continuous function $C(X,Y)$ from $X$ to $Y$ is a Hausdorff and second countable topological space with a suitable topology. Note that $C(X,Y)$ has the cardinality of continuum, therefore we already know it allows Hausdorff and second countable topologies, but we need to make sure we can give one that is actually physically meaningful in terms of experimental observations.

Suppose we have two sets $X$ and $Y$ of experimentally distinguishable elements (e.g. time and space). Suppose we need to distinguish elements within the set of all continuous function $C(X,Y)$ from $X$ to $Y$ (e.g. all possible trajectories). From a sub-base of $X$ (e.g. all open time intervals between rational numbers) we pick a set $U$ (e.g. between 1 and 2 seconds). From a sub-base of $Y$ (e.g. all open spatial intervals between rational numbers) we pick a set $V$ (e.g. between 1 and 2 meters). We can define the set $S(U,V) = \{f: X \rightarrow Y | f \in C(X,Y), f(U) \subset V\}$ of all the continuous functions such that their value remains within $V$ over the domain $U$ (e.g. all trajectories that between $1$ and $2$ seconds remain within $1$ and $2$ meters).

If we take all possible sets of functions constructed this way, and combine them with finite intersections and countable unions, we obtain a topology for the continuous function. The collections of all sets $S(U,V)$ forms a sub-base for this topology and is constructed from the sub-bases of $X$ and $Y$. This new sub-base is countable because the choices for $U$ and $V$ are countable themselves. This means that $C(X,Y)$ is second countable with this topology.

Now we need to show that we can distinguish between any two functions. Say we have two different functions $f_1$ and $f_2$ (i.e. two different trajectories). Since they are different, there will be at least one value $x \in X$ such that the values $f_1(x)\neq f_2(x)$ will be different (e.g. at $1.2$ seconds the first trajectory is at $1.1$ meters while the second trajectory is at $1.2$ meters). We can now find two disjoint elements $V_1$ and $V_2$ of the sub-base for $Y$ such that each includes either $f_1(x)$ or $f_2(x)$ (e.g. the intervals $(1.095, 1.105)$ and $(1.195, 1.205)$ meters). Since the functions are continuous, we can find an element $U$ of the sub-base for $X$ such that both functions remain within those ranges (e.g. between $(1.197, 1.203)$ seconds the first trajectory remains within $(1.095, 1.105)$ meters and the second trajectory remains within $(1.195, 1.205)$ meters). The sets $S(U, V_1)$ (e.g. all trajectories that between $(1.197, 1.203)$ seconds that remain within $(1.095, 1.105)$ meters) and $S(U, V_2)$ (e.g. all trajectories that between $(1.197, 1.203)$ seconds that remain within $(1.195, 1.205)$ meters) are disjoint and they each contain $f_1$ and $f_2$ respectively. We can distinguish between functions: the topology is Hausdorff.

This confirms that all the conceptual infrastructure is solid and self consistent. Sets of experimentally distinguishable elements are Hausdorff and second countable topological spaces. Maps between elements of such spaces are continuous maps as they preserve experimental distinguishability. The maps themselves are experimentally distinguishable and the can be given a Hausdorff and second countable topology. We can continues this recursively by constructing maps of maps (e.g. a map from a state to a trajectory) without ever going outside our definitions: everything remains experimental distinguishable.

\begin{mathSection}

For continuous functions to be physically distinguishable, we need to show they can always be given, as a set, given a topology that is Hausdorff and second-countable.

To that end, we introduce the basis-to-basis topology on the set of continuous functions from two topological spaces. This is the topology generated by the sets of functions that map a basis element of one element inside an element of the other space. 

\begin{defn} Let $X$ and $Y$ be two topological spaces. Let $C(X,Y)$ denote the set of all continuous functions from $X$ to $Y$. Let $\mathcal{B}_X$ and $\mathcal{B}_Y$ be two bases for $X$ and $Y$ respectively. The basis-to-basis topology $\mathsf{T}(C(X,Y), \mathcal{B}_X, \mathcal{B}_Y)$ on $C(X,Y)$ with respect to the basis $\mathcal{B}_X$ and $\mathcal{B}_Y$ is the topology generated by all sets of the form 
	$$
	V(U_X, U_Y) = \{f\in C(X,Y) : f(U_X)\subset U_Y\}
	$$
where $U_X \in \mathcal{B}_X$ and $U_Y \in \mathcal{B}_Y$.
\end{defn}

Now we need to show that if the two spaces $X$ and $Y$ are Hausdorff and second-countable, the basis-to-basis topology is Hausdorff and second-countable for a suitable choice of basis.

\begin{prop}
	Let $X$ and $Y$ be two Hausdorff and second-countable topological spaces. Let $C(X,Y)$ denote the set of all continuous functions from $X$ to $Y$. Let $\mathcal{B}_X$ and $\mathcal{B}_Y$ be two countable bases for $X$ and $Y$ respectively. The basis-to-basis topology $\mathsf{T}(C(X,Y), \mathcal{B}_X, \mathcal{B}_Y)$ on $C(X,Y)$ with respect to the basis $\mathcal{B}_X$ and $\mathcal{B}_Y$ is Hausdorff and second-countable. 
\end{prop}
\begin{proof}
	We claim $\mathsf{T}(C(X,Y), \mathcal{B}_X, \mathcal{B}_Y)$ is second-countable. We first note that the sub-basis $\{V(U_X, U_Y) \, |\,   U_X \in \mathcal{B}_X , U_Y \in \mathcal{B}_Y \}$ is countable since $\mathcal{B}_X$ and $\mathcal{B}_Y$ are countable. The collection $\mathcal{B}$ of all finite intersections is still countable, since it is in one-to-one correspondence with the collection of finite subsets of a countable set, which is still countable. Therefore $\mathcal{B}$ is a countable basis, which means $\mathsf{T}(C(X,Y), \mathcal{B}_X, \mathcal{B}_Y)$ is second-countable.
	
	We claim $\mathsf{T}(C(X,Y), \mathcal{B}_X, \mathcal{B}_Y)$ is Hausdorff. Let $f,g:X\to Y$ be distinct continuous functions. Then for some $x\in X$, we have $f(x)\neq g(x)$. Pick $V_1, V_2$ disjoint open subsets of $Y$ with $f(x)\in V_1$ and $g(x)\in V_2$. We may assume (possibly by shrinking $V_1$ or $V_2$) that both are basis elements for the topology of $Y$. Let $U=f^{-1}(V_1)\cap g^{-1}(V_2)$. Then $U$ is an open neighborhood of $x$. We may assume again that $U$ is a basis element for the topology on $X$ by shrinking it if necessary. Now, let $T_1$ to be the (sub-)basis element for basis-to-basis topology corresponding to $U$ and $V_1$. By construction, $f\in T_1$. Similarly, let $T_2$ to be the basis element for the basis-to-basis topology corresponding to $U$ and $V_2$ and containing $g$. Since $V_1$ and $V_2$ are disjoint, so are $T_1$ and $T_2$. $\mathsf{T}(\mathcal{C})$ is Hausdorff.
\end{proof}

We will also show that this is not in general equal to the open-open topology. 

\begin{prop}
	In general, the topology $\mathsf{T}(\mathcal{C})$ defined above depends on the bases for $X$ and $Y$. 
\end{prop}
\begin{proof}
	We will give an example of such a case. Let $X=Y=\mathbb{R}$ with the usual topology on $\mathbb{R}$. Let $\mathcal{B}_1$ be the basis for $\mathbb{R}$ consisting of all open intervals with rational endpoints, and let $\mathcal{B}_2 = \mathcal{B}_1\cup\{(0,\pi)\}$. We find a set open in our topology $\mathsf{T}(\mathcal{C})$ using $\mathcal{B}_2$ as our basis which is not open when we use $\mathcal{B}_1$. Consider the following set, open in our topology when using $\mathcal{B}_2$:
	$$
	V = \{f\in\mathcal{C}| f((0,\pi))\subset(0,1)\}
	$$
	This is clearly open because $(0,\pi)$ and $(0,1)$ are basis elements in $\mathcal{B}_2$. 
	
	Next, we show that when using $\mathcal{B}_1$ as our basis for $X=Y=\mathbb{R}$, no finite intersection and/or infinite union of sub-basis elements for $\mathsf{T}(\mathcal{C})$ will equal $V$. Henceforth, when we say ``open" subsets, unless otherwise stated, we will refer exclusively to the topology $\mathsf{T}(\mathcal{C})$ when using $\mathcal{B}_1$ for $\mathbb{R}$. The smallest sub-basis subsets containing $V$ are those of the form:
	$$
	B_r = \{f\in\mathcal{C} | f((0,r))\subset(0,1)\}
	$$
	for each $r\in\mathbb{Q}$ with $r<\pi$. By ``smallest" we mean any sub-basis elements containing $V$ which are not in the form of $B_r$ will contain some $B_r$. Now, any finite intersection of the $B_r$'s will be another $B_r$, but these strictly contain $V$, so we can never obtain $V$ from this sub-basis. 
	
	Similarly, the largest sub-basis sets contained within $V$ are of the form: 
	$$
	S_r = \{f\in\mathcal{C} | f((0,r))\subset(0,1)\}
	$$
	for each $r\in \mathbb{Q}$ with $r>\pi$. Let $S = \cup_{r>\pi}S_r$. This is the largest open set in our topology which is contained in $V$. Consider the function $f(x) = x/\pi$. This is continuous and is an element of $V$. But notice $f(\pi)=1$, so $f\notin S_r$ for all $r>\pi$, hence $f\notin S$. Thus, $V$ is a set which is open in $\mathsf{T}(\mathcal{C})$ when we use $\mathcal{B}_1$ for $X$ and $Y$, but not open when we use $\mathcal{B}_2$ for $X$ and $Y$. 
\end{proof}

TODO: add corollary environment for this one. 

\begin{prop}
The topology $\mathsf{T}(\mathcal{C})$ on the set of continuous functions $C(\mathbb{R},\mathbb{R})$ from $\mathbb{R}$ to itself is in general not equal to the open-open topology on $C(\mathbb{R},\mathbb{R})$.
\end{prop}
\begin{proof}
In the proof of the previous proposition, note that the basis for the open-open topology is independent of the basis used for the underlying spaces, and in particular is a superset of the basis for $\mathsf{T}(\mathcal{C})$ when using $\mathbb{B}_2$ to construct it. Hence the open-open topology is distinct from $\mathsf{T}(\mathcal{C})$ when using the basis $\mathcal{B}_1$ on $\mathbb{R}$. 
\end{proof}

TODO: the above propositions might generalize to any uncountable, second-countable space. This will be considered during the paper-writing process later. 

Thus we have shown one can start with two Hausdorff, second-countable spaces, and generate a new Hausdorff second-countable space consisting of all functions between them. Topological spaces with these properties make up our class of scientifically interesting spaces, so the space of continuous (distinguishability-preserving) functions between spaces of distinguishable quantities is again a physically distinguishable space. Because one can iterate this construction (since the relevant properties are preserved), this is a way to generate arbitrarily many new physically distinguishable spaces encoding important information about the ``lower-order" spaces (i.e. they consist of functions which encode a conversion between verifiable sets in different spaces).

\end{mathSection}

\section{Summary}

\begin{table}[h]
	\centering
\begin{tabular}{p{0.20\textwidth} p{0.7\textwidth}}
	Math/Topology & Science/Physics \\ 
	\hline 
	Hausdorff, second-countable topological space & Experimentally distinguishable space, whose points are the possible values and whose open sets represent the experimentally attainable levels of precision \\
	Open set & Verifiable set. We can verify experimentally that an element is within the set  \\ 
	Closed set & Refutable set. We can verify experimentally that an element is not in the set \\ 
	Basis of a topology & A collection of verifiable sets such that any verifiable set is determined by the basis sets\\
	Continuous \newline function &  A function between two sets of experimental distinguishable elements that preserves distinguishability \\
	Homeomorphism &  A perfect equivalence between experimentally distinguishable spaces. \\
\end{tabular} 
\caption{Topology to physics dictionary}
\end{table}


\chapter{Other}

	
\end{document}