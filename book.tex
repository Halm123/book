% Possible use http://www.latextemplates.com/template/the-legrand-orange-book as template? See https://www.overleaf.com/9174958nyjxxdxbchks#/33024595/

\documentclass[11pt,letterpaper,fleqn]{memoir} % Default font size and left-justified equations

\usepackage[top=3cm,bottom=3cm,left=3cm,right=3cm,headsep=10pt,letterpaper]{geometry} % Page margins

% Theorem definitions using amsthm

\usepackage{amsthm}
\usepackage{amsmath}
\usepackage{amssymb}

% Remove line spaces between items of enumerate and itemize
\usepackage{enumitem}
\setlist{noitemsep}


% Adds double bracket symbols
\usepackage{stmaryrd}

% Latex symbol guide at http://mirrors.ibiblio.org/CTAN/info/symbols/comprehensive/symbols-letter.pdf

% LOGIC symbols
% -------------

% Allows to create negation symbols
\usepackage{MnSymbol}

\DeclareMathOperator{\truth}{truth}
\DeclareMathOperator{\poss}{possibilities}
\DeclareMathOperator{\result}{result}

\def\TRUE{\textsc{true}}
\def\FALSE{\textsc{false}}

\def\SUCCESS{\textsc{success}}
\def\FAILURE{\textsc{failure}}
\def\UNDEF{\textsc{undefined}}


% Symbols for tautology and contradiction
\def\tautology{\top}
\def\contradiction{\bot}

% Symbols for "compatibility" and "incompatibility"
\def\comp{\doublefrown}
\def\ncomp{\ndoublefrown}

% Symbols for "narrower" and "wider"
\def\narrower{\subseteq}
\def\nnarrower{\nsubseteq}
\def\broader{\supseteq}
\def\nbroader{\nsupseteq}


% Symbol for "independent" and "correlated"
\def\indep{\upmodels}
\def\nindep{\nupmodels}

% Aliases for logical operations
\def\AND{\wedge}
\def\bigAND{\bigwedge}
\def\OR{\vee}
\def\bigOR{\bigvee}
\def\NOT{\neg}


% Formatting for statements
\newcommand{\stmt}[1][s] {\mathsf{#1}}
% Formatting for experimental tests
\newcommand{\expt}[1][e] {\mathsf{#1}}
% Formatting for observations
\newcommand{\obs}[1] {\mathsf{#1}}
% Formatting for observation definition
\newcommand{\obsdef}[2] {\llparenthesis #1, #2 \rrparenthesis}

% Formatting for experimental domain
\newcommand{\edomain}[1][D] {\mathcal{#1}}

% Formatting for theoretical domain
\newcommand{\tdomain}[1][D] {\bar{\mathcal{#1}}}


% Formatting for sentence statements
\newcommand{\statement}[1] {\emph{``#1"}}


\usepackage{xcolor} % Required for specifying colors by name
\definecolor{sectionNumbers}{RGB}{44, 103, 0}


\renewcommand\thesubsection{\thesection.\Alph{subsection}}
\renewcommand{\theequation}{\thechapter.\arabic{equation}}

\newtheorem{assump}{Assumption}
\renewcommand*{\theassump}{\Roman{assump}}

\newtheorem{defn}[equation]{Definition}
\newtheorem{prop}[equation]{Proposition}

%\theoremstyle{definition}

\newenvironment{rationale}{\emph{Rationale}.}{\qed}
\newenvironment{justification}{\emph{Justification}.}{\qed}
\renewenvironment{proof}{\emph{Proof}.}{\qed}

% Style for math section
\RequirePackage[framemethod=default]{mdframed} % Required for creating the theorem, definition, exercise and corollary boxes
\newmdenv[skipabove=7pt,
skipbelow=7pt,
rightline=false,
leftline=true,
topline=false,
bottomline=false,
linecolor=sectionNumbers,
backgroundcolor=black!2,
innerleftmargin=5pt,
innerrightmargin=5pt,
innertopmargin=5pt,
leftmargin=0cm,
rightmargin=0cm,
linewidth=4pt,
innerbottommargin=5pt]{mathSection}


%----------------------------------------------------------------------------------------
%	SECTION NUMBERING IN THE MARGIN
%----------------------------------------------------------------------------------------

\makeatletter
\renewcommand{\@seccntformat}[1]{\llap{\textcolor{sectionNumbers}{\csname the#1\endcsname}\hspace{1em}}}                    
\renewcommand{\section}{\@startsection{section}{1}{\z@}
	{-4ex \@plus -1ex \@minus -.4ex}
	{1ex \@plus.2ex }
	{\normalfont\large\sffamily\bfseries}}
\renewcommand{\subsection}{\@startsection {subsection}{2}{\z@}
	{-3ex \@plus -0.1ex \@minus -.4ex}
	{0.5ex \@plus.2ex }
	{\normalfont\sffamily\bfseries}}
\renewcommand{\subsubsection}{\@startsection {subsubsection}{3}{\z@}
	{-2ex \@plus -0.1ex \@minus -.2ex}
	{.2ex \@plus.2ex }
	{\normalfont\small\sffamily\bfseries}}                        
\renewcommand\paragraph{\@startsection{paragraph}{4}{\z@}
	{-2ex \@plus-.2ex \@minus .2ex}
	{.1ex}
	{\normalfont\small\sffamily\bfseries}} % Loads the book formatting

\begin{document}
	
\chapter{Scientific distinguishability}

\section{Scientific observations}

As science is the systematic study of the physical world through observation and experimentation, it seems appropriate to start our endeavor by characterizing the basic properties of scientific observations.

First we should realize that not all observations are of a scientific nature. For example, \emph{``jazz is marvelous"} or \emph{``green and red go well together"} are not scientific statements as their are subjective. That is, there is no agreed upon definition or procedure for what constitutes marvelous music or good color combination. This does not mean that these are somehow statements of a lesser nature. In fact, most of the things that make life worth living (e.g. love, passion, friendship, arts, creativity, purpose and so on) defy objective characterization and it would be a very dull world if they didn't. It simply means: they are not the subject of scientific inquiry.

While we have excluded opinions, not all facts are scientific either. For example, \emph{``the square of the hypotenuse is equal to the sum of the squares of the other two sides"} or \emph{``God is eternal"} are not scientific statements as they describe properties of entities that are not physical in nature. Again, this does not mean these concepts are of less significance. In fact, one may be more interested in them precisely because of their abstract, and therefore less transient, nature. Nonetheless, they are not the subject of scientific inquiry.

Yet, even facts about physical objects may still not be scientifically well defined. For example, \emph{``there is no extra-terrestrial life"} or \emph{``the mass of the photon is exactly zero"} are not something we can validate experimentally. In the first case, we will never know whether there is life outside the observable universe; in the second, we will always have an error bar, however small.

I encourage you to take a minute and come up with your own examples of what may or may not be a statement that can be validated experimentally. Either because of the subject or the lack of an objective and repeatable verification procedure. It should help realize that science is no golden hammer and not everything is a nail. Science is only one of the intellectual tools we have at our disposal and it does not supersede the others.

We capture this discussion with the following definition:
\begin{defn}
A \textbf{scientific observation} is a fact that can be experimentally verified. That is, there is an agreed upon procedure that, if successful, confirms the validity of the statement.
\end{defn}

\section{Logical operations}

Once we have set of verified observations, we can use logic to infer others or check for consistency. For example, once we have established that \emph{``cold reduces swelling"} and that \emph{``ice is cold"} we may infer that \emph{``ice 
deduces swelling"}. Note that this type of inference relies on the meaning of the actual observations. That is: we understand that cold is a property that has an effect and that ice posses it. As such, it cannot be generalize to all scientific observations.

What we want to understand is how scientific observations work under the common logical operations. These do not depend on the specifics of the observation and therefore have a more general character.
\begin{table}[h]
	\centering
	\begin{tabular}{p{0.2\textwidth} p{0.1\textwidth} p{0.1\textwidth} p{0.5\textwidth}}
		Operator & Gate & Symbol & Example \\ 
		\hline 
		Negation & NOT & $\neg A$ &  \emph{``the sauce is not sweet"} \\ 
		Conjunction & AND & A $\wedge B$ & \emph{``the sauce is sweet and sour"} \\ 
		Disjunction & OR & $A \vee B$ & \emph{``the sauce is at least sweet or sour"}\\
		\multicolumn{4}{c}{  $A$ = \emph{``the sauce is sweet"} and $B$ = \emph{``the sauce is sour"}}
	\end{tabular} 
	\caption{Boolean operations on scientific observations}
\end{table}

A somewhat surprising result is that the negation of a scientific observation is not always a scientific observation. For example, \emph{``there is extra-terrestrial life"} and \emph{``the mass of the neutrino is not exactly zero"} can be validated experimentally. For the first, we may send a probe to Mars and find bacteria; for the second, we may measure the mass to be within a range that excludes zero. The negation of the two statements,  \emph{``there is no extra-terrestrial life"} and \emph{``the mass of the neutrino is exactly zero"} are not for the reason we outlined before.

The ability to verify a statement, therefore, does not implies the ability to verify its negation. It implies, though, the ability to refute the negation. That is, if I verified that \emph{``there is extra-terrestrial life"} then I have refuted that \emph{``there is no extra-terrestrial life"}. This means that we can always re-frame a verification problem into a refutation problem and vice-versa. As such, we can only concentrate on verification and that is why our definition of scientific observation is solely based on that.

Conversely, note that not verifying a statement does not mean it is refuted. After seeing a flying object in the sky, you may not be have been able to verify that \emph{``it is a duck"} because it was too far and you concluded that \emph{``it is a bird"}. Yet, it may still be a duck.

These subtleties are why we are framing the problem in terms of observations and not questions. A question such that \emph{``is there extra-terrestrial life?"} only admits ``Yes" and ``I don't know" as experimental answers. A question such that \emph{``is the mass of the neutrino exactly zero?"} only admits ``No" and ``I don't know" as experimental answers. A question such that \emph{``what is that in the sky?"} may have answers at different degree of precision (\emph{``it is an animal"}, \emph{``it is a bird"}, \emph{``it is a duck"}) that are not mutually exclusive. Therefore, in this framework, we enumerate all the possible answers and keep track of the ones that can be verified experimentally.

Combining observations with conjunction, i.e. the logical AND, is more straightforward. To verify that \emph{``the sauce is sweet and sour"} we can first verify that \emph{``the sauce is sweet"} and then that \emph{``the sauce is sour"}. If both are successful, then we have verified the conjunction. That is: if we have two or more scientific observations we can always verify the logical AND just by verifying each element one at a time. Yet, the number of observations needs to be finite or we would never end the verification process.

Combining observations with disjunction, i.e. the logical OR, is also straightforward. To verify that \emph{``the sauce is at least sweet or sour"} we can first verify that \emph{``the sauce is sweet"}. If that succeeds that's enough: the sauce is at least sour. If not, we verify that \emph{``the sauce is sour"}. That is: if we have two or more scientific observations we can always verify the logical OR just by verifying each element one at a time. As soon as one is verified, the disjunction is verified. Because we stop at the first verification, the number of observations we combine in a logical OR can be countably infinite. As long as one verification succeeds, which will always be the case when the overall verification succeeds, it does not matter how many elements we are not going to verify later.

This discussion leads to the following definitions and properties.

\begin{defn}
	Let $A_1, A_2, ... , A_n, ...$ be an infinite sequence of n scientific observations. We define $\bigwedge\limits_{i=1}^{n} A_i$ as the scientific observation that is verified by verifying the first $n$ scientific observations in the sequence. We define $\bigvee\limits_{i=1}^{\infty} A_i$ as the scientific observation that is verified by verifying at least one element in the infinite sequence.
\end{defn}

\section{Summary}

\begin{table}[h]
	\centering
\begin{tabular}{p{0.20\textwidth} p{0.7\textwidth}}
	Math/Topology & Science/Physics \\ 
	\hline 
	Open set & Verifiable set. We can verify experimentally that an element is within the set  \\ 
	Closed set & Refutable set. We can verify experimentally that an element is not in the set \\ 
	Continuous \newline function &  A function between two sets of physically distinguishable elements, a function that preserves distinguishability \\
	Homeomorphism &  A bijective function between two sets of physically distinguishable elements \\
\end{tabular} 
	\caption{Topology to physics dictionary}
\end{table}

	
\end{document}