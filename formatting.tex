\usepackage[top=3cm,bottom=3cm,left=3cm,right=3cm,headsep=10pt,letterpaper]{geometry} % Page margins

% Theorem definitions using amsthm

\usepackage{amsthm}
\usepackage{amsmath}
\usepackage{amssymb}

\usepackage{multirow}

% Remove line spaces between items of enumerate and itemize
\usepackage{enumitem}
\setlist{noitemsep}


% Adds double bracket symbols
\usepackage{stmaryrd}

% Latex symbol guide at http://mirrors.ibiblio.org/CTAN/info/symbols/comprehensive/symbols-letter.pdf

% LOGIC symbols
% -------------

% Allows to create negation symbols
\usepackage{MnSymbol}

\DeclareMathOperator{\truth}{truth}
\DeclareMathOperator{\possFn}{poss}
\DeclareMathOperator{\result}{result}
\DeclareMathOperator{\idFn}{id}

\def\TRUE{\textsc{true}}
\def\FALSE{\textsc{false}}

\def\SUCCESS{\textsc{success}}
\def\FAILURE{\textsc{failure}}
\def\UNDEF{\textsc{undefined}}

% Symbols for statements set
\def\stmtSet{\mathcal{S}}
\def\vstmtSet{\mathcal{S}_\textsf{v}}
\def\dstmtSet{\mathcal{S}_\textsf{d}}


% Symbols for tautology and contradiction
\def\tautology{\top}
\def\contradiction{\bot}

% Symbols for "compatibility" and "incompatibility"
\def\comp{\doublefrown}
\def\ncomp{\ndoublefrown}

% Symbols for "narrower" and "wider"
\def\narrower{\preccurlyeq}
\def\nnarrower{\npreccurlyeq}
\def\broader{\succcurlyeq}
\def\nbroader{\nsucccurlyeq}


% Symbol for "independent" and "correlated"
\def\indep{\upmodels}
\def\nindep{\nupmodels}

% Aliases for logical operations
\def\AND{\wedge}
\def\bigAND{\bigwedge}
\def\OR{\vee}
\def\bigOR{\bigvee}
\def\NOT{\neg}


% Formatting for statements
\newcommand{\stmt}[1][s] {\mathsf{#1}}
% Formatting for experimental tests
\newcommand{\expt}[1][e] {\mathsf{#1}}
\newcommand{\exptSet}{\mathcal{E}}
% Formatting for observations
\newcommand{\obs}[1][s] {\mathsf{#1}}
% Formatting for possibilities

% Formatting for experimental domain
\newcommand{\edomain}[1][D] {\mathcal{#1}}

% Formatting for theoretical domain
\newcommand{\tdomain}[1][D] {\bar{\mathcal{#1}}}

\newcommand{\basis}[1][B] {\mathcal{#1}} % Basis

% Formatting for experimental relationships
\newcommand{\erel}[1][r] {#1}

% Formatting for sentence statements
\newcommand{\statement}[1] {\emph{``#1"}}


\usepackage{xcolor} % Required for specifying colors by name
\definecolor{sectionNumbers}{RGB}{44, 103, 0}


\renewcommand\thesubsection{\thesection.\Alph{subsection}}
\renewcommand{\theequation}{\thechapter.\arabic{equation}}

\newtheorem{assump}{Assumption}
\renewcommand*{\theassump}{\Roman{assump}}

\newtheorem{axiom}[equation]{Axiom}
\newtheorem{defn}[equation]{Definition}
\newtheorem{prop}[equation]{Proposition}
\newtheorem{coro}[equation]{Corollary}
\newtheorem{thrm}[equation]{Theorem}

%\theoremstyle{definition}

\newenvironment{rationale}{\emph{Rationale}.}{\qed}
\newenvironment{justification}{\emph{Justification}.}{\qed}
\renewenvironment{proof}{\emph{Proof}.}{\qed}

% Style for math section
\RequirePackage[framemethod=default]{mdframed} % Required for creating the theorem, definition, exercise and corollary boxes
\newmdenv[skipabove=7pt,
skipbelow=7pt,
rightline=false,
leftline=true,
topline=false,
bottomline=false,
linecolor=sectionNumbers,
backgroundcolor=black!2,
innerleftmargin=5pt,
innerrightmargin=5pt,
innertopmargin=5pt,
leftmargin=0cm,
rightmargin=0cm,
linewidth=4pt,
innerbottommargin=5pt]{mathSection}


%----------------------------------------------------------------------------------------
%	SECTION NUMBERING IN THE MARGIN
%----------------------------------------------------------------------------------------

\makeatletter
\renewcommand{\@seccntformat}[1]{\llap{\textcolor{sectionNumbers}{\csname the#1\endcsname}\hspace{1em}}}                    
\renewcommand{\section}{\@startsection{section}{1}{\z@}
	{-4ex \@plus -1ex \@minus -.4ex}
	{1ex \@plus.2ex }
	{\normalfont\large\sffamily\bfseries}}
\renewcommand{\subsection}{\@startsection {subsection}{2}{\z@}
	{-3ex \@plus -0.1ex \@minus -.4ex}
	{0.5ex \@plus.2ex }
	{\normalfont\sffamily\bfseries}}
\renewcommand{\subsubsection}{\@startsection {subsubsection}{3}{\z@}
	{-2ex \@plus -0.1ex \@minus -.2ex}
	{.2ex \@plus.2ex }
	{\normalfont\small\sffamily\bfseries}}                        
\renewcommand\paragraph{\@startsection{paragraph}{4}{\z@}
	{-2ex \@plus-.2ex \@minus .2ex}
	{.1ex}
	{\normalfont\small\sffamily\bfseries}}