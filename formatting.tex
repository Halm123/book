\usepackage[top=3cm,bottom=3cm,left=3cm,right=3cm,headsep=10pt,letterpaper]{geometry} % Page margins

\usepackage{assumptionsofphysics}

% Latex symbol guide at http://mirrors.ibiblio.org/CTAN/info/symbols/comprehensive/symbols-letter.pdf


\usepackage{xcolor} % Required for specifying colors by name
\definecolor{sectionNumbers}{RGB}{44, 103, 0}



% Style for math section
\RequirePackage[framemethod=default]{mdframed} % Required for creating the theorem, definition, exercise and corollary boxes
\newmdenv[skipabove=7pt,
skipbelow=7pt,
rightline=false,
leftline=true,
topline=false,
bottomline=false,
linecolor=sectionNumbers,
backgroundcolor=black!2,
innerleftmargin=5pt,
innerrightmargin=5pt,
innertopmargin=5pt,
leftmargin=0cm,
rightmargin=0cm,
linewidth=4pt,
innerbottommargin=5pt]{mathSection}


%----------------------------------------------------------------------------------------
%	SECTION NUMBERING IN THE MARGIN
%----------------------------------------------------------------------------------------

\makeatletter
\renewcommand{\@seccntformat}[1]{\llap{\textcolor{sectionNumbers}{\csname the#1\endcsname}\hspace{1em}}}                    
\renewcommand{\section}{\@startsection{section}{1}{\z@}
	{-4ex \@plus -1ex \@minus -.4ex}
	{1ex \@plus.2ex }
	{\normalfont\large\sffamily\bfseries}}
\renewcommand{\subsection}{\@startsection {subsection}{2}{\z@}
	{-3ex \@plus -0.1ex \@minus -.4ex}
	{0.5ex \@plus.2ex }
	{\normalfont\sffamily\bfseries}}
\renewcommand{\subsubsection}{\@startsection {subsubsection}{3}{\z@}
	{-2ex \@plus -0.1ex \@minus -.2ex}
	{.2ex \@plus.2ex }
	{\normalfont\small\sffamily\bfseries}}                        
\renewcommand\paragraph{\@startsection{paragraph}{4}{\z@}
	{-2ex \@plus-.2ex \@minus .2ex}
	{.1ex}
	{\normalfont\small\sffamily\bfseries}}