\chapter{Classical mechanics}

The standard view in physics is that classical mechanics is perfectly understood. It has three different but equivalent formulations, the oldest of which, Newtonian mechanics, is based on three laws. Classical mechanics is the theory of point particles that follow those laws. Unfortunately, this view is incorrect.

We will see that the three formulations are not equivalent, in the sense that there are physical systems that are Newtonian but not Hamiltonian and vice-versa. There are also a number of questions that have been left unanswered, such as the precise nature of the Hamiltonian or the Lagrangian, and what exactly the principle of stationary action represents physically. While shedding light on these issues, we will also find that classical mechanics already contains elements that are typically associated with other theories, such quantum mechanics (uncertainty principle, anti-particles), thermodynamics/statistical mechanics (thermodynamic and information entropy conservation) or special relativity (energy as the time-component of a four-vector). In other words, the common understanding of classical mechanics is quite shallow, and its foundations are not separate from the ones of classical statistical mechanics and special relativity.

What reverse physics shows is that the central assumption underneath classical mechanics is that of \textbf{infinitesimal reducibility}: a classical system can be thought of as made of parts, which in turn are made of parts and so on; studying the whole system is equivalent to studying all its infinitesimal parts. This assumption, together with the assumption of \textbf{independence of degrees of freedom}, is what gives us the structure of classical phase space with conjugate variables. The additional assumption of \textbf{determinism and reversibility}, the fact that the description of the system at one time is enough to predict its future or reconstruct its past, leads us to Hamiltonian mechanics. On the other hand, assuming \textbf{kinematic equivalence}, the idea that trajectories in space are enough to reconstruct the state of the system and vice-versa, leads us to Newtonian mechanics. The combination of all assumptions, instead, leads to Lagrangian mechanics and, in particular, to massive particles under (scalar and vector) potential forces.

\section{Formulations of classical mechanics}

In this section we will briefly review the three main formulations of classical mechanics. Our task is not to present them in detail, but rather to provide a brief summary of the equations so that we can proceed with the comparison. In particular, given that different conventions are used across formulations, within the same formulation and among different contexts (e.g. relativity, symplectic geometry), we will want to make the notation homogeneous to allow easier comparisons.

\subsection{Newtonian mechanics}

For all formulations, the system is modeled as a collection of point particles, though we will mostly focus on the single particle case. For a Newtonian system, the state of the system at a particular time $t$ is described by the position $x^i$ and velocity $v^i$ of all its constituents. Each particle has its mass $m$, not necessarily constant in time, and, for each particle, we define kinetic momentum as $\Pi^i = m v^i$.\footnote{We will use the letter $t$ for the time variable, $x$ for position and $v$ for velocity, which is a very common notation in Newtonian mechanics. However, we will keep using the same letters in Lagrangian mechanics as well, instead of $q$ and $\dot{q}$, for consistency. Given that the distinction between kinetic and conjugate momentum is an important one, we will denote $\Pi$ the former and $p$ the latter. The Roman letters $i,j,k,...$ will be used to span the spatial components, while we will use the Greek letters $\alpha, \beta, \gamma, ...$ to span space-time components. Unlike some texts, $x^i$ do not represent Cartesian coordinates, and therefore they should be understood already as generalized coordinates.}

The evolution of our system is given by the Newton's second law:
\begin{equation}\label{rp-cm-NewtonsSecondLaw}
	F^i(x^j, v^k, t) = d_t \Pi^i.
\end{equation}
Mathematically, if the forces $F^i$ are Lipschitz continuous, then the solution is unique. That is, given position and velocity at a given time, we can predict the position and velocity at future times. We will assume a Newtonian system has this property.

An important aspect of Newtonian mechanics is that the equations are not invariant under coordinate transformation. To distinguish between apparent forces (i.e. those dependent on the choice of frame) and the real ones, we assume the existence of inertial frames. In an inertial frame there are no apparent forces, and therefore a free system (i.e. no forces) with constant mass proceeds in a linear uniform motion, or stays still.\footnote{Recall that linear motion simply means that it describes a line in space, while uniform motion means that the speed is constant. Therefore we can have linear non-uniform notion (e.g. an object accelerated along the same direction) or a non-linear uniform motion (e.g. an object going around in a circle at constant speed).}

\subsection{Lagrangian mechanics}

The state for a Lagrangian system is also given by position $x^i$ and velocity $v^i$. The dynamics is specified by a single function $L(x^i, v^j, t)$ called Lagrangian. For each spatial trajectory $x^i(t)$ we define the action as $\mathcal{A}[x^i(t)] = \int_{t_0}^{t_1} L(x(t), d_t x(t), t) dt$.\footnote{For derivatives, we will use the shorthand $d_t$ for $\frac{d}{dt}$ and $\partial_{x^i}$ for $\frac{\partial}{\partial x^i}$. } The trajectory taken by the system is the one that makes the action stationary:
\begin{equation}
\delta \mathcal{A}[x^i(t)] = \delta \int_{t_0}^{t_1} L\left(x(t), d_t x(t), t\right) dt=0
\end{equation}
The evolution can equivalently be specified by the Euler-Lagrange equations:
\begin{equation}\label{rp-cm-EulerLagrange}
	\partial_{x^i}L=d_t \partial_{v^i} L.
\end{equation}

Note that not all Lagrangians lead to a unique solution. For example, $L=0$ will give the same action for all trajectories and therefore, strictly speaking, all trajectories are possible. The stationary action leads to a unique solution if and only if the Lagrangian is hyperregular, which means the Hessian matrix $\partial_{v^i}\partial_{v^j} L$ is invertible. Like in the Newtonian case, we will assume Lagrangian systems satisfy this property.

Unlike Newton's second law, both the Lagrangian and the Euler-Lagrange equations are invariant under coordinate transformations. This means that Lagrangian mechanics is particularly suited to study the symmetries of the system.

\subsection{Hamiltonian mechanics}

In Hamiltonian mechanics, the state of the system is given by position $q^i$ and conjugate momentum $p_i$. The dynamics is specified by a single function $H(q^i, p_j, t)$ called Hamiltonian.\footnote{We use a different symbol for position in Hamiltonian mechanics because, while it is true that $q^i = x^i$, it is also true that $\partial_{q^i} \neq \partial_{x^i}$: the first derivative is taken at constant conjugate momentum while the second is taken at constant velocity. This creates absolute confusion when mixing and comparing Lagrangian and Hamiltonian concepts, which our notation avoids completely.} The evolution is given by Hamilton's equations:
\begin{equation}\label{rp-cm-HamiltonEq}
	\begin{aligned}
		d_t q^i = \partial_{p_i} H \\
		d_t p_i = - \partial_{q^i} H \\
	\end{aligned}
\end{equation}

We will again that these equations must yield a unique soltuion, which means the Hamiltonian must be at least differentiable, and the derivatives must at least be Lipschitz continuous.

Hamilton's equations are invariant as well. The Hamiltonian itself is a scalar function which is often considered (mistakenly as we'll see later) invariant. This formulation is the most suitable for statistical mechanics as volumes of phase space correctly count the number of possible configurations.

\section{Inequivalence of formulations}

It is often stated in physics books that all three formulations of classical mechanics are equivalence. We will look at this claim in detail, and conclude that this is not the case: there are systems that can be described by one formulation and not another. More precisely, the set of Lagrangian systems is exactly the intersection of Newtonian and Hamiltonian systems.

We will consider two formalism equivalent if they can be applied exactly to the same systems. That is, Newtonian and Lagrangian mechanics are equivalent if any system that can be described using Newtonian mechanics can also be described by Lagrangian mechanics and vice-verse. In general, in physics great emphasis is put on systems that can indeed be studied by all three, leaving the impression that this always doable.\footnote{If one asks the average physicist whether Newtonian and Hamiltonian mechanics, the answer will be most of the time enthusiastically positive. If one then asks to provide the Hamiltonian for a damped harmonic oscillator, the typical reaction is annoyance to the nonsensical question (damped harmonic oscillator do not conserve energy), followed by a realization and partial retraction of the previous claim. The moral of the story is to never take these claims at face value.} However, just with a cursory glance, we realize that this can't possibly be the case.

The dynamics of a Newtonian system, in fact, is specified by three independently chosen functions of position and velocity, the forces applied to each degree of freedom. On the other hand, the dynamics of Lagrangian and Hamiltonian systems are specified by a single function of position and velocity/momentum, the Lagrangian the Hamiltonian. Intuitively, there are more choices in the dynamics for Newtonian systems than for Lagrangian and Hamiltonian.

Now, the reality is a bit trickier because the mathematical expression of the forces is not enough to fully characterize the physical system. We need to know in which frame we are, what coordinates are being used and the mass of the system, which is potentially a function of time. On the Lagrangian side, note that the Euler-Lagrange equations are homogeneous in $L$. This means that multiplying $L$ by a constant leads to the same solutions, meaning that the same system can be described by more than one Lagrangian. The converse is also true: if one system is half as massive and is subjected to a force field half as intense, the resulting Lagrangian is also simply rescaled by a constant factor. Therefore the map between Lagrangians and Lagrangian system is not one-to-one: it is many-to-may. This is why we should never look simply at mathematical structures if we want to fully understand the physics they describe.

Regardless, our task is at the moment much simpler: we only need to show that there are Newtonian systems not expressible by Lagrangian or Hamiltonian mechanics. We can therefore limit ourselves to systems with a specific constant mass $m$ in an inertial frame. Every possible expression of the force is allowed and will lead to a unique expression of the acceleration $a^i=F^i(x^i, v^i, t)/m$ which means a unique set of possible trajectories for each expression of the force. Now, we can think of trying to write a Lagrangian for each of those systems. Assuming that this is possible, the acceleration $a^i=F^i[L]/m$ is going to be some functional of that Lagrangian, which, given the Euler-Lagrange equations \ref{rp-cm-EulerLagrange} must be continuous: for small variation of the Lagrangian we must have a small variation of the equations of motion and therefore of the acceleration. But a continuous surjective map from the space of a single function (i.e. the Lagrangian) to the space of multiple functions (i.e. those that specify the forces) does not exist, and therefore there must be at least one Netwonian system with constant mass expressed in an inertial frame that is not describable using Lagrangian mechanics. The same argument applies for Hamiltonian mechanics, since the dynamics in this case is also described by a single function in the same number of arguments. We therefore reach the following conclusion:
\begin{equation}
	\textrm{Not all Netwonian systems are Lagrangian and/or Hamiltonian.}
\end{equation}

We now want to understand whether all Lagrangian systems are Newtonian. Given what we discussed, we cannot expect to reconstruct the mass and force uniquely from the expression of the Lagrangian. We consider the mass and the frame fixed by the problem, together with the Lagrangian, and therefore we must only see whether we can indeed find a unique expression for the acceleration. From the Euler-Lagrange equations \ref{rp-cm-EulerLagrange} we can write
\begin{equation}
	\begin{aligned}
	\partial_{x^i}L&=d_t \partial_{v^i} L=\partial_{q^j} \partial_{v^i} L \, d_t q^j + \partial_{v^k} \partial_{v^i} L \, d_t v^k = \partial_{q^j} \partial_{v^i} L \, v^j + \partial_{v^k} \partial_{v^i} L \, a^k \\
	\partial_{v^k} &\partial_{v^i} L \, a^k = \partial_{x^i}L - \partial_{q^j} \partial_{v^i} L \, v^j .
	\end{aligned}
\end{equation}
To be able to write the acceleration explicitly, we must be able to invert the Hessian matrix $\partial_{v^k} \partial_{v^i} L$. As we noted before, this is exactly the condition for which the principle of stationary actions leads to a unique solution, and we can better understand why. If it is not invertible at a point, the determinant is zero and therefore one eigenvalue is zero. The corresponding eigenvector correspond to a direction for which the equation tells us nothing, and therefore a variation of the acceleration in that direction will not change the action. This is why the invertibility of the Hessian is required to obtain unique solutions.

What we find, then, is that for any Lagrangian system, which we assume to have a unique solution, we can explicitly write the acceleration as a function of position, velocity and time. Therefore
\begin{equation}
	\textrm{All Lagrangian systems are Newtonian.}
\end{equation}

Now we turn our attention to Hamiltonian mechanics and, similarly, we ask whether we can express the acceleration as a function of position and velocity. We have
\begin{equation}
	\begin{aligned}
		a^i &= d_t v^i = d_t d_t q^i = d_t \partial_{p_i} H = \partial_{q^j} \partial_{p_i} H d_t q^j + \partial_{p_k} \partial_{p_i} H d_t p_k \\
		&= \partial_{q^j} \partial_{p_i} H \partial_{p_j} H - \partial_{p_k} \partial_{p_i} H \partial_{q^k} H.
	\end{aligned}
\end{equation}
This tells us that the acceleration is always an explicit function, but it is, in general, an explicit function of position and momentum, not of position and velocity. To change the expression, we need to be able to write the momentum as a function of position and velocity. Note that Hamilton's equations already give a way to express the velocity in terms of position and momentum, we just need that expression to be invertible, which means the Jacobian must be invertible. We have:
\begin{equation}
	\left|\partial_{p_i} v^j\right| = \left|\partial_{p_i}\partial_{p_j} H\right| \neq 0 .
\end{equation}
To be able to express momentum as a function of position and velocity, then, we need the Hessian of the Hamiltonian to be invertible, to have non-zero determinant.

Note that we had no such requirement for the Hamiltonian. For example, $H=0$ leads to $d_t q^i = 0$ and $d_t p_i = 0$, which have unique solutiona: both position and momentum are constants of motion. The Hessian, being the zero matrix, is not invertible, and in fact we cannot write momentum as a function of position and velocity: velocity is always zero in all cases. Though this case may not be physically interesting, it is a perfectly valid Hamiltonian system and shows that we should always check the trivial mathematical case. However, let us go through a more physically meaningful case.

\textbf{Photon as a particle}. If we want to treat the photon as a classical particle, we can write the Hamiltonian by expressing the energy as a function of momentum
\begin{equation}
	H=\hbar | \omega| = c \hbar |k| = c |p|.
\end{equation}
If we apply Hamilton's equations, we have
\begin{equation}
	\begin{aligned}
		d_t q^i &= c \frac{p^i}{|p|} \\
		d_t p_i &= 0.
	\end{aligned}
\end{equation}
That is, the norm of the velocity is always $c$, the momentum decides its direction, and the momentum itself does not change in time. This is indeed the motion of a free photon. One can confirm, through tedious calculation, that the determinant of the Hessian is indeed zero, yet it is easier and more physically instructive to see that we cannot reconstruct the momentum from the velocity. Relativistically, all photons travel along the geodesics at the same speed, therefore two photons that differ only by the magnitude of the momentum will travel the same path.

Hamiltonian systems that are also Newtonian, then, needs to satisfy this extra condition, so let us give it a name.
\renewcommand{\theassump}{KE}%
\begin{assump}[Kinematic Equivalence]\label{assum_kineq}
	The kinematics of the system is enough to reconstruct its dynamics and vice-versa.
\end{assump}
\renewcommand{\theassump}{\Roman{assump}}%
By kinematics we means the motion in space and time and by dynamics we mean the state and its evolution in phase space. We will need to analyze the difference between the two more in detail, but we should first finish our comparison between the different formulations.

Summing up, we find that
\begin{equation}
	\textrm{Not all Hamiltonian systems are Newtonian: only those for which  \ref{assum_kineq} is valid.}
\end{equation}


We now need to compare Lagrangian and Hamiltonian systems. The task is a lot easier because we already have a precise way to connect the two. If we are given a Lagrangian $L$, we define the conjugate momentum $p_i = \partial_{v^i} L$ and the Hamiltonian $H = p_i v^i - L$. If we are given a Hamiltonian $H$, we can define a Lagrangian $L = p_i v^i - H$ and a velocity $v^i = d_t q^i = \partial_{p^i} H$. The only detail that need to be understood is whether this can be done for all Lagrangian and Hamiltonian systems.

While these expression are always defined, we need to check whether we can change variables; whether we can write the Lagrangian in terms of position and velocity and the Hamiltonian in terms of position and momentum. Going from a Hamiltonian to a Lagrangian, it again means that we can write momentum as a function of velocity, and therefore assumption \ref{assum_kineq} must hold. This makes sense: if all Lagrangian systems are Newtonian, and \ref{assum_kineq} was required for a Hamiltonian system to be Newtonian, then it also required for a Hamiltonian system to be Lagrangian. But the connection is stronger: \ref{assum_kineq} is the only additional assumption we need to be able to write a Lagrangian given a Hamiltonian.

Going from a Lagrangian to a Hamiltonian, it means that we can write velocity as a function of position and momentum. Note that since we define conjugate momentum as the derivative of the Lagrangian, we can already express momentum as a function of position and velocity, which means we are simply asking that expression to be invertible. This is, again, assumption \ref{assum_kineq}, just in the opposite direction. We must have
\begin{equation}
	0 \neq \left| \partial_{v^i} p_j \right| = \left| \partial_{v_i} \partial_{v_j} L \right|.
\end{equation}
This means that assumption \ref{assum_kineq} is exactly the invertibility of the Hessian, the condition for unique solution of the Lagrangian. All Lagrangian systems that admit unique solutions, then, satisfy assumption \ref{assum_kineq}. In fact, we can see that the Hessians determinants are related
\begin{equation}
	\left| \partial_{v_i} \partial_{v_j} L \right| = \left| \partial_{v^i} p_j \right| = \left| \partial_{p_i} v^j \right|^{-1} = \left|\partial_{p_i}\partial_{p_j} H\right|^{-1}.
\end{equation}
This means that every Lagrangian admits a Hamiltonian, but not every Hamiltonian admits a Lagrangian. Only the Hamiltonian systems for which \ref{assum_kineq} is valid will also be Lagrangian system, with a guaranteed unique solution given that \ref{assum_kineq} is exactly the assumption needed for that as well. Therefore we conclude that
\begin{equation}
	\textrm{Lagrangian systems are exactly those Hamiltonian systems for which \ref{assum_kineq} is valid.}
\end{equation}

The relationship between the different formulations, then, can be summarized with the following Venn diagram.

TODO: Venn diagram of the different formulations

Lagrangian mechanics is exactly the intersection of Newtonian mechanics and Lagrangian mechanics.

We have found that \ref{assum_kineq} is a constitutive assumption of Lagrangian mechanics, and that it clearly marks which Hamiltonian systems are Newtonian/Lagrangian. By constitutive assumption we mean an assumption that must be taken, either explicitly or implicitly, for a theory to be valid. But what makes a system Hamiltonian and what makes a system Newtonian? Can we find a full set of constitutive assumptions for classical mechanics?

\section{Kinematics vs dynamics}

We have seen the importance between kinematics and dynamics. In this section we will explore this link more deeply and come to the following conclusion: the kinematics of a system is not enough to reconstruct its dynamics. 

Let us first review exactly what the kinematics and dynamics are. Given a system, its kinematics is the description of its motion in space and time. Position, velocity, and acceleration are kinematic variable because they describe the motion. Kinematics is what Galileo studied and started to give a rigorous account of. The dynamics, instead, describes the cause of such motion. Force, mass, momentum, energy are dynamic quantities as they are used to describe why a body moves in a particular way. Dynamics is what Newton introduced and his second law, expressed as $F=ma$, clearly shows the link.

The link between the two concepts seems important given the constitutive role of \ref{assum_kineq} in Lagrangian mechanics. Moreover, while both Newtonian and Hamiltonian mechanics are dynamical theories, in the sense that quantities like force and momentum are intrinsic parts of the theory, Lagrangian mechanics seems to be a purely kinematic theory, as it is described only by kinematic variables like position and velocity. Therefore it seems useful to characterize the kinematics-dynamics link as much as possible. Let's analyze a concrete example.

Suppose we are given the following equation:
\begin{equation}\label{rp-cm-frictionEquation}
	m a = - b v .
\end{equation}
The equation is in terms of kinematics variables and, given initial conditions $x_0$ and $v_0$, it admits a unique solution, a unique trajectory.
The solution is
\begin{equation}
	\begin{aligned}
	x(t)&= x_0 + v_0 \frac{m}{b} \left( 1 - e^{-\frac{b}{m}t}\right) \\
	v(t)&= v_0 e^{-\frac{b}{m}t} \\
	a(t)&= - \frac{b}{m} e^{-\frac{b}{m}t}
	\end{aligned}
\end{equation}
Can we reconstruct the forces acting on this system?

The obvious answer seems to be that the constant $m$ represents the mass of the system and $F = -bv$ the force. This is the case of a particle under linear drag:  the system is subjected to a frictional force that is proportional and opposite to the velocity. If we set the Lagrangian
\begin{equation}\label{rp-cm-frictionLagrangian}
	L = \frac{1}{2} m v^2 e^{\frac{b}{m}t}.
\end{equation}
and apply the Euler-Lagrange equation \ref{rp-cm-EulerLagrange} we have
\begin{equation}
	\begin{aligned}
	\partial_x L &= 0 = d_t \partial_v L = d_t \left(m v e^{\frac{b}{m}t} \right)=mae^{\frac{b}{m}t} - \frac{b}{m} m v e^{\frac{b}{m}t} = e^{-\frac{b}{m}t}(ma + bv) \\
	ma &= - bv.
	\end{aligned}
\end{equation}
Therefore we have a Lagrangian for the system. We can also find a Hamiltonian
\begin{equation}
	\begin{aligned}
	p &= \partial_v L = m v e^{\frac{b}{m}t} \\
		v &= \frac{p}{m} e^{-\frac{b}{m}t} \\
		H &= p v - L = p \frac{p}{m} e^{-\frac{b}{m}t} - \frac{1}{2} m \left( \frac{p}{m} e^{-\frac{b}{m}t} \right)^2 e^{\frac{b}{m}t} = \frac{p^2}{m}  e^{-\frac{b}{m}t} - \frac{1}{2} \frac{p^2}{m}  e^{-\frac{b}{m}t} \\ 
		&=\frac{1}{2} \frac{p^2}{m}  e^{-\frac{b}{m}t}
	\end{aligned}
\end{equation}
and apply Hamilton's equations \ref{rp-cm-HamiltonEq}
\begin{equation}
	\begin{aligned}
		d_t q &= \partial_p H = \frac{p}{m}  e^{-\frac{b}{m}t} \\
		d_t p &= - \partial_q H = 0. 
	\end{aligned}
\end{equation}
The second equation tells us momentum is constant, and therefore the first equation reduced to simply the derivative of position. We have
\begin{equation}
	\begin{aligned}
	q(t) &= q_0 + \frac{p_0}{b} \left( 1 - e^{-\frac{b}{m}t}\right) \\
	p(t) &= p_0.
	\end{aligned}
\end{equation}

The kinematics works perfectly, but the dynamics seems off. First of all, based on the physics, one would expect the conjugate momentum to be decreasing in time
\begin{equation}
	p(t)=m v(t) = m v_0 e^{-\frac{b}{m}t}.
\end{equation}
However, conjugate momentum is a constant of motion. For the energy, we would expect the Hamiltonian to match the kinetic energy
\begin{equation}
	E(t)=\frac{1}{2} m v^2(t) = \frac{1}{2} m v_0^2 e^{-2\frac{b}{m}t}
\end{equation}
but if we express the Hamiltonian in terms of velocity we have
\begin{equation}
	H(t)=\frac{1}{2} \frac{p^2}{m} e^{-\frac{b}{m}t} = \frac{1}{2} \frac{1}{m} \left( m v(t) e^{\frac{b}{m}t} \right)^2 e^{-\frac{b}{m}t}= \frac{1}{2} m v^2(t) e^{\frac{b}{m}t} = \frac{1}{2} m v_0^2 e^{-\frac{b}{m}t}.
\end{equation}
That is, the energy decreases more slowly than it should. This is not good.

Now, it is true that conjugate momentum in not the same as kinetic momentum. But the difference, as we will see much more clearly later, is caused by non-inertial non-Cartesian coordinate systems and/or the presence of vector potential forces.\footnote{The relationship is $p_i = m g_{ij} v^j + q A_i$. This reduces to $p_i = m v^i$ if and only if we are in an inertial frame with Cartesian coordinates (i.e. $g_{ij}=\delta_{ij}$) and no forces $A_i = 0$} We are not at all in that case. Also, note that at time $t=0$ the momentum and the energy do match what we expect, but not after. Therefore imagine a situation where friction is non-negligible only in a particular region. We would expect $p=mv$ to be valid before it enters, but not when it comes out. But wouldn't it come out in another region where we would expect $p=mv$ to work? This is strange. How should we proceed?

As it is typical in reverse physics, we will assume that things work in a reasonable way and that we simply have the wrong connection between physics and math. Recall that we started just with an equation, and we then interpreted $m$ to be the mass of the system. Let's just assume that $m$ is a constant with units of mass and define the actual mass of the system as the ratio between conjugate momentum and velocity. We have 
\begin{equation}
	\begin{aligned}
	\hat{m}(t) &= p(t) / v(t) = m e^{\frac{b}{m}t} \\
	p(t) &= mv(t)e^{\frac{b}{m}t} = \hat{m}(t) v(t) \\
	H(t) &= \frac{1}{2} \frac{p^2(t)}{m}  e^{-\frac{b}{m}t} = \frac{1}{2} \frac{p^2(t)}{\hat{m}} = \frac{1}{2} \hat{m}(t) v^2(t) = E(t)
	\end{aligned}
\end{equation}
Now everything actually works perfectly: the relationship between kinetic and conjugate momentum is respected, the Hamiltonian matches the kinetic energy. We just have a variable mass system. How and why does this work exactly?

Let us expand Newton's second law for a variable mass system. We have:
\begin{equation}
	\begin{aligned}
		F^i &= d_t (\hat{m}v^i) = d_t \hat{m} \, v^i + \hat{m} a^i \\
		\hat{m} a^i &= F^i - d_t \hat{m} v^i
	\end{aligned}
\end{equation}
In particular, for our one dimensional case, let us set $F=0$ and substitute $\hat{m}$
\begin{equation}
	\begin{aligned}
		m e^{\frac{b}{m}t} a &= 0 - d_t m e^{\frac{b}{m}t} v = -\frac{b}{m} m e^{\frac{b}{m}t} v \\
		ma &= -bv.
	\end{aligned}
\end{equation}
Therefore the same equation, the same kinematics, applies to a variable mass system that increases the mass over time. You can imagine, for example, a body that is absorbing mass from all directions, so that the balance of forces on the body is zero. The body, then, is not slowing down because of friction. It is slowing down because energy and momentum are conserved, and if the mass is increasing, the velocity must be decreasing.

In Newtonian mechanics, we can readily distinguish these two cases because we have to be explicit about forces and masses. In Hamiltonian mechanics things are a bit more difficult because, as we will see later more precisely, conjugate momentum is not exactly kinetic momentum and the Hamiltonian is not exactly energy. Yet, conjugate momentum and Hamiltonian are not kinematic quantities, they are dynamic quantities and therefore we can see that these would be different in different cases. In Lagrangian mechanics this is even more difficult to see because it looks like a purely kinematic theory, while it is not: the Lagrangian itself is not a purely kinematic entity. As we saw, Lagrangian mechanics implicitly assumes \ref{assum_kineq}, which is a condition on the dynamics as well, and the Lagrangian itself is used to reconstruct conjugate momentum and the Hamiltonian. Moreover, if Lagrangian mechanics were a purely kinematic theory, and told us nothing about forces, energy and momentum, it would not be a complete formulation of classical mechanics.

So we have seen that the same kinematic equation can describe a constant mass dissipative system or variable mass non-dissipative system. Is that it? Not quite. Recall that we mentioned that kinetic and conjugate momentum will differ in non-inertial frames. Note that we implicitly assumed that $x$ and $t$ represented the variable for an inertial observer, in the same way that we originally assumed $m$ was the mass of the system. Could the same equation, then, be describing yet another system but in a non-inertial frame?

Let's compare the motion of a particle traveling at constant velocity in an inertial frame, using $t$ as time variable, and the motion of a particle decelerating exponentially, using $\hat{t}$ as time variables
\begin{equation}
	\begin{aligned}
		x(t) &= x_0 + v_0 t \\
		x(\hat{t}) &= x_0 + v_0 \frac{m}{b}\left(1-e^{-\frac{b}{m}\hat{t}}\right)
	\end{aligned}
\end{equation}
Note the striking similarity: we can simply set
\begin{equation}
	t = \frac{m}{b} \left(1-e^{-\frac{b}{m}\hat{t}}\right)
\end{equation}
which clearly takes us to a non-inertial frame since uniform motion is no longer uniform in the new frame.

Let's study how Newtown's second law changes if we make a change of time variable while keeping the position variables unchanged
\begin{equation}
	\begin{aligned}
		\hat{t}&=\hat{t}(t) \\
		F^i &= d_t  (m \, v^i) = d_t  (m \, d_t x^i) = d_t \hat{t} \, d_{\hat{t}}  (m \, d_t \hat{t} \, d_{\hat{t}} x^i) \\ &= d_t \hat{t} \, d_{\hat{t}}  (\hat{m} \, d_{\hat{t}} x^i) = d_t \hat{t} \, d_{\hat{t}}  (\hat{m} \, \hat{v}^i) = d_t \hat{t} \hat{F}^i \\
		\hat{m} &= m \, d_t \hat{t}.
	\end{aligned}
\end{equation}
This tells us that the second observer will see an effective mass rescaled exactly by the ratio of the time variables. Note that this is exactly what happens in special relativity: the clock for a boosted observer is dilated by a factor of $\gamma$ which is exactly the factor used in the relativistic mass.\footnote{It may be surprising to see a proto-relativistic effect showing up given that no assumption on space-time has been made. As we will see, these types of connections between different theories come up often in reverse physics.} If $t$ is the time variable for an inertial frame and $t(\hat{t})$ is a non-linear function, the resulting frame will be non-inertial and the observer will see and effective variable mass system.

If we look at our problem this way, the rescaling of the mass, then, is not due by a truly variable mass, but a variable effective mass due to the slowing down of the clock. The body slows down because the non-inertial time is slowing down and the body appears to stop because the clock becomes infinitely slow. While this might sound like a contrived case, these are exactly the type of situation a fully relativistic theory (i.e. one that works for all definitions of time and space variables) needs to take into account.

We can verify that this gives us the correct effective mass
\begin{equation}
	\begin{aligned}
	d_{\hat{t}} t  &=d_{\hat{t}} \left( \frac{m}{b} (1-e^{-\frac{b}{m}\hat{t}}) \right) =\frac{m}{b} d_{\hat{t}} (1-e^{-\frac{b}{m}\hat{t}}) = - \frac{m}{b} d_{\hat{t}} e^{-\frac{b}{m}\hat{t}} = + \frac{m}{b} \frac{b}{m} e^{-\frac{b}{m}\hat{t}} = e^{-\frac{b}{m}\hat{t}} \\
	\hat{m} &= m d_t \hat{t} = m (d_{\hat{t}} t)^{-1} = m e^{\frac{b}{m}\hat{t}}.
	\end{aligned}
\end{equation}
And we can verify that we get the same equation by plugging in the time transformation in Newton's second law with a zero force
\begin{equation}
	\begin{aligned}
		0 &= d_t  (m \, v) = d_t  (m \, d_t x) = d_t \hat{t} \, d_{\hat{t}}  (m \, d_t \hat{t} \, d_{\hat{t}} x) \\ &= e^{\frac{b}{m}\hat{t}} \, d_{\hat{t}}  (m e^{\frac{b}{m}\hat{t}} \, \hat{v}) = e^{\frac{b}{m}\hat{t}} \left( m \frac{b}{m} e^{\frac{b}{m}\hat{t}} \hat{v} + m e^{\frac{b}{m}\hat{t}} \hat{a} \right)  = e^{2\frac{b}{m}\hat{t}} \left( b \hat{v} + m \hat{a} \right) \\
		m \hat{a} &= - b \hat{v}.
	\end{aligned}
\end{equation}
Note that the expressions for momentum and energy will match because the system in the non-inertial frame does look like a variable mass system.

This last case highlights a more subtle issue. In the two previous cases we were in the same inertial frame, we saw the same trajectory, the same kinematics, but we couldn't tell whether we were looking at a fixed mass system under linear drag or a variable mass system: we couldn't tell the dynamics. Now, we have the same system, a constant mass particle under no forces, described in two different frames, one inertial and one not. The motion of the system will naturally have different representation in the different frames, but this does not mean the motion or the causes of motion are different: it's the same object. Therefore we have the same kinematics even though we have different expressions for the same trajectory. The expression $x(t)$, then, is not enough to define the kinematics if we do not know exactly what $x$ and $t$ represent physically, if the frame is not given.

While typically one proceeds by defining the frame first, here we have followed a different approach: we defined the system first (i.e. constant mass system under no forces proceeding in uniform linear motion) and then found the corresponding frame that matches the kinematics non-uniform motion. Given that Lagrangian and Hamiltonian mechanics are frame invariant, an intrinsic characterization of the system themselves is exactly what we should be looking for. Saying, for example, that a system is subjected to no forces or to a linear drag is not frame invariant because forces are not frame invariant.

It is clear that the type of apparent variable mass due to non-inertial frames is unavoidable if we want to have a consistent theory with invariant laws. Therefore both Lagrangian and Hamiltonian mechanics must include these cases. However, it is not exactly clear what to do for true variable mass systems. From a cursory look, it would seem that everything is fine and there is no harm in including them. Yet again, from a cursory look we seemed to have a Lagrangian for a particle under linear drag. As we will see later, there are implicit connections between Lagrangian/Hamiltonian mechanics on one side and thermodynamics, statistical mechanics and special relativity on the other. Given that it is not clear to us whether these connections hold or not,\footnote{For example, areas of phase space are connected to entropy. Does this connection hold with a variable mass system?} we will concentrate on the constant mass case from now on.

Let's recap what we learned. The biggest point is that we can't simply look at the kinematics and understand what are the causes of motion. The different formulations have different way to relate the dynamics and the kinematics. Newtonian mechanics is the most clear about the dynamics as it makes us clearly spell out what is going on. This, however, comes at a cost: the equations are not coordinate invariant. The second law, in fact, is valid only for inertial frames with Cartesian coordinates. It is only in these frames, in fact, that a body will proceed in uniform motion if no forces are applied to it. If we are in polar coordinates, for example, the trajectory expressed in radius $\rho$ and angle $\theta$ will not be linear. Even the notion of force is, if one looks closely, a bit ambivalent. In principle, we want to write both the second law $F=ma$ and the expression for work $dW = F dx$. If $dW$ is invariant under change of position variables, the force should be a covector and therefore $dW = F_i dx^i$. But since the acceleration $a$ will change like a vector, we also have $F^i = m a^i$. The notion of force in the second law and in the infinitesimal work are slightly different, and they coincide only if we are in an inertial frame and Cartesian coordinates.

On the other side, Hamiltonian and Lagrangian mechanics are coordinate independent: the laws remain the same if we change position variables. This makes them more useful in many contexts. Lagrangian mechanics is more useful when trying to study the symmetries of the system. Hamiltonian mechanics is more useful for statistical mechanics and to better separate degrees of freedom. However, this comes at a price. Hamiltonian and Lagrangian mechanics apply in fewer cases than Newtonian mechanics. As we saw, linear drag may look like it has a valid Hamiltonian/Lagrangian, but it doesn't. For quadratic drag or friction due to normal force, one cannot find a suitable trick, and is forced to use Rayleigh’s dissipation functions which modify the Euler-Lagrange equations. This is not a coincidence: while Newtonian mechanics links kinematics and dynamics by choosing a particular frame, Hamiltonian and Lagrangian mechanics do so by fixing a type of system. It is the implicit knowledge of the type of system that allows to reconstruct the dynamics just by looking at the kinematics in an unknown frame. What we need to understand, then, is what exactly is this restriction.

\section{Reversing Hamiltonian mechanics}

We now turn our attention to Hamiltonian mechanics and try to understand exactly what types of systems it focuses on. We will find seven equivalent formulations of Hamiltonian mechanics that link ideas from vector calculus, differential geometry, statistical mechanics, thermodynamics, information theory and plain statistics. The overall result is that Hamiltonian mechanics focuses on systems that are deterministic and reversible. We will see how the physical significance of that assumption differ from mathematically naive characterizations.

To simplify our discussion, we will concentrate on a single degree of freedom. The first characterization is of course in terms of the equations
\begin{equation}
	\begin{aligned}
		d_t q &= \partial_p H \\
		d_t p &= - \partial_q H.
	\end{aligned}
\end{equation}
We want to think of phase space as a space with two coordinates, like we would for physical space. Since the term coordinates depends on the space (i.e. manifold) one is considering, we will refer to the collection of position and momentum as state variables and will note them as $\xi^a = [q, p]$. We can now define the displacement field
\begin{equation}\label{rp-cm-displacement1d}
	S^a = d_t \xi^a = [d_t q, d_t p]
\end{equation}
which is a vector field that defines the evolution of the system in time. Hamilton's equation, then, can be expressed as
\begin{equation}\label{rp-cm-displacementCurl1d}
	\tag{HM-1}
	\begin{aligned}
		S^q &= \partial_p H \\
		S^p &= - \partial_q H.
	\end{aligned}
\end{equation}
Geometrically, this tells us that the displacement field is the gradient of the Hamiltonian rotated by a right angle.\footnote{The notion of angle is technically ill-defined in phase space, but this slight imprecision makes it easier to get the point across.} Since the gradient is the only perpendicular direction to the regions at constant energy, a right angle rotation will give us a tangent to those regions, making it geometrically evident why the value of the Hamiltonian is a constant of motion.

Equation \ref{rp-cm-displacementCurl1d} looks very similar to a curl, except that it is expressed in two dimensions. In vector calculus, a vector field is the curl of another field if and only if its divergence is zero.\footnote{We will leave for now topological requirements as they would be a distraction from the overall point.} This holds here as well. First, we can verify that
\begin{equation}
	\partial_a S^a = \partial_q S^q + \partial_p S^p = \partial_q \partial_p H - \partial_p \partial_q H = 0.
\end{equation}
Geometrically, this means that the flow of $S^a$ through a closed region $\oint S^a \times d\xi^a = \oint S^q dp - S^p dq = 0$ is always zero.

Moreover, each divergenceless field in two dimension admits a stream function $H$ such that \ref{rp-cm-displacementCurl1d}. Geometrically, we can construct $H$ in the following way. Take a reference point $O$ and assign $H(O) = 0$. For any other point $P$, consider the flow of $S$ through any two lines that connect $O$ and $P$ given that the flow through the region contoured by those lines must be zero, the flow through each line must be equal. Therefore the flow through a line that connect $O$ and $P$ only depends on the points, it is path independent. We can assign $H(P) = - \int_{OP} S^a \times d\xi^a$. We can see that the differentials agree
\begin{equation}
	dH = \partial_q H dq + \partial_p H dp = - S^a \times d\xi^a = S^p dq - S^q dp
\end{equation}
and this will be a valid stream function. Therefore, at least in this simple case, we can understand the different of the Hamiltonian between two points as the flow of the displacement field between them.

We conclude that the following condition
\begin{equation}\label{rp-cm-divergenceless1d}
	\tag{HM-2}
	\partial_a S^a = 0
\end{equation}
is equivalent to \ref{rp-cm-displacementCurl1d}.

Having looked at the flow through a region, we turn our attention to how regions themselves are transported by the flow. Liouville's theorem states that volumes of phase space are preserved during Hamiltonian evolution, which in our case will be simply the area of the $q-p$ plane. To see this, let us review the transformation rules for infinitesimal volumes:
\begin{equation}\label{rp-cm-volumeTransformation1d}
	\begin{aligned}
		\hat{\xi^i} &= \hat{\xi^i}(\xi^j) \\
		d\hat{\xi}^n &= \left| \partial_{\xi^j} \hat{\xi^i} \right| d\xi^n \\
		d\hat{q} d\hat{p} &= \left|\begin{array}{ c c }
			\partial_q \hat{q} & \partial_p \hat{q} \\
			\partial_q \hat{p} & \partial_p \hat{p} \\
		\end{array} \right| dq dp \\
	\end{aligned}	
\end{equation}

Mathematically, a transformation is volume preserving if the determinant of the Jacobian $\partial_{\xi^j} \hat{\xi^i}$ is unitary. If $\hat{q}$ and $\hat{p}$ represent the evolution of $q$ and $p$ after an infinitesimal time step $\delta t$, we have
\begin{equation}
	\begin{aligned}
	\hat{q} &= q + S^q \delta t \\ 
	\hat{p} &= p + S^p \delta t \\ 
	d\hat{q} d\hat{p} &= (dq + \partial_q S^q dq \delta t) (dp + \partial_p S^p dp \delta t) = dqdp + \left(\partial_q S^q + \partial_p S^p \right) dq dp \delta t + O(\delta t^2). 
	\end{aligned}
\end{equation}
Note that the first order term is proportional to the divergence of the displacement field, therefore the Jacobian determinant is equal to one if and only if the displacement is divergenceless. In other words,
\begin{equation}\label{rp-cm-unitaryJacobian1d}
	\tag{HM-3}
	\left| \partial_{\xi^j} \hat{\xi}^i \right| = 1.
\end{equation}
and \ref{rp-cm-divergenceless1d} are equivalent conditions, and we have a third way to characterize Hamiltonian evolution.


Note that densities will transform in the opposite way of areas precisely because a density is a quantity divided an infinitesimal area. That is,
\begin{equation}\label{rp-cm-densityTransformation1d}
	\begin{aligned}
		\rho(\xi^j) &= \left| \partial_{\xi^j} \hat{\xi^i} \right| \hat{\rho}(\hat{\xi}^i).
	\end{aligned}	
\end{equation}
This means that the same condition, unitarity of the Jacobian, means that both areas and densities are transported unchanged by the evolution. That is, the density at an initial state is equal to the density at the corresponding final state.


It is useful to look more closely at the definition of the Poisson bracket
\begin{equation}
	\{f, g\} = \partial_q f \, \partial_p g - \partial_p f \, \partial_q g = \left|\begin{array}{ c c }
		\partial_q f & \partial_p f \\
		\partial_q g & \partial_p g \\
	\end{array} \right|.
\end{equation}
For a single degree of freedom, the Poisson bracket coincide with the Jacobian determinant. It essentially tells us how the volume changes if we change state variable from $[q, p]$ to $[f, g]$. Canonical coordinates and transformations, then, are those that do not change the units of area.

The three characterizations \ref{rp-cm-displacementCurl1d}, \ref{rp-cm-divergenceless1d} and \ref{rp-cm-unitaryJacobian1d} are all in terms of the mathematical description. While these are useful, the final goal of reverse physics is to find physical assumptions, not equivalent mathematical definitions. So it is time to step back and try to understand what the math is really about.

Let us first reflect to what we just found out: the defining characteristic of Hamiltonian mechanics is not the transport of points, but the transport of areas. If classical Hamiltonian mechanics were really about and only about point particles, there would be no reason for it to be characterized by \ref{rp-cm-divergenceless1d} or \ref{rp-cm-unitaryJacobian1d}. In fact, there would be no reason for the equations of motions \ref{rp-cm-displacementCurl1d} to be differentiable. Differentiable equations are exactly needed if we need to define the transport of areas, or of densities defined on those areas. Classical point particles, then, are more aptly thought not as points, but as infinitesimal regions of phase space, as distributions so peaked that only the mean value is important.

This, in retrospect, matches how classical mechanics is used in practice: planets, cannonballs, pendulums, beads on a wire, all the objects we study with classical mechanics are not point-like object. They can be thought as point-like if their size is negligible compared to the scale of the problem. If the distance between two celestial bodies is smaller than the sum of their radius, the point particle approximation clearly fails. This is also consistent with fluid-dynamics and continuous mechanics, where we are literally studying the motion of infinitesimal parts of a material. It is interesting to see echoes of these considerations present in the mathematics, and we will investigate this link in more detail later. At the moment, this insight is just motivation to look at things differently and more broadly.

If we do that, we realize that in statistical mechanics we already have a physical interpretation for volumes of regions phase space: they represent the number of states. Hamiltonian mechanics, then, maps regions while preserving the number of states. This means that, for each initial state there is one and only one final state, which leads to the following condition
\begin{equation}\label{rp-cm-detrev1d}
	\tag{HM-4}
	\text{The evolution is deterministic and reversible.}	
\end{equation}
Note that by reversible here we mean that given the final state we can reconstruct the initial state. Given that areas measure the number of states, \ref{rp-cm-detrev1d} is equivalent to \ref{rp-cm-unitaryJacobian1d}, which means this is another characterization of Hamiltonian mechanics.

As we saw, a dissipative system, like a particle under linear drag, is not a Hamiltonian system. We can now understand why that is true in general. A dissipative system will have an attractor: a point or a region to which the system will tend to given enough time. This means that, in time, the area around the attractor must shrink, but this is exactly what Hamiltonian systems cannot do. Therefore Hamiltonian systems cannot have attractors, they cannot be dissipative. The converse is also true: they can't have unstable points or regions from which the system always goes away from.

Now, the motion of a particle under linear drag may seem reversible, in the sense that we are able to, given the final position and momentum, reconstruct the initial values. Mathematically, it maps points one-to-one and would seem to satisfy \ref{rp-cm-detrev1d}, even though it is not a Hamiltonian system. This is a perfect example of how focusing on just the points leads to the wrong physical intuition. Physically, we would say that a one meter range of position allows for more configurations than a one centimeter range, even though technically they have the same number of points. If we understand that states are infinitesimal areas of phase space, we can see that a dissipative systems, though it does map the center points of infinitesimal areas one-to-one, it does not map the full infinitesimal area one-to-one. In this sense dissipative systems fail to be reversible.

Let that sink in: we found that, if the system is deterministic and reversible, it admits a Hamiltonian, a notion of energy, and that energy is conserved over time. This may seem like a surprising and unexpected result. In retrospect, we can make an argument for it based on familiar physics consideration. If a system is deterministic and reversible it means that its evolution only depends on the state of the system itself. This means that it does not depend on the state of anything else. A system whose evolution does not depend on anything else is and isolated system. Therefore a deterministic and reversible system is isolated, and from thermodynamic we know that an isolated system conserves energy. It should not be surprising, then, that a deterministic and reversible system conserves energy. However, we found that not only it conserves energy, it defines it. Therefore this link between mechanics and thermodynamics is actually deeper than we may think at first, and we should explore it more.

The idea that a dissipative system is not reversible sounds true on thermodynamic grounds. But thermodynamic reversibility is not the ability to reconstruct the initial state, but rather the existence of a process that can undo the change. Alternatively, a process is  thermodynamically reversible if it conserves thermodynamic entropy, which is a more precise characterization.\footnote{The actual existence of a reverse process is not something that can always be guaranteed.} We should not, then, confuse the two notions of reversibility but we can easily show the tow are related. The fundamental assumption of statistical mechanics tells us that the thermodynamic entropy $S = \log W$ is the logarithm of the count of states, which corresponds to volume in phase space. Conservation of areas of phase space, then, is equivalent to conservation of entropy. Therefore
\begin{equation}\label{rp-cm-thermorev1d}
	\tag{HM-5}
	\text{The evolution is deterministic and thermodynamic reversible.}	
\end{equation}
is yet another characterization of Hamiltonian mechanics.

However, there is another type of entropy that is also fundamental in both statistical mechanics and information theory: the Gibbs/Shannon entropy $I[\rho(\xi^i)]=\int \rho \log \rho \, d\xi^n$ which is defined for each distribution $\rho(\xi^i)$. Recalling the transformation rules for both volumes \ref{rp-cm-volumeTransformation1d} and densities \ref{rp-cm-densityTransformation1d}, we have
\begin{equation}
	\begin{aligned}
	I[\rho(\xi^j)] &= - \int \rho(\xi^j) \log \rho(\xi^j) \, d\xi^n \\
&= - \int  \hat{\rho}(\hat{\xi}^i) \left| \partial_{\xi^j} \hat{\xi^i} \right| \log \left( \hat{\rho}(\hat{\xi}^i) \left| \partial_{\xi^j} \hat{\xi^i} \right| \right) \, d\xi^n \\
&= - \int \hat{\rho}(\hat{\xi}^i) \log \left( \hat{\rho}(\hat{\xi}^i) \left| \partial_{\xi^j} \hat{\xi^i} \right| \right) \, d\hat{\xi}^n \\
&= - \int \hat{\rho}(\hat{\xi}^i) \log \hat{\rho}(\hat{\xi}^i) \, d\hat{\xi}^n - \int \hat{\rho}(\hat{\xi}^i) \log \left| \partial_{\xi^j} \hat{\xi^i} \right| \, d\hat{\xi}^n \\
&= I[\hat{\rho}(\hat{\xi}^i)] - \int \hat{\rho}(\hat{\xi}^i) \log \left| \partial_{\xi^j} \hat{\xi^i} \right| \, d\hat{\xi}^n.
	\end{aligned}
\end{equation}
Information entropy, then, remains constant if and only if the logarithm of the Jacobian determinant is zero, which means the Jacobian determinant is one. Therefore
\begin{equation}\label{rp-cm-information1d}
	\tag{HM-6}
	\text{The evolution conserves information entropy}	
\end{equation}
is equivalent to \ref{rp-cm-unitaryJacobian1d} and is yet another characterization of Hamiltonian mechanics.

The fact that determinism and reversibility is equivalent to conservation of information entropy should not be, in retrospect, surprising. Given a distribution, its information entropy quantifies the average amount of information needed to specify a particular element chosen according to that distribution. If the evolution is deterministic and reversible, giving the initial state is equivalent to giving the final state and therefore the information to describe one or the other must be the same. Determinism and reversibility, then, can be understood as the informational equivalence between past and future descriptions.

Lastly, given that entropy is often associated with uncertainty, it may be useful to understand how Hamiltonian evolution affects uncertainty. Given a multivariate distribution, the uncertainty is characterized by the covariance matrix
\begin{equation}
	cov(\xi^i, \xi^j) = \left[\begin{array}{ c c }
		\sigma^2_q & cov_{q, p} \\
		cov_{p, q} & \sigma^2_p \\
	\end{array} \right].
\end{equation}
The determinant gives us a coordinate independent quantity to characterize the uncertainty. If the distribution is narrow enough, we can use the linearized transformation to see how the uncertainty evolves after an infinitesimal time step $\delta t$. We have
\begin{equation}
	\left| cov(\hat{\xi}^i, \hat{\xi}^j) \right| = \left| \partial_{\xi^k} \hat{\xi}^i  cov(\xi^k, \xi^l) \partial_{\xi^l} \hat{\xi}^j  \right| = \left| \partial_{\xi^k} \hat{\xi}^i \right| \left| cov(\xi^k, \xi^l) \right| \left| \partial_{\xi^l} \hat{\xi}^j  \right|.
\end{equation}
which means the uncertainty remains unchanged if and only if the Jacobian is unitary. So
\begin{equation}\label{rp-cm-uncertainty1d}
	\tag{HM-7}
	\text{The evolution conserves the uncertainty of peaked distributions}	
\end{equation}
is equivalent to \ref{rp-cm-unitaryJacobian1d} and another characterization of Hamiltonian mechanics.

This connection gives us yet another insight on the nature of determinism and reversibility in physics. Given that all physically meaningful descriptions are finite precision, a system is deterministic and reversible in a physically meaningful sense if and only if the past/future descriptions can be reconstructed/predicted at the same level of precision. This gives us another perspective as to why areas must be conserved.

We have found seven equivalent characterizations that link Hamiltonian mechanics, vector calculus, differential geometry, statistical mechanics, thermodynamics, information theory and plain statistics. Though we only talked about the case of a single degree of freedom, it gives us a much better idea of what systems Hamiltonian mechanics is supposed to describe, those that undergo deterministic and reversible evolution. We can see how this concept is implemented mathematically: it is not simply a one-to-one map between points. Classical particles should be more properly thought of as infinitesimal regions of phase space. Conceptually, the count of states, the thermodynamic entropy and information entropy are all conserved, and are all equivalent characterizations of determinism and reversibility. In terms of physical measurement, past and future states are given at the same level of uncertainty. But the most important lesson is that the foundations of classical mechanics are not disconnected from the foundations of all other disciplines we connected. Therefore we will have to understand as much as possible those connections as well.

