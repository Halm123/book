\chapter{Classical mechanics}

The standard view in physics is that classical mechanics is perfectly understood. It has three different equivalent formulations, the oldest of which, Newtonian mechanics, is based on three laws. Classical mechanics is the theory of point particles that follow that laws. Unfortunately, this view is incorrect.

We will see that the three formulations are not equivalent, in the sense that there are physical systems that are Newtonian but not Hamiltonian and vice-versa. There are also a number of questions that are left unanswered, such as the precise nature of the Hamiltonian or the Lagrangian, and what exactly the principle of stationary actions represents physically. While shedding light on these issues, we will also find that classical mechanics already contains elements that are typically associated with other theories, such quantum mechanics (uncertainty principle, anti-particles), thermodynamics/statistical mechanics (thermodynamic and information entropy conservation) or special relativity (energy as the time-component of a four-vector). In other words, the common understanding of classical mechanics is quite shallow, and its foundations are not separate to the ones of classical statistical mechanics and special relativity.

What reverse physics shows is that the central assumption underneath classical mechanics is that of \textbf{infinitesimal reducibility}: a classical system can be thought as made of parts, which in terms are made of parts and so on; studying the whole system is equivalent to studying all its infinitesimal parts. This assumptions, together with the assumptions of \textbf{independence of degrees of freedom}, is what gives us the structure of classical phase space with conjugate variables. The additional assumption of \textbf{determinism and reversibility}, the fact that the description of the system at one time is enough to predict its future or reconstruct its past, leads us to Hamiltonian mechanics. Conversely, assuming \textbf{kinematic equivalence}, the idea that trajectories in space are enough to reconstruct the state of the system and vice-versa, leads us to Newtonian mechanics. The combination of all assumption, instead, leads to Lagrangian mechanics and, in particular, to massive particles under (scalar and vector) potential forces.

\section{Formulations of classical mechanics}

\subsection{Newtonian mechanics}

\subsection{Lagrangian mechanics}

\subsection{Hamiltonian mechanics}