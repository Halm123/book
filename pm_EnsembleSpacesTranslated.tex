\documentclass[11pt,letterpaper,fleqn]{memoir}

\usepackage[top=3cm,bottom=3cm,left=3cm,right=3cm,headsep=10pt,letterpaper]{geometry} % Page margins

% Theorem definitions using amsthm

\usepackage{amsthm}
\usepackage{amsmath}
\usepackage{amssymb}

% Remove line spaces between items of enumerate and itemize
\usepackage{enumitem}
\setlist{noitemsep}


% Adds double bracket symbols
\usepackage{stmaryrd}

% Latex symbol guide at http://mirrors.ibiblio.org/CTAN/info/symbols/comprehensive/symbols-letter.pdf

% LOGIC symbols
% -------------

% Allows to create negation symbols
\usepackage{MnSymbol}

\DeclareMathOperator{\truth}{truth}
\DeclareMathOperator{\poss}{possibilities}
\DeclareMathOperator{\result}{result}

\def\TRUE{\textsc{true}}
\def\FALSE{\textsc{false}}

\def\SUCCESS{\textsc{success}}
\def\FAILURE{\textsc{failure}}
\def\UNDEF{\textsc{undefined}}


% Symbols for tautology and contradiction
\def\tautology{\top}
\def\contradiction{\bot}

% Symbols for "compatibility" and "incompatibility"
\def\comp{\doublefrown}
\def\ncomp{\ndoublefrown}

% Symbols for "narrower" and "wider"
\def\narrower{\subseteq}
\def\nnarrower{\nsubseteq}
\def\broader{\supseteq}
\def\nbroader{\nsupseteq}


% Symbol for "independent" and "correlated"
\def\indep{\upmodels}
\def\nindep{\nupmodels}

% Aliases for logical operations
\def\AND{\wedge}
\def\bigAND{\bigwedge}
\def\OR{\vee}
\def\bigOR{\bigvee}
\def\NOT{\neg}


% Formatting for statements
\newcommand{\stmt}[1][s] {\mathsf{#1}}
% Formatting for experimental tests
\newcommand{\expt}[1][e] {\mathsf{#1}}
% Formatting for observations
\newcommand{\obs}[1] {\mathsf{#1}}
% Formatting for observation definition
\newcommand{\obsdef}[2] {\llparenthesis #1, #2 \rrparenthesis}

% Formatting for experimental domain
\newcommand{\edomain}[1][D] {\mathcal{#1}}

% Formatting for theoretical domain
\newcommand{\tdomain}[1][D] {\bar{\mathcal{#1}}}


% Formatting for sentence statements
\newcommand{\statement}[1] {\emph{``#1"}}


\usepackage{xcolor} % Required for specifying colors by name
\definecolor{sectionNumbers}{RGB}{44, 103, 0}


\renewcommand\thesubsection{\thesection.\Alph{subsection}}
\renewcommand{\theequation}{\thechapter.\arabic{equation}}

\newtheorem{assump}{Assumption}
\renewcommand*{\theassump}{\Roman{assump}}

\newtheorem{defn}[equation]{Definition}
\newtheorem{prop}[equation]{Proposition}

%\theoremstyle{definition}

\newenvironment{rationale}{\emph{Rationale}.}{\qed}
\newenvironment{justification}{\emph{Justification}.}{\qed}
\renewenvironment{proof}{\emph{Proof}.}{\qed}

% Style for math section
\RequirePackage[framemethod=default]{mdframed} % Required for creating the theorem, definition, exercise and corollary boxes
\newmdenv[skipabove=7pt,
skipbelow=7pt,
rightline=false,
leftline=true,
topline=false,
bottomline=false,
linecolor=sectionNumbers,
backgroundcolor=black!2,
innerleftmargin=5pt,
innerrightmargin=5pt,
innertopmargin=5pt,
leftmargin=0cm,
rightmargin=0cm,
linewidth=4pt,
innerbottommargin=5pt]{mathSection}


%----------------------------------------------------------------------------------------
%	SECTION NUMBERING IN THE MARGIN
%----------------------------------------------------------------------------------------

\makeatletter
\renewcommand{\@seccntformat}[1]{\llap{\textcolor{sectionNumbers}{\csname the#1\endcsname}\hspace{1em}}}                    
\renewcommand{\section}{\@startsection{section}{1}{\z@}
	{-4ex \@plus -1ex \@minus -.4ex}
	{1ex \@plus.2ex }
	{\normalfont\large\sffamily\bfseries}}
\renewcommand{\subsection}{\@startsection {subsection}{2}{\z@}
	{-3ex \@plus -0.1ex \@minus -.4ex}
	{0.5ex \@plus.2ex }
	{\normalfont\sffamily\bfseries}}
\renewcommand{\subsubsection}{\@startsection {subsubsection}{3}{\z@}
	{-2ex \@plus -0.1ex \@minus -.2ex}
	{.2ex \@plus.2ex }
	{\normalfont\small\sffamily\bfseries}}                        
\renewcommand\paragraph{\@startsection{paragraph}{4}{\z@}
	{-2ex \@plus-.2ex \@minus .2ex}
	{.1ex}
	{\normalfont\small\sffamily\bfseries}} % Loads the book formatting
\usepackage{tikz, pgfplots}
\usetikzlibrary{positioning,calc,arrows,arrows.meta, shapes, decorations.markings}

% Adding pdf index and links within the document
% Custom formatting to remove default red boxes
\usepackage{hyperref}
\hypersetup{
	colorlinks=true,
	urlcolor=blue,
	linkcolor=blue
}
\urlstyle{same}
% ---------------------------


\begin{document}

\def\>{\rangle}
\def\<{\langle}

\newcommand\mix{\mathrm{mix}}
\newcommand\component{\mathrm{comp}}
\newcommand\cospan{\mathrm{cospan}}

\newcommand{\ens}[1][e] {\mathsf{#1}} % Ensemble
\newcommand{\Ens}[1][E] {\mathcal{#1}} % Ensemble space

\section{Convex subspaces}

Since the goal is to use convex spaces to create a unified theory for all types of mechanics, we need a construction that generalizes the notion of support/subspaces of ensembles.

\begin{defn}
	Let $\Ens$ be a convex space and $X \subseteq \Ens$ be a subset. We say that $X$ is a \textbf{subspace} of $\Ens$ if it contains all the convex combinations and all the components of its elements. That is, for every $\ens_1, \ens_2, \ens_3 \in \Ens$ and $\lambda \in (0,1)$ such that $c_\lambda(\ens_1, \ens_2) = \ens_3$ we have:
	\begin{itemize}
	\item $\ens_1, \ens_2 \in X$ implies $\ens_3 \in X$
	\item $\ens_3 \in X$ implies $\ens_1, \ens_2 \in X$
\end{itemize}
\end{defn}

\begin{defn}
	Let $\Ens$ be a convex space and $X \subseteq \Ens$.  The \textbf{convex span} of $X$, noted $\cospan(X)$, is the smallest subspace containing $X$.
\end{defn}

\begin{remark}
	As defined, the convex span of two elements will include all their possible mixtures (i.e. the segment that connects them), all possible decompositions (i.e. all lines that pass through them) plus, recursively, all other mixtures and decompositions that can be reached from those. Physically, the idea is that not all ensembles, pure states in particular, cannot be physically realized. Therefore, the inclusion of pure states in our convex space stems from a theoretical idealization useful to decompose and study the problem. It makes sense, then, that a subspace comes with all its idealizations.
\end{remark}

\begin{example} Classical discrete subspaces.
	Let $S$ be a set of $n$ possible discrete states and let $\Ens$ the space of probability distributions over the set $S$ (i.e. $\Ens$ is an $n$-simplex and $S$ are its extreme points). A subspace $X$ of $\Ens$ is a convex hull of a subset $U$ of $S$. That is, a subspace of $\Ens$ is the space of probability distributions over a subset of the cases. Geometrically, it is one of the sides (possibly recursively) of the simplex.
	
	To see this, first note that the convex hull $X$ of any subset $U$ of extreme points $S$ is a subspace. In fact, it will contain all convex combinations of $U$, and any element can only be decomposed in convex combinations of $U$. Second, note that only convex hulls of a subset of extreme points can be a subspace. In fact, any element of $\Ens$ can be expressed as a non-trivial convex combination of a set of extreme points $U$. Therefore, if an element is present in a subspace $X$, then $U \subset X$, which means all elements of the convex hull of $U$ are in $X$.
\end{example}

\begin{remark}
	The definition does not entirely work in the continuous case. Let $S$ be a symplectic manifold and let $\Ens$ be the space of probability distributions over $S$ (i.e. the space of continuous functions over $S$ that are integrable and integrate to one). We would like to have a definition for which a subspace of $\Ens$ is a set of functions whose support is a subset of an open region $U \subseteq S$. That is, a subspace contains all probability distributions defined over a subset of the cases. One problem is that continuity forces the function to go to zero on the boundary of $U$, and the speed of the convergence cannot be change with a finite convex combination. That is, the convex combination of functions that go down like an exponential will also go down like an exponential. The issue seems to be related to finding the correct closure for infinite convex combinations in the definition of convex space.
\end{remark}

\begin{example} Quantum subspaces.
	Let $\mathcal{H}$ be an $n$-dimensional Hilbert space and let $\Ens$ be the space of density matrices (i.e. positive semi-definite self-adjoint operators with trace one). A subspace $X$ of $\Ens$ is the space of density matrices of a subspace $U$ of $\mathcal{H}$. That is, a subspace of $\Ens$ is the space of mixed states over a subspace of pure states.
	
	To see this, first note that the space of density matrices $X$ of a subspace $U$ of $\mathcal{H}$ is a subspace of $\Ens$. In fact, $X$ it will contain all convex combinations of its elements. Moreover, any element $x \in X$ can only be decomposed in a convex combination of pure states of $U$. Therefore any convex decomposition of $x$ has all its elements in $X$. Second, note that only the space of density matrices $X$ of a subspace $U$ of $\mathcal{H}$ is a subspace of $\Ens$. In fact, any element $x$ of $\Ens$ can be expressed as a non-trivial convex combination of orthogonal pure states, its eigenstates. These elements will span a subspace $U$ of $\mathcal{H}$. From those elements, we can construct an equal mixture which represents the maximally mixed states and, mathematically, is the identity operator $I/m$ divided by the number of elements $m \leq n$ of $U$. The equal mixture of any orthogonal basis of $U$ will also give the maximally mixed state. Therefore, given an element $x$, any subspace that contain $x$ will also contain a basis of $U$, the maximally mixed state $I/n$, all possible basis of $U$, which means all the pure states, and finally all convex combinations of the pure states, which means all possible density matrices, all possible mixed states.
\end{example}

\begin{defn}
	A convex space $\mathcal{E}$ is \textbf{closed} if it contains all its extreme points.
\end{defn}

\begin{defn}
	Let $\mathcal{E}$ be a convex space and $X \subset \mathcal{E}$ be a subspace of $\mathcal{E}$. The \textbf{dimension} of $X$, noted $\dim(X)$, is the minimum number of elements whose convex span is $X$.
\end{defn}

\begin{conj}
	Let $\mathcal{E}$ be closed convex space of finite dimensions. Then:
	\begin{itemize}
		\item $\mathcal{E}$ is a simplex if and only if every two dimensional subspace is a line segment
		\item the set of extreme points of $\mathcal{E}$ is a complex projective space if and only if every two dimensional subspace is a two dimensional sphere ($\mathcal{E}$ is the space of density matrices).
	\end{itemize}
\end{conj}

\begin{remark}
	The only if is easy to show based on the discussion above.
\end{remark}
\end{document}
