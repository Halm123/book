\textbf{Reverse physics} is an approach to the foundations of physics that seeks to determine which principles or assumptions are required to derive the different physical theories. This the analogue of \textbf{reverse mathematics}, a program in mathematical logic that seeks to determine which axioms are required to prove mathematical theorems.

While some physical theories, such as Newtonian mechanics, thermodynamics and special relativity, are indeed founded on laws or principles, most of the more modern theories, such as Hamiltonian and Lagrangian mechanics, quantum mechanics, or general relativity, are based on mathematical relationships that are simply postulated without a strict physical justification. The goal of reverse physics, then, is to find suitable physical premises that can function as a more proper foundations for each theory.

We stress that the premises have to be of physical nature, such as ``the system under study is isolated'' or ``the quantity is additive under system composition''. That is, they must be principles or assumptions that express some physical idea, not some abstract mathematical notion. The goal is, in fact, to fully understand what the mathematical structures we use in physics are supposedly representing, what connections can be made across different theories and what are exactly the true limit of applicability of the different physical theories.\footnote{When presented with a physics problem, we typically do not have a principled way to reason which physical theory to use, but rather we match the problem with patterns we previously studied.}

What reverse physics teaches us is that there are some core ideas that are shared by all physical theories, and that, in fact, physical theories are not as disconnected from each other as it is generally thought. Reverse physics functions as the main source of insights and ideas that later we formalize rigorously through \textbf{physical mathematics}.
