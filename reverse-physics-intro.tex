\textbf{Reverse physics} is an approach to the foundations of physics that analyzes known theories to identify those physical principles and assumptions that can be taken as their conceptual foundation. This is the analogue of \textbf{reverse mathematics}, a program in mathematical logic that seeks to determine which axioms are required to prove mathematical theorems.

While some physical theories, such as Newtonian mechanics, thermodynamics and special relativity, are indeed founded on laws or principles, many theories, such as Hamiltonian and Lagrangian mechanics, quantum mechanics, and general relativity, are based on mathematical relationships that are simply postulated without a strict physical justification. Most modern theories are of the latter type, as theoretical physicists have increasingly focused on mathematical ideas rather than physical starting points. Therefore we do not know why the state of a quantum system can be represented by a ray in a Hilbert space, or what exact relationship is represented by the Einstein field equations. The goal of reverse physics, then, is to find suitable physical premises that can function as a more proper foundation for each theory.

We stress that the premises have to be of a physical nature, such as ``the system under study is isolated'' or ``the quantity is additive under system composition''. That is, they must be principles or assumptions that express some physical idea, not some abstract mathematical notion. Physics is currently a patchwork of different theories, and one is trained to match patterns and examples to decide which problems (or what aspects of a larger problem) should be treated with a particular theory. The goal here is to fully understand what physical situation each mathematical structure can suitably describe, what connections can be made across different theories and what exactly are the true limits of applicability of the different physical theories. One of the main results of reverse physics is that there are some core ideas that are shared by all physical theories, and that, in fact, physical theories are not as disconnected from each other as generally thought.

By nature, reverse physics does not aim to be mathematically precise, but rather conceptually precise. While the goal is to create a dictionary between physical concepts and their mathematical representation, this mapping cannot be absolutely perfect for a simple reason: we have no guarantee that the current mathematical structures are the correct ones to capture the physical ideas. Conversely, we cannot give a precise mathematical characterization of the physical ideas if these are not conceptually clear. Therefore conceptual clarity has to come before formal precision. Yet, it is true that to reach full conceptual clarity the ideas have to be sufficiently refined to allow a formal definition. This task is the purview of \textbf{physical mathematics}, which aims to find physically meaningful mathematical structures starting from definitions and axioms that can be fully supported by physical requirements and assumptions. Yet, it would be premature to engage in such activity without having thoroughly tested the conceptual framework beforehand.
