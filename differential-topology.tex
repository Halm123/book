\chapter{Differential geometry}

\section{Differentiability}

The main idea is to just use the vector space structure of $\mathbb{R}^n$ to define a notion of differential, and then recover differentiability as maps that preserve that notion.

\subsubsection{Differentials}

We define differentials as equivalence classes of those sequences for which, in the limit, each element is roughly half of the previous element.

\begin{defn}
Let $V$ be a vector space and let $2=1+1$ be the sum of the unit of the field with itself. A \textbf{halving sequence} is a sequence of vectors that can be written as $\{(1/2)^i v\}_{i=1}^{\infty}$ for some $v \in V$.
\end{defn}

\begin{prop}
	The space of all halving sequences is isomorphic to $V$.
\end{prop}

\begin{proof}
	By construction, for each vector $v \in V$ there is one and only one halving sequence. The map from the vector to each halving sequence induces a vector space structure and therefore the space of all halving sequences is isomorphic to $V$.
\end{proof}

\begin{defn}
A sequence of vectors $\{v_i\}_{i=1}^{\infty}$ convergences to the halving sequence of $v$ if 
$$ \lim\limits_{i \to \infty} \frac{v_i}{(1/2)^i} = v.$$
Given a vector $v$, the \textbf{differential} $dv$ is the set of all sequences of vectors that converge to the halving sequence of $v$. The \textbf{differential space} of $V$, noted $dV$, is the space of all differentials.
\end{defn}

\begin{prop}
The space of all halving sequences is isomorphic to the differential space of $V$, which means $V$ and $dV$ are isomorphic.
\end{prop}

\begin{proof}
	Note that the halving sequence of $v$ converges to the halving sequence of $v$. In fact:
	$$ \lim\limits_{i \to \infty} \frac{v_i}{(1/2)^i} = \lim\limits_{i \to \infty} \frac{(1/2)^i v}{(1/2)^i} = \lim\limits_{i \to \infty} v = v.$$ Therefore we have a bijection between halving sequences and differentials.
	
	Convergence to halving sequences preserves linear combination in the following sense: if $\{v_i\}_{i=1}^{\infty} \in dv$ and $\{w_i\}_{i=1}^{\infty} \in dw$, then $\{a v_i + b w_i\}_{i=1}^{\infty} \in d(av+bw)$. In fact:
	$$\lim\limits_{i \to \infty} \frac{a v_i + b w_i}{(1/2)^i} = a\lim\limits_{i \to \infty} \frac{v_i}{(1/2)^i} + b\lim\limits_{i \to \infty} \frac{w_i}{(1/2)^i} = a v + b w$$. This means that the space of halving sequences is isomorphic to the space of differentials.
	
	Since the space of halving sequences is isomorphic to both the vector space $V$ and the space of differentials $dV$, then $V$ and $dV$ are isomorphic.
\end{proof}

\begin{prop}
	Let $v \in V$ and let $dv \in dV$ be the differential corresponding to the halving sequence of $v$. Any sequence $\{v_i\}_{i=1}^{\infty} \in dv$ can be written as $v_i = (1/2)^i(v + w_i)$ where $\lim\limits_{i \to \infty} w_i = 0$.
\end{prop}

\begin{proof}
	Note that we can write $v_i = (1/2)^i v + (1/2)^i \left( \frac{v_i}{(1/2)^i} - v\right)$ for all $i$. We can set $w_i = \frac{v_i}{(1/2)^i} - v$ and write $v_i = (1/2)^i(v + w_i)$. We have:
	$$\lim\limits_{i \to \infty} \frac{v_i}{(1/2)^i} = \lim\limits_{i \to \infty} \frac{(1/2)^i v}{(1/2)^i} + \lim\limits_{i \to \infty} \frac{(1/2)^i w_i}{(1/2)^i} = \lim\limits_{i \to \infty} v + \lim\limits_{i \to \infty} w_i$$
	The sequence on the left converges to $v$, the first sequence on the right converges to $v$, therefore the second sequence on the right converges to zero.
\end{proof}

\begin{remark}
	Let $x$ be a variable over $\mathbb{R}$. We can express the differential $dx=\{(1/2)^i x\}_{i=1}^{\infty}$. In $\mathbb{R}^n$, we can express the differential in two ways. First as $dX = \{(1/2)^i X\}_{i=1}^{\infty} = d\lambda X$, that is a vector multiplied by a differential for a real parameter lambda. Second as $\{(1/2)^i x^j e_j\}_{i=1}^{\infty} = dx^j e_j$, that is one differential for each component multiplied by the base vectors of $X$. When we write $x+dx$, the choice of value $x$ and differential $dx$ is independent, Therefore we can have $x+dx=\{3 + (1/2)^i 2\}_{i=1}^{\infty}$ or $x+dx=\{\pi + (1/2)^i e\}_{i=1}^{\infty}$.
\end{remark}


\subsubsection{Differentiability}

We now define differentiability as the property of a function between vector spaces to induce a linear map between differentials. OPEN QUESTION: is linearity an additional property, or can it be derived from the existence of the map? That is, can a map between vector spaces induce a non-linear map between differentials? It would seem that it would break the equivalence class.

\begin{defn}
	Let $V$ and $W$ be two vector spaces. We say that a map $f: V \to W$ is \textbf{differentiable} at $v \in V$ if it induces a linear map from differentials of $V$ to differentials of $W$. That is, there exists a linear map $\left.\frac{df}{dV} \right|_v : dV \to dW$ called \textbf{derivative} such that, given a differential $dv \in dV$, $\{f(v +v_i) - f(v)\}_{i=1}^{\infty} \in \frac{df}{dV}(dv)$  for all $\{v_i\}_{i=1}^{\infty} \in dv$.
\end{defn}

\begin{prop}
	Let $f:\mathbb{R} \to \mathbb{R}$. Then the standard analytical notions of differentiability and derivative coincide.
\end{prop}
\begin{proof}
	Let $x \in \mathbb{R}$ and $dx=\{x_i\}_{i=1}^{\infty} \in d\mathbb{R}$. We can write:
	$$ \lim\limits_{i \to \infty} \frac{f(x +x_i) - f(x)}{(1/2)^i} = \lim\limits_{i \to \infty} \frac{f(x +x_i) - f(x)}{x_i}\frac{x_i}{(1/2)^i}.$$
	We can see that the sequence on the left is equal to the product of two sequences. The sequence on the left converges because the map is differentiable. The second sequence to the right converges because $dx$ is a differential. This means that the first sequence to the right must also converge. That is,  $\lim\limits_{i \to \infty} \frac{f(x_i +x) - f(x)}{x_i}$ converges. But since $x_i$ is a sequence that goes to zero, this evaluates the derivative. Therefore the derivative exists.
	
	Conversely, suppose we do not know that $\{f(v_i +v) - f(v)\}_{i=1}^{\infty}$ is a differential. We still have that the left sequence is a product of the two sequence on the right. These must both converge because the derivative exist and $dx$ is a differential. This means that the sequence on the left must also converge and therefore $f$ is differentiable in the new sense.
\end{proof}

\begin{remark}
	To convince ourselves that the equivalence class actually works, let us calculate the derivative of $x^2$ at a generic point $x_0$ with a generic differential $dx = \{(1/2)^i(k + w_i)\}$.
	$$ \lim\limits_{i \to \infty} \frac{(x_0 + (1/2)^i(k + w_i))^2 - x_0^2}{(1/2)^i(k + w_i)} = \lim\limits_{i \to \infty} \frac{x_0^2 + 2 x_0 (1/2)^i(k + w_i) + (1/2)^{2i}(k + w_i) - x_0^2}{(1/2)^i(k + w_i)}.$$
	$$= \lim\limits_{i \to \infty} \frac{2 x_0 (1/2)^i(k + w_i) + (1/2)^{2i}(k + w_i)}{(1/2)^i(k + w_i)}$$
	$$= \lim\limits_{i \to \infty} \frac{2 x_0 (k + w_i) + (1/2)^{i}(k + w_i)}{(k + w_i)}$$
	Note that $k$ is constant, while $w_i$ and terms multiplied by $(1/2)^{i}$ converge to zero. The sequence then converges to $2x_0$, which is indeed the correct result.
	
	What this shows is that higher orders here are simply halving sequences that proceed with a higher power: at each step we divide by two not once, but as many times corresponding to the order.
\end{remark}

\begin{remark}
	Consider the $n$ vectors of components
	$$ \left[\begin{array}{c} 1\\ 1\\ 1\\ ...\\ 1 \end{array}\right] 
	\left[\begin{array}{c} 1\\ 1/2\\ 1/4\\ ...\\ (1/2)^{(n-1)1} \end{array}\right]
	\left[\begin{array}{c} 1\\ 1/4\\ 1/16\\ ...\\ (1/2)^{(n-1)2} \end{array}\right]
	\left[\begin{array}{c} ...\\ ...\\ ...\\ ...\\ ... \end{array}\right]
	\left[\begin{array}{c} 1\\ (1/2)^{1(n-1)}\\ (1/2)^{2(n-1)}\\ ...\\ (1/2)^{(n-1)(n-1)} \end{array}\right].$$
	They are all linearly independent, and therefore they form a basis. That is, any sequence of $n$ elements can be written as $v_i = \sum c_j (1/2)^{ij}$. If we take the limit as $n$ goes to infinity, this does not work: all the sequences $\{(1/2)^{ij}\}_{i=0}^\infty$ converge for any $j$, but it is not true that all sequences converge. In the light of our definition of differentials in terms of halving sequences, is smooth of order zero if it converges, and therefore we can assign the zeroth coefficient $c_0$. It is smooth of first order if it is a differential, and therefore we can assign the first coefficient $c_1$. And so on. An infinitely smooth sequence is one that can be expressed with the infinite sum.
	
	Since in our definitions $dx$ and $df$ are sequences, $df = \sum c_j dx^j$ maps to the Taylor expansion. If the function is not infinitely smooth, it means that, at some point, one of the sequences at higher order does not converge. An infinitely smooth function retains, in a sense, the property that the halving sequences of all orders provide a basis for all sequences.
\end{remark}


\begin{prop}[Chain rule]
	Let $U$, $V$ and $W$ be three vector spaces. Let $f : U \to V$ and $g : V \to W$ be two differentiable maps and $h = g \circ f$ their composition. Then $\frac{dh}{dU} = \frac{dg}{dV} \circ \frac{df}{dU}$.
\end{prop}

\begin{proof}
	Given that $f$ and $g$ are differentiable, a differential of $U$ will be mapped to a differential of $V$ which will be mapped to a differential of $W$. The composite map $h$ will map a differential of $U$ to a differential of $W$ and therefore it is differentiable and the derivative exists. The differential returned by $h$ will be the one returned by $g$ and $f$ combined, therefore the derivative of $h$ is the function combination of the derivatives of $g$ and $f$.
\end{proof}

\begin{remark}
	The chain rule in this framework is just function combination. Given that these are linear function, it reduces to multiplication of the derivative on $\mathbb{R}$ and on matrix multipliciation on $\mathbb{R}^N$.
\end{remark}


\subsubsection{Partial derivatives}

Partial derivatives are recovered studying how differentials behave under vector space composition. If $f : V \to W$ is differentiable and $V = V_1 \times V_2$, we can ask for the map from a differential of $V_1$ to a differential of $W$ through $f$. This is exactly a variation of $V$ that keeps $V_2$ constant. Conceptually, given that $dv = (dv_1, dv_2)$ and the derivative is linear, we have $dw = \frac{df}{dV} dv = \frac{df}{dV} (dv_1, dv_2) = \frac{df}{dV} (dv_1, 0) + \frac{df}{dV} (0, dv_2) = \frac{df}{dV_1} dv_1 + \frac{df}{dV_2} dv_2$.
	
Suppose that $V = V_1 \times V_2$, then $dV = dV_1 \times dV_2$. This means that we can decompose a differential $dv= dv_1 + dv_2$ as the sum of two differentials in the respective spaces (mapped into the composite through the injection map). Suppose that $W = W_1 \times W_2$ and $f : V \to W$ is differentiable. Then we also have $dw= dw_1 + dw_2$ and we can study of $f$ maps differentials of $V_1$ and $V_2$ to differentials of $W_1$ and $W_2$. We will find $dw = dw_1 + dw_2 = \frac{dw_1}{dV} dv + \frac{dw_2}{dV} dv = \frac{\partial w_1}{\partial V_1} dv_1 + \frac{\partial w_1}{\partial V_2} dv_2 + \frac{\partial w_2}{\partial V_1} dv_1 + \frac{\partial w_2}{\partial V_2} dv_2$. If $U = \mathbb{R}^n$ and $V = \mathbb{R}^m$ we find the usual definitions.

\begin{prop}[Direct product and direct sum]
	The differential space $dV$ of the direct product $V=\prod V_i$ of a family of vector spaces $V_i$ is the direct product $\prod dV_i$ of the respective differential spaces. The same is true for the direct sum.
\end{prop}

\begin{proof}
	Let $v \in V$ and consider its halving sequence $\{(1/2)^i v\}_{i=1}^{\infty}$. Given that $V$ is a direct product, $v=(v_j)$ we have $\{(1/2)^i (v_j)\}_{i=1}^{\infty} = \{((1/2)^i v_j)\}_{i=1}^{\infty}$. Conversely, if we pick one vector for each space $V_j$ and construct their respective halving sequences, this becomes a halving sequence of a vector $v \in V$. Therefore the space of halving sequences over $V$ is the direct product of the halving sequences over the respective $V_j$. Given that the spaces of halving sequences are isomorphic to the respective differential spaces, the differential space of the direct product is the direct product of the differential spaces.
	
	The same argument works for direct sum.
\end{proof}

\begin{defn}
	Let $V$ and $W$ be two vector spaces and let $V=\prod V_i$ be the direct product of a family of vector spaces $V_i$. Let $f: V \to W$ be a differentiable map. The \textbf{partial derivative} of $f$ with respect to $W_i$, noted $\frac{\partial f}{\partial V_i}$, is the map from a differential of $V_i$ to the differential of $W$ through $f$. That is, $\frac{\partial f}{\partial V_i} (dv_i) = \frac{\partial f}{\partial V} \left( (0, ..., 0, dv_i, 0, ...) \right)$.
\end{defn}

\begin{prop}
	Let $f:\mathbb{R}^m \to \mathbb{R}^n$. Then the standard analytical notions of differentiability and partial derivatives coincide with the new ones.
\end{prop}

\begin{proof}
\end{proof}


\begin{remark}
	Partial derivatives should be recoverable through vector space composition. Suppose that $V = V_1 \times V_2$, then $dV = dV_1 \times V_2$. This means that we can decompose a differential $dv= dv_1 + dv_2$ as the sum of two differentials in the respective spaces (mapped into the composite through the injection map). Suppose that $W = W_1 \times W_2$ and $f : V \to W$ is differentiable. Then we also have $dw= dw_1 + dw_2$ and we can study of $f$ maps differentials of $V_1$ and $V_2$ to differentials of $W_1$ and $W_2$. We will find $dw = dw_1 + dw_2 = \frac{dw_1}{dV} dv + \frac{dw_2}{dV} dv = \frac{\partial w_1}{\partial V_1} dv_1 + \frac{\partial w_1}{\partial V_2} dv_2 + \frac{\partial w_2}{\partial V_1} dv_1 + \frac{\partial w_2}{\partial V_2} dv_2$. If $U = \mathbb{R}^n$ and $V = \mathbb{R}^m$ we find the usual definitions.
\end{remark}


\section{Linear functionals}

Review notation. Let $M$ be a differentiable manifold of dimension $n$.

\begin{defn}
	A $k$-surface is a $k$-dimensional smooth submanifold of $M$. We denote by $S^k$ the set of all  $k$-surfaces of dimension $k$ and by $S = \bigcup_{k=0}^n S^k$ the set of all smooth surfaces of all dimensions.
\end{defn}

\begin{defn}
	Given a $k$-surface $\sigma^k \in S^k$, the \textbf{boundary} of $\sigma^k$, denoted by $\partial\sigma^k \in S^{k-1}$ is the limit of varied coordinates. The \textbf{boundary operator} $\partial : S \to S$ is a map from a $k$-surface to its boundary. A surface is \textbf{closed} if has no boundary.
\end{defn}

\begin{coro}
	Boundary of smooth surfaces are smooth surfaces. Boundaries do not have boundaries. That is, $\partial\partial \sigma^k = \emptyset$ for all $\sigma^k \in S^k$.
\end{coro}


\begin{defn}
	A \textbf{$k$-functional} is a linear function of $k$-surfaces. That is, it is a function $f_k : S^k \to \mathbb{R}$ with the following properties:
	\begin{description}
		\item[Linear] $f_k(\sigma^k_1 \cup \sigma^k_2) = f_k(\sigma^k_1) + f_k(\sigma^k_2)$ for every $\sigma^k_1, \sigma^k_2 \in S^k$ such that $\sigma^k_1 \cap \sigma^k_2 = \emptyset$
		\item[No contribution from boundary] $f_k(\sigma^k) = f_k(\sigma^k \setminus \partial \sigma^k)$ for every $\sigma^k \in S^k$
		\item[Commutes with the limit] $\lim\limits_{i \to \infty} f_k(\sigma_i^k) = f_k(\lim\limits_{i \to \infty}\sigma_i^k)$
	\end{description}
	We denote by $F_k$ the set of all $k$-functionals of dimension $k$ and by $F = \bigcup_{k=0}^nF_k$ is the set of all $k$-functionals.
\end{defn}

\begin{coro}
	Any $k$-functional applied to the empty set returns zero. That is, for any $f_k \in F$, $f_k(\emptyset) = 0$.
\end{coro}

\begin{defn}
	The \textbf{zero $k$-functional}, noted $0_k \in F_k$, is the $k$-functional that always returns zero. That is, $0_k(\sigma^k) = 0$ for all $\sigma^k \in S^k$.
\end{defn}


\begin{defn}
	Given a $k$-functional $f_k \in F_k$, the \textbf{boundary functional} $\partial f_k \in F_{k+1}$ is a $(k+1)$-functional that applies $f_k$ on the boundary. That is, $\partial f_k(\sigma^{k+1}) = f_k(\partial \sigma^{k+1})$. 
\end{defn}

\begin{coro}
	The boundary functional of the boundary functional is the zero functional. That is, for any $k$-functional $f_k \in F$, $\partial \partial f_k = 0_{k+2}$.
\end{coro}

\begin{proof}
	$\partial \partial f_k (\sigma ^{f+2}) = \partial f_k (\partial \sigma ^{f+2}) = f_k (\partial \partial \sigma ^{f+2}) = f_k(\emptyset) = 0$
\end{proof}

\begin{defn}
	A $k$-surface $\sigma^k \in S^k$ is \textbf{contractible} if it can be continuously shrunk to a point. That is, the inclusion map $\iota : \sigma^k \to X$ is null-homotopic.
\end{defn}

\begin{defn}
	An \textbf{exact functional} is a $k$-functional that returns zero on all closed $k$-surfaces. That is, $f_k(\sigma^k) = 0$ for all $\sigma^k \in S^k$ such that $\partial\sigma^k = \emptyset$. A \textbf{closed functional} returns zero on all contractible closed surfaces.
\end{defn}

\begin{remark}
	Names are chosen to agree with exact/closed forms... Should we find better names?
\end{remark}

\begin{prop}
	Let $f \in F_k$ be an exact $k$-functional. Then there exists some $(k-1)$-functional $g \in F_{k-1}$ such that $f = \partial g$. We say $g$ is the \textbf{potential} of $f$.
\end{prop}

\begin{remark}
	The aim here is to prove the theorem on finite surfaces, without using standard differentiable calculus.
	
	As a model, we should use the standard proof used in physics for irrotational fields. Suppose we have an exact 1-functional, that is it gives zero for all closed lines. Then, one can show that two lines that share the same boundary must have the same value. Then pick a point and assign zero to that point. To any other point, assign the value given by the functional over a line that starts at the zero point and ends that the new point. The potential is given by those assignments.
	
	The way to generalize is to realize that a $k$-surface is half a boundary of a $k+1$-surface. For example, a point is half a boundary of a line, which constitutes of 2 points. The boundary of a surface is a closed line, which can be understood as two lines that share the same boundary, but opposite orientation. Like the line integral can be understood as ``going from`` one point (i.e. half boundary) to the other, the surface integral can be understood as ``going from'' one line (i.e. half boundary) to the other.
	
	For example, suppose we have an exact 2-functional, that is it gives zero for all closed surfaces. Then two surfaces that share the same boundary have the same value. Pick a reference point. Pick a family of lines such that they all start from the reference point, all end at different point (covering the whole space) and never form two paths to the same point. For example, in local coordinates, change one coordinate at a time (i.e. first increase the x, then increase the y, then the z, ...). Now pick a scalar function (a (2 - 2)-form) and assign to each line the difference at the boundary (this arbitrary choice is the equivalent of choosing the constant function in the previous case, and effectively maps to the choice of gauge). Given any other line, we can use the family to find two lines to form a closed loop. A closed loop identifies, which can now be given a value based on the 2-functional.
	
	This should be generalizable with the following sketch. Take a set of surfaces $R \subset S^{k-1}$, called references, that:
	\begin{enumerate}
		\item includes the empty surface $\emptyset$
		\item the union of a family of surface is in $R$
		\item subsurface of a surface is in $R$
		\item no two surfaces share the same boundary.
	\end{enumerate}
	(Note: some care needs to be done with the definition for $k=1$) Therefore, for any $\sigma^{k-1} \in S^{k-1}$ we find a unique $R(\sigma^{k-1}) \in R$ such that $\partial R(\sigma^{k-1}) = \partial \sigma^{k-1}$. That is, for any surface we find a reference surface with the same boundary, and together $\partial \sigma^{k-1} \cup R(\sigma^{k-1})$ they form a closed surface. Since $f$ is exact, $f(\sigma^k)$ depends only on the boundary of $\sigma^k$: two surfaces with equal boundary can be joined together to form a surface with no boundary, for which $f$ is zero. Therefore we can define $\hat{f}_{k-1} : S^{k-1} \to \mathbb{R}$ such that $\hat{f}_{k-1}(\sigma^{k-1}) = f_k(\hat{\sigma}^{k})$ where $\hat{\sigma}^{k}$ is any surface such that $\partial \hat{\sigma}^{k} = \sigma^{k-1}$. Now take an exact functional $v \in F_{k-1}$. Define $g \in F_{k-1}$ such that $g(\sigma^{k-1}) = \hat{f}_{k-1}(\sigma^{k-1} \cup R(\sigma^{k-1})) + v(R(\sigma^{k-1}))$. We have $\partial g(\sigma^{k}) = g(\partial \sigma^{k}) = \hat{f}_{k-1}(\partial \sigma^{k} \cup R(\partial \sigma^{k})) + v(R(\partial \sigma^{k}))$. Since $\partial \partial \sigma^k = \emptyset$, $R(\partial \sigma^{k}) = \emptyset$ because that is the only surface in $R$ with an empty boundary. Therefore $\partial g(\sigma^k) = \hat{f}_{k-1}(\partial \sigma^k \cup \emptyset) + v(\emptyset) = \hat{f}_{k-1}(\partial \sigma^k) = f_k(\sigma^k)$. Which means $\partial g = f$.
\end{remark}


\iffalse

\section{Differential forms}

This section needs to show that vectors and differential forms are infinitesimal counterparts of surfaces and functional. 


\begin{defn}
	TODO: Define a \textbf{$k$-vector} $v^k \in V^k$ as an infinitesimal parallelepiped. We note $V^k$ as the set of all vectors of rank k and $V = \bigcup_{k=0}^n V^k$ as the set of all $k$-vectors. 
\end{defn}


\begin{defn}
	The \textbf{wedge product} $\wedge : V^k\times V^j \to V^{k+j}$ returns the $(k+j)$-vector that represents the parallelopiped formed by the sides represented by the given $k$-vector and $j$-vector. 
\end{defn}


\begin{remark}
	Notation for a generic vector. Infinitesimal displacement $dP$ is a vector and can be expressed as: $dP = dx \frac{\partial P}{\partial x^i}$. We can set $e_i = \frac{\partial P}{\partial x^i}$ so $dP = dx^i e_i$.
	
	Every infinitesimal $k$-surface $d\sigma^k$ can be expressed, in terms of the wedge product, as $d\sigma^k= dx^{i_1}dx^{i_2}...dx^{i_k}\frac{\partial P}{\partial x^1} \wedge \frac{\partial P}{\partial x^2} \wedge ... \wedge \frac{\partial P}{\partial x^k} = dx^{i_1}dx^{i_2}...dx^{i_k} e_1 \wedge e_2 \wedge ... \wedge e_k$.
	
	Suppose we have a $k$-surface in terms of $k$ coordinates $s^j$. We will have a differentiable function $x^i = x^i(s^j)$ that maps the parametrization of the $k$-surface into the manifold. At each point $P$, we can write $dx^i = \frac{\partial x^i}{\partial s^j} ds^j$. Therefore we have $d\sigma^k = dx^{i_1}dx^{i_2}...dx^{i_k} e_1 \wedge e_2 \wedge ... \wedge e_k$.
	
	For example:
	\begin{align*}
		x^i &= \{x,y,z\} \\
		s^j &= \{\varphi, \theta\} \\
		x &= \sin \varphi \cos \theta \\ 
		y &= \sin \varphi \sin \theta \\ 
		z &= \cos \varphi \\ 
		d\sigma &= d\varphi d\theta (e_\varphi \wedge e_\theta) \\
		&=d\varphi d\theta \left(\frac{\partial x}{\partial \varphi} e_x + \frac{\partial y}{\partial \varphi} e_y + \frac{\partial z}{\partial \varphi} e_z\right) \wedge \left(\frac{\partial x}{\partial \theta} e_x + \frac{\partial y}{\partial \theta} e_y + \frac{\partial z}{\partial \theta} e_z\right) \\
		&=d\varphi d\theta (
\frac{\partial x}{\partial \varphi} e_x \wedge \frac{\partial x}{\partial \theta} e_x +
\frac{\partial x}{\partial \varphi} e_x \wedge \frac{\partial y}{\partial \theta} e_y +
\frac{\partial x}{\partial \varphi} e_x \wedge \frac{\partial z}{\partial \theta} e_z + \\
&\frac{\partial y}{\partial \varphi} e_y \wedge \frac{\partial x}{\partial \theta} e_x +
\frac{\partial y}{\partial \varphi} e_y \wedge \frac{\partial y}{\partial \theta} e_y +
\frac{\partial y}{\partial \varphi} e_y \wedge \frac{\partial z}{\partial \theta} e_z + \\
&\frac{\partial z}{\partial \varphi} e_z \wedge \frac{\partial x}{\partial \theta} e_x +
\frac{\partial z}{\partial \varphi} e_z \wedge \frac{\partial y}{\partial \theta} e_y + 
\frac{\partial z}{\partial \varphi} e_z \wedge \frac{\partial z}{\partial \theta} e_z ) \\
		&=d\varphi d\theta ((
\frac{\partial x}{\partial \varphi} \frac{\partial y}{\partial \theta} - \frac{\partial y}{\partial \varphi}\frac{\partial x}{\partial \theta}) e_x \wedge e_y + \\
& (\frac{\partial y}{\partial \varphi} \frac{\partial z}{\partial \theta} - \frac{\partial z}{\partial \varphi} \frac{\partial y}{\partial \theta}) e_y \wedge e_z + \\
&\frac{\partial z}{\partial \varphi} \frac{\partial x}{\partial \theta} - \frac{\partial x}{\partial \varphi} \frac{\partial z}{\partial \theta}) e_z \wedge e_x ) \\
		&=d\varphi d\theta ((
\cos \varphi \cos \theta \sin \varphi \cos \theta - \cos \varphi \sin \theta \sin \varphi (-\sin \theta)) e_x \wedge e_y + \\
& (\cos \varphi \sin \theta  0 - (- \sin \varphi) \sin \varphi \cos \theta) e_y \wedge e_z + \\
& - \sin \varphi \sin \varphi (-\sin \theta) - \cos \varphi \cos \theta 0) e_z \wedge e_x ) \\
		&=d\varphi d\theta (\cos \varphi \sin \varphi e_x \wedge e_y +
\sin^2 \varphi \cos \theta e_y \wedge e_z +
\sin^2 \varphi \sin \theta e_z \wedge e_x )
	\end{align*}
\end{remark}

\begin{defn}
	A \textbf{$k$-form} $\omega_k : V^k \to \mathbb{R}$ is a linear function of a vector. We note $\Omega_k$ as the set of all $k$-forms of dimension k and $\Omega = \cup_{k=0}^n\Omega_k$ as the set of all forms. 
\end{defn}

\begin{prop}
	TODO Show that every k-functional has a corresponding k-form, such that for $f_k = \int_{\sigma^k} \omega_k(d\sigma^k)$. In words, a linear functional applied over a k-surface is the same as an integral of a k-form over the infinitesimal parallelepipedes of that k-surface. 
\end{prop}

\fi




